
\documentclass[12pt]{book}

\usepackage[dvipsnames]{xcolor}
\usepackage{amssymb,latexsym}
\usepackage{graphicx}

\usepackage[spanish,mexico,es-nolayout]{babel}
\usepackage[utf8]{inputenc}
\usepackage{amsmath,amscd}
%\usepackage{amssymb}
\usepackage{amsthm}
%\usepackage{graphicx}
\usepackage{color}
\usepackage{tikz}
\usepackage{tkz-berge}
\usepackage{makeidx}
\usepackage{url}
\usepackage{xspace}
\usepackage{tocbibind}
\usepackage{rotating}
% ver http://gilmation.com/articles/latex-margins-for-book-binding/
% y http://tex.stackexchange.com/questions/50258/margins-of-book-class
\usepackage[margin=3.5cm]{geometry}
\geometry{bindingoffset=1cm}

\usepackage{babelbib}
\usepackage{rotating}

\usepackage{changepage}                 % adjust margins for selected portions

% wide page for side by side figures, tables, etc
% http://tex.stackexchange.com/a/154766/250
\newlength{\offsetpage}
\setlength{\offsetpage}{2.0cm}
\newenvironment{widepage}{\begin{adjustwidth}{-\offsetpage}{-\offsetpage}%
    \addtolength{\textwidth}{2\offsetpage}}%
{\end{adjustwidth}}


\usetikzlibrary{positioning,shapes,fit,arrows,decorations.pathmorphing}
\definecolor{myblue}{RGB}{56,94,141}


\newtheorem{theorem}{Teorema}[section]
\newtheorem{corollary}[theorem]{Corolario}
\newtheorem{proposition}[theorem]{Proposición}

\theoremstyle{definition}

\newtheorem{definition}[theorem]{Definición}
\newtheorem{notation}[theorem]{Notación}
\newtheorem{example}[theorem]{Ejemplo}
\newtheorem{lemma}[theorem]{Lema}

\DeclareMathOperator{\im}{im}
\DeclareMathOperator{\sgn}{sgn}

\newcounter{in}
\newcounter{ini}

\makeindex

\newcommand{\elespacio}{1.4cm}

\begin{document}
\mainmatter 
\begin{titlepage}
  \begin{center}
    \null
    \vspace*{\fill}

    \includegraphics[scale=1.2,bb=55 20 0 0]{escudouaeh.pdf}

    \vspace*{\elespacio}

    \textsc{Universidad Autónoma del Estado de Hidalgo}

    \textsc{Instituto de Ciencias Básicas e Ingeniería}

    \textsc{Área Académica de Matemáticas y Física}

    \vspace*{\elespacio}

    {\Huge\bfseries Representaciones del grupo simétrico en homologías\par}

    \vspace*{\elespacio}

    {\large Tesis que para obtener el título de}

    \vspace*{\elespacio}

    {\Large\textsc{Licenciada en Matemáticas Aplicadas}}

    \vspace*{\elespacio}

    {\large presenta}

    \vspace*{\elespacio}

    {\Huge Briseida Guadalupe Trejo Escamilla}

    \vspace*{\elespacio}

    {\large bajo la dirección de}

    \bigskip

    {\Large Dr.~Rafael Villarroel Flores}

    \bigskip

    {Pachuca, Hidalgo. Junio de 2013.}

    \vspace*{\fill}

  \end{center}
\end{titlepage}

\thispagestyle{empty}
\begin{flushleft}
  {\bfseries\Large Resumen}
\end{flushleft}

El presente trabajo tiene como objetivo analizar la estructura de
la homología de complejos simpliciales construidos a partir de una
gráfica simple finita en donde actúa el grupo simétrico $S_{n}$.

Se define la gráfica $M(G)$ de emparejamientos de la gráfica simple
$G$ como la gráfica cuyos vértices son las aristas de $G$ y dos
vértices son adyacentes si las correspondientes aristas son ajenas. En
la gráfica $G_{n}=M(K_{n})$, el grupo $S_{n}$ actúa de manera natural, por
lo que sus homologías con coeficientes complejos definen
representaciones de $S_{n}$. 

La descomposición en irreducibles de tales homologías ha sido exhibida
por Bouc (1992). En el presente trabajo se muestra el resultado
correspondiente para la gráfica de clanes $K(G_{6})$.

\vspace{2cm}

\begin{flushleft}
  {\bfseries\Large Abstract}
\end{flushleft}

In this thesis blah blah blah blah blah blah blah blah blah
blah blah blah blah blah blah blah blah blah.


\chapter*{Introducción}

(Resumen extendido)

En el primer capítulo...

\tableofcontents


 \newpage \thispagestyle{empty}

\chapter{Repaso de álgebra lineal}
%\label{cha:repaso-algebra-line}
\section{Espacios vectoriales}

En este capítulo se revisan algunos resultados de álgebra lineal, las
pruebas se omiten pero se puede consultar algún texto de álgebra
lineal como \cite{friedberg1982algebra}
  para más  detalles. En este trabajo todos los espacios vectoriales
  considerados serán dimensionalmente finitos y sobre los complejos.

\begin{definition}
  Sea $W$ un subespacio de un espacio vectorial $V$. Para toda $v\in V$ el conjunto $\{v\}+W=\{v+w:w\in W\}$ se
  llama \textbf{clase lateral de $W$ que contiene a $v$}. Es frecuente
  expresar esta clase lateral como $v+W$ en vez de $\{v\}+W$. 
\end{definition}

Se puede demostrar lo siguiente:
 
$v_{1}+W=v_{2}+W$ si y solo si $v_{1}-v_{2}\in W.$

La suma y el producto puede definirse en el conjunto $S=\{v+W:v\in
V\}$ de todos las clases laterales de $W$ como: 

$(v_{1}+W)+(v_{2}+W)=(v_{1}+v_{2})+W$ para toda $v_{1}$, $v_{2}\in V$ y

$a(v+W)=av+W$ para toda $v\in V$ y $a\in \mathbb{C}$.

Se puede demostrar que las operaciones anteriores están bien definidas.

\begin{definition}
  El conjunto $S$ es un espacio vectorial bajo las operaciones
  definidas anteriormente y se llama \textbf{espacio cociente de $V$ módulo $W$} y se denota mediante $V/W$. 
\end{definition}

\begin{theorem}
  \label{dim-esp-coc}
  Sea $(V/W)$ el espacio cociente de $V$ módulo $W$, se tiene:
  $$\dim(V/W)=\dim(V)-\dim(W)$$
\end{theorem}

\begin{theorem}
  \label{dim-esp-vec}
  Sean $W_{1}$ y $W_{2}$ subespacios de un espacio vectorial
  $V$. Entonces, 
  $$\dim(W_{1}+W_{2})=\dim(W_{1})+\dim(W_{2})-\dim(W_{1}\cap W_{2}).$$ 
\end{theorem}

\begin{definition}
  Si $W_{1}$ y $W_{2}$ son dos subconjuntos no vacíos de un espacio
  vectorial $V$, entonces la \textbf{suma} de $W_{1}$ y $W_{2}$, que se
  expresa como $W_{1}+W_{2}$, es el conjunto $\{x+y:x\in W_{1}$ y $y\in
  W_{_2}\}$. La suma de cualquier número finito de subconjuntos no
  vacíos de $V$, $W_{1},\cdots,W_{n}$, se define análogamente como el
  conjunto
  $$W_{1}+\ldots+W_{n}=\{x_{1}+\ldots+x_{n}: x_{i}\in W_{i} \mbox{ para }i=1,2,\ldots,n\}$$
\end{definition}

\begin{definition}
  Un espacio vectorial $V$ es la \textbf{suma directa de $W_{1}$ y
  $W_{2}$}, denotada como $V=W_{1}\oplus W_{2}$, si $W_{1}$ y $W_{2}$
son subespacios de $V$ tales que $W_{1}\cap W_{2}=\{0\}$ y $W_{1}+W_{2}=V.$ 
\end{definition}

\begin{theorem}
  Sean $W_{1}$ y $W_{2}$ subespacios de $V$ tales que
  $V=W_{1}+W_{2}$. Luego, $V=W_{1}\oplus W_{2}$ si y solo si 
  $$\dim(V)=\dim(W_{1})+\dim(W_{2}).$$
\end{theorem}

\begin{theorem}
  \label{clunica}
  Sea $V$ un espacio vectorial y $\beta=\{v_{1},\dots,v_{n}\}$ un
  subconjunto de $V$. Luego $\beta$ es una base de $V$ si y solo si
  cada vector $y\in V$ puede ser expresado de manera única como una
  combinación lineal de vectores de $\beta$, es decir, puede ser
  expresado de la forma
  $$y=a_{1}v_{1}+\ldots+a_{n}v_{n}$$
  para escalares únicos $a_{1},\ldots,a_{n}.$
\end{theorem}

\begin{theorem}
  \label{esp-iguales}
  Sea $W$ un subespacio de un espacio vectorial $V$ de dimensión
  $n$. Entonces, $W$ es dimensionalmente finito y $\dim(W)\leq
  n$. Además, si $\dim(W)=n$, entonces $W=V$.
\end{theorem}

\section{Transformaciones lineales}

\begin{theorem}
  \label{imT}
  Sean $V$ y $W$ espacios vectoriales y sea $T:V \rightarrow W$
  lineal. Si $V$ tiene una base $\beta=\{v_{1},\ldots,v_{n}\}$,
  entonces $\im(T)=\langle T(v_{1}),\ldots,T(v_{n})\rangle.$  
\end{theorem}

\begin{theorem}
  \label{esp-isomorfos}
  Sean $V$ y $W$ espacios vectoriales (sobre el mismo campo
  $F$). Entonces $V$ es isomorfo a $W$ si y solo si $\dim (V)=\dim(W).$ 
\end{theorem}

\begin{theorem}{(\textbf{Teorema de Isomorfismo})}
  \label{teorema-isomorfismo-esp}
  Si $f:V\rightarrow W$ es una transformación lineal, el espacio
  cociente $V/\ker(f)\cong \im(f)$. Un isomorfismo entre estos dos
  espacios es el siguiente:
  $$\phi:V/\ker(f)\rightarrow \im(f)$$
  definido por $\phi(v+\ker(f))=f(v).$
\end{theorem}
\begin{proof}[Demostración.]
  Hay que demostrar que $\phi$ está bien definida, es decir, que si
  $v_1+\ker(f)=v_2+\ker(f)$, entonces
  $f(v_1)=f(v_{2}).$ Pero
  $v_1+\ker(f)=v_2+\ker(f)$ si y sólo si
  $v_{1}-v_{2}\in \ker(f)$, es decir,
  $f(v_{1}-v_{2})=f(v_1)-f(v_2)=0$,
  como se quería. Queda por demostrar de forma directa que $\phi$ es
  una transformación lineal; y para demostrar que es isomorfismo falta
  demostrar que es inyectiva y sobreyectiva, lo cual también se hace
  de forma directa.
\end{proof}

\chapter{Representaciones de grupos}
\label{Representaciones de grupos}

\section{Grupos}

\begin{definition}
  Una \textbf{operación binaria} en un conjunto $G$ es una función
  de la forma $G  \times G \rightarrow G$. Para cada $(a,b)\in G
  \times G$, denotaremos al elemento $*((a,b))\in G$ por $ab$. 
\end{definition} 

\begin{definition} 
  Un \textbf{grupo} es un conjunto no vacío $G$, junto con una
  \textbf{operación binaria} $*$ que satisface las siguientes condiciones:
    \begin{enumerate}
    \item La operación es asociativa, es decir, $$a(bc)=(ab)c$$ para todo $a,b,c \in G$.
    \item Existe un elemento neutro $1 \in G$ que
      satisface $$a1=1a=a$$ para todo $a \in G$.
    \item Para cada elemento $a \in G$ existe otro elemento $a' \in G$
      tal que $$aa'=a'a=1$$ Al elemento $a'$ se le llama inverso del elemento $a$.
    \end{enumerate}
\end{definition}

\begin{definition}
  Si $G$ y $H$ son grupos, entonces un \textbf{homomorfismo} de $G$
  en $H$ es una función $\phi:G\rightarrow H$ la cual
  satisface $$\phi(ab)=\phi(a)\phi(b)$$ para todo $a,b \in G$
\end{definition}

\begin{theorem}
  Sea $\rho:G\rightarrow G^{'}$ un homomorfismo de grupos. Entonces
  \begin{enumerate}
    \item $\rho(1)=1^{'}$, donde $1\in G$ y  $1^{'}\in G^{'}$ son los
    neutros respectivos. 
    \item Si $g\in G$, entonces $\rho(g^{-1})=\rho (g)^{-1}$.
  \end{enumerate}
\end{theorem}

La demostración del teorema anterior es elemental, por lo tanto la omitiremos.

\begin{definition}
  Cuando un homomorfismo de grupos $\phi:G\rightarrow H$ es biyectivo,
  diremos que $\phi$ es un  \textbf{isomorfismo}. También diremos que
  $G$ y $H$ son grupos  \textbf{isomorfos} cuando exista un
    isomorfismo entre ellos, usaremos la notación $G\cong H$.
\end{definition}

\begin{theorem}
  Supongamos que $G$ y $H$ son grupos y sea $\phi:G\rightarrow H$ un
  homomorfismo. Entonces $$G/\ker \phi\cong \im \phi$$ 
\end{theorem}

  Denotemos $GL(n,\mathbb{C})$ al grupo de matrices
  invertibles  $n \times n$ con entradas en $\mathbb{C}$.  
  Y a $GL(V)$ a los operadores lineales invertibles en un
  $\mathbb{C}$-espacio vectorial $V$ de dimensión finita $n$.
  Además $GL(n,\mathbb{C})\cong GL(V)$ es un isomorfismo de grupos.

\section{Acciones de grupos}

\begin{definition}
  Sea $G$ un grupo y $Y$ un conjunto no vacío. Una  \textbf{acción de $G$
  en $Y$} es una función $*:G \times Y \rightarrow Y$ tal que
\begin{enumerate}
\item $1*x=x$ para todo $x\in Y.$
\item $(g_{1}g_{2})*x=g_{1}*(g_{2}*x)$ para todo $x\in Y$ y $g_{1},g_{2}\in G.$
\end{enumerate}
   Bajo estas condiciones, $Y$ es un $G$-conjunto. En ocasiones, por
   abuso de notación, en lugar de $g*x$ usaremos la notación $gx.$
\end{definition}

\begin{definition}\textbf{Acciones lineales.}
  Sea $G$ un grupo. Una acción de $G$ en un $K$-espacio
  vectorial $V$ de dimensión finita es una función
 $$*:G\times V \rightarrow V $$
que satisface los axiomas:
\begin{enumerate}
\item $1v=v$, para todo $v\in V$ (donde $1$ es el neutro de $G$).
\item $(g_{1}g_{2})v=g_{1}(g_{2}v)$ para todos $v\in V$ y
  $g_{1},g_{2}\in G$.
\item $g_{1}(v+w)=g_{1}v+g_{1}w$, para $g_{1}\in G$ y $v,w \in V .$
\item $g_{1}(\lambda v)=\lambda(g_{1}v)$, para $\lambda \in K$,
  $v\in V$ y $g_{1}\in G.$
\end{enumerate}
Diremos que la acción es $G$-lineal y que $V$ es un $G$-espacio
vectorial.
\end{definition} 

\begin{theorem}
  Existe una correspondencia biyectiva entre el conjunto de acciones
  lineales de un grupo $G$ en un $K$-espacio vectorial $V$ y el conjunto
  de homomorfismos de $G$ en $GL(V)$
\end{theorem}

  \textit{Demostración.} Supongamos primero que se tiene una acción lineal 
  $$*:G\times V \rightarrow V$$
  para cada $g\in G$, usando esta acción definamos la función $\rho
  g:V \rightarrow V$ dada por $(\rho g)v=gv$ para todo $v\in V$.

  Primero observemos que la función $\rho:G\rightarrow GL(V)$ es un
  homomorfismo de grupos ya que si $g,h\in G$, entonces para todo
  $v\in V$:

  $$(\rho(gh))v=(gh)v=g(hv)=g((\rho h)v)=(\rho g)((\rho h)v)=(\rho g \circ \rho h)v$$
  por lo que $\rho(gh)=\rho g \circ \rho h$

  Ahora demostremos que $\rho g$ es una transformación lineal
  invertible. Sean $v,w \in V$, $\lambda \in K$.

  \begin{flalign*}
   &(\rho g)(v+w)=g(v+w)=gv+gw=(\rho g)v+(\rho g)w\\
   &(\rho g)(\lambda v)=g(\lambda v)=\lambda(gv)=\lambda((\rho g)v)       
  \end{flalign*}

  Así que $\rho g$ es lineal, ahora veamos que es invertible. Como $G$
  es un grupo, existe $g^{-1}\in G$.

  \begin{eqnarray*}
     ((\rho g)(\rho g)^{-1})v=(\rho g)((\rho g)^{-1}v)=(\rho g)((\rho g^{-1})v)=(\rho g)(g^{-1}v)=g(g^{-1}v)=(gg^{-1})v=v\\
     ((\rho g)^{-1}(\rho g))v=(\rho g)^{-1}((\rho g)v)=(\rho g^{-1})((\rho g)v)=(\rho g^{-1})(gv)=g^{-1}(gv)=(g^{-1}g)v=v
   \end{eqnarray*}

Suponga ahora que se tiene un homomorfismo de grupos $\rho:G\rightarrow GL(V)$, por demostrar que $gv:=(\rho g)v$, define un
acción lineal de $G$ en $V$. 

\begin{enumerate} 
   \item $1$ es el neutro de $G$, entonces $$1v=(\rho 1)v=id_{V}=v$$ para
     todo $v\in V$. 
   \item Si $g,h\in G$ y $v\in V$, entonces $$(gh)v=(\rho g h)v=(\rho
     g \circ \rho h)v=\rho g((\rho h)v)=\rho g(hv)=g(hv)$$
   \item Para $g\in G$ y $v,w\in V$, $$g(v+w)=(\rho g)(v+w)=(\rho g)v+(\rho g)w=gv+gw$$
   \item Para $\lambda\in K$, $v\in V$ y $g\in G$, 
    $$g(\lambda v)=(\rho g)(\lambda v)=\lambda((\rho g)v)=\lambda(gv)$$
\end{enumerate}

$\qed$

\section{Representaciones de grupos}

\begin{definition}
  Si $G$ es un grupo y $V$ es un $K$-espacio vectorial, una
  \textbf{representación lineal} de $G$ en $V$ es un homomorfismo 
     $$\rho:G\rightarrow GL(V).$$
  Se denotará $(\rho,V)$ para enfatizar el hecho de que la
  representación $\rho$ de $G$  es sobre el espacio lineal $V$. A
  dicho espacio lineal $V$ se le llamará \emph{espacio de representación} y a
  su dimensión el \emph{grado} de la representación 
\end{definition}

Estudiar las representaciones lineales de un grupo $G$ en un espacio
vectorial $V$, es equivalente a estudiar las acciones lineales de $G$
en $V$. 

En este trabajo estudiaremos representaciones de grupos finitos en
espacios vectoriales complejos y de dimensión finita. 

\begin{example}(\textbf{Representación trivial})
  La representación trivial de un grupo $G$ es el homomorfismo
  $\rho:G\rightarrow \mathbb{C^{*}}$ dado por $\rho(g)=1$ para todo
  $g\in G$.
\end{example}

\begin{example}(\textbf{Representación signo})
  El grupo simétrico sobre $\mathbb{I}_{n}=\{1,2,\ldots,n\}$, denotado
  por $\mathcal{S}_{n}$ es el grupo de todas las permutaciones de $\mathbb{I}_{n}$ 
  Si $\pi=\tau_{1}\tau_{2}\cdots\tau_{k}$, donde $\tau_{i}$ son
  transposiciones, definimos la función signo
  $\sgn:\mathcal{S}_{n} \rightarrow\{\pm1\}$ mediante $$\sgn(\pi):=(-1)^{k}.$$
\end{example}

\begin{example}(\textbf{Representación regular})
  Sea $G$ cualquier grupo de orden $n$ y sea V el espacio vectorial de
  dimensión $n$ con base $\mathcal{B}=\{v_{g}\}_{g\in G}$ indexada por
  los elementos del grupos. Para cada $\sigma\in G$ definamos la función
  $\rho(\sigma):\mathcal{B}\rightarrow \mathcal{B}$ mediante $\rho(\sigma)(v_{g})=v_{\sigma g}$.
  Mostremos que la función $$\rho:G\rightarrow GL(V)$$ dada por
  $\sigma\mapsto\rho(\sigma)$ es un homomorfismo. Si $\sigma,\tau \in G$
  y si $v_{g}\in \mathcal{B}$, entonces $$\rho(\sigma
  \tau)v_{g}=v_{(\sigma\tau)g}=v_{\sigma(\tau g)}=\rho(\sigma)v_{\tau
    g}=\rho(\sigma)(\rho(\tau)v_{g})=\rho(\sigma)\rho(\tau)v_{g} $$
  así que $\rho(\sigma\tau)=\rho(\sigma)\rho(\tau)$. Por definición el 
  grado de la representación es el orden de $G$.
\end{example}

\section{Módulos irreducibles y submódulos}
\label{mod-irr-submodulos}
Denote $G$ un grupo y $V$ un espacio vectorial sobre
el campo de los complejos. 

\begin{definition}
  Sea $V$ un $G$-módulo no trivial. Decimos que $V$ es \textbf{irreducible} si
  los únicos submódulos de $V$ son $0$ y $V$.
\end{definition}

\begin{definition}
  Sea $V$ un $G$-módulo. Un \textbf{submódulo} de $V$ es un subespacio
  $W$ que es invariante bajo la acción de $G$, es decir, $ gw\in W$
  para todos $w\in W$, $g\in G$, tal que $W$ junto con la acción de
  $G$ es en sí mismo un $G$-módulo. Escribiremos $W\leq V$ si $W$ es
  submódulo de $V$. 
\end{definition}

\begin{theorem}
  \label{interseccion-submodulos}
  Sea $V$ un $G$-módulo. Entonces la intersección de cualquier
  colección de submódulos de $V$ es un submódulo de $V$.
\end{theorem}

\begin{definition}
  Sea $V$ un $G$-módulo. Al submódulo de $V$ generado por las combinaciones lineales de los
  elementos de $W$ lo llamaremos \textbf{submódulo generado} por $W$ y lo denotaremos como $\langle W\rangle$.
\end{definition}

Sean $U$ y $V$ dos espacios vectoriales. Llamaremos \textbf{suma directa} de
$U$ y $V$ al conjunto $U\times V=\{(u,v):u\in U,v\in V\}$ con las operaciones
$$(u,v)+(u_{1},v_{1})=(u+u_{1},v+v_{1})$$
$$k(u,v)=(ku,kv),$$
donde $u,u_{1}\in U$, $v,v_{1}\in V$ y $k\in \mathbb{C}$. Con estas
operaciones $U\times V$ es un espacio vectorial, que designaremos por
$$U\oplus V.$$

Así que dados $V$ y $W$ $G$-módulos, podemos formar un tercer módulo a partir
de la suma directa $V\oplus W$, definiendo la acción como
$a(v,w)=(av,aw)$, con $a\in G$. Podemos extender esta definición a cualquier
cantidad finita de $G$-módulos.

Por otro lado, si $V$ es un $G$-módulo, se llama la \textbf{suma} de $W_{1}$ y $W_{2}$ y se denota con
$W_{1}+W_{2}$ al submódulo de $V$ generado por $W_{1}\cup W_{2}$. Ésta definición se puede extender a una
colección arbitraria $\{W_{i}\}_{i\in I}$ es
decir, $\sum_{i\in I}W_{i}$ es el submódulo generado por $\bigcup_{i\in I}W_{i}$.


\textbf{\emph{Observación:}} En el caso de que $W_{1}$ y $W_{2}$ son
submódulos de $V$ con $W_{1}\cap W_{2}=0$ tenemos que
$W_{1}+W_{2}\cong W_{1}\oplus W_{2}$.

% dado por el siguiente isomorfismo:
% \begin{align*}
%   \phi:W_{1}+W_{2}&\rightarrow W_{1}\oplus W_{2}\\
%   w & \mapsto  (w_{1},w_{2})
% \end{align*}
% sean
% $w^{1}_{1},w^{2}_{1},\ldots,w^{n}_{1},w^{1}_{2},w^{2}_{2},\ldots,w^{m}_{2}$
% los elementos de $W_{1}\cup W_{2}$ donde $w^{i}_{1}\in W_{1}$ y
% $w^{i}_{2}\in W_{2}$, tomemos $w\in W_{1}+W_{2}$, es decir,
% \begin{align*}
%   w&=a_{1}w^{1}_{1}+a_{2}w^{2}_{1}+\ldots+a_{n}w^{n}_{1}+a_{n+1}w^{1}_{2}+a_{n+2}w^{2}_{2}+\ldots+a_{m}w^{m}_{2}\\
%   &=w_{1}+w_{2}
% \end{align*}
% donde $w_{1}=a_{1}w^{1}_{1}+a_{2}w^{2}_{1}+\ldots+a_{n}w^{n}_{1}$ y
% $w_{2}=a_{n+1}w^{1}_{2}+a_{n+2}w^{2}_{2}+\ldots+a_{m}w^{m}_{2}$,
% $w_{1}\in W_{1}$ puesto que $w_{1}$ es combinación lineal de los
% elementos de $W_{1}$, de forma análoga concluimos que $w_{2}\in
% W_{2}$. Ahora veamos que $\phi$ está bien definida, supongamos que
% $w=w_{1}+w_{2}=w^{'}_{1}+w^{'}_{2}$, así que
% $w_{1}-w^{'}_{1}=w^{'}_{2}-w_{2}$ y notamos que $w_{1}-w^{'}_{1}\in W_{1}$ y
% $w^{'}_{2}-w_{2}\in W_{2}$, así que
% $w_{1}-w^{'}_{1}=w^{'}_{2}-w_{2}\in W_{1}\cap W_{2}$, entonces
% $w_{1}-w^{'}_{1}=w^{'}_{2}-w_{2}=0$ por hipótesis, por lo tanto
% $w_{1}=w^{'}_{1}$ y $w^{'}_{2}=w_{2}$.
\begin{proposition}
  \label{modulos-iguales}
  Supongamos que $S\leq M$, $S$ irreducible, si $M=S\oplus T$ donde $S$
  tiene multiplicidad uno y sea $S^{'}\leq M$, tal que $S^{'}\cong S$, entonces $S=S^{'}$.
\end{proposition}

\begin{proof}[Demostración.]
  Consideremos $S\cap S^{'}\leq S$, así que $S\cap S^{'}=0$ o $S\cap
  S^{'}=S$, pues $S$ es irreducible.

  Si $S\cap S^{'}=0$, consideremos $U$ el submódulo generado por $S\cup S'$ al cual
  denotamos como $S+S^{'}$ y como $S\cap S^{'}=0$, entonces
  $S+S^{'}\cong S\oplus S^{'}$ como notamos en la observación anterior,
  así que $M$ tendría un submódulo isomorfo a dos copias de S, lo cual
  no es posible por hipótesis. 
  
 Por otro lado, si $S\cap S^{'}=S$, entonces $\dim (S\cap
 S^{'})=\dim(S)$, además por hipótesis $S\cong S^{'}$ así que
 $\dim(S)=\dim(S^{'})$. Por lo tanto $\dim(S\cap S^{'})=\dim(S^{'})$ y
 sea $S\cap S^{'}\leq S^{'}$ por el teorema \ref{esp-iguales} tenemos
 que $S\cap S^{'}=S^{'}$, con lo que $S=S^{'}$ como se quería demostrar.
\end{proof}

\begin{theorem}[Lema de Schur]
  \label{lema-schur}
  Si $V$ y $W$ son $G$-módulos irreducibles, y $\phi:V\rightarrow W$
  es un morfismo de $G$-módulos no trivial, entonces $\phi$ es un isomorfismo.
\end{theorem}

\begin{proof}[Demostración.]
  Como $\ker \phi\leq V$, y $V$ es irreducible, entonces $\ker \phi=0$
  o $\ker\phi=V$, pero  $\ker\phi\neq V$ pues $\phi$ es no trivial,
  así que $\ker \phi=0$ y por lo tanto $\phi$ es inyectiva. 

  Análogamente, $\im\phi\leq W$ y $\im \phi\neq 0$, así que $\im
  \phi=W$ y entonces $\phi$ es suprayectiva.
\end{proof}

\begin{proposition}
  \label{im-mod-irreducible}
   Si $V$ y $W$ son $G$-módulos, $f:V\rightarrow W$ un morfismo de $G$-módulos, $S\leq V$ con $S$
  irreducible, entonces $f(S)\cong S$ o $f(S)=0$.
\end{proposition}

\begin{proof}[Demostración.]
  Tomemos el siguiente morfismo de módulos
  $S\stackrel{i}{\hookrightarrow} V\stackrel{f}{\rightarrow}W$ donde
  $i$ es la función inclusión. Así que $S/\ker(f\circ i)\cong\im(f\circ
  i)$ por el teorema \ref{teorema-isomorfismo-mod}. Como $\ker(f\circ
  i)\leq S$ entonces $\ker(f\circ i)=0$ o $\ker(f\circ i)=S$, pues $S$ es irreducible.

  Si $\ker(f\circ i)=0$, se sigue que
  $$S\cong S/0\cong\im(f\circ i)=f(S)$$

  Por otro lado, si $\ker(f\circ i)=S$, tenemos
  $$0=S/S\cong\im(f\circ i)=f(S)$$
\end{proof}
\bigskip
\begin{theorem}[Teorema de isomorfismo de módulos]
  \label{teorema-isomorfismo-mod}
  Sean.....
\end{theorem}
\chapter{Homología de complejos simpliciales}
%\label{cha:primer-capitulo}

\section{Complejos simpliciales abstractos}

\begin{definition}
Un \textbf{complejo simplicial abstracto} es una colección finita
$\Delta$ de conjuntos no vacíos, tal que si $A$ es un elemento de $\Delta$,
cada subconjunto no vacío de $A$ pertenece a $\Delta$.
\end{definition}

El elemento $A$ de $\Delta$ es llamado \textbf{simplejo} de
$\Delta$; Si $A$ tiene $p+1$ elementos, decimos que $A$ es un
\emph{p-simplejo} y su dimensión es $p$. La dimensión de $\Delta$
es el máximo de las dimensiones de los simplejos de $\Delta$. Cada
subconjunto no vacío de $A$ es llamado \textbf{cara} de $A$. El
conjunto \textbf{vértices} $V$ de $\Delta$ es la unión de los
elementos de un punto de $\Delta$; no hacemos distinción entre los
vértices $v\in V$ y los $0$-simplejos $\{v\}\in \Delta$. Una
subcolección de $\Delta$, que es a su vez es un complejo, se llama
\textbf{subcomplejo} de $\Delta$.

\begin{definition}
  El subcomplejo de $\Delta$ que consiste de todos los simplejos de
  $\Delta$ de dimensión a lo más $p$, se llama \textbf{$\boldsymbol{p}$-esqueleto} de
  $\Delta$ y se denota por $\Delta^{(p)}$. El $1$-esqueleto
  $\Delta^{1}$ de cualquier complejo simplicial de $\Delta$ es una
  gráfica.
\end{definition}

\begin{definition}
  Un complejo simplicial $\Delta$ es \textbf{conexo} si
  $\Delta^{1}$ es una gráfica conexa, es decir, si existe un camino en
  $\Delta^{1}$  para cualquiera dos vértices.
\end{definition}

Primero introducimos la idea de un \emph{n-simplejo orientado}. Un
\textbf{0-simplejo orientado} es un punto $v$. Un \textbf{1-simplejo
  orientado} es un segmento de línea dirigido $v_{1}v_{2}$ uniendo los
puntos $v_{1}$ y $v_{2}$ en dirección de $v_{1}$ a $v_{2}$ ver \setlength{\fboxsep}{0pt}\colorbox{green}{FIGURA(referencia)}, así que
$v_{1}v_{2}\neq v_{2}v_{1}$, pero estaremos de acuerdo en que
$v_{1}v_{2}=-v_{2}v_{1}$. Un \textbf{2-simplejo orientado} es una
región triangular $v_{1}v_{2}v_{3}$ con dirección de $v_{1}$ a $v_{2}$
a $v_{3}$ ver \setlength{\fboxsep}{0pt}\colorbox{green}{FIGURA(referencia)}, claramente $v_{1}v_{2}v_{3}$ tiene el mismo orden que
$v_{2}v_{3}v_{1}$ y $v_{3}v_{1}v_{2}$, pero con orientación opuesta a
$v_{1}v_{3}v_{2}$, $v_{3}v_{2}v_{1}$ y $v_{2}v_{1}v_{3}$, es decir, estaremos de acuerdo que:
$$v_{1}v_{2}v_{3}=v_{2}v_{3}v_{1}=v_{3}v_{1}v_{2}=-v_{1}v_{3}v_{2}=-v_{3}v_{2}v_{1}=-v_{2}v_{1}v_{3}$$

Note que $v_{i}v_{j}v_{k}$ es igual a $v_{1}v_{2}v_{3}$ si

\[ \left(
  \begin{array}{ccc}
    1 & 2 & 3 \\
    i & j & k 
  \end{array} 
\right)\] 

es una permutación par y es igual a $-v_{1}v_{2}v_{3}$ si la
permutación es impar.

Un \textbf{3-simplejo orientado} está dado por una secuencia ordenada
$v_{1}v_{2}v_{3}v_{4}$ de cuatro vértices de un tetraedro sólido
ver \setlength{\fboxsep}{0pt}\colorbox{green}{FIGURA(referencia)} y
acordaremos que $v_{1}v_{2}v_{3}v_{4}=\pm v_{i}v_{j}v_{r}v_{s}$,
dependiendo si la permutación es par o impar. Así que dos
\emph{n-simplejos}, son equivalentes si difieren el uno del otro
por una permutación par. 

\begin{center}
  \begin{tikzpicture}
    \GraphInit[vstyle=Classic][scale=.2]
    \Vertex[x=0,y=0,Math]{v_{0}}
    \Vertex[x=1.5,y=1.5,Math]{v_{1}}
    \Edge[style={->}](v_{0})(v_{1})
  \end{tikzpicture}\quad
  \begin{tikzpicture}[scale=.6]
    \draw[help lines] (-2,0);% grid (2,3);
    \SetGraphUnit{2}
    \GraphInit[vstyle=Classic]
    \Vertex[x=-2,y=0,Math,LabelOut,Lpos=180,]{v_{0}}
    \Vertex[x=2,y=0,Math]{v_{1}}
    \Vertex[x=0,y=2.5,Math]{v_{2}}
    \Edge[style={->}](v_{0})(v_{1})
    \Edge[style={->}](v_{1})(v_{2})
    \Edge[style={->}](v_{2})(v_{0})
 \end{tikzpicture}\qquad
\begin{tikzpicture}[scale=.25]
  \SetVertexNoLabel
  \GraphInit[vstyle=Classic]
  \grTetrahedral[RA=4]
\end{tikzpicture}
\end{center}

\begin{definition}
  Sea $\Delta$ un complejo simplicial. Una \textbf{\emph{p}-cadena} en
  $\Delta$ es una función $c$ de el conjunto de $p$-simplejos de
  $\Delta$ a los complejos, tal que:
  \begin{enumerate}
    \item $c(\sigma)=-c(\sigma^{'})$ si $\sigma$ y $\sigma^{'}$ tienen
      direcciones opuestas del mismo simplejo.
    \item $c(\sigma)=0$ para todo $p$-simplejo orientado $\sigma$,
      excepto en un número finito de ellos.
  \end{enumerate} 
\end{definition}

Dado que en nuestro caso sólo estudiamos complejos simpliciales cuyo
conjunto de vértices es finito, la segunda condición no se aplica.

Sumamos $p$-cadenas sumando sus valores; el $\mathbb{C}$-espacio vectorial resultante es
denotado por $C_{p}(\Delta)$ y es llamado el \textbf{espacio de
  \emph{p}-cadenas (orientadas)} de $\Delta$. Si $p<0$ o $p>dim \Delta$,
$C_{p}(\Delta)$ denota al espacio trivial.

\begin{definition}
  Si $\sigma$ es un simplejo orientado, la \textbf{cadena elemental} $c$
  correspondiente a $\sigma$ es la función definida como:
  \[ 
  \begin{array}{cl}
    c(\sigma)=1, & \\
    c(\sigma^{'})=-1 & \mbox{si $\sigma^{'}$ tiene orientación opuesta de $\sigma$}, \\
    c(\tau)=0 & \mbox{para todos los otros simplejos orientados $\tau$}, 
  \end{array}\] 
  \end{definition}

Por abuso de notación, muchas veces usamos el símbolo $\sigma$ para
denotar no solo a un simplejo, o a un simplejo orientado, también
denotamos a la $p$-cadena elemental $c$ correspondiente al simplejo
orientado $\sigma$. Con esta convención, si $\sigma$ y $\sigma^{'}$
tienen orientaciones opuestas del mismo simplejo, entonces podemos
escribir $\sigma^{'}=-\sigma$, pues ésta ecuación se mantiene cuando
nos referimos a $\sigma$ y $\sigma^{'}$ como cadenas elementales.

\begin{lemma}
   Una base para $C_{p}(\Delta)$ se puede obtener
   tomando una orientación por cada $p$-simplejo y usando las
   correspondientes cadenas elementales como elementos de la base.
\end{lemma}

\textit{Demostración.} Orientando (arbitrariamente) a cada
\emph{p}-simplejo de $\Delta$, toda \emph{p}-cadena se puede escribir
de manera única como una combinación lineal finita
$$c=\sum n_{i}\sigma_{i},$$
de las correspondientes cadenas elementales $\sigma_{i}$. La
cadena $c$ asigna el valor $(n_{i})$ al \emph{p}-simplejo orientado
$\sigma_{i}$ , el valor $(-n_{i})$ a la orientación opuesta de
$\sigma_{i}$ y el valor $0$ a todo \emph{p}-simplejo orientado que no
aparece en la suma, se sigue del teorema $\ref{clunica}$. $\qed$

\begin{corollary}
  Toda función $f$ de los p-simplejos orientados de $\Delta$ en
  un espacio vectorial $V$ puede extenderse de manera única a una
  trasformación lineal de $C_{p}(\Delta)\rightarrow V$, tal que $f(-
  \sigma)=-f(\sigma)$ para todo p-simplejo orientado $\sigma$.
\end{corollary}

\begin{definition}
  Sea $\sigma=(v_{0},\ldots ,v_{p})$ un simplejo orientado con $p>0$
  (sin embargo también será denotado como $\sigma=v_{0}\ldots v_{p}$
  cuando no tengamos problemas de confusión),
  definimos la trasformación lineal $\partial_{p}:C_{p}(\Delta)\rightarrow
  C_{p-1}(\Delta)$ como:

  \begin{equation}
    \label{ofrontera}
    \partial_{p}(\sigma)=\partial_{p}(v_{0}\ldots
    v_{p})=\sum^{p}_{i=0}(-1)^{i}(v_{0}\ldots \widehat v_{i}\ldots v_{p}),
  \end{equation}
  al que llamamos el \textbf{\emph{p}-ésimo operador frontera}, donde
  $\widehat v_{i}$ indica que el vértice $v_{i}$ es borrado del arreglo.
\end{definition}

Puesto que $C_{p}(\Delta)$ es el espacio trivial para $p<0$, diremos
que $\partial_{p}$ es la \textit{trasformación cero} para $p\leq
0$. Mostremos ahora que $\partial_{p}$ está bien definido y que
$\partial_{p}(-\sigma)=-\partial_{p}(\sigma)$. Para esto es suficiente
mostrar que la ecuación $(\ref{ofrontera})$ cambia de signo si intercambiamos dos
vértices adyacentes en el orden $v_{0}\ldots v_{p}$, entonces,
debemos comparar las expresiones:
$$\partial_{p}(v_{0}\ldots v_{j} v_{j+1} \ldots v_{p}) \mbox{ y } \partial_{p}(v_{0}\ldots v_{j+1} v_{j} \ldots v_{p}).$$

Para $i\neq j$, $j+1$, el $i$-ésimo término de estas dos expresiones
difieren precisamente por un signo; los términos son idénticos,
excepto que $v_{j}$ y $v_{j+1}$ aparecen intercambiados. Veamos que
sucede sobre el i-ésimo término cuando $i=j$ y $i=j+1$. En la primera
expresión tenemos que:
$$(-1)^{j}(\ldots v_{j-1} \widehat v_{j}v_{j+1}v_{j+2}\ldots)+(-1)^{j+1}(\ldots v_{j-1}v_{j}\widehat v_{j+1}v_{j+2}\ldots).$$
en la segunda expresión tenemos:
$$(-1)^{j}(\ldots v_{j-1}\widehat v_{j+1}v_{j}v_{j+2}\ldots)+(-1)^{j+1}(\ldots v_{j-1}v_{j+1}\widehat v_{j}v_{j+2}\ldots).$$
Comparando estas dos expresiones observamos que solo difieren por un signo.

\begin{example}
  De acuerdo a lo anterior, tenemos que
  \begin{enumerate}
  \item para un \emph{1-simplejo}: $\partial_{1}(v_{0}v_{1})= v_{1}-v_{0}$,
  \item para un \emph{2-simplejo}: $\partial_{2}(v_{0}v_{1}v_{2})=v_{1}v_{2}-v_{0}v_{2}+v_{0}v_{1}$,
  \item para un \emph{3-simplejo}:
    $\partial_{3}(v_{0}v_{1}v_{2}v_{3})=v_{1}v_{2}v_{3}-v_{0}v_{2}v_{3}+v_{0}v_{1}v_{3}-v_{0}v_{1}v_{2}$. 
  \end{enumerate}
\end{example}

\section{$\partial^{2}=0$ y Homología simplicial}

\begin{definition}
   El kernel de $\partial_{p}:C_{p}(\Delta)\rightarrow
   C_{p-1}(\Delta)$ es llamado el espacio de
   \textbf{\emph{p}-ciclos} y denotado por $Z_{p}(\Delta)$. La imagen
   de $\partial_{p+1}:C_{p+1}(\Delta)\rightarrow C_{p}(\Delta)$ es
   llamado el espacio de \textbf{\emph{p}-fronteras} y es denotado por $B_{p}(\Delta)$.
\end{definition}

\begin{definition}
  Sea $\varepsilon:C_{0}\rightarrow \mathbb{C}$ la transformación
  lineal sobreyectiva definida por $\varepsilon(v)=1$ para cada
  vértice $v\in \Delta$. Entonces si $c$ es una $0$-cadena,
  $\varepsilon(c)$ es igual a la suma de los valores de $c$ en los
  vértices de $\Delta$, es decir:
  $$\varepsilon(\sum \lambda_{i}v_{i})=\sum\lambda_{i}$$
  $\varepsilon$ es llamada la \textbf{función aumento} para
  $C_{0}(\Delta)$.
\end{definition}

\begin{theorem}
  $\partial_{p-1}\circ\partial_{p}=0$ para cualquier $p$ y también $\varepsilon\circ\partial_{1}=0$.
\end{theorem}

\textit{Demostración.} Calculamos 
\begin{align*}
  \partial_{p-1}\partial_{p}(v_{0}\ldots
  v_{p})&=\sum_{i=0}^{p}(-1)^{i}\partial_{p-1}(v_{0}\ldots \widehat v_{i}\ldots v_{p})\\
  &=\sum_{j<i}(-1)^{i}(-1)^{j}(\ldots \widehat v_{j} \ldots \widehat v_{i} \ldots)\\
  &+\sum_{j>i}(-1)^{i}(-1)^{j-1}(\ldots\widehat v_{i}\ldots \widehat v_{j}\ldots).
\end{align*}

Los términos de estas dos sumas se cancelan a pares. $\qed$

\begin{corollary}
  $B_{p}(\Delta)$ es subespacio de $Z_{p}(\Delta)$.
\end{corollary}

\textit{Demostración.} Primero veamos que $B_{n}(\Delta)\subseteq Z_{n}(\Delta)$, tenemos que
$B_{n}(\Delta)=\partial[C_{n+1}(\Delta)]$, si $b\in B_{n}(\Delta)$,
podemos escribir a $b$ como $b=\partial_{n+1}(c)$ para algún $c\in
C_{n+1}(\Delta)$. Así que
$$\partial_{n}(b)=\partial_{n}(\partial_{n+1}(c))=0$$

por lo tanto $b\in Z_{n}(\Delta)$.
Las otras condiciones se siguen de que $\partial_{n+1}$ es una
transformación lineal. $\qed$

De forma análoga se demuestra que $B_{0}(\Delta)$ es subespacio de $\ker(\varepsilon)$.
% \begin{enumerate}
%   \item Tomemos $0\in C_{n+1}$, así que $\partial_{n+1}(0)=\widehat 0$,
%     por lo tanto $\widehat 0\in B_{n}(\Delta)$
%   \item Sean $b_{1}\in B_{n}(\Delta)$ y $b_{2}\in B_{n}(\Delta)$,
%     donde $b_{1}=\partial_{n+1}(c_{1})$ y $b_{2}=\partial_{n+1}(c_{2})$ para
%     algún $c_{1},c_{2}\in C_{n+1}(\Delta)$, así que $b_{1}+b_{2}\in B_{n}$
%     pues $b_{1}+b_{2}=\partial_{n+1}(c_{1})+\partial_{n+1}(c_{2})=\partial_{n+1}(c_{1}+c_{2})$
%     ya que $c_{1}+c_{2}\in C_{n+1}(\Delta)$.
%   \item Por último sea $a\in \mathbb{C}$, y $b\in B_{n}(\Delta)$ con
%     $b=\partial_{n+1}(c)$ para algún $c\in C_{n+1}(\Delta)$, notemos
%     que $ac\in C_{n+1}(\Delta)$, así que $ab\in B_{n}(\Delta)$ pues $ab=a\partial_{n+1}(c)=\partial_{n+1}(ac)$.
% \end{enumerate}
\begin{definition}
   Definimos al espacio
   $$H_{p}(\Delta)=Z_{p}(\Delta)/B_{p}(\Delta)$$
   al cual llamamos la \textbf{\emph{p}-ésima homología de $\Delta$}
\end{definition}

\begin{definition}
  Definimos la \textbf{homología reducida} de $\Delta$ en
  dimensión $0$, denotado por $\widetilde H_{0}(\Delta)$, como
  \begin{equation*}
    \widetilde H_{0}(\Delta)=\ker\varepsilon/\im \partial_{1}
  \end{equation*}
  (Si $p>0$, $\widetilde H_{p}(\Delta)$ denota el espacio usual
  $H_{p}(\Delta)$.)
\end{definition}

\begin{theorem}
  Sea $\Delta$ un complejo simplicial no vacío,
  $$\widetilde H_{0}(\Delta)\cong \widetilde H_{0}(\Delta)\oplus
  \mathbb{C}.$$
  Así $\widetilde H_{0}(\Delta)=0$ si $\Delta$ es conexo.
\end{theorem}
La demostración puede ser encontrada en \cite{munkres1984elements}.

\begin{example}
  Calculemos para $n=0, 1, 2$ los espacios $Z_{n}(\Delta)$,
  $B_{n}(\Delta)$ y $H_{n}$ para la superficie $\Delta$ del tetraedro.

  Como $C_{-1}(\Delta)=0$ por definición, se sigue que
  $$\boldsymbol{Z_{0}(\Delta)}=C_{0}(\Delta)$$
  Por otro lado, $C_{1}(\Delta)=\langle v_{0}v_{1},v_{0}v_{2},v_{0}v_{3},v_{1}v_{2},v_{1}v_{3},v_{2}v_{3}\rangle$,
  así que por el teorema $\ref{imT}$ tenemos 
  \begin{align}
    \label{frontera0}
    \boldsymbol{B_{0}(\Delta)}=&\partial_{1}[C_{1}(\Delta)]\nonumber\\
    &=\langle \partial_{1}(v_{0}v_{1}),\partial_{1}(v_{0}v_{2}),\partial_{1}(v_{0}v_{3}),\partial_{1}(v_{1}v_{2}),\partial_{1}(v_{1}v_{3}),\partial_{1}(v_{2}v_{3})\rangle\nonumber\\
    &=\langle v_{1}-v_{0},v_{2}-v_{0},v_{3}-v_{0},v_{2}-v_{1},v_{3}-v_{1},v_{3}-v_{2}\rangle\nonumber\\
    &=\langle v_{1}-v_{0},v_{2}-v_{0},v_{3}-v_{0}\rangle
  \end{align} 
  pues los vectores $v_{2}-v_{1}, v_{3}-v_{1}, v_{3}-v_{2}$ son
  combinación lineal de los vectores de $\ref{frontera0}$, es decir:
  $$v_{2}-v_{1}=(v_{2}-v_{0})-(v_{1}-v_{0})$$
  $$v_{3}-v_{1}=(v_{3}-v_{0})-(v_{1}-v_{0})$$
  $$v_{3}-v_{2}=(v_{3}-v_{0})-(v_{2}-v_{0})$$
  Así que $\dim(B_{0}(\Delta))=3$, además $C_{0}(\Delta)=\langle
  v_{0},v_{1},v_{2},v_{3}\rangle$, por lo que la
  $\dim(Z_{0}(\Delta))=\dim(C_{0}(\Delta))=4$, luego de el teorema
  $\ref{dim-esp-coc}$ tenemos que
  $\dim(Z_{0}(\Delta)/B_{0}(\Delta))=1$, así mismo la
  $\dim(\mathbb{C})=1$, por el resultado $\ref{esp-isomorfos}$ se sigue
  que $Z_{0}(\Delta)/B_{0}(\Delta)\cong \mathbb{C}$

  Por lo tanto:
  $$\boldsymbol{H_{0}(\Delta)}=Z_{0}(\Delta)/B_{0}(\Delta)\cong \mathbb{C}$$
  Un razonamiento similar se sigue para calcular las homologías restantes.
  Ahora calculemos $B_{1}(\Delta)$. Sabemos que:
  $$C_{2}(\Delta)=\langle
  v_{0}v_{1}v_{2},v_{0}v_{2}v_{3},v_{0}v_{1}v_{3},v_{1}v_{2}v_{3}\rangle$$
  Nuevamente por el teorema $\ref{imT}$ tenemos:
  \begin{align}  
    \label{generadores-B1}
    &\boldsymbol{B_{1}(\Delta)}=\partial_{2}[C_{2}(\Delta)]=\langle\partial_{2}(v_{0}v_{1}v_{2}),\partial_{2}(v_{0}v_{2}v_{3}),\partial_{2}(v_{0}v_{1}v_{3}),\partial_{2}
    (v_{1}v_{2}v_{3})\rangle \nonumber\\
    &=\langle v_{1}v_{2}-v_{0}v_{2}+v_{0}v_{1},v_{2}v_{3}-v_{0}v_{3}+v_{0}v_{2},\nonumber\\
    &\phantom{{}=v_{1}v_{2}-v_{0}v_{2}+v_{0}v_{1},v_{2}v_{3}}v_{1}v_{3}-v_{0}v_{3}+v_{0}v_{1},v_{2}v_{3}-v_{1}v_{3}+v_{1}v_{2}\rangle\nonumber\\
    &=\langle v_{1}v_{2}-v_{0}v_{2}+v_{0}v_{1},v_{2}v_{3}-v_{0}v_{3}+v_{0}v_{2},v_{1}v_{3}-v_{0}v_{3}+v_{0}v_{1}\rangle
  \end{align} 
 
  A continuación veremos que el conjunto generador de $Z_{1}(\Delta)$
  es el mismo que $B_{1}(\Delta)$. Sea $c\in C_{1}(\Delta)$, es decir,
 $$c=n_{1}v_{0}v_{1}+n_{2}v_{0}v_{2}+n_{3}v_{0}v_{3}+n_{4}v_{1}v_{2}+n_{5}v_{1}v_{3}+n_{6}v_{2}v_{3}$$
 tal que:
 \begin{align*}
   \partial_{1}(c)&=n_{1}(v_{1}-v_{0})+n_{2}(v_{2}-v_{0})+n_{3}(v_{3}-v_{0})\\
   &\phantom{{}=n_{1}}+n_{4}(v_{2}-v_{1})+n_{5}(v_{3}-v_{1})+n_{6}(v_{3}-v_{2})\\
   % \nonumber \\
   &=(-n_{1}-n_{2}-n_{3})v_{0}+(n_{1}-n_{4}-n_{5})v_{1}\\
   &\phantom{{}=-n_{1}}+(n_{2}+n_{4}-n_{6})v_{2}+(n_{3}+n_{5}+n_{6})v_{3}\\
   &=0
 \end{align*}
 Así que tenemos que resolver el siguiente sistema de ecuaciones:
 \[\begin{array}{rrrrr}
   -n_{1} & -n_{2} & -n_{3} & = & 0 \\
   n_{1} & -n_{4} & -n_{5} & = & 0 \\
   n_{2} & +n_{4} & -n_{6} & = & 0 \\
   n_{3} & +n_{5} & +n_{6} & = & 0 
 \end{array}\]
 Lo representamos en forma matricial.
 \[ \left(
   \begin{array}{rrrrrr}
  % n_{1} & n_{2} & n_{3} & n_{4} & n_{5} & n_{6} \\
     -1  & -1    & -1   & 0    & 0     & 0 \\
     1   & 0     &    0 & -1   & -1    & 0 \\
     0   & 1     &    0 & 1   & 0    & -1 \\
     0   & 0     &    1 & 0   & 1    & 1 
   \end{array} 
 \right)\]
 y lo llevamos a su forma escalonada reducida
 \[ \left(
   \begin{array}{rrrrrr}
     % n_{1} & n_{2} & n_{3} & n_{4} & n_{5} & n_{6} \\
     1     &    0  & 0     & -1    & -1    & 0 \\
     0     &    1  & 0     &  1    & 0     & -1 \\
     0     &    0  & 1     & 0     & 1     & 1 \\
     0     &    0  & 0     & 0     & 0     & 0 
   \end{array} 
 \right)\]
 De lo cual concluimos:
 \[\begin{array}{rrrrr}
   % n_{1}& = & n_{4} & n_{5} & n_{6} \\  
   n_{1} & = & n_{4} & +n_{5} & \\
   n_{2} & = & -n_{4} &      &+n_{6} \\
   n_{3} & = &        &-n_{5}&-n_{6} 
 \end{array}\]
 Entonces podemos escribir a $c\in Z_{1}(\Delta)$ como:
 \begin{align}
   \label{generadores-Z1}
   c&=(n_{4}+n_{5})v_{0}v_{1}+(-n_{4}+n_{6})v_{0}v_{2}+(-n_{5}-n_{6})v_{0}v_{3}\nonumber\\
   &\phantom{{}=n_{4}}+n_{4}v_{1}v_{2}+n_{5}v_{1}v_{3}+n_{6}v_{2}v_{3}\nonumber\\
   &=n_{4}(v_{0}v_{1}-v_{0}v_{2}+v_{1}v_{2})+n_{5}(v_{0}v_{1}-v_{0}v_{3}+v_{1}v_{3})\nonumber\\
   &\phantom{{}=n_{4}}+n_{6}(v_{0}v_{2}-v_{0}v_{3}+v_{2}v_{3})
 \end{align}
 Por lo que de la ecuación anterior $\ref{generadores-Z1}$ tenemos:
 $$\boldsymbol{Z_{1}(\Delta)}=\langle v_{0}v_{1}-v_{0}v_{2}+v_{1}v_{2},v_{0}v_{1}-v_{0}v_{3}+v_{1}v_{3},v_{0}v_{2}-v_{0}v_{3}+v_{2}v_{3}\rangle$$
 Notemos de las ecuaciones $\ref{generadores-B1}$ y
 $\ref{generadores-Z1}$ que  $Z_{1}(\Delta)$ y $B_{1}(\Delta)$ tienen
 el mismo conjunto generador, por lo tanto
 $Z_{1}(\Delta)=B_{1}(\Delta)$, en consecuencia,
 $$\boldsymbol{H_{1}(\Delta)}=Z_{1}(\Delta)/B_{1}(\Delta)=0$$
 Los simplejos de dimensión más alta son los \emph{2-simplejos}, así
 que $C_{3}(\Delta)=0$ por lo que 
 $$\boldsymbol{B_{2}(\Delta)}=\partial_{3}[C_{3}(\Delta)]=0$$ 
 Por determinar $Z_{2}(\Delta)$. Si $c\in C_{2}(\Delta)$, es decir 
 $$c=n_{1}v_{0}v_{1}v_{2}+n_{2}v_{0}v_{2}v_{3}+n_{3}v_{0}v_{1}v_{3}+n_{4}v_{1}v_{2}v_{3}$$
 tal que
 \begin{align*}
   \partial_{2}(c)&=n_{1}(v_{1}v_{2}-v_{0}v_{2}+v_{0}v_{1})+n_{2}(v_{2}v_{3}-v_{0}v_{3}+v_{0}v_{2})\\
   &+n_{3}(v_{1}v_{3}-v_{0}v_{3}+v_{0}v_{1})+n_{4}(v_{2}v_{3}-v_{1}v_{3}+v_{1}v_{2})\\
   &=(n_{1}+n_{3})v_{0}v_{1}+(-n_{1}+n_{2})v_{0}v_{2}+(-n_{2}-n_{3})v_{0}v_{3}\\
   & +(n_{1}+n_{4})v_{1}v_{2}+(n_{3}-n_{4})v_{1}v_{3}+(n_{2}+n_{4})v_{2}v_{3}\\
   &=0
 \end{align*}
 entonces $n_{1}=n_{2}=-n_{3}=-n_{4}$, así que podemos escribir a $c$
 de la siguiente forma: 
 $$c=n_{1}(v_{0}v_{1}v_{2}+v_{0}v_{2}v_{3}-v_{0}v_{1}v_{3}-v_{1}v_{2}v_{3})$$
 de donde vemos que
 $$Z_{2}(\Delta)=\langle v_{0}v_{1}v_{2}+v_{0}v_{2}v_{3}-v_{0}v_{1}v_{3}-v_{1}v_{2}v_{3}\rangle$$
 es decir, $\boldsymbol{Z_{2}(\Delta)}\cong \mathbb{C}$. Luego  $\boldsymbol{H_{2}}(\Delta)=Z_{2}(\Delta)/B_{2}(\Delta)\cong\mathbb{C}$.
\end{example}

\section{Complejo de cadenas}

\begin{definition}
  Un \textbf{complejo de cadenas} $(A,\partial)$ es una secuencia
  $$A=\{\cdots,A_{2},A_{1},A_{0},A_{-1},A_{-2},\cdots\}$$
  de espacios vectoriales $A_{k}$, junto con una colección
  $\partial=\{\partial_{k}\mid k \in \mathbb{Z}\}$ de transformaciones
  lineales tales que $\partial_{k}:A_{k}\rightarrow A_{k-1}$ y
  $\partial_{k-1}\partial_{k}=0.$

  Al complejo de cadenas en donde $A_{-1}=\mathbb{C}$ y $\partial_{0}=\varepsilon$,
  donde $\varepsilon$ es la función aumento, le llamamos \textbf{complejo de cadenas aumentado}.
\end{definition}

\begin{theorem}
  Si $(A,\partial)$ es un complejo de cadenas, entonces la imagen bajo
  $\partial_{k}$ es un subespacio de el kernel de $\partial_{k-1}$
\end{theorem}
\begin{proof}[Demostración.]
  Considere
  \begin{small}
    \[
    \begin{array}{ccccc}
      A_{k} & \stackrel{\partial_{k}}{\longrightarrow} & A_{k-1} &
      \stackrel{\partial_{k-1}}{\longrightarrow} & A_{k-2}.
    \end{array} 
    \]
  \end{small}
  $\partial_{k-1}\partial_{k}=0,$ pues $(A,\partial)$ es un complejo
  de cadenas. Esto es $\partial_{k-1}[\partial_{k}[A_{k}]]=0.$ Así que
  $\partial_{k}[A_{k}]$ está contenido en el kernel de
  $\partial_{k-1}$, como se quería demostrar.
\end{proof}

\begin{definition}
  Si $(A,\partial)$ es un complejo de cadenas, entonces el kernel
  $Z_{k}(A)$ de $\partial_{k}$ es el\textbf{ espacio de} $\boldsymbol{k}$\textbf{-ciclos,} y
  la imagen $B_{k}(A)=\partial_{k+1}[A_{k+1}]$ es el \textbf{espacio
    de} $\boldsymbol{k}$\textbf{-fronteras.} El espacio cociente $H_{k}(A)=Z_{k}(A)/B_{k}(A)$
  es la $\boldsymbol{k}$\textbf{-ésima homología de A.}
\end{definition}

\begin{theorem}
  Sean $(A,\partial)$ y $(A^{'},\partial^{'})$ complejos de cadenas, y
  supongamos que hay una colección $f$ de transformaciones lineales
  $f_{k}:A_{k}\rightarrow A^{'}_{k}$ como se indica en el diagrama  
  \[
  \begin{CD}
    \cdots @>{\partial_{k+2}}>> A_{k+1} @>{\partial_{k+1}}>> A_{k} @>{\partial_{k}}>> A_{k-1} @>{\partial_{k-1}}>> \cdots\\
    @.   @VVf_{k+1}V   @VVf_{k}V   @VVf_{k-1}V    \\
    \cdots @>{\partial^{'}_{k+2}}>> A^{'}_{k+1} @>{\partial^{'}_{k+1}}>> A^{'}_{k} @>{\partial^{'}_{k}}>> A^{'}_{k-1} @>{\partial^{'}_{k-1}}>> \cdots
  \end{CD}
  \]
  Supongamos, además que
  $$f_{k-1}\partial_{k}=\partial^{'}_{k}f_{k}$$
  para todo $k$. Entonces $f_{k}$ induce una transformación lineal
  $f_{*k}:H_{k}(A)\rightarrow H_{k}(A^{'}).$
\end{theorem}
\begin{proof}[Demostración.]
  Sea $z\in Z_{k}(A)$. Ahora
  \begin{equation*}
    \partial^{'}_{k}(f_{k}(z))=f_{k-1}(\partial_{k}(z))=f_{k-1}(0)=0,
  \end{equation*}
  así que $f_{k}(z)\in Z_{k}(A^{'})$. Definamos
  $f_{*k}:H_{k}(A)\rightarrow H_{k}(A^{'})$ por
  \begin{equation}
    \label{trans-lin-homologias}
    f_{*k}(z+B_{k}(A))=f_{k}(z)+B_{k}(A^{'}).
  \end{equation}
  
  Primero debemos mostrar que $f_{*k}$ está bien definida, es decir,
  independientemente de la elección de un representante de
  $z+B_{k}(A).$ Supongamos que $z_{1}\in (z+B_{k}(A)).$ Entonces
  $(z_{1}-z)\in B_{k}(A)$, así que existe $c\in A_{k+1}$ tal que
  $z_{1}-z=\partial_{k+1}(c).$ Pero entonces
  $$f_{k}(z_{1})-f_{k}(z)=f_{k}(z_{1}-z)=f_{k}(\partial_{k+1}(c))=\partial^{'}_{k+1}(f_{k+1}(c))$$
  este último término es un elemento de
  $\partial^{'}_{k+1}[A^{'}_{k+1}]=B_{k}(A^{'}).$ Por lo tanto
  $$f_{k}(z_{1})\in (f_{k}(z)+B_{k}(A^{'}))$$
  Así dos representantes de la misma clase lateral en
  $H_{k}(A)=Z_{k}(A)/B_{k}(A)$ son enviados en representantes de la misma
  clase lateral en $H_{k}(A^{'})=Z_{k}(A^{'})/B_{k}(A^{'}).$ Esto muestra
  que $f_{*k}:H_{k}(A)\rightarrow H_{k}(A^{'})$ está bien definida por
  la ecuación $(\ref{trans-lin-homologias})$.
  
  Demostrar que $f_{*k}$ es una transformación lineal se sigue de la
  linealidad de $f$.
\end{proof}
Notemos que si $f_{k}$ es un isomorfismo, entonces $f_{k}$ induce un
isomorfismo $f_{*k}$, es decir, $H_{k}(A)\cong H_{k}(A^{'})$.

\section{Homomorfismos inducidos por mapas homotópicos}
En esta sección $I$ denota el intervalo $[0,1]$.

\begin{definition}
  Sean $X$, $Y$ espacios topológico, dos funciones continuas
  $h,k:X\rightarrow Y$ son \textbf{homotópicas} si existe una función continua
  $$F:X\times I\rightarrow Y,$$
  tal que $F(x,0)=h(x)$ y $F(x,1)=k(x)$ para todo $x\in X.$ Si $h$ y
  $k$ son homotópicas, lo denotamos como $h\simeq k$. Pensamos a $F$
  como una forma de ``deformar'' a $h$ continuamente en $k$, conforme
  $t$ varía de $0$ a $1$.
\end{definition}

% \begin{theorem}
%   Si $h,k:|K|\rightarrow |L|$ son homotópicas, entonces
%   $h_{*},k_{*}:H_{p}(K)\rightarrow H_{p}(L)$ son iguales. El mismo
%   resultado se preserva para homologías reducidas.
% \end{theorem}

Si dos espacios son homeomorfos, tienen homologías isomorfas. Una
condición más débil que homeomorfismo que implica el mismo resultado,
es la de equivalencia homotópica.
 
\begin{definition}
  Dos espacios topológicos $X$ y $Y$ se dicen \textbf{equivalentes homotópicos}, si
  existen las funciones
  $$f:X\rightarrow Y \quad \mbox{ y }\quad g:Y\rightarrow X$$
  tales que $g\circ f\simeq i_{X}$ y $f\circ g\simeq i_{Y}$. Las
  funciones $f$ y $g$ son llamadas\textbf{ equivalencias homotópicas},
  y $g$ es la inversa homotópica de $f$. 
\end{definition}
Si $X$ es homotópico a un punto, se dice que $X$ es
\textbf{contraíble}.

\begin{theorem}
  Sean $X$, $Y$ espacios topológicos, si $X$ y $Y$ son equivalentes
  homotópicos, entonces $\widetilde H_{n}(X)\cong \widetilde
  H_{n}(Y)$. En particular, si $X$ es contraíble, entonces $\widetilde
  H_{n}(X)=0$ para $n\geq 1$.
\end{theorem}
La demostración se encuentra en \cite{munkres1984elements}.
 
\chapter{Gráfica de clanes}
%\label{cha:primer-capitulo}
\section{Gráficas}

\begin{definition}
  Una \textbf{gráfica} $G$ consiste de un conjunto finito no vacío $V(G)$
  de $p$ \emph{vértices} junto con un conjunto $E(G)$ de $q$ pares no
  ordenados de vértices distintos de $V$. Cada par $e=\{u,v\}$ de vértices
  en $E$ es una \emph{arista} de $G$, y se dice que $e$ une a $u$ y
  $v$. Escribimos a $e=uv$ y decimos que $u$ y $v$ son \emph{vértices
    adyacentes}; un vértice $u$ y una arista $e$ son \emph{incidentes}
  si $u\in e$. Dos aristas $e$ y $f$ son \emph{aristas adyacentes}
  si $e\cap f\neq\emptyset$. Una gráfica con $p$ vértices y $q$
  aristas es llamada una $(p,q)$ gráfica.

  Una manera natural de ver una gráfica es poner puntos para los
  vértices y una línea entre dos puntos si los vértices correspondientes
  son adyacentes.
\end{definition}

\begin{definition}
  Decimos que $G$ es una gráfica \textbf{completa} si para todos
  $u,v\in G$ con $u\neq v$ se tiene que $uv$. Una grafica completa con
  $n$ vértices se denota como $K_{n}$.
\end{definition}

\begin{figure}[h]
  \centering
  \begin{tikzpicture}[rotate=90,scale=.8]
    \GraphInit[vstyle=Classic] 
    \SetUpVertex[MinSize=1pt]
    \SetVertexNoLabel 
    \grComplete[RA=2.5]{5}
  \end{tikzpicture}
  
  \caption{$K_{5}$}
\label{fig:K5}
\end{figure}

\begin{definition}
  Una gráfica $H$ se llama \textbf{subgráfica} de $G$ si tiene todos
  sus vértices y aristas en $G$, es decir, $V(H)\subseteq V (G)$ y
  $E(H)\subseteq E(G)$. 
\end{definition}

\section{Gráfica de intersección}

\begin{definition}
  Sea $S$ un conjunto y $F=\{S_{1},\cdots,S_{p}\}$ una familia de
  subconjuntos distintos no vacíos de $S$ cuya unión es $S$. La
  \textbf{gráfica de intersección} de $F$ es denotada por $\Omega(F)$
  y definida por $V(\Omega(F))=F$, con $S_{i}$ y $S_{j}$ adyacentes
  siempre que $i\neq j$ y $S_{i}\cap S_{j}\neq\emptyset$
\end{definition}

\begin{definition}
  Un \textbf{clan} de una gráfica es una subgráfica completa.  Un clan
  es \textbf{maximal} si no es subconjunto propio de ningún
  otro clan.
\end{definition}

\begin{definition}
  La \textbf{gráfica de clanes} de una gráfica dada $G$ es la gráfica
  de intersección de la familia de clanes maximales de $G$. Denotemos a la
  gráfica de clanes de $G$ como $K(G)$. Por ejemplo la gráfica $K_{4}$
  de la figura \ref{KG} es la gráfica de clanes de $G$ de la figura
  \ref{G}.
\end{definition}

\begin{center}
  \begin{figure}[h]
    \begin{minipage}[h]{0.45\linewidth}
      \centering
      \begin{tikzpicture}[scale=.8]
        \GraphInit[vstyle=Classic] \SetUpVertex[MinSize=1pt]
        \SetVertexNoLabel \grTriangularGrid[prefix=G,Math,RA=1.5]{3}%
      \end{tikzpicture}

      \caption{$G$}
      \label{G}
    \end{minipage}
    \begin{minipage}[h]{0.45\linewidth}
      \centering
      \begin{tikzpicture}[scale=.8]
        \GraphInit[vstyle=Classic] \SetUpVertex[MinSize=1pt]
        \SetVertexNoLabel \grTetrahedral[RA=1.9]
      \end{tikzpicture}
    
      \caption{$K(G)$}
      \label{KG}
    \end{minipage}
  \end{figure}
\end{center}

\begin{definition}
  La \textbf{gráfica bipartita clánica} de $G$ es definida como la gráfica
  $BK(G)$ con $V(BK(G))=V(G)\cup V(K(G))$, donde $x\in V(G)$,
  $c\in (K(G))$ son adyacentes si $x\in c$.
\end{definition}

\chapter{Homologías del complejo de emparejamiento}
%\label{cha:primer-capitulo}
\section{Complejo de emparejamiento}

\begin{definition}
Consideremos la gráfica completa de $n$ vértices $K_{n}$, tales
vértices son etiquetados como $1,2,\ldots,n$, además
$\overline{ij}$ denotará la arista que une al vértice $i$ con el
vértice $j$ (donde $\overline{ij}=\overline{ji}$). Llamaremos \textbf{complejo de emparejamiento} de orden
$n$ al complejo simplicial $M_{n}$ de dimensión $n$ tal que:

\begin{enumerate}
  \item Su conjunto de vértices $V$ consta de las aristas de la gráfica
  $K_{n}$. 
  \item Si $v_{i}=\overline{pq}$ y $v_{j}=\overline{rs}$ están en
  $V$, $\{v_{i},v_{j}\}$ es  un 1-simplejo de $M_{n}$ si $v_{i}$
  y $v_{j}$ son ajenas.
\end{enumerate} 
La \textbf{gráfica de emparejamiento} de $K_{n}$ será denotada por $G_{n}$.
\end{definition}

\begin{example}
  \label{ejemploM4}
  Considérese la gráfica $K_{4}$; construiremos el complejo de
  emparejamiento $M_{4}$ dado por el conjunto de vértices
  $$V=\{\overline{12},\overline{13},\overline{14},\overline{23},\overline{24},\overline{34}\},$$
  que es el conjunto de aristas de la gráfica de $K_{4}$. La familia de
  $1$-simplejos estará dada por el siguiente conjunto:
  $$\{\{\overline{12},\overline{34}\},\{\overline{13},\overline{24}\},\{\overline{14},\overline{23}\}\}.$$ 
  Es decir,
\begin{equation*}
  M_{4}=\{\{\overline{12}\},\{\overline{13}\},\{\overline{14}\},\{\overline{23}\},\{\overline{24}\},\{\overline{34}\},\{\overline{12},\overline{34}\},\{\overline{13},\overline{24}\},\{\overline{14},\overline{23}\}\}.
\end{equation*}

\begin{center}
  \begin{minipage}{0.3\linewidth}
    \centering
    \begin{tikzpicture}[x=0.8 cm,y=0.8 cm]
      \draw[help lines] (-2,0);% grid (0,2);
      \GraphInit[vstyle=Classic] \SetUpVertex[MinSize=1pt]
      \Vertex[x=-2,y=0,Math,LabelOut,Lpos=180]{2}
      \Vertex[x=0,y=0,Math]{3}
      \Vertex[x=-2,y=2,Math,LabelOut,Lpos=180]{1}
      \Vertex[x=0,y=2,Math]{4} \Edge(1)(2) \Edge(1)(3) \Edge(1)(4)
      \Edge(2)(3) \Edge(2)(4) \Edge(3)(4)
    \end{tikzpicture}
  
    $K_{4}$
  \end{minipage}
  \begin{minipage}{0.3\linewidth}
    \centering
    \begin{tikzpicture}[x=0.8 cm,y=0.8 cm]
      \draw[help lines] (-2,0);% grid (0,2);
      \GraphInit[vstyle=Classic] \SetUpVertex[MinSize=1pt]
      \Vertex[x=-2,y=2,Math,LabelOut,Lpos=90,L=\overline{12}]{12}
      \Vertex[x=-2,y=0,Math,LabelOut,Lpos=-90,L=\overline{34}]{34}
      \Vertex[x=-1,y=0,Math,LabelOut,Lpos=-90,L=\overline{24}]{24}
      \Vertex[x=-1,y=2,Math,LabelOut,Lpos=90,L=\overline{13}]{13}
      \Vertex[x=0,y=2,Math,LabelOut,Lpos=90,L=\overline{14}]{14}
      \Vertex[x=0,y=0,Math,LabelOut,Lpos=-90,L=\overline{23}]{23} \Edge(12)(34)
      \Edge(24)(13) \Edge(14)(23)
    \end{tikzpicture}
  
    $G_{4}$
  \end{minipage}
  % \begin{minipage}{0.3\linewidth}
  %   \centering
  %   \begin{tikzpicture}[x=1 cm,y=0.8 cm]
  %     \draw[help lines] (-2,0);% grid (2,0);
  %     \GraphInit[vstyle=Classic] \SetUpVertex[MinSize=1pt]
  %     \Vertex[x=-2,y=0,Math,LabelOut,Lpos=-90]{12,34} 
  %     \Vertex[x=0,y=0,Math,LabelOut,Lpos=-90]{24,13} 
  %     \Vertex[x=2,y=0,Math,LabelOut,Lpos=-90]{14,23}
  %   \end{tikzpicture}
  
  %   $K(G_{4})$
  % \end{minipage}
\end{center}

Tomando la notación de $p$-simplejos orientados tenemos los espacios
de cadenas:
\begin{equation*}
  C_{0}(M_{4})=\langle(\overline{12}),(\overline{13}),(\overline{14}),(\overline{23}),(\overline{24}),(\overline{34})\rangle.
\end{equation*}
\begin{equation*}
 C_{1}(M_{4})=\langle(\overline{12},\overline{34}),(\overline{13},\overline{24}),(\overline{14},\overline{23})\rangle
\end{equation*}
Sean
\begin{center}
  \begin{tabular}{ccc}
    $a_{1}=\overline{12}$, & $a_{4}=\overline{23}$, & $b_{1}=(\overline{12},\overline{34})$,\\
    $a_{2}=\overline{13}$, & $a_{5}=\overline{24}$, & $b_{2}=(\overline{13},\overline{24})$,\\
    $a_{3}=\overline{14}$, & $a_{6}=\overline{34}$, & $b_{3}=(\overline{14},\overline{23})$,\\
  \end{tabular}
\end{center}
Consideremos a $\beta_{0}=\{a_{1},a_{2},a_{3},a_{4},a_{5},a_{6}\}$ y
$\beta_{1}=\{b_{1},b_{2},b_{3}\}$ como las bases de $C_{0}(M_{4})$ y
$C_{1}(M_{4})$ respectivamente. Notemos que $\dim C_{0}(M_{4})=6$ y
$\dim C_{1}(M_{4})=3$. A continuación se tiene la acción un
representante de cada clase de conjugación del grupo
simétrico $S_{4}$ sobre los elementos de la base de $C_{0}(M_{4})$ y
$C_{1}(M_{4})$.
\begin{center}
  \begin{tabular}{llll}
    $(12)a_{1}=a_{1}$ & $(123)a_{1}=a_{4}$ & $(1234)a_{1}=a_{4}$ & $(12)(34)a_{1}=a_{1}$ \\
    $(12)a_{2}=a_{4}$ & $(123)a_{2}=a_{1}$ & $(1234)a_{2}=a_{5}$ & $(12)(34)a_{2}=a_{5}$ \\
    $(12)a_{3}=a_{5}$ & $(123)a_{3}=a_{5}$ & $(1234)a_{3}=a_{1}$ & $(12)(34)a_{3}=a_{4}$ \\
    $(12)a_{4}=a_{2}$ & $(123)a_{4}=a_{2}$ & $(1234)a_{4}=a_{6}$ & $(12)(34)a_{4}=a_{3}$ \\
    $(12)a_{5}=a_{3}$ & $(123)a_{5}=a_{6}$ & $(1234)a_{5}=a_{2}$ & $(12)(34)a_{5}=a_{2}$ \\
    $(12)a_{6}=a_{6}$ & $(123)a_{6}=a_{3}$ & $(1234)a_{6}=a_{3}$ & $(12)(34)a_{6}=a_{6}$ \\
  \end{tabular}
\end{center}

\begin{center}
  \begin{tabular}{llll}
    $(12)b_{1}=b_{1}$  & $(123)b_{1}=-b_{3}$ & $(1234)b_{1}=-b_{3}$ & $(12)(34)b_{1}=b_{1}$ \\
    $(12)b_{2}=-b_{3}$ & $(123)b_{2}=b_{1}$  & $(1234)b_{2}=-b_{2}$ & $(12)(34)b_{2}=-b_{2}$ \\
    $(12)b_{3}=-b_{2}$ & $(123)b_{3}=-b_{2}$ & $(1234)b_{3}=b_{1}$  & $(12)(34)b_{3}=-b_{3}$ \\
  \end{tabular}
\end{center}
Entonces tenemos las representaciones $\theta_{1}$ y $\theta_{2}$ de
$S_{4}$ en $C_{0}(M_{4})$ y $C_{1}(M_{4})$ respectivamente como se
muestra enseguida:

\begin{center}
  $\theta_{1}(12)= \left(
    \begin{array}{rrrrrr}
      1 & 0 & 0 & 0 & 0 & 0\\
      0 & 0 & 0 & 1 & 0 & 0\\
      0 & 0 & 0 & 0 & 1 & 0\\
      0 & 1 & 0 & 0 & 0 & 0\\
      0 & 0 & 1 & 0 & 0 & 0\\
      0 & 0 & 0 & 0 & 0 & 1\\
    \end{array} 
  \right)$\quad 
  $\theta_{1}(123)= \left(
    \begin{array}{rrrrrr}
      0 & 1 & 0 & 0 & 0 & 0\\
      0 & 0 & 0 & 1 & 0 & 0\\
      0 & 0 & 0 & 0 & 0 & 1\\
      1 & 0 & 0 & 0 & 0 & 0\\
      0 & 0 & 1 & 0 & 0 & 0\\
      0 & 0 & 0 & 0 & 1 & 0\\
    \end{array} 
  \right)$
\end{center}

\begin{center}
  $\theta_{1}(1234)= \left(
    \begin{array}{rrrrrr}
      0 & 0 & 1 & 0 & 0 & 0\\
      0 & 0 & 0 & 0 & 1 & 0\\
      0 & 0 & 0 & 0 & 0 & 1\\
      1 & 0 & 0 & 0 & 0 & 0\\
      0 & 1 & 0 & 0 & 0 & 0\\
      0 & 0 & 0 & 1 & 0 & 0\\
    \end{array} 
  \right)$ \quad
  $\theta_{1}(12)(34)= \left(
    \begin{array}{rrrrrr}
      1 & 0 & 0 & 0 & 0 & 0\\
      0 & 0 & 0 & 0 & 1 & 0\\
      0 & 0 & 0 & 1 & 0 & 0\\
      0 & 0 & 1 & 0 & 0 & 0\\
      0 & 1 & 0 & 0 & 0 & 0\\
      0 & 0 & 0 & 1 & 0 & 1\\
    \end{array} 
  \right)$ 
\end{center}

\begin{center}
  $\theta_{2}(12)= \left(
    \begin{array}{rrr}
      1 & 0 & 0 \\
      0 & 0 & -1 \\
      0 & -1 & 0 \\
    \end{array} 
  \right)$ \quad
  $\theta_{2}(123)= \left(
    \begin{array}{rrr}
      0 & 1 & 0 \\
      0 & 0 & -1 \\
      -1 & 0 & 0 \\
    \end{array} 
  \right)$
\end{center}

\begin{center}
  $\theta_{2}(1234)= \left(
    \begin{array}{rrr}
      0 & 0 & 1 \\
      0 & -1 & 0 \\
      -1 & 0 & 0 \\
    \end{array} 
  \right)$ \quad
  $\theta_{2}(12)(34)= \left(
    \begin{array}{rrr}
      1 & 0 & 0 \\
      0 & -1& 0 \\
      0 & 0 & -1 \\
    \end{array} 
  \right)$
\end{center}
de tal forma que

\begin{tabular}{r r r}
  $\chi_{C_{0}(M_{4})}((1))=6$, & $\chi_{C_{0}(M_{4})}((12))=2$, & $\chi_{C_{0}(M_{4})}((123))=0$, \\
  $\chi_{C_{0}(M_{4})}((1234))=0$, & $\chi_{C_{0}(M_{4})}((12)(34))=2$, & \\
\end{tabular}
\bigskip	

\begin{tabular}{r r r}
  $\chi_{C_{1}(M_{4})}((1))=3$, & $\chi_{C_{1}(M_{4})}((12))=1$, & $\chi_{C_{1}(M_{4})}((123))=0$, \\
  $\chi_{C_{1}(M_{4})}((1234))=-1$, & $\chi_{C_{1}(M_{4})}((12)(34))=-1$. & \\
\end{tabular}
\medskip

Añadiendo estos dos últimos caracteres a la tabla de caracteres de
$S_{4}$ tenemos:
\begin{table}[htpb]
  \centering
  \begin{tabular}{c|r r r r r}
    No. Elementos & 1 & 6 & 8 & 6 & 3 \\
    Clase & (1) & (12) & (123) & (1234) &(12)(34)\\
    \hline
    $\chi_{\mathbb{C}}$ & 1 & 1 & 1 & 1 & 1 \\
    $\chi_{S^{(1,1,1,1)}}$ & 1 & -1 & 1 & -1 & 1\\
    $\chi_{S^{(3,1)}}$ & 3 & 1 & 0 & -1 & -1\\
    $\chi_{S^{(2,1,1)}}$ & 3 & -1 & 0 & 1 & -1 \\
    $\chi_{S^{(2,2)}}$ & 2 & 0 & -1 & 0 & 2 \\
    \hline
    $\chi_{C_{0}(M_{4})}$ & 6 & 2 & 0 & 0 & 2 \\
    $\chi_{C_{1}(M_{4})}$ & 3 & 1 & 0 & -1 & -1
  \end{tabular}

\caption{Tabla de caracteres de $S_{4}$, $C_{0}(M_{4})$ y $C_{1}(M_{4})$.}
\label{tab:S_4}
\end{table}

Queremos escribir a $C_{0}(M_{4})$ como suma directa de módulos
irreducibles de $S_{4}$, así que calculemos el producto interno de $\chi_{C_{0}(M_{4})}$ con los
caracteres de los módulos irreducibles de $S_{4}$ para conocer la
multiplicidad con la que aparecen estos últimos.
(\setlength{\fboxsep}{0pt}\colorbox{green}{COROLARIO(referencia)todavía
no escrito})
\begin{align*}
  \langle\chi_{C_{0}(M_{4})},\chi_{\mathbb{C}}\rangle &=\frac{1}{24}((1)(1\cdot6)+(6)(1\cdot2)+(8)(1\cdot0)+(6)(1\cdot0)+(3)(1\cdot2))\\
  &=\frac{1}{24}(6+12+6)=1\\
  \langle\chi_{C_{0}(M_{4})},\chi_{S^{(1,1,1,1)}}\rangle &=\frac{1}{24}((1)(1\cdot6)+(6)(-1\cdot2)+(8)(1\cdot0)+(6)(-1\cdot0)+(3)(1\cdot2))\\
  &=\frac{1}{24}(6-12+6)=0\\
  \langle\chi_{C_{0}(M_{4})},\chi_{S^{(3,1)}}\rangle &=\frac{1}{24}((1)(3\cdot6)+(6)(1\cdot2)+(8)(0\cdot0)+(6)(-1\cdot0)+(3)(-1\cdot2))\\
  &=\frac{1}{24}(18+12-6)=1\\
  \langle\chi_{C_{0}(M_{4})},\chi_{S^{(2,1,1)}}\rangle &=\frac{1}{24}((1)(3\cdot6)+(6)(-1\cdot2)+(8)(0\cdot0)+(6)(1\cdot0)+(3)(-1\cdot2))\\
  &=\frac{1}{24}(18-12-6)=0\\
  \langle\chi_{C_{0}(M_{4})},\chi_{S^{(2,2)}}\rangle &=\frac{1}{24}((1)(2\cdot6)+(6)(0\cdot2)+(8)(-1\cdot0)+(6)(0\cdot0)+(3)(2\cdot2))\\
  &=\frac{1}{24}(12+12)=1
\end{align*}
De lo anterior se sigue:
\begin{equation}
\label{eq:C_0(M_4)}
C_{0}(M_{4})\cong \mathbb{C}\oplus S^{(3,1)}\oplus S^{(2,2)}
\end{equation}
De la tabla \ref{tab:S_4}
podemos observar que $\chi_{C_{1}(M_{4})}=\chi_{S^{(3,1)}}$, así que  
\begin{equation}
\label{eq:C_1(M_4)}
C_{1}(M_{4})\cong S^{(3,1)}
\end{equation}
Con lo cual tenemos el siguiente complejo de cadenas aumentado
donde $\partial_{1},\partial_{2},\varepsilon$ son los correspondientes
operadores frontera y la función aumento:

\begin{small}
  \[
    \begin{array}{ccccccccccccc}
      \dots 0 & \rightarrow & 0 &
      \stackrel{\partial_{2}}{\rightarrow} & C_{1}(M_{4}) &
      \stackrel{\partial_{1}}{\rightarrow} & C_{0}(M_{4}) & \stackrel{\varepsilon}{\rightarrow} &
      \mathbb{C} & \rightarrow  & 0 & \rightarrow & 0 \dots
    \end{array} 
    \]
  \end{small}
Sean $f_{0}$ y $f_{1}$ los isomorfimos obtenidos de las expresiones
\ref{eq:C_0(M_4)}, \ref{eq:C_1(M_4)} respectivamente, y definimos en
general al morfismo $\widehat\partial_{k}$ como
$\widehat\partial_{k}=f_{k-1}\circ \partial_{k}\circ f^{-1}_{k}$, con
lo cual tenemos el complejo de cadenas aumentado: 

\begin{small}
    \[
    \begin{array}{ccccccccccccc}
      \dots 0 & \rightarrow & 0 &
      \stackrel{\widehat\partial_{2}}{\rightarrow} &  S^{(3,1)} &
      \stackrel{\widehat\partial_{1}}{\rightarrow} & \mathbb{C} \oplus
      S^{(3,1)}\oplus S^{(2,2)} & \stackrel{\widehat\varepsilon}{\rightarrow} &
      \mathbb{C} & \rightarrow  & 0 & \rightarrow & 0 \dots
    \end{array} 
    \]
  \end{small}

En la figura \ref{fig:diagrama-conmutativo4} se muestra diagrama
conmutativo de los complejos de cadenas anteriores.
\begin{figure}[!hbtp]
  \centering
  \[
  \begin{CD}
    0 @>{\partial_{2}}>> C_{1}(M_{4}) @>{\partial_{1}}>> C_{0}(M_{4}) @>{\varepsilon}>> \mathbb{C}\\
    @VVV   @V{f_{1}}VV   @V{f_{0}}VV   @VVV    \\
    0 @>{\widehat\partial_{2}}>> S^{(3,1)} @>{\widehat\partial_{1}}>>
    \mathbb{C} \oplus S^{(3,1)}\oplus S^{(2,2)} @>{\widehat
      \varepsilon}>> \mathbb{C}
  \end{CD}
  \]
  
  \caption{Diagrama conmutativo de los complejos de cadenas de $M_{4}$}
\label{fig:diagrama-conmutativo4}
\end{figure}

Ahora calculemos las homologías reducidas $\widetilde H_{0}(M_{4})$ y
$\widetilde H_{1}(M_{4})$.

Por el teorema \ref{teorema-isomorfismo-mod} y como
$\widehat\varepsilon$ es suprayectiva tenemos:
$$(\mathbb{C} \oplus S^{(3,1)}\oplus S^{(2,2)})/\ker\widehat\varepsilon\cong\im\widehat\varepsilon=\mathbb{C}$$
entonces
\begin{equation}
\label{ker-0-4}
\ker\widehat\varepsilon\cong S^{(3,1)}\oplus S^{(2,2)}
\end{equation}

Por otra parte tenemos que $\im\widehat\partial_{1}\cong S^{(3,1)}$ o $\im\widehat\partial_{1}=0$ por
la proposición \ref{im-mod-irreducible}

Como $\partial_{1}((\overline{12},\overline{34}))=\overline{34}-\overline{12}$
tenemos que $\im\partial_{1}\neq 0$ (en general $\partial_{k}\neq 0$
por definición del operador frontera), se sigue que
$\im\widehat\partial_{1}\neq 0$ puesto que el diagrama es
conmutativo. Por lo tanto 

% Si $\partial_{1}(C_{1}(M_{4}))=0$ entonces $f_{0}(\partial_{1}(C_{1}(M_{4})))=0$,
% pues $f_{0}$ es morfismo, como el diagrama es conmutativo se tiene
% $f_{0}(\partial_{1}(C_{1}(M_{4})))=\widehat\partial_{1}(f_{1}(C_{1}(M_{4})))$,
% además $\widehat\partial_{1}(f_{1}(C_{1}(M_{4})))=\widehat\partial_{1}(S^{(3,1)})$,
% pues $f_{1}$ es suprayectiva, por lo tanto $\widehat\partial_{1}(S^{(3,1)})=0$.
\begin{equation}
\label{im-1-4}
\im\widehat\partial_{1}\cong S^{(3,1)}
\end{equation}

De las expresiones \ref{ker-0-4}, \ref{im-1-4} y de la
proposición \ref{modulos-iguales} concluimos:
\begin{equation}
\label{ker-0-4=}
\ker\widehat\varepsilon=S^{(3,1)}\oplus S^{(2,2)}
\end{equation}
\begin{equation}
\label{im-1-4=}
\im\widehat\partial_{1}=S^{(3,1)}
\end{equation}

Nuevamente por el teorema \ref{teorema-isomorfismo-mod}
$$S^{(3,1)}/\ker\widehat\partial_{1}\cong\im\widehat\partial_{1}= S^{(3,1)} $$
entonces
\begin{equation}
\ker\widehat\partial_{1}=0
\label{ker-1-4}
\end{equation}

$\widehat\partial_{2}$ es un morfismo de módulos, así que
\begin{equation}
\im\widehat\partial_{2}=\widehat\partial_{2}(0)=0
\label{im-2-4}
\end{equation}
Por lo tanto, de la ecuaciones \ref{ker-0-4=}, \ref{im-1-4=},
\ref{ker-1-4} y \ref{im-2-4} obtenemos:
\begin{align*}
\widetilde H_{0}(M_{4})&=\ker \widehat\varepsilon/\im
\widehat\partial_{1}=S^{(3,1)}\oplus S^{(2,2)}/S^{(3,1)}=S^{(2,2)},\\
\widetilde H_{1}(M_{4})&=\ker \widehat\partial_{1}/\im \widehat\partial_{2}=0/0=0.
\end{align*}

\end{example}

\begin{example}
  \label{ejemploM5}
  Considérese el complejo de emparejamiento $M_{5}$, que
  está dado por el conjunto de vértices (aristas de la gráfica de $K_{5}$):
  $$V=\{a_{1},a_{2},a_{3},a_{4},a_{5},a_{6},a_{7},a_{8},a_{9},a_{10}\}$$
  donde
  \begin{table}[!hbtp]
    \centering
    \begin{tabular}{lllll}
    $a_{1}=\overline{12}$ & $a_{2}=\overline{13}$ & $a_{3}=\overline{14}$ & $a_{4}=\overline{15}$ & $a_{5}=\overline{23}$ \\
    $a_{6}=\overline{24}$ & $a_{7}=\overline{25}$ & $a_{8}=\overline{34}$ & $a_{9}=\overline{35}$ & $a_{10}=\overline{45}$
  \end{tabular}
\end{table}

Entonces la familia de $1$-simplejos orientados estará dada por:
\begin{center}
  \begin{tabular}[h]{lll}
    $b_{1}=(a_{1},a_{8})$ & $b_{6}=(a_{2},a_{10})$ & $b_{11}=(a_{4},a_{6})$  \\
    $b_{2}=(a_{1},a_{9})$ & $b_{7}=(a_{3},a_{5})$ & $b_{12}=(a_{4},a_{8})$  \\
    $b_{3}=(a_{1},a_{10})$ & $b_{8}=(a_{3},a_{7})$ & $b_{13}=(a_{5},a_{10})$  \\
    $b_{4}=(a_{2},a_{6})$ & $b_{9}=(a_{3},a_{9})$ & $b_{14}=(a_{6},a_{9})$  \\
    $b_{5}=(a_{2},a_{7})$ & $b_{10}=(a_{4},a_{5})$ & $b_{15}=(a_{7},a_{8})$  
  \end{tabular}
\end{center}
\begin{figure}[!hbtp]
  \centering
  \begin{tikzpicture}[rotate=90,scale=0.75]
    \newcommand{\aset}[2]{$\{#1,#2\}$} \GraphInit[vstyle=Classic]
    \SetUpVertex[MinSize=17pt] \SetVertexNoLabel \SetVertexMath
    \grPetersen[RA=3,RB=1.5]
    \AssignVertexLabel{a}{\textsl{$\overline{12}$},\textsl{$\overline{34}$},\textsl{$\overline{15}$},\textsl{$\overline{23}$},\textsl{$\overline{45}$}}
    \AssignVertexLabel{b}{\textsl{$\overline{35}$},\textsl{$\overline{25}$},\textsl{$\overline{24}$},\textsl{$\overline{14}$},\textsl{$\overline{13}$}}
  \end{tikzpicture}
  
  \caption{Gráfica de emparejamiento $G_{5}$}
  \label{fig:G_5}
\end{figure}
\begin{table}[!hbtp]
  \centering
  \begin{small}
    \begin{tabular}{c |r r r r r r r}
      No. Elementos& 1 & 10 & 20 & 30 & 24 & 15 & 20  \\
      Clase & (1) & (12) & (123) & (1234) & (12345) & (12)(34) & (123)(45) \\
      \hline
      $\chi_{S^{(5)}}$       & 1 & 1 & 1 & 1 & 1 & 1 & 1 \\
      $\chi_{S^{(1,1,1,1,1)}}$ & 1 & -1 & 1 & -1 & 1 & 1 & -1\\
      $\chi_{S^{(4,1)}}$      & 4 & 2 & 1 & 0 & -1 & 0 & -1\\
      $\chi_{S^{(2,1,1,1)}}$   & 4 & -2 & 1 & 0 & -1 & 0 & 1 \\
      $\chi_{S^{(3,1,1)}}$    & 6 & 0 & 0 & 0 & 1 & -2 & 0 \\
      $\chi_{S^{(3,2)}}$     & 5 & 1 & -1 & -1 & 0 & 1 & 1 \\
      $\chi_{S^{(2,2,1)}}$   & 5 & -1 & -1 & 1 & 0 & 1 & -1 
    \end{tabular}
  \end{small}

  \caption{Tabla de caracteres de $S_{5}$}
  \label{tab:S_5}
\end{table}

Con el teorema de la reciprocidad de Frobenius
\setlength{\fboxsep}{0pt}\colorbox{green}{(referencia)} obtendremos la
descomposición de los espacios de cadenas $C_{0}(M_{5})$ y
$C_{1}(M_{5})$ en módulos de Specht con un menor
número de cálculos. 

Consideremos:
\begin{eqnarray*}
  V_{0}&=&\langle\overline{45}\rangle\\
  H_{0}&=&\{g\in S_{5}\mid gV_{0}=V_{0}\}=\{\langle(1),(12),(123),(45),(12)(45),(123)(45)\rangle\}\\
\end{eqnarray*}

\begin{table}[!hbtp]
  \centering
    \begin{tabular}{c |r r r r r r}
      No. Elementos& 1 & 1 & 3 & 3 & 2 & 2 \\
      Clase & (1) & (45) & (12) & (12)(45) & (123) & (123)(45) \\
      \hline
      $\chi_{S^{(5)}}$       & 1 & 1 & 1 & 1 & 1 & 1 \\
      $\chi_{S^{(1,1,1,1,1)}}$ & 1 & -1 & -1 & 1 & 1 & -1 \\
      $\chi_{S^{(4,1)}}$      & 4 & 2 & 2 & 0 & 1 & -1 \\
      $\chi_{S^{(2,1,1,1)}}$   & 4 & -2 & -2 & 0 & 1 & 1 \\
      $\chi_{S^{(3,1,1)}}$     & 6 & 0 & 0 & -2 & 0 & 0 \\
      $\chi_{S^{(3,2)}}$      & 5 & 1 & 1 & 1 & -1 & 1 \\
      $\chi_{S^{(2,2,1)}}$    & 5 & -1 & -1 & 1 & -1 & -1 \\
      \hline
      $\chi_{V_{0}}$ & 1 & 1 & 1 & 1 & 1 & 1 \\
    \end{tabular}

\caption{Caracteres de $S_{5}$ restringidos a $H_{0}$ y carácter de $V_{0}$}
\label{tab:restriccion-H_0}
\end{table}

{\scriptsize
  \begin{align*}
    \langle\chi_{C_{0}(M_{5})},\chi_{S^{(5)}}\rangle_{S_{5}}&=\langle\chi_{V_{0}\uparrow^{S_{5}}_{H_0}},\chi_{S^{(5)}}\rangle_{S_{5}}=\langle\chi_{V_{0}},\chi_{S^{(5)}}\downarrow_{H_{0}}\rangle_{H_{0}}\\ 
    &=\frac{1}{12}(1+1+3+3+2+2)=1\\ 
    \langle\chi_{C_{0}(M_{5})},\chi_{S^{(1,1,1,1,1)}}\rangle_{S_{5}}&=\langle\chi_{V_{0}\uparrow^{S_{5}}_{H_0}},\chi_{S^{(1,1,1,1,1)}}\rangle_{S_{5}}=\langle\chi_{V_{0}},\chi_{S^{(1,1,1,1,1)}}\downarrow_{H_{0}}\rangle_{H_{0}}\\
    &=\frac{1}{12}(1-1-3+3+2-2)=0 \\
    \langle\chi_{C_{0}(M_{5})},\chi_{S^{(4,1)}}\rangle_{S_{5}}&=\langle\chi_{V_{0}\uparrow^{S_{5}}_{H_0}},\chi_{S^{(4,1)}}\rangle_{S_{5}}=\langle\chi_{V_{0}},\chi_{S^{(4,1)}}\downarrow_{H_{0}}\rangle_{H_{0}}\\
    &=\frac{1}{12}(4+2+6+0+2-2)=1 \\
    \langle\chi_{C_{0}(M_{5})},\chi_{S^{(2,1,1,1)}}\rangle_{S_{5}}&=\langle\chi_{V_{0}\uparrow^{S_{5}}_{H_0}},\chi_{S^{(2,1,1,1)}}\rangle_{S_{5}}=\langle\chi_{V_{0}},\chi_{S^{(2,1,1,1)}}\downarrow_{H_{0}}\rangle_{H_{0}}\\
    &=\frac{1}{12}(4-2-6+0+2+2)=0 \\
    \langle\chi_{C_{0}(M_{5})},\chi_{S^{(3,1,1)}}\rangle_{S_{5}}&=\langle\chi_{V_{0}\uparrow^{S_{5}}_{H_0}},\chi_{S^{(3,1,1)}}\rangle_{S_{5}}=\langle\chi_{V_{0}},\chi_{S^{(3,1,1)}}\downarrow_{H_{0}}\rangle_{H_{0}}\\
    &=\frac{1}{12}(6+0+0-6+0+0)=0 \\
    \langle\chi_{C_{0}(M_{5})},\chi_{S^{(3,2)}}\rangle_{S_{5}}&=\langle\chi_{V_{0}\uparrow^{S_{5}}_{H_0}},\chi_{S^{(3,2)}}\rangle_{S_{5}}=\langle\chi_{V_{0}},\chi_{S^{(3,2)}}\downarrow_{H_{0}}\rangle_{H_{0}}\\
    &=\frac{1}{12}(5+1+3+3-2+2)=1 \\
    \langle\chi_{C_{0}(M_{5})},\chi_{S^{(2,2,1)}}\rangle_{S_{5}}&=\langle\chi_{V_{0}\uparrow^{S_{5}}_{H_0}},\chi_{S^{(2,2,1)}}\rangle_{S_{5}}=\langle\chi_{V_{0}},\chi_{S^{(2,2,1)}}\downarrow_{H_{0}}\rangle_{H_{0}}\\
    &=\frac{1}{12}(5-1-3+3-2-2)=0 
  \end{align*}}

De los productos internos calculados anteriormente obtenemos:
\begin{equation}
  \label{eq:C0-M5}
  C_{0}(M_{5})\cong \mathbb{C}\oplus S^{(4,1)} \oplus S^{(3,2)} 
\end{equation}
Ahora consideremos los siguientes conjuntos, nuevamente para usar
reciprocidad de Frobenius.
\begin{eqnarray*}
  V_{1}&=&\langle(\overline{13},\overline{24})\rangle\\
  H_{1}&=&\{g\in S_{5}\mid gV_{1}=V_{1}\}
\end{eqnarray*}

  \begin{table}[!hbtp]
    \centering
    \begin{small}
      \begin{tabular}{c |r r r r r}
        & & (24) & (1432) & (14)(23) & \\
        Elementos & (1) & (13) & (1234) & (12)(34) & (13)(24) \\
        \hline
        $\chi_{S^{(5)}}$       & 1 & 1 & 1 & 1 & 1 \\
        $\chi_{S^{(1,1,1,1,1)}}$ & 1 & -1 & -1 & 1 & 1 \\
        $\chi_{S^{(4,1)}}$      & 4 & 2 & 0 & 0 & 0 \\
        $\chi_{S^{(2,1,1,1)}}$   & 4 & -2 & 0 & 0 & 0 \\
        $\chi_{S^{(3,1,1)}}$     & 6 & 0 & 0 & -2 & -2 \\
        $\chi_{S^{(3,2)}}$      & 5 & 1 & -1 & 1 & 1 \\
        $\chi_{S^{(2,2,1)}}$    & 5 & -1 & 1 & 1 & 1 \\
        \hline
        $\chi_{V_{1}}$ & 1 & 1 & -1 & -1 & 1 \\
      \end{tabular}

    \end{small}
    \caption{Caracteres de $S_5$ restringidos a $H_{1}$ y carácter de $V_{1}$}
    \label{tab:restriccion-H_1}
  \end{table}

{\scriptsize
  \begin{align*}
    \langle\chi_{C_{1}(M_{5})},\chi_{S^{(5)}}\rangle_{S_{5}}&=\langle\chi_{V_{1}\uparrow^{S_{5}}_{H_1}},\chi_{S^{(5)}}\rangle_{S_{5}}=\langle\chi_{V_{1}},\chi_{S^{(5)}}\downarrow_{H_{1}}\rangle_{H_{1}}\\
    &=\frac{1}{8}(1+2-2-2+1)=0\\
    \langle\chi_{C_{1}(M_{5})},\chi_{S^{(1,1,1,1,1)}}\rangle_{S_{5}}&=\langle\chi_{V_{1}\uparrow^{S_{5}}_{H_1}},\chi_{S^{(1,1,1,1,1)}}\rangle_{S_{5}}=\langle\chi_{V_{1}},\chi_{S^{(1,1,1,1,1)}}\downarrow_{H_{1}}\rangle_{H_{1}}\\
    &=\frac{1}{8}(1-2+2-2+1)=0 \\
    \langle\chi_{C_{1}(M_{5})},\chi_{S^{(4,1)}}\rangle_{S_{5}}&=\langle\chi_{V_{1}\uparrow^{S_{5}}_{H_1}},\chi_{S^{(4,1)}}\rangle_{S_{5}}=\langle\chi_{V_{1}},\chi_{S^{(4,1)}}\downarrow_{H_{1}}\rangle_{H_{1}}\\
    &=\frac{1}{8}(4+4+0+0+0)=1 \\
    \langle\chi_{C_{1}(M_{5})},\chi_{S^{(2,1,1,1)}}\rangle_{S_{5}}&=\langle\chi_{V_{1}\uparrow^{S_{5}}_{H_1}},\chi_{S^{(2,1,1,1)}}\rangle_{S_{5}}=\langle\chi_{V_{1}},\chi_{S^{(2,1,1,1)}}\downarrow_{H_{1}}\rangle_{H_{1}}\\
    &=\frac{1}{8}(4-4+0+0+0)=0 \\
    \langle\chi_{C_{1}(M_{5})},\chi_{S^{(3,1,1)}}\rangle_{S_{5}}&=\langle\chi_{V_{1}\uparrow^{S_{5}}_{H_1}},\chi_{S^{(3,1,1)}}\rangle_{S_{5}}=\langle\chi_{V_{1}},\chi_{S^{(3,1,1)}}\downarrow_{H_{1}}\rangle_{H_{1}}\\
    &=\frac{1}{8}(6+0+0+4-2)=1 \\
    \langle\chi_{C_{1}(M_{5})},\chi_{S^{(3,2)}}\rangle_{S_{5}}&=\langle\chi_{V_{1}\uparrow^{S_{5}}_{H_1}},\chi_{S^{(3,2)}}\rangle_{S_{5}}=\langle\chi_{V_{1}},\chi_{S^{(3,2)}}\downarrow_{H_{1}}\rangle_{H_{1}}\\
    &=\frac{1}{8}(5+2+2-2+1)=1 \\
    \langle\chi_{C_{1}(M_{5})},\chi_{S^{(2,2,1)}}\rangle_{S_{5}}&=\langle\chi_{V_{1}\uparrow^{S_{5}}_{H_1}},\chi_{S^{(2,2,1)}}\rangle_{S_{5}}=\langle\chi_{V_{1}},\chi_{S^{(2,2,1)}}\downarrow_{H_{1}}\rangle_{H_{1}}\\
    &=\frac{1}{8}(5-2-2-2+1)=0
  \end{align*}}

De donde obtenemos:
\begin{equation}
  \label{eq:C1-M5}
  C_{1}(M_{5})\cong S^{(4,1)}\oplus S^{(3,1,1)}\oplus S^{(3,2)}
\end{equation}
En la figura \ref{fig:diagrama-conmutativo5} se muestra el diagrama
conmutativo de los complejos de cadena de $M_{5}$, donde $f_{0}$ y
$f_{1}$ son los isomorfismos obtenidos de las expresiones
\ref{eq:C0-M5} y \ref{eq:C1-M5}

\begin{figure}[h]
  \centering
    \[
    \begin{CD}
      0 @>{\partial_{2}}>> C_{1}(M_{5}) @>{\partial_{1}}>> C_{0}(M_{5}) @>{\varepsilon}>> \mathbb{C}\\
      @VVV   @Vf_{1} VV   @Vf_{0} VV   @VVV    \\
      0 @>{\widehat\partial_{2}}>> S^{(4,1)}\oplus S^{(3,1,1)}\oplus
      S^{(3,2)} @>{\widehat\partial_{1}}>> \mathbb{C}\oplus S^{(4,1)}
      \oplus S^{(3,2)} @>{\widehat \varepsilon}>> \mathbb{C}
    \end{CD}
    \]

    \caption{Diagrama conmutativo de los complejos de cadenas de
      $M_{5}$}
\label{fig:diagrama-conmutativo5}
\end{figure}

Calculemos las homologías reducidas $\widetilde H_{0}(M_{4})$ y
$\widetilde H_{1}(M_{4})$.

Como $\widehat\varepsilon$ es suprayectiva y por el teorema \ref{teorema-isomorfismo-mod} tenemos:
$$(\mathbb{C}\oplus S^{(4,1)} \oplus
S^{(3,2)})/\ker\widehat\varepsilon\cong\im\widehat\varepsilon=\mathbb{C}$$
así que
\begin{equation*}
\label{ker0-5}
\ker\widehat\varepsilon\cong S^{(4,1)} \oplus S^{(3,2)}
\end{equation*}
Sabemos que $\widetilde H_{0}(M_{5})=\ker \widehat\varepsilon/\im
\widehat\partial_{1}=0$ pues $M_{5}$  es
conexo, se sigue que $\ker \widehat\varepsilon\cong
\im\widehat\partial_{1}$, con lo cual:
\begin{equation}
\label{im1-5}
\im \widehat\partial_{1}\cong S^{(4,1)} \oplus S^{(3,2)}
\end{equation}
Además el teorema \ref{teorema-isomorfismo-mod} y la expresión \ref{im1-5} se tiene:
$$(S^{(4,1)}\oplus S^{(3,1,1)}\oplus S^{(3,2)})/\ker
\widehat\partial_{1}\cong \im \widehat\partial_{1}$$
De lo anterior y la proposición \ref{modulos-iguales} tenemos:
\begin{equation}
\label{ker1-5}
\ker \widehat\partial_{1}= S^{(3,1,1)}
\end{equation}

Por otro lado
\begin{equation}
\im\widehat\partial_{2}=\widehat\partial_{2}(0)=0
\label{im-2-4}
\end{equation}
pues $\widehat\partial_{2}$ es un morfismo de módulos.

De las ecuaciones \ref{ker1-5} y \ref{im-2-4} concluimos:
\begin{equation*}
  \widetilde H_{1}(M_{5})=\ker \widehat\partial_{1}/\im \widehat\partial_{2}=S^{(3,1,1)}/0=S^{(3,1,1)}.
\end{equation*}
 
Anteriormente usamos el hecho de que $M_{5}$ es conexo para concluir que $\widetilde
H_{0}(M_{5})=0$, sin embargo notemos que no es necesario, por lo que
necesitamos calcular $\im \widehat\partial_{1}$. 

Verifiquemos primero que $\dim(\im \partial_{1})=9$. Del teorema
\ref{imT}, sabemos que
$\im \partial_{1}=\langle\partial_{1}(b_{1}),\ldots,\partial_{1}(b_{15})\rangle$,
veamos que el número de vectores linealmente independientes de la
$\im \partial_{1}$ es 9, es
decir, si
$$\lambda_{1}\partial_{1}(b_{1})+\lambda_{2}\partial_{1}(b_{2})+\ldots+\lambda_{15}\partial_{1}(b_{15})=0$$
así que:
\begin{align*}
  &\lambda_{1}(a_{8}-a_{1})+\lambda_{2}(a_{9}-a_{1})+\lambda_{3}(a_{10}-a_{1})+\lambda_{4}(a_{6}-a_{2})+\lambda_{5}(a_{7}-a_{2})\\
  &\qquad
  {}+\lambda_{6}(a_{10}-a_{2})+\lambda_{7}(a_{5}-a_{3})+\lambda_{8}(a_{7}-a_{3})+\lambda_{9}(a_{9}-a_{3})+\lambda_{10}(a_{5}-a_{4})\\
  &\qquad{}+\lambda_{11}(a_{6}-a_{4})+\lambda_{12}(a_{8}-a_{4})+\lambda_{13}(a_{10}-a_{5})+\lambda_{14}(a_{9}-a_{6})+\lambda_{15}(a_{8}-a_{7})\\
  &=(-\lambda_{1}-\lambda_{2}-\lambda_{3})a_{1}+(-\lambda_{4}-\lambda_{5}-\lambda_{6})a_{2}+(-\lambda_{7}-\lambda_{8}-\lambda_{9})a_{3}\\
  &\qquad{}+(-\lambda_{10}-\lambda_{11}-\lambda_{12})a_{4}+(\lambda_{7}+\lambda_{10}-\lambda_{13})a_{5}+(\lambda_{4}+\lambda_{11}-\lambda_{14})a_{6}\\
  &\qquad{}+(\lambda_{5}+\lambda_{8}-\lambda_{15})a_{7}+(\lambda_{1}+\lambda_{12}+\lambda_{15})a_{8}+(\lambda_{2}+\lambda_{9}+\lambda_{14})a_{9}\\
  &\qquad{}+(\lambda_{3}+\lambda_{6}+\lambda_{13})a_{10}=0
\end{align*}
con lo cual obtenemos el siguiente sistema de ecuaciones:
 \[\begin{array}{rrrrr}
   -\lambda_{1} & -\lambda_{2} & -\lambda_{3} & = & 0 \\
   -\lambda_{4} & -\lambda_{5} & -\lambda_{6} & = & 0 \\
   -\lambda_{7} & -\lambda_{8} & -\lambda_{9} & = & 0 \\
   -\lambda_{10} & -\lambda_{11} & -\lambda_{12} & = & 0 \\
   \lambda_{7} & +\lambda_{10} & -\lambda_{13} & = & 0 \\
   \lambda_{4} & +\lambda_{11} & -\lambda_{14} & = & 0 \\
   \lambda_{5} & +\lambda_{8} & -\lambda_{15} & = & 0 \\
   \lambda_{1} & +\lambda_{12} & +\lambda_{15} & = & 0 \\
   \lambda_{2} & +\lambda_{9} & +\lambda_{14} & = & 0 \\
   \lambda_{3} & +\lambda_{6} & +\lambda_{13} & = & 0 
 \end{array}\]
al que representamos en su forma matricial:
\[ \left(
  \begin{tabular}{rrrrrrrrrrrrrrr}
    -1 & -1 & -1 & 0  & 0  & 0  & 0  & 0  & 0  & 0  & 0  & 0  & 0  & 0  & 0  \\
    0  & 0  & 0  & -1 & -1 & -1 & 0  & 0  & 0  & 0  & 0  & 0  & 0  & 0  & 0  \\
    0  & 0  & 0  & 0  & 0  & 0  & -1 & -1 & -1 & 0  & 0  & 0  & 0  & 0  & 0  \\
    0  & 0  & 0  & 0  & 0  & 0  & 0  & 0  & 0  & -1 & -1 & -1 & 0  & 0  & 0  \\
    0  & 0  & 0  & 0  & 0  & 0  & 1  & 0  & 0  & 1  & 0  & 0  & -1 & 0  & 0  \\
    0  & 0  & 0  & 1  & 0  & 0  & 0  & 0  & 0  & 0  & 1  & 0  & 0  & -1 & 0  \\
    0  & 0  & 0  & 0  & 1  & 0  & 0  & 1  & 0  & 0  & 0  & 0  & 0  & 0  & -1 \\
    1  & 0  & 0  & 0  & 0  & 0  & 0  & 0  & 0  & 0  & 0  & 1  & 0  & 0  & 1  \\
    0  & 1  & 0  & 0  & 0  & 0  & 0  & 0  & 1  & 0  & 0  & 0  & 0  & 1  & 0  \\
    0  & 0  & 1  & 0  & 0  & 1  & 0  & 0  & 0  & 0  & 0  & 0  & 1  & 0  & 0 
  \end{tabular}
\right)\] 
lo llevamos a su forma escalonada reducida:
\[ \left(
  \begin{tabular}{rrrrrrrrrrrrrrr}
    1 & 1 & 1 & 0 & 0 & 0 & 0 & 0  & 0  & 0  & 0  & 0  & 0  & 0  & 0  \\
    0 & 1 & 1 & 0 & 0 & 0 & 0 & 0  & 0  & 0  & 0  & -1 & 0  & 0  & -1 \\
    0 & 0 & 1 & 0 & 0 & 0 & 0 & 0  & -1 & 0  & 0  & -1 & 0  & -1 & -1 \\
    0 & 0 & 0 & 1 & 0 & 0 & 0 & 0  & 0  & 0  & 1  & 0  & 0  & -1 & 0  \\
    0 & 0 & 0 & 0 & 1 & 0 & 0 & 1  & 0  & 0  & 0  & 0  & 0  & 0  & -1 \\
    0 & 0 & 0 & 0 & 0 & 1 & 0 & -1 & 0  & 0  & -1 & 0  & 0  & 1  & 1  \\
    0 & 0 & 0 & 0 & 0 & 0 & 1 & 0  & 0  & 1  & 0  & 0  & -1 & 0  & 0  \\
    0 & 0 & 0 & 0 & 0 & 0 & 0 & 1  & 1  & -1 & 0  & 0  & 1  & 0  & 0  \\
    0 & 0 & 0 & 0 & 0 & 0 & 0 & 0  & 0  & 1  & 1  & 1  & 0  & 0  & 0  \\
    0 & 0 & 0 & 0 & 0 & 0 & 0 & 0  & 0  & 0  & 0  & 0  & 0  & 0  & 0 
  \end{tabular}
\right)\]
Por lo tanto $\dim(\im\partial_{1})=9$, lo cual implica que
$\dim(\im\widehat\partial_{1})=9$ (se sigue de la definición de
$\widehat\partial_{1}$).

Por el teorema \ref{im-mod-irreducible} tenemos:
  $$\widehat\partial_{1}(S^{(4,1)})=S^{(4,1)} \quad \mbox{o }\quad \widehat\partial_{1}(S^{(4,1)})=0$$
  $$\widehat\partial_{1}(S^{(3,1,1)})=0$$
  $$\widehat\partial_{1}(S^{(3,2)})=S^{(4,1)} \quad \mbox{o }\quad \widehat\partial_{1}(S^{(3,2)})=0$$
además $\dim(S^{(4,1)})=4$ y $\dim(S^{(3,2)})=5$, luego
$$\im\widehat\partial_{1}=\widehat\partial_{1}(S^{(4,1)}\oplus S^{(3,1,1)}\oplus S^{(3,2)})=S^{(4,1)}\oplus S^{(3,2)}$$
\end{example}

\begin{example}
  \label{ejemploKM5}
  Consideremos el complejo de clanes de la gráfica de emparejamiento
  $M_{5}$, al que denotamos como $K(M_{5})$, que está dado por el
  conjunto de vértices (aristas de la gráfica de
  $M_{5}$):
\begin{center}
  \begin{tabular}[h]{lll}
    $a_{1}=(\overline{12},\overline{34})$ & $a_{6}=(\overline{13},\overline{45})$ & $a_{11}=(\overline{15},\overline{24})$  \\
    $a_{2}=(\overline{12},\overline{35})$ & $a_{7}=(\overline{14},\overline{23})$ & $a_{12}=(\overline{15},\overline{34})$  \\
    $a_{3}=(\overline{12},\overline{45})$ & $a_{8}=(\overline{14},\overline{25})$ & $a_{13}=(\overline{23},\overline{45})$  \\
    $a_{4}=(\overline{13},\overline{24})$ & $a_{9}=(\overline{14},\overline{35})$ & $a_{14}=(\overline{24},\overline{35})$  \\
    $a_{5}=(\overline{13},\overline{25})$ & $a_{10}=(\overline{15},\overline{23})$ & $a_{15}=(\overline{25},\overline{34})$  
  \end{tabular}
\end{center}

La familia de $1$-simplejos orientados estará dada por:
\begin{center}
  \begin{tabular}[h]{llll}
    $b_{1}=(a_{1},a_{2})$ & $b_{9}=(a_{3},a_{13})$ & $b_{17}=(a_{6},a_{13})$ & $b_{25}=(a_{10},a_{11})$ \\
    $b_{2}=(a_{1},a_{3})$ & $b_{10}=(a_{4},a_{5})$ & $b_{18}=(a_{7},a_{8})$ & $b_{26}=(a_{10},a_{12})$ \\
    $b_{3}=(a_{1},a_{12})$ & $b_{11}=(a_{4},a_{6})$ & $b_{19}=(a_{7},a_{9})$ & $b_{27}=(a_{10},a_{13})$ \\
    $b_{4}=(a_{1},a_{15})$ & $b_{12}=(a_{4},a_{11})$ & $b_{20}=(a_{7},a_{10})$ & $b_{28}=(a_{11},a_{12})$ \\
    $b_{5}=(a_{2},a_{3})$ & $b_{13}=(a_{4},a_{14})$ & $b_{21}=(a_{7},a_{13})$ & $b_{29}=(a_{11},a_{14})$ \\
    $b_{6}=(a_{2},a_{9})$ & $b_{14}=(a_{5},a_{6})$ & $b_{22}=(a_{8},a_{9})$ & $b_{30}=(a_{12},a_{15})$ \\
    $b_{7}=(a_{2},a_{14})$ & $b_{15}=(a_{5},a_{8})$ & $b_{23}=(a_{8},a_{15})$ &  \\
    $b_{8}=(a_{3},a_{6})$ & $b_{16}=(a_{5},a_{15})$ & $b_{24}=(a_{9},a_{14})$ &
  \end{tabular}
\end{center}
Los $2$-simplejos orientados son:

\begin{tabular}[h]{llll}
  $c_{1}=(a_{1},a_{2},a_{3})$ & $c_{4}=(a_{3},a_{6},a_{13})$ &$c_{7}=(a_{5},a_{8},a_{15})$ &$c_{9}=(a_{7},a_{10},a_{13})$ \\
  $c_{2}=(a_{1},a_{12},a_{15})$ & $c_{5}=(a_{4},a_{5},a_{6})$ &$c_{8}=(a_{7},a_{8},a_{9})$ &$c_{10}=(a_{10},a_{11},a_{12})$ \\
  $c_{3}=(a_{2},a_{9},a_{14})$ & $c_{6}=(a_{4},a_{11},a_{14})$ &  &
\end{tabular}

\begin{figure}[!hbtp]
  \centering
  \begin{tikzpicture}[scale=.85]%[rotate=90,scale=1]
    \newcommand{\aset}[2]{$\{#1,#2\}$} \GraphInit[vstyle=Classic]
    % \tikzset{VertexStyle/.style={draw,circle}}
    \SetVertexNoLabel \SetVertexMath \SetUpVertex[MinSize=17pt]
    \grEmptyCycle[RA=1,rotation=-90]{5}
    \grEmptyCycle[RA=2.5,prefix=w,rotation=-90]{5}
    \grCycle[RA=3.8,prefix=z,rotation=90]{5} \EdgeInGraphMod{a}{5}{2}
    \EdgeMod{a}{w}{5}{1} \EdgeMod{a}{w}{5}{-1} \EdgeMod{w}{z}{5}{2}
    \EdgeMod{w}{z}{5}{-2}
    \AssignVertexLabel{z}{\textsl{$a_{1}$},\textsl{$a_{3}$},\textsl{$a_{6}$},\textsl{$a_{5}$},\textsl{$a_{15}$}}
    \AssignVertexLabel{w}{\textsl{$a_{4}$},\textsl{$a_{8}$},\textsl{$a_{12}$},\textsl{$a_{2}$},\textsl{$a_{13}$}}
    \AssignVertexLabel{a}{\textsl{$a_{7}$},\textsl{$a_{11}$},\textsl{$a_{9}$},\textsl{$a_{10}$},\textsl{$a_{14}$}}
  \end{tikzpicture}
 
  \caption{Gráfica de clanes $K(G_{5})$}
  \label{fig:KG_5}
\end{figure}

A continuación obtendremos la descomposición de los espacios de
cadenas $C_{0}(K(M_{5}))$, $C_{1}(K(M_{5}))$ y $C_{2}(K(M_{5}))$ en
módulos de Specht por medio de el teorema de la reciprocidad de Frobenius.
\setlength{\fboxsep}{0pt}\colorbox{green}{(referencia)}

Consideremos:
\begin{eqnarray*}
V_{0}=\langle a_{1}\rangle=\langle
(\overline{12},\overline{34})\rangle\\
H_{0}=\{g\in S_{5}\mid gV_{0}=V_{0}\}
\end{eqnarray*}

  \begin{table}[!hbtp]
    \centering
    \resizebox*{!}{5.2cm}{
    \begin{tabular}{c |r r r r}
      &     &      &        & (12)(34) \\
      &     & (12) & (1324) & (13)(24) \\
      Elementos & (1) & (34) & (1423) & (14)(23) \\
      \hline
      $\chi_{S^{(5)}}$       & 1 & 1  & 1  & 1  \\
      $\chi_{S^{(1,1,1,1,1)}}$ & 1 & -1 & -1 & 1  \\
      $\chi_{S^{(4,1)}}$      & 4 & 2  & 0  & 0  \\
      $\chi_{S^{(2,1,1,1)}}$   & 4 & -2 & 0  & 0  \\
      $\chi_{S^{(3,1,1)}}$    & 6 & 0  & 0  & -2 \\
      $\chi_{S^{(3,2)}}$     & 5 & 1  & -1  & 1  \\
      $\chi_{S^{(2,2,1)}}$   & 5 & -1  & 1  & 1  \\
      \hline
      $\chi_{V_{0}}$ & 1 & 1 & 1 & 1 
    \end{tabular}}
    
    \caption{Caracteres de $S_{5}$ restringidos a $H_{0}$ y carácter de $V_{0}$}
    \label{tab:clanes-H_0-5}
  \end{table}

\begin{eqnarray*}
  \langle\chi_{C_{0}(K(M_{5}))},\chi_{S^{(5)}}\rangle_{S_{5}}=\langle\chi_{V_{0}\uparrow^{S_{5}}_{H_0}},\chi_{S^{(5)}}\rangle_{S_{5}}=\langle\chi_{V_{0}},\chi_{S^{(5)}}\downarrow_{H_{0}}\rangle_{H_{0}}=1\\
 \langle\chi_{C_{0}(K(M_{5}))},\chi_{S^{(1,1,1,1,1)}}\rangle_{S_{5}}=\langle\chi_{V_{0}\uparrow^{S_{5}}_{H_0}},\chi_{S^{(1,1,1,1,1)}}\rangle_{S_{5}}=\langle\chi_{V_{0}},\chi_{S^{(1,1,1,1,1)}}\downarrow_{H_{0}}\rangle_{H_{0}}=0\\
\langle\chi_{C_{0}(K(M_{5}))},\chi_{S^{(4,1)}}\rangle_{S_{5}}=\langle\chi_{V_{0}\uparrow^{S_{5}}_{H_0}},\chi_{S^{(4,1)}}\rangle_{S_{5}}=\langle\chi_{V_{0}},\chi_{S^{(4,1)}}\downarrow_{H_{0}}\rangle_{H_{0}}=1\\
\langle\chi_{C_{0}(K(M_{5}))},\chi_{S^{(2,1,1,1)}}\rangle_{S_{5}}=\langle\chi_{V_{0}\uparrow^{S_{5}}_{H_0}},\chi_{S^{(2,1,1,1)}}\rangle_{S_{5}}=\langle\chi_{V_{0}},\chi_{S^{(2,1,1,1)}}\downarrow_{H_{0}}\rangle_{H_{0}}=0\\
\langle\chi_{C_{0}(K(M_{5}))},\chi_{S^{(3,1,1)}}\rangle_{S_{5}}=\langle\chi_{V_{0}\uparrow^{S_{5}}_{H_0}},\chi_{S^{(3,1,1)}}\rangle_{S_{5}}=\langle\chi_{V_{0}},\chi_{S^{(3,1,1)}}\downarrow_{H_{0}}\rangle_{H_{0}}=0\\
\langle\chi_{C_{0}(K(M_{5}))},\chi_{S^{(3,2)}}\rangle_{S_{5}}=\langle\chi_{V_{0}\uparrow^{S_{5}}_{H_0}},\chi_{S^{(3,2)}}\rangle_{S_{5}}=\langle\chi_{V_{0}},\chi_{S^{(3,2)}}\downarrow_{H_{0}}\rangle_{H_{0}}=1\\
\langle\chi_{C_{0}(K(M_{5}))},\chi_{S^{(2,2,1)}}\rangle_{S_{5}}=\langle\chi_{V_{0}\uparrow^{S_{5}}_{H_0}},\chi_{S^{(2,2,1)}}\rangle_{S_{5}}=\langle\chi_{V_{0}},\chi_{S^{(2,2,1)}}\downarrow_{H_{0}}\rangle_{H_{0}}=1\\
\end{eqnarray*}
Con los productos internos calculados obtenemos:
\begin{equation}
C_{0}(K(M_{5}))\cong \mathbb{C}\oplus S^{(4,1)}\oplus S^{(3,2)}\oplus
S^{(2,2,1)}
\label{C0-KM5}
\end{equation}
Consideremos los siguientes conjuntos:
\begin{eqnarray*}
V_{1}=\langle b_{1}\rangle=\langle (a_{1},a_{2})\rangle=\langle((\overline{12},\overline{34}),(\overline{12},\overline{35}))\rangle\\
H_{1}=\{g\in S_{5}\mid gV_{1}=V_{1}\}=\{(1),(12),(45),(12)(45)\}
\end{eqnarray*}
% \begin{align*}
%   (1)b_{1}&=b_{1}\\
%   (12)b_{1}&=b_{1}\\
%   (45)b_{1}&=-b_{1}\\
%   (12)(45)b_{1}&=-b_{1}
% \end{align*}
\begin{table}[!hbtp]
  \centering
  \begin{tabular}{c |r r r r}
    Elementos & (1) & (12) & (45) & (12)(45) \\
    \hline
    $\chi_{S^{(5)}}$ & 1 & 1  & 1  & 1 \\
    $\chi_{S^{(1,1,1,1,1)}}$ & 1 & -1 & -1 & 1  \\
    $\chi_{S^{(4,1)}}$ & 4 & 2  & 2  & 0  \\
    $\chi_{S^{(2,1,1,1)}}$ & 4 & -2 & -2 & 0  \\
    $\chi_{S^{(3,1,1)}}$ & 6 & 0  & 0  & -2 \\
    $\chi_{S^{(3,2)}}$ & 5 & 1  & 1  & 1  \\
    $\chi_{S^{(2,2,1)}}$ & 5 & -1 & -1 & 1  \\
    \hline
    $\chi_{V_{1}}$ & 1 & 1 & -1 & -1 \\
  \end{tabular}

  \caption{Caracteres de $S_{5}$ restringidos a $H_{1}$ y carácter de $V_{1}$}
  \label{tab:clanes-H_1-5}
\end{table}

De la tabla \ref{tab:clanes-H_1-5} calculamos los productos internos:
\begin{eqnarray*}
  \langle\chi_{C_{1}(K(M_{5}))},\chi_{S^{(5)}}\rangle_{S_{5}}=\langle\chi_{V_{1}\uparrow^{S_{5}}_{H_1}},\chi_{S^{(5)}}\rangle_{S_{5}}=\langle\chi_{V_{1}},\chi_{S^{(5)}}\downarrow_{H_{1}}\rangle_{H_{1}}=0\\
  \langle\chi_{C_{1}(K(M_{5}))},\chi_{S^{(1,1,1,1,1)}}\rangle_{S_{5}}=\langle\chi_{V_{1}\uparrow^{S_{5}}_{H_1}},\chi_{S^{(1,1,1,1,1)}}\rangle_{S_{5}}=\langle\chi_{V_{1}},\chi_{S^{(1,1,1,1,1)}}\downarrow_{H_{1}}\rangle_{H_{1}}=0\\
  \langle\chi_{C_{1}(K(M_{5}))},\chi_{S^{(4,1)}}\rangle_{S_{5}}=\langle\chi_{V_{1}\uparrow^{S_{5}}_{H_1}},\chi_{S^{(4,1)}}\rangle_{S_{5}}=\langle\chi_{V_{1}},\chi_{S^{(4,1)}}\downarrow_{H_{1}}\rangle_{H_{1}}=1\\
  \langle\chi_{C_{1}(K(M_{5}))},\chi_{S^{(2,1,1,1)}}\rangle_{S_{5}}=\langle\chi_{V_{1}\uparrow^{S_{5}}_{H_1}},\chi_{S^{(2,1,1,1)}}\rangle_{S_{5}}=\langle\chi_{V_{1}},\chi_{S^{(2,1,1,1)}}\downarrow_{H_{1}}\rangle_{H_{1}}=1\\
  \langle\chi_{C_{1}(K(M_{5}))},\chi_{S^{(3,1,1)}}\rangle_{S_{5}}=\langle\chi_{V_{1}\uparrow^{S_{5}}_{H_1}},\chi_{S^{(3,1,1)}}\rangle_{S_{5}}=\langle\chi_{V_{1}},\chi_{S^{(3,1,1)}}\downarrow_{H_{1}}\rangle_{H_{1}}=2\\
  \langle\chi_{C_{1}(K(M_{5}))},\chi_{S^{(3,2)}}\rangle_{S_{5}}=\langle\chi_{V_{1}\uparrow^{S_{5}}_{H_1}},\chi_{S^{(3,2)}}\rangle_{S_{5}}=\langle\chi_{V_{1}},\chi_{S^{(3,2)}}\downarrow_{H_{1}}\rangle_{H_{1}}=1\\
  \langle\chi_{C_{1}(K(M_{5}))},\chi_{S^{(2,2,1)}}\rangle_{S_{5}}=\langle\chi_{V_{1}\uparrow^{S_{5}}_{H_1}},\chi_{S^{(2,2,1)}}\rangle_{S_{5}}=\langle\chi_{V_{1}},\chi_{S^{(2,2,1)}}\downarrow_{H_{1}}\rangle_{H_{1}}=1
\end{eqnarray*}
De donde resulta:
\begin{equation}
C_{1}(K(M_{5}))\cong S^{(4,1)}\oplus S^{(2,1,1,1)}\oplus
2S^{(3,1,1)}\oplus S^{(3,2)} \oplus S^{(2,2,1)}
\label{C1-KM5}
\end{equation}
Ahora tomemos los siguientes conjuntos:
\begin{eqnarray*}
V_{2}=\langle c_{1}\rangle=\langle
(a_{1},a_{2},a_{3})\rangle&=&\langle((\overline{12},\overline{34}),(\overline{12},\overline{35}),(\overline{12},\overline{45}))\rangle\\
H_{2}&=&\{g\in S_{5}\mid gV_{2}=V_{2}\}
\end{eqnarray*}
Con lo cual conseguimos la tabla \ref{tab:clanes-H_2-5}.
\begin{table}[!hbtp]
  \centering
  \begin{tabular}{c |r r r r r r}
    &     &      & (34) & (12)(34) &       &  \\
    &     &      & (35) & (12)(35) & (345) & (12)(345) \\
    Elementos & (1) & (12) & (45) & (12)(45) & (354) & (12)(354) \\
    \hline
    $\chi_{S^{(5)}}$ & 1 & 1  & 1  & 1 & 1 & 1 \\
    $\chi_{S^{(1,1,1,1,1)}}$ & 1 & -1 & -1 & 1 & 1 & -1\\
    $\chi_{S^{(4,1)}}$ & 4 & 2  & 2  & 0 & 1 & -1\\
    $\chi_{S^{(2,1,1,1)}}$ & 4 & -2 & -2 & 0 & 1 & 1\\
    $\chi_{S^{(3,1,1)}}$ & 6 & 0  & 0  & -2& 0 & 0\\
    $\chi_{S^{(3,2)}}$ & 5 & 1  & 1  & 1 & -1& 1\\
    $\chi_{S^{(2,2,1)}}$ & 5 & -1 & -1 & 1 & -1& -1\\
    \hline
    $\chi_{V_{2}}$ & 1 & 1 & -1 & -1& 1 & 1\\
  \end{tabular}

  \caption{Caracteres de $S_{5}$ restringidos a $H_{2}$ y carácter de $V_{2}$}
  \label{tab:clanes-H_2-5}
\end{table}

La cual nos ayuda a calcular los siguientes productos internos:
\begin{eqnarray*}
  \langle\chi_{C_{2}(K(M_{5}))},\chi_{S^{(5)}}\rangle_{S_{5}}=\langle\chi_{V_{2}\uparrow^{S_{5}}_{H_{2}}},\chi_{S^{(5)}}\rangle_{S_{5}}=\langle\chi_{V_{2}},\chi_{S^{(5)}}\downarrow_{H_{2}}\rangle_{H_{2}}=0\\
  \langle\chi_{C_{2}(K(M_{5}))},\chi_{S^{(1,1,1,1,1)}}\rangle_{S_{5}}=\langle\chi_{V_{2}\uparrow^{S_{5}}_{H_{2}}},\chi_{S^{(1,1,1,1,1)}}\rangle_{S_{5}}=\langle\chi_{V_{2}},\chi_{S^{(1,1,1,1,1)}}\downarrow_{H_{2}}\rangle_{H_{2}}=0\\
  \langle\chi_{C_{2}(K(M_{5}))},\chi_{S^{(4,1)}}\rangle_{S_{5}}=\langle\chi_{V_{2}\uparrow^{S_{5}}_{H_{2}}},\chi_{S^{(4,1)}}\rangle_{S_{5}}=\langle\chi_{V_{2}},\chi_{S^{(4,1)}}\downarrow_{H_{2}}\rangle_{H_{2}}=0\\
  \langle\chi_{C_{2}(K(M_{5}))},\chi_{S^{(2,1,1,1)}}\rangle_{S_{5}}=\langle\chi_{V_{2}\uparrow^{S_{5}}_{H_{2}}},\chi_{S^{(2,1,1,1)}}\rangle_{S_{5}}=\langle\chi_{V_{2}},\chi_{S^{(2,1,1,1)}}\downarrow_{H_{2}}\rangle_{H_{2}}=1\\
  \langle\chi_{C_{2}(K(M_{5}))},\chi_{S^{(3,1,1)}}\rangle_{S_{5}}=\langle\chi_{V_{2}\uparrow^{S_{5}}_{H_{2}}},\chi_{S^{(3,1,1)}}\rangle_{S_{5}}=\langle\chi_{V_{2}},\chi_{S^{(3,1,1)}}\downarrow_{H_{2}}\rangle_{H_{2}}=1\\
  \langle\chi_{C_{2}(K(M_{5}))},\chi_{S^{(3,2)}}\rangle_{S_{5}}=\langle\chi_{V_{2}\uparrow^{S_{5}}_{H_{2}}},\chi_{S^{(3,2)}}\rangle_{S_{5}}=\langle\chi_{V_{2}},\chi_{S^{(3,2)}}\downarrow_{H_{2}}\rangle_{H_{2}}=0\\
  \langle\chi_{C_{2}(K(M_{5}))},\chi_{S^{(2,2,1)}}\rangle_{S_{5}}=\langle\chi_{V_{2}\uparrow^{S_{5}}_{H_{2}}},\chi_{S^{(2,2,1)}}\rangle_{S_{5}}=\langle\chi_{V_{2}},\chi_{S^{(2,2,1)}}\downarrow_{H_{2}}\rangle_{H_{2}}=0\\
\end{eqnarray*}

De donde obtenemos:
\begin{equation}
\label{C2-KM5}
C_{2}(K(M_{5}))\cong S^{(2,1,1,1)}\oplus S^{(3,1,1)}
\end{equation}

En la figura \ref{fig:diagrama-conmutativo-clanes5} se muestra el diagrama
conmutativo de los complejos de cadena de $K(M_{5})$, donde $f_{0}$,
$f_{1}$ y $f_{2}$ son los isomorfismos obtenidos de las expresiones
\ref{C0-KM5}, \ref{C1-KM5} y \ref{C2-KM5}.

\begin{sidewaysfigure}%[h]
  %\centering
  {\small
  \[
  \begin{CD}
    0 @>{\partial_{3}}>> C_{2}(K(M_{5})) @>{\partial_{2}}>> C_{1}(K(M_{5})) @>{\partial_{1}}>> C_{0}(K(M_{5})) @>{\varepsilon}>> \mathbb{C}\\
    @VVV   @Vf_{2}VV   @Vf_{1}VV  @Vf_{0}VV  @VVV    \\
    0 @>{\widehat\partial_{3}}>> S^{(2,1,1,1)}\oplus S^{(3,1,1)}
    @>{\widehat\partial_{2}}>> S^{(4,1)}\oplus S^{(2,1,1,1)}\oplus
    2S^{(3,1,1)}\oplus S^{(3,2)} \oplus S^{(2,2,1)}
    @>{\widehat\partial_{1}}>> \mathbb{C}\oplus S^{(4,1)}\oplus
    S^{(3,2)}\oplus S^{(2,2,1)}@>{\widehat \varepsilon}>> \mathbb{C}
  \end{CD}
  \]
   }
  
  \caption{Diagrama conmutativo de los complejos de cadenas de $K(M_{5})$}
  \label{fig:diagrama-conmutativo-clanes5}
\end{sidewaysfigure}

Calculemos las homologías reducidas $\widetilde H_{0}(K(M_{5}))$,
$\widetilde H_{1}(K(M_{5}))$ y $\widetilde H_{2}(K(M_{5}))$.

Como $\widehat\varepsilon$ es suprayectiva y por el teorema \ref{teorema-isomorfismo-mod} tenemos:
\begin{equation*}
  (\mathbb{C}\oplus S^{(4,1)}\oplus S^{(3,2)}\oplus
  S^{(2,2,1)})/\ker\widehat\varepsilon\cong \im \widehat\varepsilon=\mathbb{C}
\end{equation*}
así que
\begin{equation*}
  \label{ker0-KM5}
  \ker\widehat\varepsilon\cong S^{(4,1)} \oplus S^{(3,2)}\oplus S^{(2,2,1)}
\end{equation*}
Sabemos que $\widetilde H_{0}(K(M_{5}))=\ker \widehat\varepsilon/\im
\widehat\partial_{1}=0$ pues $K(M_{5})$ es conexo, se sigue que $\ker \widehat\varepsilon\cong
\im\widehat\partial_{1}$, con lo cual:
\begin{equation}
  \label{im1-KM5}
  \im \widehat\partial_{1}\cong S^{(4,1)} \oplus S^{(3,2)}\oplus S^{(2,2,1)}
\end{equation}

Además el teorema \ref{teorema-isomorfismo-mod} y la expresión \ref{im1-KM5} se tiene:
$$(S^{(4,1)}\oplus S^{(2,1,1,1)}\oplus 2S^{(3,1,1)}\oplus S^{(3,2)}
\oplus S^{(2,2,1)})/\ker \widehat\partial_{1}\cong \im \widehat\partial_{1}$$
De lo anterior y la proposición \ref{modulos-iguales} tenemos:
\begin{equation}
\label{ker1-KM5}
\ker \widehat\partial_{1}=S^{(2,1,1,1)}\oplus 2S^{(3,1,1)}
\end{equation}
Por otro lado
\begin{equation*}
\im\widehat\partial_{3}=\widehat\partial_{3}(0)=0
\label{im3-KM5}
\end{equation*}
pues $\widehat\partial_{3}$ es un morfismo de módulos. Ya que $M_{5}\simeq K(M_{5})$
\setlength{\fboxsep}{0pt}\colorbox{green}{(referencia)} por el teorema
\setlength{\fboxsep}{0pt}\colorbox{green}{(referencia)} tenemos que
$\widetilde H_{2}(M_{5})=\widetilde H_{2}(K(M_{5}))$ y  $\widetilde H_{2}(M_{5})=0$, así que:
\begin{equation*}
\widetilde H_{2}(K(M_{5}))=\ker \widehat\partial_{2}/\im \widehat\partial_{3}=0.
\end{equation*}
por lo que $\ker \widehat\partial_{2}=0$. De el teorema
\ref{teorema-isomorfismo-mod} tenemos:
$$(S^{(2,1,1,1)}\oplus S^{(3,1,1)})/\ker \widehat\partial_{2}\cong \im
\widehat\partial_{2}$$
De lo anterior y la proposición \ref{modulos-iguales} tenemos:
\begin{equation}
\im \widehat\partial_{2}=S^{(2,1,1,1)}\oplus S^{(3,1,1)}
\label{im2-KM5}
\end{equation}

De las ecuaciones \ref{ker1-KM5} y \ref{im2-KM5} tenemos:

\begin{eqnarray*}
  \widetilde H_{1}(K(M_{5}))&=&\ker \widehat\partial_{1}/\im
  \widehat\partial_{2}\\
  &=&(S^{(2,1,1,1)}\oplus 2S^{(3,1,1)})/(S^{(2,1,1,1)}\oplus S^{(3,1,1)})=S^{(3,1,1)}.
\end{eqnarray*}

Anteriormente usamos el hecho de que $M_{5}\simeq K(M_{5})$  para
concluir que $\widetilde H_{2}(K(M_{5}))=\widetilde H_{2}(M_{5})=0$, sin embargo veamos que no es necesario, por lo que
necesitamos calcular $\im \widehat\partial_{2}$. 

Verifiquemos primero que $\dim(\im \partial_{2})=10$. Del teorema
\ref{imT}, sabemos que
$\im \partial_{2}=\langle\partial_{2}(c_{1}),\ldots,\partial_{2}(c_{10})\rangle$,
veamos que el número de vectores linealmente independientes de la
$\im \partial_{2}$ es 10, es
decir, si
$$\lambda_{1}\partial_{2}(c_{1})+\lambda_{2}\partial_{2}(c_{2})+\ldots+\lambda_{10}\partial_{2}(c_{10})=0$$
así que:

\begin{footnotesize}
  \begin{align*}
    &\lambda_{1}(a_{2}a_{3}-a_{1}a_{3}+a_{1}a_{2})+\lambda_{2}(a_{12}a_{15}-a_{1}a_{15}+a_{1}a_{12})+\lambda_{3}(a_{9}a_{14}-a_{2}a_{14}+a_{2}a_{9})\\
    &+\lambda_{4}(a_{6}a_{13}-a_{3}a_{13}+a_{3}a_{6})+\lambda_{5}(a_{5}a_{6}-a_{4}a_{6}+a_{4}a_{5})+\lambda_{6}(a_{11}a_{14}-a_{4}a_{14}+a_{4}a_{11})\\
    &+\lambda_{7}(a_{8}a_{15}-a_{5}a_{15}+a_{5}a_{8})+\lambda_{8}(a_{8}a_{9}-a_{7}a_{9}+a_{7}a_{8})+\lambda_{9}(a_{10}a_{13}-a_{7}a_{13}+a_{7}a_{10})\\
    &+\lambda_{10}(a_{11}a_{12}-a_{10}a_{12}+a_{10}a_{11})\\
    &=0
  \end{align*}
\end{footnotesize}
Claramente $\lambda_{1}=\lambda_{2}=\cdots=\lambda_{10}=0$,
intuituvamente esto se puede deducir de la gráfica \ref{fig:KG_5}.
Sea $b_{j}$ la frontera de algún $2$-simplejo $c_{i}$, las aristas
de $b_{j}$ no pertenece a ningún otro triángulo distinto de $c_{i}$, por
lo tanto $\partial_{2}(c_{j})$ no es combinación lineal de las fronteras
$\partial_{2}(c_{i})$ con $i\neq j$.

Así que $\dim(\im\partial_{2})=10$, lo cual implica que
$\dim(\im\widehat\partial_{2})=10$ (se sigue de la definición de
$\widehat\partial_{2}$).

Ahora, por el teorema \ref{im-mod-irreducible} tenemos:
  $$\widehat\partial_{2}(S^{(2,1,1,1)})=S^{(2,1,1,1)} \quad \mbox{o }\quad \widehat\partial_{2}(S^{(2,1,1,1)})=0$$
  $$\widehat\partial_{2}(S^{(3,1,1)})=S^{(3,1,1)} \quad \mbox{o }\quad \widehat\partial_{2}(S^{(3,1,1)})=0$$
Además $\dim(S^{(2,1,1,1)})=4$ y $\dim(S^{(3,1,1)})=6$, entonces:
$$\im\widehat\partial_{2}=\widehat\partial_{2}(S^{(2,1,1,1)}\oplus S^{(3,1,1)})=S^{(2,1,1,1)}\oplus S^{(3,1,1)}$$

\end{example}

\begin{example}
  \label{ejemploM6}
  Consideremos el complejo de emparejamiento $M_{6}$, que está dado
  por el conjunto de vértices (aristas de la gráfica de $K_{6}$):
  \begin{center}
    \begin{tabular}[h]{lllll}
      $a_{1}=\overline{12}$ & $a_{4}=\overline{15}$ & $a_{7}=\overline{24}$ & $a_{10}=\overline{34}$ & $a_{13}=\overline{45}$  \\
      $a_{2}=\overline{13}$ & $a_{5}=\overline{16}$ & $a_{8}=\overline{25}$ & $a_{11}=\overline{35}$ & $a_{14}=\overline{46}$  \\
      $a_{3}=\overline{14}$ & $a_{6}=\overline{23}$ & $a_{9}=\overline{26}$ & $a_{12}=\overline{36}$ & $a_{15}=\overline{56}$  
    \end{tabular}
  \end{center}
  los $1$-simplejos:
  \begin{center}
    \begin{tabular}[h]{llll}
      $b_{1}=(a_{1},a_{10})$ & $b_{13}=(a_{3},a_{6})$ & $b_{25}=(a_{5},a_{6})$ & $b_{37}=(a_{8},a_{10})$ \\
      $b_{2}=(a_{1},a_{11})$ & $b_{14}=(a_{3},a_{8})$ & $b_{26}=(a_{5},a_{7})$ & $b_{38}=(a_{8},a_{12})$ \\
      $b_{3}=(a_{1},a_{12})$ & $b_{15}=(a_{3},a_{9})$ & $b_{27}=(a_{5},a_{8})$ & $b_{39}=(a_{8},a_{14})$ \\
      $b_{4}=(a_{1},a_{13})$ & $b_{16}=(a_{3},a_{11})$ & $b_{28}=(a_{5},a_{10})$ & $b_{40}=(a_{9},a_{10})$ \\
      $b_{5}=(a_{1},a_{14})$ & $b_{17}=(a_{3},a_{12})$ & $b_{29}=(a_{5},a_{11})$ & $b_{41}=(a_{9},a_{11})$ \\
      $b_{6}=(a_{1},a_{15})$ & $b_{18}=(a_{3},a_{15})$ & $b_{30}=(a_{5},a_{13})$ & $b_{42}=(a_{9},a_{13})$ \\
      $b_{7}=(a_{2},a_{7})$ & $b_{19}=(a_{4},a_{6})$ & $b_{31}=(a_{6},a_{13})$ & $b_{43}=(a_{10},a_{15})$ \\
      $b_{8}=(a_{2},a_{8})$ & $b_{20}=(a_{4},a_{7})$ & $b_{32}=(a_{6},a_{14})$ & $b_{44}=(a_{11},a_{14})$ \\
      $b_{9}=(a_{2},a_{9})$ & $b_{21}=(a_{4},a_{9})$ & $b_{33}=(a_{6},a_{15})$ & $b_{45}=(a_{12},a_{13})$ \\
      $b_{10}=(a_{2},a_{13})$ & $b_{22}=(a_{4},a_{10})$ & $b_{34}=(a_{7},a_{11})$ &  \\
      $b_{11}=(a_{2},a_{14})$ & $b_{23}=(a_{4},a_{12})$ & $b_{35}=(a_{7},a_{12})$ &  \\
      $b_{12}=(a_{2},a_{15})$ & $b_{24}=(a_{4},a_{14})$ & $b_{36}=(a_{7},a_{15})$ &  
    \end{tabular}
  \end{center}
  y los $2$-simplejos:
  \begin{center}
    \begin{tabular}[h]{lll}
      $c_{1}=(a_{1},a_{10},a_{15})$ & $c_{6}=(a_{2},a_{9},a_{13})$ & $c_{11}=(a_{4},a_{7},a_{12})$  \\
      $c_{2}=(a_{1},a_{11},a_{14})$ & $c_{7}=(a_{3},a_{6},a_{15})$ & $c_{12}=(a_{4},a_{9},a_{10})$  \\
      $c_{3}=(a_{1},a_{12},a_{13})$ & $c_{8}=(a_{3},a_{8},a_{12})$ & $c_{13}=(a_{5},a_{6},a_{13})$  \\
      $c_{4}=(a_{2},a_{7},a_{15})$ & $c_{9}=(a_{3},a_{9},a_{11})$ & $c_{14}=(a_{5},a_{7},a_{11})$  \\
      $c_{5}=(a_{2},a_{8},a_{14})$ & $c_{10}=(a_{4},a_{6},a_{14})$ & $c_{15}=(a_{5},a_{8},a_{10})$  
    \end{tabular}
  \end{center}

\begin{figure}[h]
  \centering
  \begin{tikzpicture}[scale=.5]
    \GraphInit[vstyle=Normal] \SetVertexNoLabel
    \grStar[RA=3.7,Math]{7}%
    \grEmptyCycle[RA=8,prefix=c]{24} \EdgeInGraphMod*{a}{6}{1}{0}{2}
    \EdgeInGraphMod*{c}{24}{1}{0}{2} \EdgeFromOneToSeq{a}{c}{1}{2}{5}
    \EdgeFromOneToSeq{a}{c}{2}{6}{9}
    \EdgeFromOneToSeq{a}{c}{3}{10}{13}
    \EdgeFromOneToSeq{a}{c}{4}{14}{17}
    \EdgeFromOneToSeq{a}{c}{5}{18}{21}
    \EdgeFromOneToSeq{a}{c}{0}{22}{23}
    \EdgeFromOneToSeq{a}{c}{0}{0}{1}
    \AssignVertexLabel{a}{\textsl{$a_{15}$},\textsl{$a_{10}$},\textsl{$a_{13}$},\textsl{$a_{12}$},\textsl{$a_{14}$},\textsl{$a_{11}$},\textsl{$a_{1}$}}
    \AssignVertexLabel{c}{\textsl{$a_{7}$},\textsl{$a_{2}$},\textsl{$a_{8}$},\textsl{$a_{5}$},\textsl{$a_{9}$},\textsl{$a_{4}$},\textsl{$a_{9}$},\textsl{$a_{2}$},\textsl{$a_{6}$},\textsl{$a_{5}$},\textsl{$a_{7}$},\textsl{$a_{4}$},\textsl{$a_{8}$},\textsl{$a_{3}$},\textsl{$a_{6}$},\textsl{$a_{4}$},\textsl{$a_{8}$},\textsl{$a_{2}$},\textsl{$a_{7}$},\textsl{$a_{5}$},\textsl{$a_{9}$},\textsl{$a_{3}$},\textsl{$a_{6}$},\textsl{$a_{3}$}}
  \end{tikzpicture}
  
  \caption{Gráfica de emparejamiento $G_{6}$}
  \label{fig:emparejamiento6}
\end{figure}

\begin{table}[!hbtp]
  \resizebox*{!}{5.5cm}{
    \centering
    \begin{tabular}{c |r r r r r r r r r r r}
      No. de Elementos  & 720 & 48 & 18 & 16 & 8 & 6 & 5 & 48 & 18 & 8 &6\\
      Clase&(1)& (2) & (3) & (2,2) & (4)& (3,2) & (5) & (2,2,2) & (3,3)
      & (4,2)  & (6) \\
      \hline
      $\chi_{S^{{(6)}}}$         & 1 & 1  & 1  & 1 & 1 & 1 & 1 & 1 & 1 & 1&1 \\
      $\chi_{S^{{(1,1,1,1,1,1)}}}$ & 1 & -1 & 1  & 1 & -1&-1 & 1 &-1 & 1& 1&-1 \\
      $\chi_{S^{{(5,1)}}}$       & 5 & 3  & 2  & 1 & 1 & 0 & 0 &-1 &-1&-1&-1 \\
      $\chi_{S^{{(2,1,1,1,1)}}}$  & 5 & -3 &  2 & 1 &-1 & 0 & 0 & 1 &-1&-1&1  \\
      $\chi_{S^{{(4,1,1)}}}$     & 10& 2  & 1  & -2& 0 &-1 & 0 &-2 & 1& 0&1  \\
      $\chi_{S^{{(3,1,1,1)}}}$    & 10&-2  & 1  &-2 & 0 & 1 & 0 & 2 & 1& 0&-1 \\
      $\chi_{S^{{(4,2)}}}$       & 9 & 3  & 0  & 1 & -1& 0 &-1 & 3 & 0& 1&0 \\
      $\chi_{S^{{(2,2,1,1)}}}$    & 9 & -3 & 0  & 1 & 1 & 0 &-1 &-3 & 0& 1&0 \\
      $\chi_{S^{{(3,3)}}}$       & 5 & 1  & -1  & 1 &-1 &1 & 0 & -3& 2& -1&0 \\
      $\chi_{S^{{(2,2,2)}}}$     & 5  & -1& -1  & 1 & 1 & -1& 0& 3 & 2& -1&0 \\
      $\chi_{S^{{(3,2,1)}}}$     & 16 & 0 & -2  & 0 & 0 & 0 & 1& 0 &-2& 0 &0
    \end{tabular}}
    
    \caption{Tabla de caracteres de $S_{6}$}
    \label{tab:S_6}
  \end{table}
  La tabla \ref{tab:S_6} nos ayudará a construir las tablas de
  caracteres restringidas a los conjuntos que se establecen a
  continuación, esto para obtener la descomposición de los espacios de
  cadenas $C_{0}(M_{6})$, $C_{1}(M_{6})$ y $C_{2}(M_{6})$ en
  módulos de Specht de forma análoga como se hizo en los ejemplos anteriores.
  
  Consideremos:
  $$V_{0}=\langle a_{1}\rangle=\langle (\overline{12})\rangle$$

  \begin{footnotesize}
    \begin{align*}
      H_{0}=&\{g\in S_{6}\mid
      gV_{0}=V_{0}\}=\{(1),(12),(34),(35),(36),(45),(46),(56),(345),(346),\\
      &(354),(356),(364),(365),(456),(465),(12)(34),(12)(35),(12)(36),(12)(45),\\
      &(12)(46),(12)(56),(34)(56),(35)(46),(36)(45),(3456),(3465),(3546),(3564),\\
      &(3645),(3654),(345)(12),(346)(12),(354)(12),(356)(12),(364)(12),(365)(12),\\
      &(456)(12),(465)(12),(12)(34)(56),(12)(35)(46),(12)(36)(45),(3456)(12),\\
      &(3465)(12),(3546)(12),(3564)(12),(3645)(12),(3654)(12)\}
    \end{align*}
  \end{footnotesize}

 \begin{table}[!hbtp]
    \centering
    \begin{tabular}{c |r r r r r r r r}
      No. de Elementos  & 1 & 7 & 8 & 9 & 6 & 8 & 3 & 6  \\
      Clase&(1)& (2) & (3) & (2,2) & (4)& (3,2) & (2,2,2) & (4,2)\\
      \hline
      $\chi_{S^{{(6)}}}$         & 1 & 1  & 1  & 1 & 1 & 1 & 1 & 1 \\
      $\chi_{S^{{(1,1,1,1,1,1)}}}$ & 1 & -1 & 1  & 1 & -1&-1 &-1 & 1 \\
      $\chi_{S^{{(5,1)}}}$       & 5 & 3  & 2  & 1 & 1 & 0 &-1 &-1 \\
      $\chi_{S^{{(2,1,1,1,1)}}}$  & 5 & -3 &  2 & 1 &-1 & 0 & 1 &-1  \\
      $\chi_{S^{{(4,1,1)}}}$     & 10& 2  & 1  & -2& 0 &-1 &-2 & 0  \\
      $\chi_{S^{{(3,1,1,1)}}}$    & 10&-2  & 1  &-2 & 0 & 1 & 2 & 0 \\
      $\chi_{S^{{(4,2)}}}$       & 9 & 3  & 0  & 1 & -1& 0 & 3 & 1 \\
      $\chi_{S^{{(2,2,1,1)}}}$    & 9 & -3 & 0  & 1 & 1 & 0 &-3 & 1 \\
      $\chi_{S^{{(3,3)}}}$       & 5 & 1  & -1  & 1 &-1 &1 & -3& -1 \\
      $\chi_{S^{{(2,2,2)}}}$     & 5  & -1& -1  & 1 & 1 & -1& 3 & -1 \\
      $\chi_{S^{{(3,2,1)}}}$     & 16 & 0 & -2  & 0 & 0 & 0 & 0 & 0 \\
      \hline
      $\chi_{V_{0}}$        & 1 & 1 & 1 & 1 & 1 & 1 &1 & 1  \\
    \end{tabular}

    \caption{Caracteres de $S_{6}$ restringidos a $H_{0}$ y carácter de $V_{0}$}
    \label{tab:restriccion-H_0-M-6}
  \end{table}

  De los productos internos calculados a partir de la la tabla
  \ref{tab:restriccion-H_0-M-6} obtenemos:
  \begin{equation}
    C_{0}(M_{6})\cong \mathbb{C}\oplus S^{(5,1)}\oplus S^{(4,2)}
    \label{C0-M6}
  \end{equation}
  Ahora tomemos:
  \begin{equation*}
    V_{1}=\langle b_{1}\rangle =\langle
    (a_{1},a_{2})\rangle=\langle(\overline{12},\overline{34})\rangle
  \end{equation*}
  \begin{small}
    \begin{align*}
      H_{1}=&\{g \in S_{6}\mid
      gV_{1}=V_{1}\}=\{(1),(12),(34),(56),(12)(34),(12)(56),(34)(56),\\
      &(13)(24),(14)(23),(12)(34)(56),(13)(24)(56),(14)(23)(56),(2314),\\
      &(2413),(2314)(56),(2413)(56)\}
    \end{align*}
  \end{small}
  Con $V_{1}$ y $H_{1}$ construimos la tabla
  \ref{tab:restriccion-H-1-M-6} y calculamos los productos internos
  correspondientes con la reciprocidad de Frobenius de forma similar a los ejemplos anteriores.
  
 \begin{table}[!hbtp]
    \resizebox*{!}{5.8cm}{
      \centering
      \begin{tabular}{c |r r r r r r r r}
        &   & (12) & (12)(34) &          &             &              & &  \\        
        &   & (34) & (12)(56) &(13)(24)  &             & (13)(24)(56) & (2314)& (2314)(56)\\
        Elementos &(1)& (56) & (34)(56) & (14)(23) & (12)(34)(56)& (14)(23)(56) &(2413)&(2413)(56)\\
        \hline
        $\chi_{S^{{(6)}}}$           & 1 & 1  & 1  & 1 & 1 & 1 & 1 & 1 \\
        $\chi_{S^{{(1,1,1,1,1,1)}}}$ & 1 & -1 & 1  & 1 &-1 &-1 &-1 & 1  \\
        $\chi_{S^{{(5,1)}}}$         & 5 & 3  & 1  & 1 &-1 &-1 & 1 &-1  \\
        $\chi_{S^{{(2,1,1,1,1)}}}$   & 5 & -3 &  1 & 1 & 1 & 1 &-1 &-1  \\
        $\chi_{S^{{(4,1,1)}}}$       & 10& 2  & -2 & -2&-2 &-2 & 0 & 0  \\
        $\chi_{S^{{(3,1,1,1)}}}$     & 10&-2  & -2 &-2 & 2 & 2 & 0 & 0  \\
        $\chi_{S^{{(4,2)}}}$         & 9 & 3  & 1  & 1 & 3 &3  &-1 & 1  \\
        $\chi_{S^{{(2,2,1,1)}}}$     & 9 & -3 & 1  & 1 & -3&-3 & 1 & 1  \\
        $\chi_{S^{{(3,3)}}}$         & 5 & 1  & 1  & 1 &-3 &-3 &-1 & -1 \\
        $\chi_{S^{{(2,2,2)}}}$       & 5 & -1 & 1  & 1 & 3 & 3 & 1 & -1 \\
        $\chi_{S^{{(3,2,1)}}}$       & 16& 0  & 0  & 0 & 0 & 0 & 0 &  0 \\
        \hline
        $\chi_{V_{1}}$            & 1 & 1  & 1  & -1& 1 & -1&-1 &-1  \\
      \end{tabular}}
    
    \caption{Caracteres de $S_{6}$ restringidos a $H_{1}$ y carácter de $V_{1}$}
    \label{tab:restriccion-H-1-M-6}
  \end{table}  
  \begin{footnotesize}
    De donde obtenemos:
    \begin{equation}
      C_{1}(M_{6})=S^{(5,1)}\oplus S^{{(4,1,1)}} \oplus S^{{(4,2)}}
      \oplus S^{{(3,3)}} \oplus S^{{(3,2,1)}}
      \label{C1-M6}
    \end{equation}
    Por último consideremos:
    \begin{equation*}
      V_{2}=\langle c_{1}\rangle=\langle(\overline{12},\overline{34},\overline{56})\rangle
    \end{equation*}
  \end{footnotesize}
  \begin{tiny}
    \begin{align*}
      H_{2}=&\{g\in S_{6}\mid
      gV_{2}=V_{2}\}=\{(1),(12),(34),(56),(12)(34),(12)(56),(34)(56),(13)(24),(14)(23),(15)(26),(16)(25),\\
      &(35)(46),(36)(45),(12)(34)(56),(13)(24)(56),(14)(23)(56),(15)(26)(34),(16)(25)(34),(35)(46)(12),\\
      &(36)(45)(12),(2314),(2413),(1625),(1526),(3546),(3645),(2314)(56),(2413)(56),(1625)(34),(1526)(34),\\
      &(3546)(12),(3645)(12),(135)(246),(136)(245),(145)(236),(146)(235),(153)(264),(154)(263),(163)(254),\\
      &(164)(253),(135246),(136245),(145236),(146235),(153264),(154263),(163254),(164253)\}
    \end{align*}
  \end{tiny}

  \begin{table}[h]
    \centering
    \resizebox*{!}{5.8cm}{
      \begin{tabular}{c |r r r r r r r r r r}
        No. de Elementos  & 1 & 3 & 3 & 6 & 6 & 1 & 6 & 8 & 6 & 8 \\
        Clase &(1)& (2) & (2,2) & (2,2) & (4)& (2,2,2) & (2,2,2) & (3,3) & (4,2) & (6)\\
        \hline
        $\chi_{S^{{(6)}}}$         & 1 & 1  & 1  & 1 & 1 &1  & 1 & 1 & 1 & 1 \\
        $\chi_{S^{{(1,1,1,1,1,1)}}}$ & 1 & -1 & 1  & 1 & -1&-1 &-1 & 1 & 1 & -1 \\
        $\chi_{S^{{(5,1)}}}$       & 5 & 3  & 1  & 1 & 1 &-1 &-1 &-1 &-1 & -1 \\
        $\chi_{S^{{(2,1,1,1,1)}}}$  & 5 & -3 &  1 & 1 &-1 & 1 & 1 &-1  &-1 & 1 \\
        $\chi_{S^{{(4,1,1)}}}$     & 10& 2  & -2 & -2& 0 &-2 &-2 & 1  & 0 & 1 \\
        $\chi_{S^{{(3,1,1,1)}}}$    & 10&-2  & -2 &-2 & 0 & 2 & 2 & 1 & 0 & -1 \\
        $\chi_{S^{{(4,2)}}}$       & 9 & 3  & 1  & 1 & -1& 3 & 3 & 0 & 1 &  0 \\
        $\chi_{S^{{(2,2,1,1)}}}$    & 9 & -3 & 1  & 1 & 1 &-3 &-3 & 0 & 1 &  0 \\
        $\chi_{S^{{(3,3)}}}$       & 5 & 1  & 1  & 1 &-1 &-3 & -3& 2 &-1 &  0 \\
        $\chi_{S^{{(2,2,2)}}}$     & 5  & -1& 1  & 1 & 1 & 3 & 3 & 2 & -1 & 0 \\
        $\chi_{S^{{(3,2,1)}}}$     & 16 & 0 & 0  & 0 & 0 & 0 & 0 & -2 & 0 & 0 \\
        \hline
        $\chi_{V_{2}}$ & 1 & 1 & 1 & -1 & -1 & 1 &-1 & 1 &- 1 & 1  \\
      \end{tabular}}

    \caption{Caracteres de $S_{6}$ restringidos a $H_2$ y carácter de $V_{2}$}
    \label{tab:restriccion-H_2-M-6}
  \end{table}
  De la tabla \ref{tab:restriccion-H_2-M-6} obtenemos:
  \begin{equation}
    C_{2}(M_{6})\cong S^{(4,1,1)}\oplus S^{(3,3)}
    \label{C2-M6}
  \end{equation}

  En la figura \ref{fig:diagrama-conmutativo6} se muestra el diagrama
  conmutativo de los complejos de cadena de $M_{6}$, donde $f_{0}$,
  $f_{1}$ y $f_{2}$ son los isomorfismos obtenidos de las expresiones
  \ref{C0-M6}, \ref{C1-M6} y \ref{C2-M6}.

  \begin{figure}[h]
    \centering
    \begin{widepage}
      \scriptsize{
        \[
        \begin{CD}
          0 @>{\partial_{3}}>> C_{2}(M_{6}) @>{\partial_{2}}>> C_{1}(M_{6}) @>{\partial_{1}}>> C_{0}(M_{6}) @>{\varepsilon}>> \mathbb{C}\\
          @VVV @Vf_{2}VV   @Vf_{1}VV  @Vf_{0}VV  @VVV    \\
          0 @>{\widehat\partial_{3}}>> S^{(4,1,1)}\oplus S^{(3,3)}
          @>{\widehat\partial_{2}}>> S^{(5,1)}\oplus S^{(4,1,1)}\oplus
          S^{(4,2)}\oplus S^{(3,3)}\oplus S^{(3,2,1)}
          @>{\widehat\partial_{1}}>> \mathbb{C} \oplus S^{(5,1)}\oplus
          S^{(4,2)} @>{\widehat \varepsilon}>> \mathbb{C}
        \end{CD}
        \]
      }
    \end{widepage}
\caption{Diagrama conmutativo de los complejos de cadenas de $M_{6}$}
\label{fig:diagrama-conmutativo6}
\end{figure}

Por calcular las homologías reducidas $\widetilde
H_{0}(M_{6})$, $\widetilde H_{1}(M_{6})$ y $\widetilde H_{2}(M_{6})$.

Como $\widehat\varepsilon$ es suprayectiva y por el teorema \ref{teorema-isomorfismo-mod} tenemos:
\begin{equation*}
  (\mathbb{C} \oplus S^{(5,1)}\oplus S^{(4,2)})/\ker\widehat\varepsilon\cong \im \widehat\varepsilon=\mathbb{C}
\end{equation*}

así que
\begin{equation*}
  \label{ker0-M6}
  \ker\widehat\varepsilon\cong  S^{(5,1)}\oplus S^{(4,2)} 
\end{equation*}
Sabemos que $\widetilde H_{0}(M_{6})=\ker \widehat\varepsilon/\im
\widehat\partial_{1}=0$ pues $M_{6}$ es conexo, se sigue que $\ker \widehat\varepsilon\cong
\im\widehat\partial_{1}$, con lo cual:
\begin{equation}
  \label{im1-M6}
  \im \widehat\partial_{1}\cong  S^{(5,1)}\oplus S^{(4,2)} 
\end{equation}

Además de el teorema \ref{teorema-isomorfismo-mod} y la expresión \ref{im1-M6} se tiene:
$$(S^{(5,1)}\oplus S^{(4,1,1)}\oplus S^{(4,2)}\oplus S^{(3,3)}\oplus S^{(3,2,1)})/\ker \widehat\partial_{1}\cong \im \widehat\partial_{1}$$
De lo anterior y la proposición \ref{modulos-iguales} tenemos:
\begin{equation}
\label{ker1-M6}
\ker \widehat\partial_{1}=S^{(4,1,1)}\oplus S^{(3,3)}\oplus S^{(3,2,1)}
\end{equation}
\begin{figure}[h]
  \centering
  \begin{center}
    \begin{minipage}{0.45\linewidth}
      % \centering
      % \begin{figure}[h]
      \centering
      \begin{tikzpicture}[scale=.3]
        \SetVertexNoLabel \GraphInit[vstyle=Classic]
        \SetUpVertex[MinSize=1pt] \grStar[RA=4,Math]{7}
        \grEmptyCycle[RA=8,prefix=c]{24}
        \EdgeInGraphMod*{a}{6}{1}{0}{2}
        \EdgeInGraphMod*{c}{24}{1}{0}{2}
        \EdgeFromOneToSeq{a}{c}{1}{2}{5}
        \EdgeFromOneToSeq{a}{c}{2}{6}{9}
        \EdgeFromOneToSeq{a}{c}{3}{10}{13}
        \EdgeFromOneToSeq{a}{c}{4}{14}{17}
        \EdgeFromOneToSeq{a}{c}{5}{18}{21}
        \EdgeFromOneToSeq{a}{c}{0}{22}{23}
        \EdgeFromOneToSeq{a}{c}{0}{0}{1}
      \end{tikzpicture}

      \caption{$G_{6}$}
      \label{fig:emparejamiento6-1}
      % \end{figure}
    \end{minipage}%\quad
    \begin{minipage}{0.45\linewidth}
      % \begin{figure}[h]
      \centering
      \begin{tikzpicture}[scale=.3]
        \SetVertexNoLabel \GraphInit[vstyle=Classic]
        \SetUpVertex[MinSize=1pt] \grStar[RA=2,Math,rotation=270]{4}
        \grEmptyCycle[RA=4,Math,prefix=b]{6}
        \grEmptyCycle[RA=6,Math,prefix=c,rotation=-15]{12}
        \grEmptyCycle[RA=8,Math,prefix=d,rotation=-22]{24}
        \EdgeFromOneToSeq{a}{b}{1}{0}{1}
        \EdgeFromOneToSeq{a}{b}{2}{2}{3}
        \EdgeFromOneToSeq{a}{b}{0}{4}{5}
        \EdgeFromOneToSeq{b}{c}{0}{0}{1}
        \EdgeFromOneToSeq{b}{c}{1}{2}{3}
        \EdgeFromOneToSeq{b}{c}{2}{4}{5}
        \EdgeFromOneToSeq{b}{c}{3}{6}{7}
        \EdgeFromOneToSeq{b}{c}{4}{8}{9}
        \EdgeFromOneToSeq{b}{c}{5}{10}{11}
        \EdgeFromOneToSeq{c}{d}{0}{0}{1}
        \EdgeFromOneToSeq{c}{d}{1}{2}{3}
        \EdgeFromOneToSeq{c}{d}{2}{4}{5}
        \EdgeFromOneToSeq{c}{d}{3}{6}{7}
        \EdgeFromOneToSeq{c}{d}{4}{8}{9}
        \EdgeFromOneToSeq{c}{d}{5}{10}{11}
        \EdgeFromOneToSeq{c}{d}{6}{12}{13}
        \EdgeFromOneToSeq{c}{d}{7}{14}{15}
        \EdgeFromOneToSeq{c}{d}{8}{16}{17}
        \EdgeFromOneToSeq{c}{d}{9}{18}{19}
        \EdgeFromOneToSeq{c}{d}{10}{20}{21}
        \EdgeFromOneToSeq{c}{d}{11}{22}{23}
      \end{tikzpicture}

      \caption{Gráfica de .....}
      \label{grafica-contraible-M6}
      % \end{figure};
    \end{minipage}
  \end{center}
\end{figure}

Por otro lado, $\widehat\partial_{3}$ es un morfismo de módulos, así que:
\begin{equation*}
\im\widehat\partial_{3}=\widehat\partial_{3}(0)=0
\label{im3-KM5}
\end{equation*}
También de
\setlength{\fboxsep}{0pt}\colorbox{green}{(referencia[Bouc]????)} sabemos:
\begin{equation*}
\widetilde H_{2}(M_{6})=\ker \widehat\partial_{2}/\im \widehat\partial_{3}=0.
\end{equation*}
por lo que $\ker \widehat\partial_{2}=0$. De el teorema
\ref{teorema-isomorfismo-mod} tenemos:
$$(S^{(4,1,1)}\oplus S^{(3,3)})/\ker \widehat\partial_{2}\cong \im \widehat\partial_{2}$$
De lo anterior y la proposición \ref{modulos-iguales} tenemos:
\begin{equation}
\im \widehat\partial_{2}=S^{(4,1,1)}\oplus S^{(3,3)}
\label{im2-M6}
\end{equation}

De las ecuaciones \ref{ker1-M6}, \ref{im2-M6} y la proposición
\ref{modulos-iguales} tenemos:
\begin{eqnarray*}
  \widetilde H_{1}(M_{6})&=&\ker \widehat\partial_{1}/\im
  \widehat\partial_{2}\\
  &=&(S^{(4,1,1)}\oplus S^{(3,3)}\oplus
  S^{(3,2,1)})/(S^{(4,1,1)}\oplus S^{(3,3)})\\
  &=&S^{(3,2,1)}
\end{eqnarray*}

En lo anterior usamos el hecho de que $\widetilde H_{2}(M_{6})=0$ de el
resultado de \setlength{\fboxsep}{0pt}\colorbox{green}{(referencia[Bouc]????)}, sin embargo veamos que no es necesario, por lo que
necesitamos calcular $\im \widehat\partial_{2}$. 

Verifiquemos primero que $\dim(\im \partial_{2})=15$. Del teorema
\ref{imT}, sabemos que $\im \partial_{2}=\langle\partial_{2}(c_{1}),\ldots,\partial_{2}(c_{15})\rangle$,
veamos que el número de vectores linealmente independientes de la
$\im \partial_{2}$ es 15, es decir, si
$$\lambda_{1}\partial_{2}(c_{1})+\lambda_{2}\partial_{2}(c_{2})+\ldots+\lambda_{15}\partial_{2}(c_{15})=0$$
así que:

\begin{footnotesize}
  \begin{align*}
    &\lambda_{1}(a_{10}a_{15}-a_{1}a_{15}+a_{1}a_{10})+\lambda_{2}(a_{11}a_{14}-a_{1}a_{14}+a_{1}a_{11})+\lambda_{3}(a_{12}a_{13}-a_{1}a_{13}+a_{1}a_{12})\\
    &+\lambda_{4}(a_{7}a_{15}-a_{2}a_{15}+a_{2}a_{7})+\lambda_{5}(a_{8}a_{14}-a_{2}a_{14}+a_{2}a_{8})+\lambda_{6}(a_{9}a_{13}-a_{2}a_{13}+a_{2}a_{9})\\
    &+\lambda_{7}(a_{6}a_{15}-a_{3}a_{15}+a_{3}a_{6})+\lambda_{8}(a_{8}a_{12}-a_{3}a_{12}+a_{3}a_{8})+\lambda_{9}(a_{9}a_{11}-a_{3}a_{11}+a_{3}a_{9})\\
    &+\lambda_{10}(a_{6}a_{14}-a_{4}a_{14}+a_{4}a_{6})+\lambda_{11}(a_{7}a_{12}-a_{4}a_{12}+a_{4}a_{7})+\lambda_{12}(a_{9}a_{10}-a_{4}a_{10}+a_{4}a_{9})\\
    &+\lambda_{13}(a_{6}a_{13}-a_{5}a_{13}+a_{5}a_{6})+\lambda_{14}(a_{7}a_{11}-a_{5}a_{11}+a_{5}a_{7})+\lambda_{15}(a_{8}a_{10}-a_{5}a_{10}+a_{5}a_{8})\\
    &=0
  \end{align*}
\end{footnotesize}
Claramente $\lambda_{1}=\lambda_{2}=\cdots=\lambda_{15}=0$,
intuituvamente esto se puede deducir de la gráfica
\ref{fig:emparejamiento6}. Sea $b_{j}$ la frontera de algún $2$-simplejo $c_{i}$, las aristas
de $b_{j}$ no pertenece a ningún otro triángulo distinto de $c_{i}$, por
lo tanto $\partial_{2}(c_{j})$ no es combinación lineal de las fronteras
$\partial_{2}(c_{i})$ con $i\neq j$.

Así que $\dim(\im\partial_{2})=15$, lo cual implica que
$\dim(\im\widehat\partial_{2})=15$ (se sigue de la definición de
$\widehat\partial_{2}$).

Ahora, por el teorema \ref{im-mod-irreducible} tenemos:
  $$\widehat\partial_{2}(S^{(4,1,1)})=S^{(4,1,1)} \quad \mbox{o }\quad \widehat\partial_{2}(S^{(4,1,1)})=0$$
  $$\widehat\partial_{2}(S^{(3,3)})=S^{(3,3)} \quad \mbox{o }\quad \widehat\partial_{2}(S^{(3,3)})=0$$
  Además $\dim(S^{(4,1,1)})=10$ y $\dim(S^{(3,3)})=5$, entonces:
  $$\im\widehat\partial_{2}=\widehat\partial_{2}(S^{(4,1,1)}\oplus
  S^{(3,3)})=S^{(4,1,1)}\oplus S^{(3,3)}$$

\end{example}

\begin{example}
  \label{ejemploKM6}
  Consideremos el complejo de clanes de la gráfica de emparejamiento
  $M_{6}$, al que denotamos como $K(M_{6})$, que está dado por el
  conjunto de vértices (triángulos de la gráfica de
  $M_{6}$):

\begin{center}
  \begin{tabular}[h]{lll}
    $a_{1}=(\overline{12},\overline{34},\overline{56})$&$a_{6}=(\overline{13},\overline{26},\overline{45})$&$a_{11}=(\overline{15},\overline{24},\overline{36})$  \\
    $a_{2}=(\overline{12},\overline{35},\overline{46})$&$a_{7}=(\overline{14},\overline{23},\overline{56})$&$a_{12}=(\overline{15},\overline{26},\overline{34})$  \\
    $a_{3}=(\overline{12},\overline{36},\overline{45})$&$a_{8}=(\overline{14},\overline{25},\overline{36})$&$a_{13}=(\overline{16},\overline{23},\overline{45})$  \\
    $a_{4}=(\overline{13},\overline{24},\overline{56})$&$a_{9}=(\overline{14},\overline{26},\overline{35})$&$a_{14}=(\overline{16},\overline{24},\overline{35})$ \\
    $a_{5}=(\overline{13},\overline{25},\overline{46})$&$a_{10}=(\overline{15},\overline{23},\overline{46})$&$a_{15}=(\overline{16},\overline{25},\overline{34})$  
  \end{tabular}
\end{center}

el conjunto $1$-simplejos:
\begin{center}
  \begin{tabular}[h]{llll}
    $b_{1}=(a_{1},a_{2})$ & $b_{13}=(a_{3},a_{8})$ & $b_{25}=(a_{4},a_{14})$ & $b_{37}=(a_{9},a_{12})$ \\     
    $b_{2}=(a_{1},a_{3})$ & $b_{14}=(a_{3},a_{11})$ &$b_{26}=(a_{5},a_{8})$ & $b_{38}=(a_{10},a_{11})$\\     
    $b_{3}=(a_{1},a_{4})$ & $b_{15}=(a_{3},a_{13})$ & $b_{27}=(a_{5},a_{6})$&$b_{39}=(a_{10},a_{12})$\\     
    $b_{4}=(a_{1},a_{7})$ & $b_{16}=(a_{4},a_{7})$ & $b_{28}=(a_{5},a_{15})$& $b_{40}=(a_{10},a_{13})$ \\     
    $b_{5}=(a_{1},a_{12})$ & $b_{17}=(a_{5},a_{10})$ & $b_{29}=(a_{6},a_{9})$& $b_{41}=(a_{11},a_{12})$\\     
    $b_{6}=(a_{1},a_{15})$ & $b_{18}=(a_{6},a_{13})$ & $b_{30}=(a_{6},a_{12})$&$b_{42}=(a_{11},a_{14})$\\ 
    $b_{7}=(a_{2},a_{3})$ & $b_{19}=(a_{8},a_{11})$ & $b_{31}=(a_{7},a_{8})$&$b_{43}=(a_{13},a_{14})$\\
    $b_{8}=(a_{2},a_{5})$ & $b_{20}=(a_{9},a_{14})$ & $b_{32}=(a_{7},a_{9})$&$b_{44}=(a_{13},a_{15})$\\
    $b_{9}=(a_{2},a_{9})$ & $b_{21}=(a_{12},a_{15})$ & $b_{33}=(a_{7},a_{10})$&$b_{45}=(a_{14},a_{15})$\\
    $b_{10}=(a_{2},a_{10})$ & $b_{22}=(a_{4},a_{5})$ & $b_{34}=(a_{7},a_{13})$&\\
    $b_{11}=(a_{2},a_{14})$ & $b_{23}=(a_{4},a_{6})$ & $b_{35}=(a_{8},a_{9})$&\\
    $b_{12}=(a_{3},a_{6})$ & $b_{24}=(a_{4},a_{11})$ & $b_{36}=(a_{8},a_{15})$&
  \end{tabular}
\end{center}

y los $2$-simplejos:
\begin{center}
  \begin{tabular}[h]{lll}
    $c_{1}=(a_{1},a_{2},a_{3})$ & $c_{6}=(a_{3},a_{6},a_{13})$ & $c_{11}=(a_{13},a_{14},a_{15})$   \\
    $c_{2}=(a_{1},a_{4},a_{7})$ & $c_{7}=(a_{3},a_{8},a_{11})$ & $c_{12}=(a_{7},a_{10},a_{13})$\\
    $c_{3}=(a_{1},a_{12},a_{15})$& $c_{8}=(a_{4},a_{5},a_{6})$ & $c_{13}=(a_{4},a_{11},a_{14})$\\
    $c_{4}=(a_{2},a_{5},a_{10})$& $c_{9}=(a_{7},a_{8},a_{9})$ & $c_{14}=(a_{5},a_{8},a_{15})$\\
    $c_{5}=(a_{2},a_{9},a_{14})$& $c_{10}=(a_{10},a_{11},a_{12})$ & $c_{15}=(a_{6},a_{9},a_{12})$
  \end{tabular}
\end{center}

\begin{figure}[h]
  \centering
  \begin{tikzpicture}[scale=.8]
    \GraphInit[vstyle=Normal] \SetVertexNoLabel
    \grCycle[RA=1,prefix=a,rotation=-90]{3}
    \grEmptyCycle[RA=3,prefix=b,rotation=-10]{12}
    \EdgeInGraphMod*{b}{12}{1}{0}{2} \EdgeFromOneToSeq{a}{b}{1}{0}{3}
    \EdgeFromOneToSeq{a}{b}{2}{4}{7} \EdgeFromOneToSeq{a}{b}{0}{8}{11}
    \AssignVertexLabel{a}{\textsl{$a_{2}$},\textsl{$a_{1}$},\textsl{$a_{3}$}}
    \AssignVertexLabel{b}{\textsl{$a_{7}$},\textsl{$a_{4}$},\textsl{$a_{15}$},\textsl{$a_{12}$},\textsl{$a_{6}$},\textsl{$a_{13}$},\textsl{$a_{11}$},\textsl{$a_{8}$},\textsl{$a_{10}$},\textsl{$a_{5}$},\textsl{$a_{14}$},\textsl{$a_{9}$}}
  \end{tikzpicture}
  
  \caption{Gráfica de clanes $K(M_{6})$}
  \label{fig:KM6}
\end{figure}

A continuación obtendremos la descomposición de los espacios de
cadenas $C_{0}(K(M_{6}))$, $C_{1}(K(M_{6}))$ y $C_{2}(K(M_{6}))$ en
módulos de Specht por medio de el teorema de la reciprocidad de Frobenius.
\setlength{\fboxsep}{0pt}\colorbox{green}{(referencia)}

Consideremos:
  $$V_{0}=\langle a_{1}\rangle=\langle(\overline{12},\overline{34},\overline{56})\rangle$$
  \begin{footnotesize}
    \begin{align*}
      H_{0}=&\{g\in S_{6}\mid
      gV_{0}=V_{0}\}=\{(1),(12),(34),(56),(12)(34),(12)(56),(13)(24),(14)(23),\\
      &(15)(26),(16)(25),(34)(56),(35)(46),(36)(45),(1423),(1324),(1625),(1526),\\
      &(3645),(3546),(12)(34)(56),(13)(24)(56),(14)(23)(56),(15)(26)(34),(16)(25)(34),\\
      &(35)(46)(12),(36)(45)(12),(153)(264),(135)(246),(145)(236),(154)(263),\\
      &(136)(245),(163)(254),(146)(235),(164)(253),(1423)(56),(1324)(56),(1625)(34),\\
      &(1526)(34),(3645)(12),(3546)(12),(135246),(136245),(145236),(146235),\\
      &(153264),(154263),(163254),(164253)\}
    \end{align*}
  \end{footnotesize}

  
  \begin{table}[!hbtp]
    \centering
    \begin{tabular}{c |r r r r r r r r}
      No. de Elementos  & 1 & 3 & 9 & 6 & 7 & 8 & 6 & 8  \\
      Clase&(1)& (2) & (2,2) & (4) & (2,2,2)& (3,3) & (4,2) & (6)\\
      \hline
      $\chi_{S^{{(6)}}}$         & 1 & 1  & 1 & 1 & 1 & 1 & 1 & 1\\
      $\chi_{S^{{(1,1,1,1,1,1)}}}$ & 1 & -1 & 1 & -1&-1 & 1 & 1 &-1\\
      $\chi_{S^{{(5,1)}}}$       & 5 & 3  & 1 & 1 &-1 &-1 &-1 &-1\\
      $\chi_{S^{{(2,1,1,1,1)}}}$  & 5 & -3 & 1 &-1 & 1 &-1  &-1 & 1\\
      $\chi_{S^{{(4,1,1)}}}$     & 10& 2  & -2& 0 &-2 & 1  & 0 & 1\\
      $\chi_{S^{{(3,1,1,1)}}}$    & 10&-2  &-2 & 0 & 2 & 1 & 0 &-1\\
      $\chi_{S^{{(4,2)}}}$       & 9 & 3  & 1 & -1& 3 & 0 & 1 & 0\\
      $\chi_{S^{{(2,2,1,1)}}}$    & 9 & -3 & 1 & 1 &-3 & 0 & 1 & 0\\
      $\chi_{S^{{(3,3)}}}$       & 5 & 1  & 1 &-1 & -3& 2 & -1& 0\\
      $\chi_{S^{{(2,2,2)}}}$     & 5  & -1 & 1 & 1 & 3 & 2 &-1& 0\\
      $\chi_{S^{{(3,2,1)}}}$     & 16 & 0  & 0 & 0 & 0 &-2 & 0& 0\\
      \hline
      $\chi_{V_{0}}$        & 1 & 1 & 1 & 1 & 1 & 1 &1 & 1  \\
    \end{tabular}

    \caption{Caracteres de $S_{6}$ restringidos a $H_{0}$ y carácter de $V_{0}$}
    \label{tab:clanes-H_0-6}
  \end{table}
  Con los productos internos calculados a partir de la tabla
  \ref{tab:clanes-H_0-6} obtenemos:
  \begin{equation}
    C_{0}(K(M_{6}))\cong \mathbb{C}\oplus S^{(4,2)}\oplus S^{(2,2,2)}
    \label{C0-KM6}
  \end{equation}
Consideremos los siguientes conjuntos:
  $$V_{1}=\langle b_{1}\rangle=\langle (a_{1},a_{2})\rangle=\langle(\overline{12},\overline{34},\overline{56}),(\overline{12},\overline{35},\overline{46})\rangle$$
  \begin{footnotesize}
    \begin{align*}
      H_{1}&=\{g\in S_{6}\mid gV_{1}=V_{1}\}\\
      &=\{(1),(12),(36),(45),(12)(36),(12)(45),(34)(56),(35)(46),(36)(45),(12)(34)(56),\\
      &\qquad{}(12)(35)(46),(12)(36)(45),(12)(3564),(12)(3465),(3564),(3465)\}
    \end{align*}
  \end{footnotesize}
  Los conjuntos $V_{1}$ y $H_{1}$ nos ayudan a construir la tabla \ref{tab:clanes-H_1-6}.
  
  \begin{table}[!hbtp]
    \centering
    \begin{tabular}{c |r r r r r r r r}
      No. de Elementos  & 1 & 1 & 2 & 3 & 2 & 2 & 2 & 3  \\
      Clase&(1)& (2) & (2) & (2,2) & (2,2)& (4) & (4,2) & (2,2,2)\\
      \hline
      $\chi_{S^{{(6)}}}$         & 1 & 1 &1  & 1 & 1 & 1 & 1 & 1\\
      $\chi_{S^{{(1,1,1,1,1,1)}}}$ & 1 & -1&-1 & 1 & 1 &-1 & 1 &-1\\
      $\chi_{S^{{(5,1)}}}$       & 5 & 3 & 3 & 1 & 1 & 1 &-1 &-1\\
      $\chi_{S^{{(2,1,1,1,1)}}}$  & 5 & -3& -3 & 1 &1 &-1 &-1 & 1\\
      $\chi_{S^{{(4,1,1)}}}$     & 10& 2  & 2 &-2 &-2& 0 & 0 & -2\\
      $\chi_{S^{{(3,1,1,1)}}}$    & 10&-2 & -2&-2 & -2& 0 & 0 &2\\
      $\chi_{S^{{(4,2)}}}$       & 9 & 3 & 3 & 1 & 1 & -1& 1 & 3\\
      $\chi_{S^{{(2,2,1,1)}}}$    & 9 &-3 & -3&1  & 1 & 1 & 1 & -3\\
      $\chi_{S^{{(3,3)}}}$       & 5 & 1 & 1 &1  &1  & -1& -1& -3\\
      $\chi_{S^{{(2,2,2)}}}$     & 5 &-1 & -1&1  & 1 & 1 &-1& 3\\
      $\chi_{S^{{(3,2,1)}}}$     & 16 &0 &0  & 0 & 0 & 0 & 0& 0\\
      \hline
      $\chi_{V_{1}}$        & 1 & 1 & -1 & 1 & -1 & -1 &-1 & 1  \\
    \end{tabular}
    
    \caption{Caracteres de $S_{6}$ restringidos a $H_{1}$ y carácter de $V_{1}$}
    \label{tab:clanes-H_1-6}
  \end{table}
  Después de calcular los productos internos por medio de la tabla
  anterior resulta:
  \begin{equation}
    C_{1}(K(M_{6}))\cong S^{(2,1,1,1,1)}\oplus S^{(3,1,1,1)}\oplus
    S^{(4,2)} \oplus S^{(2,2,2)}\oplus S^{(3,2,1)}
    \label{C1-KM6}
  \end{equation}
  Ahora tomemos los siguientes conjuntos:
  $$V_{2}=\langle c_{1}\rangle=\langle
  (a_{1},a_{2},a_{3})\rangle=\langle(\overline{12},\overline{34},\overline{56}),(\overline{12},\overline{35},\overline{46}),(\overline{12},\overline{36},\overline{45})\rangle$$

\begin{footnotesize}
  \begin{align*}
    H_{2}=&\{g\in S_{6}\mid
    gV_{2}=V_{2}\}=\{(1),(12),(34),(35),(36),(45),(46),(56),(345),(346),\\
    &(354),(356),(364),(365),(456),(465),(12)(34),(12)(35),(12)(36),(12)(45),\\
    &(12)(46),(12)(56),(34)(56),(35)(46),(36)(45),(3456),(3465),(3546),(3564),\\
    &(3645),(3654),(345)(12),(346)(12),(354)(12),(356)(12),(364)(12),(365)(12),\\
    &(456)(12),(465)(12),(12)(34)(56),(12)(35)(46),(12)(36)(45),(3456)(12),\\
    &(3465)(12),(3546)(12),(3564)(12),(3645)(12),(3654)(12)\}
  \end{align*}
\end{footnotesize}

\begin{table}[!hbtp]
  \centering
  \resizebox*{!}{6cm}{
  \begin{tabular}{c |r r r r r r r r r r}
    No. de Elementos  & 1 & 1 & 6 & 8 & 3 & 6 & 6 & 8 & 3 & 6  \\
    Clase &(1)& (2) & (2) & (3) & (2,2)& (2,2) & (4) & (3,2) & (2,2,2) & (4,2)\\
    \hline
    $\chi_{S^{{(6)}}}$         &  1 &  1 &  1 &  1 &  1 &  1 &  1 &  1 &  1 &  1 \\ 
    $\chi_{S^{{(1,1,1,1,1,1)}}}$ &  1 & -1 & -1 &  1 &  1 &  1 & -1 & -1 & -1& 1\\ 
    $\chi_{S^{{(5,1)}}}$       &  5 &  3 &  3 &  2 &  1 &  1 &  1 &  0 & -1&-1\\
    $\chi_{S^{{(2,1,1,1,1)}}}$  &  5 & -3 & -3 &  2 &  1 &  1 & -1 &  0 &  1&-1\\  
    $\chi_{S^{{(4,1,1)}}}$     & 10 &  2 &  2 &  1 & -2 & -2 &  0 & -1 & -2 &0\\
    $\chi_{S^{{(3,1,1,1)}}}$    & 10 & -2 & -2 &  1 & -2 & -2 &  0 &  1 &  2 &0\\
    $\chi_{S^{{(4,2)}}}$       &  9 &  3 &  3 &  0 &  1 &  1 & -1 &  0 &  3 & 1\\ 
    $\chi_{S^{{(2,2,1,1)}}}$    &  9 & -3 & -3 &  0 &  1 &  1 &  1 &  0 & -3 & 1\\   
    $\chi_{S^{{(3,3)}}}$       &  5 &  1 &  1 & -1 &  1 &  1 & -1 &  1 & -3 &-1\\
    $\chi_{S^{{(2,2,2)}}}$      &  5 & -1 & -1 & -1 &  1 &  1 &  1 & -1 &  3 &-1\\
    $\chi_{S^{{(3,2,1)}}}$      & 16 &  0 &  0 & -2 &  0 &  0 &  0 &  0 &  0 &0 \\
    \hline
    $\chi_{V_{2}}$        & 1 & 1 & -1 & 1 & 1 & -1 &-1 & 1 & 1 & -1 \\
  \end{tabular}}

\caption{Caracteres de $S_{6}$ restringidos a $H_{2}$ y carácter de $V_{2}$}
\label{tab:clanes-H_2-6}
\end{table}

De donde obtenemos:

\begin{equation}
  C_{2}(K(M_{6}))\cong S^{(2,1,1,1,1)}\oplus S^{(3,1,1,1)}
  \label{C2-KM6}
\end{equation}

En la figura \ref{fig:diagrama-conmutativo-clanes5} se muestra el diagrama
conmutativo de los complejos de cadena de $K(M_{6})$, donde $f_{0}$,
$f_{1}$ y $f_{2}$ son los isomorfismos obtenidos de las expresiones
\ref{C0-KM6}, \ref{C1-KM6} y \ref{C2-KM6}.

\begin{sidewaysfigure}%[h]
  {\small
    \[
    \begin{CD}
      0 @>{\partial_{3}}>> C_{2}(K(M_{6})) @>{\partial_{2}}>> C_{1}(K(M_{6})) @>{\partial_{1}}>> C_{0}(K(M_{6})) @>{\varepsilon}>> \mathbb{C}\\
      @VVV   @Vf_{2}VV   @Vf_{1}VV   @Vf_{0}VV   @VVV    \\
      0  @>{\widehat\partial_{3}}>> S^{(2,1,1,1,1)}\oplus S^{(3,1,1,1)} @>{\widehat\partial_{2}}>>
      S^{(2,1,1,1,1)}\oplus S^{(3,1,1,1)}\oplus S^{(4,2)}\oplus
      S^{(2,2,2)}\oplus S^{(3,2,1)} @>{\widehat\partial_{1}}>>
      \mathbb{C} \oplus S^{(4,2)}\oplus S^{(2,2,2)} @>{\widehat
        \varepsilon}>> \mathbb{C}
    \end{CD}
    \]
  }
  
  \caption{Diagrama conmutativo de los complejos de cadenas de $K(M_{6})$}
  \label{fig:diagrama-conmutativo-clanes6}
\end{sidewaysfigure}

Calculemos las homologías reducidas $\widetilde H_{0}(K(M_{6}))$,
$\widetilde H_{1}(K(M_{6}))$ y $\widetilde H_{2}(K(M_{6}))$.

Como $\widehat\varepsilon$ es suprayectiva y por el teorema
\ref{teorema-isomorfismo-mod} tenemos:



\begin{equation*}
  (\mathbb{C} \oplus S^{(4,2)}\oplus S^{(2,2,2)})/\ker\widehat\varepsilon\cong \im \widehat\varepsilon=\mathbb{C}
\end{equation*}
así que
\begin{equation*}
  \label{ker0-KM6}
  \ker\widehat\varepsilon\cong S^{(4,2)}\oplus S^{(2,2,2)}
\end{equation*}
Sabemos que $\widetilde H_{0}(K(M_{6}))=\ker \widehat\varepsilon/\im
\widehat\partial_{1}=0$ pues $K(M_{6})$ es conexo, se sigue que $\ker \widehat\varepsilon\cong
\im\widehat\partial_{1}$, con lo cual:
\begin{equation}
  \label{im1-KM6}
  \im \widehat\partial_{1}\cong S^{(4,2)}\oplus S^{(2,2,2)}
\end{equation}

Además el teorema \ref{teorema-isomorfismo-mod} y la expresión \ref{im1-KM6} se tiene:
$$(S^{(2,1,1,1,1)}\oplus S^{(3,1,1,1)}\oplus S^{(4,2)}\oplus
      S^{(2,2,2)}\oplus S^{(3,2,1)})/\ker \widehat\partial_{1}\cong \im \widehat\partial_{1}$$
De lo anterior y la proposición \ref{modulos-iguales} tenemos:
\begin{equation}
\label{ker1-KM6}
\ker \widehat\partial_{1}=S^{(2,1,1,1,1)}\oplus S^{(3,1,1,1)}\oplus S^{(3,2,1)}
\end{equation}
Por otro lado, $\widehat\partial_{3}$ es un morfismo de módulos, así que:
\begin{equation*}
\im\widehat\partial_{3}=\widehat\partial_{3}(0)=0
\label{im3-KM6}
\end{equation*}
Como $M_{6}\simeq K(M_{6})$
\setlength{\fboxsep}{0pt}\colorbox{green}{(referencia)} por el teorema
\setlength{\fboxsep}{0pt}\colorbox{green}{(referencia)} tenemos que
$\widetilde H_{2}(M_{6})=\widetilde H_{2}(K(M_{6}))$ y  $\widetilde
H_{2}(M_{6})=0$, así que:

\begin{equation*}
\widetilde H_{2}(K(M_{6}))=\ker \widehat\partial_{2}/\im \widehat\partial_{3}=0.
\end{equation*}
por lo que $\ker \widehat\partial_{2}=0$. De el teorema
\ref{teorema-isomorfismo-mod} tenemos:
$$(S^{(2,1,1,1,1)}\oplus S^{(3,1,1,1)})/\ker \widehat\partial_{2}\cong \im
\widehat\partial_{2}$$
De lo anterior y la proposición \ref{modulos-iguales} tenemos:
\begin{equation}
\im \widehat\partial_{2}=S^{(2,1,1,1,1)}\oplus S^{(3,1,1,1)}
\label{im2-KM6}
\end{equation}

De las ecuaciones \ref{ker1-KM6} y \ref{im2-KM6} tenemos:

\begin{eqnarray*}
  \widetilde H_{1}(K(M_{6}))&=&\ker \widehat\partial_{1}/\im
  \widehat\partial_{2}\\
  &=&(S^{(2,1,1,1,1)}\oplus S^{(3,1,1,1)}\oplus
  S^{(3,2,1)})/(S^{(2,1,1,1,1)}\oplus S^{(3,1,1,1)})\\
  &=&S^{(3,2,1)}
\end{eqnarray*}




Anteriormente usamos el hecho de que $M_{6}\simeq K(M_{6})$  para
concluir que $\widetilde H_{2}(K(M_{6}))=\widetilde H_{2}(M_{6})=0$, sin embargo veamos que no es necesario, por lo que
necesitamos calcular $\im \widehat\partial_{2}$. 

Verifiquemos primero que $\dim(\im \partial_{2})=15$. Del teorema
\ref{imT}, sabemos que
$\im \partial_{2}=\langle\partial_{2}(c_{1}),\ldots,\partial_{2}(c_{15})\rangle$,
veamos que el número de vectores linealmente independientes de la
$\im \partial_{2}$ es 15, es
decir, si
$$\lambda_{1}\partial_{2}(c_{1})+\lambda_{2}\partial_{2}(c_{2})+\ldots+\lambda_{15}\partial_{2}(c_{15})=0$$
así que:




  



\begin{footnotesize}
  \begin{align*}
    &\lambda_{1}(a_{2}a_{3}-a_{1}a_{3}+a_{1}a_{2})+\lambda_{2}(a_{4}a_{7}-a_{1}a_{7}+a_{1}a_{4})+\lambda_{3}(a_{12}a_{15}-a_{1}a_{15}+a_{1}a_{12})\\
    &+\lambda_{4}(a_{5}a_{10}-a_{2}a_{10}+a_{2}a_{5})+\lambda_{5}(a_{9}a_{14}-a_{2}a_{14}+a_{2}a_{9})+\lambda_{6}(a_{6}a_{13}-a_{3}a_{13}+a_{3}a_{6})\\
    &+\lambda_{7}(a_{8}a_{11}-a_{3}a_{11}+a_{3}a_{8})+\lambda_{8}(a_{5}a_{6}-a_{4}a_{6}+a_{4}a_{5})+\lambda_{9}(a_{8}a_{9}-a_{7}a_{9}+a_{7}a_{8})\\
    &+\lambda_{10}(a_{11}a_{12}-a_{10}a_{12}+a_{10}a_{11})+\lambda_{11}(a_{14}a_{15}-a_{13}a_{15}+a_{13}a_{14})\\
    &+\lambda_{12}(a_{10}a_{13}-a_{7}a_{13}+a_{7}a_{10})+\lambda_{13}(a_{11}a_{14}-a_{4}a_{14}+a_{4}a_{11})\\
    &+\lambda_{14}(a_{8}a_{15}-a_{5}a_{15}+a_{5}a_{8})\lambda_{15}(a_{9}a_{12}-a_{6}a_{12}+a_{6}a_{9})\\
    &=0
  \end{align*}
\end{footnotesize}


Claramente $\lambda_{1}=\lambda_{2}=\cdots=\lambda_{15}=0$,
intuituvamente esto se puede deducir de la gráfica \ref{fig:KG_5}.
Sea $b_{j}$ la frontera de algún $2$-simplejo $c_{i}$, las aristas
de $b_{j}$ no pertenece a ningún otro triángulo distinto de $c_{i}$, por
lo tanto $\partial_{2}(c_{j})$ no es combinación lineal de las fronteras
$\partial_{2}(c_{i})$ con $i\neq j$.

Así que $\dim(\im\partial_{2})=15$, lo cual implica que
$\dim(\im\widehat\partial_{2})=15$ (se sigue de la definición de
$\widehat\partial_{2}$).

Ahora, por el teorema \ref{im-mod-irreducible} tenemos:
  $$\widehat\partial_{2}(S^{(2,1,1,1,1)})=S^{(2,1,1,1,1)} \quad \mbox{o }\quad \widehat\partial_{2}(S^{(2,1,1,1,1)})=0$$
  $$\widehat\partial_{2}(S^{(3,1,1,1)})=S^{(3,1,1,1)} \quad \mbox{o }\quad \widehat\partial_{2}(S^{(3,1,1,1)})=0$$
Además $\dim(S^{(2,1,1,1,1)})=5$ y $\dim(S^{(3,1,1,1)})=10$, entonces:
$$\im\widehat\partial_{2}=\widehat\partial_{2}(S^{(2,1,1,1,1)}\oplus S^{(3,1,1,1)})=S^{(2,1,1,1,1)}\oplus S^{(3,1,1,1)}$$

\end{example}
%Vamos a citar \cite{MR0225619}

\bibliographystyle{plain}
\bibliography{labiblio}

\printindex


\end{document}
