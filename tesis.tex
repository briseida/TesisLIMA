
\documentclass[12pt]{book}

\usepackage[dvipsnames]{xcolor}
\usepackage{amssymb,latexsym}
\usepackage{graphicx}


\usepackage[spanish,mexico,es-nolayout]{babel}
\usepackage[utf8]{inputenc}
\usepackage{amsmath,amscd}
%\usepackage{amssymb}
\usepackage{amsthm}
%\usepackage{graphicx}
\usepackage{color}
\usepackage{tikz}
\usepackage{tkz-berge}
\usetikzlibrary{positioning,backgrounds}
\usepackage{makeidx}
\usepackage{url}
\usepackage{xspace}
\usepackage{tocbibind}
\usepackage{rotating}
\usepackage{young}
\usepackage{ytableau}
% ver http://gilmation.com/articles/latex-margins-for-book-binding/
% y http://tex.stackexchange.com/questions/50258/margins-of-book-class
\usepackage[margin=3.5cm]{geometry}
\geometry{bindingoffset=1cm}

\usepackage{babelbib}
\usepackage{rotating}

\usepackage{changepage}                 % adjust margins for selected portions

% wide page for side by side figures, tables, etc
% http://tex.stackexchange.com/a/154766/250
\newlength{\offsetpage}
\setlength{\offsetpage}{2.0cm}
\newenvironment{widepage}{\begin{adjustwidth}{-\offsetpage}{-\offsetpage}%
    \addtolength{\textwidth}{2\offsetpage}}%
{\end{adjustwidth}}


\usetikzlibrary{positioning,shapes,fit,arrows,decorations.pathmorphing}
\definecolor{myblue}{RGB}{56,94,141}


\newtheorem{theorem}{Teorema}[section]
\newtheorem{corollary}[theorem]{Corolario}
\newtheorem{proposition}[theorem]{Proposición}

\theoremstyle{definition}

\newtheorem{definition}[theorem]{Definición}
\newtheorem{notation}[theorem]{Notación}
\newtheorem{example}[theorem]{Ejemplo}
\newtheorem{lemma}[theorem]{Lema}

\DeclareMathOperator{\im}{im}
\DeclareMathOperator{\sgn}{sgn}
\DeclareMathOperator{\tr}{tr}
\DeclareMathOperator{\End}{End}

\newcounter{in}
\newcounter{ini}

\makeindex

\newcommand{\elespacio}{1.4cm}

\begin{document}
\mainmatter 
\begin{titlepage}
  \begin{center}
    \null
    \vspace*{\fill}

    \includegraphics[scale=1.2,bb=55 20 0 0]{escudouaeh.pdf}

    \vspace*{\elespacio}

    \textsc{Universidad Autónoma del Estado de Hidalgo}

    \textsc{Instituto de Ciencias Básicas e Ingeniería}

    \textsc{Área Académica de Matemáticas y Física}

    \vspace*{\elespacio}

    {\Huge\bfseries Representaciones del grupo simétrico en homologías\par}

    \vspace*{\elespacio}

    {\large Tesis que para obtener el título de}

    \vspace*{\elespacio}

    {\Large\textsc{Licenciada en Matemáticas Aplicadas}}

    \vspace*{\elespacio}

    {\large presenta}

    \vspace*{\elespacio}

    {\Huge Briseida Guadalupe Trejo Escamilla}

    \vspace*{\elespacio}

    {\large bajo la dirección de}

    \bigskip

    {\Large Dr.~Rafael Villarroel Flores}

    \bigskip

    {Pachuca, Hidalgo. Junio de 2013.}

    \vspace*{\fill}

  \end{center}
\end{titlepage}

\thispagestyle{empty}
\begin{flushleft}
  {\bfseries\Large Resumen}
\end{flushleft}

El presente trabajo tiene como objetivo analizar la estructura de
los módulos de homología de ciertos complejos simpliciales construidos a partir de una
gráfica simple finita en donde actúa el grupo simétrico $S_{n}$.

Se define la gráfica de emparejamientos $M(G)$ de la gráfica simple
$G$ como la gráfica cuyos vértices son las aristas de $G$ y dos
vértices son adyacentes si las correspondientes aristas son ajenas. En
la gráfica $G_{n}=M(K_{n})$, el grupo~$S_{n}$ actúa de manera natural, por
lo que sus módulos de homología con coeficientes en el campo de los números complejos definen
representaciones de $S_{n}$. 

La descomposición en irreducibles de tales módulos de homología ha sido exhibida
por Bouc (1984). En este trabajo se muestra el resultado correspondiente para la
gráfica de clanes $K(G_{n})$, donde $n\leq 6$.

\vspace{2cm}

\begin{flushleft}
  {\bfseries\Large Abstract}
\end{flushleft}

In this work we seek to analyze the structure of the homology module of
certain simplicial complexes built from a finite graph where the
symmetric group~$S_{n}$ acts. 

We define the matching graph $M(G)$ from a simple graph $G$ as 
the graph whose vertices are the edges of $G$ and two vertices are
adjacent if the corresponding edges are disjoint. In the graph
$G_{n}=M(K_{n})$, the group~$S_{n}$ acts naturally, so their homology module
with coefficients in the field of complex numbers define representations of $S_{n}$. 

The decomposition into irreducibles of such homology modules has been shown
by Bouc (1984). In this work, we show the corresponding calculations for
the clique graph $K(G_{n})$, where $n\leq 6$.

\chapter*{Introducción}

En matemáticas como en otras ciencias es importante descomponer cierto
objeto en piezas y mejor aún si tales piezas son las más pequeñas. Por
ejemplo, de el teorema fundamental de la aritmética sabemos que todo número
entero positivo puede ser expresado como producto de potencias de
primos. Enseguida se muestran dos ejemplos más de álgebra lineal. 

Una transformación lineal $T$ en un espacio vectorial $V$ de dimensión
finita, es diagonalizable si y solo si $V$ es suma directa de los
espacios propios de $T$, es decir,
$$V=E_{\lambda_{1}}\oplus E_{\lambda_{2}}\oplus \cdots \oplus E_{\lambda_{k}},$$
donde $\lambda_{1},\lambda_{2},\dots,\lambda_{k}$ son los distintos
valores propios de $T$.
%ver [\cite{friedberg1982algebra}, teorema 5.14, p.257].

Dada una transformación lineal $T$ en un espacio vectorial $V$ de
dimensión finita. Es deseable descomponer a $V$ en una suma directa de
subespacios $T$-invariantes propios, es decir, cuando se aplica la
transformación a cada elemento del subespacio lo envía a otro elemento
del mismo subespacio. Esta descomposición de $V$ resulta útil, puesto que el
comportamiento de $T$ puede ser inferido a través de su comportamiento
en cada uno de los sumandos directos, es decir, de la restricción de
$T$ a un subespacio $T$-invariante $W$, la cual es la transformación
lineal $T$ en $W$. El siguiente ejemplo enuncia como son los espacios
$T$-invariantes bajo la condición de que $T$ sea
diagonalizable: Si $T$ es diagonalizable, entonces $V$ puede
descomponerse en una suma directa de subespacios $T$-invariantes
unidimensionales, que son los subespacios generados por los vectores
de la base formada por los vectores propios de $T$. 
%[\cite{friedberg1982algebra}, ejercicio 7, p.303]

En álgebra lineal se estudian espacios vectoriales sobre campos
arbitrarios. El concepto de módulo es una generalización de un espacio
vectorial. Se obtiene la generalización simplemente reemplazando el
campo por algún anillo. Ahora hagamos ciertas analogías relacionando
algunos conceptos de la teoría de representaciones con el lenguaje ya
conocido de álgebra lineal: asociamos módulo con espacio vectorial, submódulo con
subespacio y submódulo irreducible con subespacio $T$-invariante.

El teorema fundamental de la aritmética es análogo al teorema de
Maschke [\cite{sagan2001symmetric}, teorema 1.5.3, p.16], el cual
sigue la idea de descomponer a un módulo en sus piezas más
pequeñas. El teorema de Maschke enuncia que, un módulo es completamente reducible,
es decir, es suma directa de submódulos irreducibles.
%[\cite{fulton1991representation}, proposición 1.8, p.7]. pag 24 teorema de Machke.

A cada campo $F$ y cada grupo $G$ se le asocia un anillo denotado por
$FG$ llamado el anillo de grupo. Estudiar estructuras de módulo sobre el
anillo de grupo $FG$, es equivalente a estudiar $G$-módulos, que a su
vez, es equivalente a estudiar representaciones de $G$ en
$V$, donde $V$ es un espacio vectorial sobre $F$. Usaremos
tales conceptos según sea conveniente, por ejemplo, nos referimos a
una representación de $G$ en $V$ cuando consideramos caracteres y
$G$-módulo cuando descomponemos tal módulo en submódulos
irreducibles. 

Nos interesa la llamada teoría ordinaria de representaciones de grupos finitos
actuando en un espacio vectorial de dimensión finita sobre el campo de
los números complejos y más generalmente sobre campos de
característica que no divide al orden del grupo, esta restricción en
la característica implica que todo módulo es completamente
reducible. Por otra parte, la teoría modular, estudia módulos sobre campos cuya
característica divide al orden de el grupo, en este caso pueden
existir módulos que no son completamente reducibles, como se
muestra en el ejemplo \ref{ej-no-reducibilidad}. En este trabajo no consideraremos la teoría
modular, ya que en ese contexto no necesariamente se cumple el teorema de Maschke. 

Un complejo simplicial $\Delta$ está formado por un conjunto de
vértices y una colección de subconjuntos de vértices, llamados
simplejos, tal que si $A$ es un elemento de $\Delta$, cada subconjunto
no vacío de $A$ pertenece a $\Delta$. A cada complejo simplicial se le
puede asociar un espacio topológico $|\Delta|$, llamado su realización
geométrica. Los complejos simpliciales son una herramienta fundamental
para introducir conceptos topológicos dentro de la combinatoria.

Consideraremos complejos simpliciales obtenidos a partir de
gráficas. Es decir, a cada gráfica $G$ simple finita se le asocia un
espacio topológico por medio del complejo simplicial $\Delta(G)$, cuyos
simplejos son las subgráficas completas. 
% Por lo tanto, a la gráfica $G$ se le asocian conceptos topológicos por
% medio de $|\Delta(G)|$. Por ejemplo, si $G_{1}$ y $G_{2}$ son gráficas
% finitas, diremos que $G_{1}$ y $G_{2}$ son homeomorfas si
% $|\Delta(G_{1})|$ y $|\Delta(G_{2})|$ lo son.

Estamos interesados en la graficas de emparejamientos $G_{n}$ las
cuales se construyen a partir  de gráficas completas de $n$ vértices,
a su vez, $K(G_{n})$ denota las gráficas de clanes obtenidas de
$G_{n}$. Con estas gráficas se definen los complejos simpliciales
$M_{n}=\Delta(G_{n})$ y $K(M_{n})=\Delta(K(G_{n}))$. Todos los cálculos que
realizamos son para $n=4,5,6$ y sus respectivas gráficas aparecen abajo.

\begin{center}
  \begin{minipage}{0.3\linewidth}
    \centering
    \begin{tikzpicture}[x=0.8 cm,y=0.8 cm]
      \draw[help lines] (-2,0);% grid (0,2);
      \GraphInit[vstyle=Classic] \SetUpVertex[MinSize=1pt]
      \SetVertexNoLabel \Vertex[x=-2,y=2,Math,LabelOut,Lpos=90]{12}
      \Vertex[x=-2,y=0,Math,LabelOut,Lpos=-90]{34}
      \Vertex[x=-1,y=0,Math,LabelOut,Lpos=-90]{24}
      \Vertex[x=-1,y=2,Math,LabelOut,Lpos=90]{13}
      \Vertex[x=0,y=2,Math,LabelOut,Lpos=90]{14}
      \Vertex[x=0,y=0,Math,LabelOut,Lpos=-90]{23} \Edge(12)(34)
      \Edge(24)(13) \Edge(14)(23)
    \end{tikzpicture}
  
    $G_{4}$
  \end{minipage}
  \begin{minipage}{0.3\linewidth}
    \centering
    \begin{tikzpicture}[x=0.8 cm,y=0.8 cm]
      \draw[help lines] (-2,0);% grid (2,0);
      \GraphInit[vstyle=Classic] \SetUpVertex[MinSize=1pt]
      \SetVertexNoLabel
      \Vertex[x=-2,y=0,Math,LabelOut,Lpos=-90]{12,34} \Vertex[x=0,
      y=0,Math,LabelOut,Lpos=-90]{24,13} \Vertex[x=2,
      y=0,Math,LabelOut,Lpos=-90]{14,23}
    \end{tikzpicture}
  
    $K(G_{4})$
  \end{minipage}
  \bigskip

\begin{minipage}{0.33\linewidth}
  \centering
  \begin{tikzpicture}[rotate=90,scale=0.75]
    \newcommand{\aset}[2]{$\{#1,#2\}$} \GraphInit[vstyle=Classic]
    \SetUpVertex[MinSize=1pt] \SetVertexNoLabel
    \grPetersen[RA=2.8,RB=1.3]
  \end{tikzpicture}

  $G_{5}$
\end{minipage}\quad
\begin{minipage}{0.33\linewidth}
  \centering
  \begin{tikzpicture}[scale=.75]
    \newcommand{\aset}[2]{$\{#1,#2\}$} \GraphInit[vstyle=Classic]
    \SetUpVertex[MinSize=1pt] \SetVertexNoLabel
    \grEmptyCycle[RA=1,rotation=-90]{5}
    \grEmptyCycle[RA=1.8,prefix=w,rotation=-90]{5}
    \grCycle[RA=2.8,prefix=z,rotation=90]{5} \EdgeInGraphMod{a}{5}{2}
    \EdgeMod{a}{w}{5}{1} \EdgeMod{a}{w}{5}{-1} \EdgeMod{w}{z}{5}{2}
    \EdgeMod{w}{z}{5}{-2}
  \end{tikzpicture}
 
  $K(G_{5})$
\end{minipage}
\bigskip
  
\begin{minipage}{0.33\linewidth}
  \centering
  \begin{tikzpicture}[scale=.5]
    \GraphInit[vstyle=Normal] \SetUpVertex[MinSize=1pt]
    \SetVertexNoLabel \grStar[RA=2,Math]{7}%
    \grEmptyCycle[RA=4,prefix=c]{24} \EdgeInGraphMod*{a}{6}{1}{0}{2}
    \EdgeInGraphMod*{c}{24}{1}{0}{2} \EdgeFromOneToSeq{a}{c}{1}{2}{5}
    \EdgeFromOneToSeq{a}{c}{2}{6}{9}
    \EdgeFromOneToSeq{a}{c}{3}{10}{13}
    \EdgeFromOneToSeq{a}{c}{4}{14}{17}
    \EdgeFromOneToSeq{a}{c}{5}{18}{21}
    \EdgeFromOneToSeq{a}{c}{0}{22}{23}
    \EdgeFromOneToSeq{a}{c}{0}{0}{1}
  \end{tikzpicture}
  
  $G_{6}$
\end{minipage}\quad
\begin{minipage}{.33\linewidth}
  \centering
  \begin{tikzpicture}[scale=.5]
    \GraphInit[vstyle=Normal] \SetUpVertex[MinSize=1pt]
    \SetVertexNoLabel \grCycle[RA=1.7,prefix=a,rotation=-90]{3}
    \grEmptyCycle[RA=4,prefix=b,rotation=-10]{12}
    \EdgeInGraphMod*{b}{12}{1}{0}{2} \EdgeFromOneToSeq{a}{b}{1}{0}{3}
    \EdgeFromOneToSeq{a}{b}{2}{4}{7} \EdgeFromOneToSeq{a}{b}{0}{8}{11}
  \end{tikzpicture}
  
  $K(G_{6})$
\end{minipage}
\end{center}
% El grupo simétrico $S_{n}$ actúa en el espacio de $p$-cadenas
% $C_{p}(M_{n})$ con coeficientes en el campo de los
% números complejos, así $C_{p}(M_{n})$ es un $S_{n}$-módulo el cual
% descomponemos en suma directa de submódulos irreducibles. Se define el $p$-ésimo
% operador frontera $\partial_{p}$, el cual es un homomorfismo de
% $C_{p}(M_{n})$ en $C_{p-1}(M_{n})$, del mismo modo como notamos que
% resulta útil descomponer un espacio vectorial en suma directa de
% subespacios invariantes, es útil descomponer un módulo en submódulos irreducibles,
% puesto que el comportamiento de $\partial_{p}$ puede ser inferido
% mediante su comportamiento en cada uno de los sumandos directos.

Aunque hay varias teorías de homología [\cite{munkres1984elements},
{\setlength{\fboxsep}{0pt}\colorbox{green}{...??}}], la homología que
usamos en este trabajo es la homología simplicial, pues... 

Los módulos de homología de $\Delta$ con coeficientes en el campo de los números
complejos denotadas por $H_{k-1}(\Delta)$ definen módulos de el grupo $G$.

La idea de este trabajo surge a partir del teorema de Bouc
[\cite{MR756517}, proposición 4, p.172], el cual describe
la descomposición del $S_{n}$-módulo $\widetilde H_{k-1}(M_{n})$ en
suma directa de submódulos irreducibles, en donde actúa el grupo
simétrico~$S_{n}$, para todo $n,k\geq1$.

En este trabajo reproducimos el resultado de Bouc para $n=4,5,6,$ por
medio de argumentos elementales de representaciones de grupos y
topología. Además se calcula la descomposición en módulos irreducibles
de $\widetilde H(K(M_{n}))$, es decir, de los módulos de homología reducida de
los complejos simpliciales obtenidos a partir de la gráfica de clanes
$K(G_{n})$ en donde también actúa el grupo simétrico $S_{n}$, con
$n=5,6$.

% La idea de la presente tesis surge a partir del Teorema de Bouc
% \cite{MR756517}: 
% \begin{equation*}
%   \widetilde H_{k-1}(M_{n})\cong_{S_{n}}\bigoplus_{\substack{\lambda:\lambda\vdash n\\
%       \lambda=\lambda^{'}\\d(\lambda)=n-2k}} S^{\lambda},
% \end{equation*}
% para todo $n,k\geq1$. Tal teorema describe la descomposición en
% submódulos irreducibles del $S_{n}$-módulo $\widetilde
% H_{k-1}(M_{n})$, donde $\widetilde H_{k-1}(M_{n})$ es la homología
% reducida de complejos simpliciales $M_{n}$ construidos a partir de ciertas
% gráficas, (llamadas \emph{gráficas de emparejamientos}) en donde actúa
% el grupo simétrico~$S_{n}$.

% Se define la gráfica de emparejamientos $M(G)$ de la gráfica simple
% $G$ como la gráfica cuyos vértices son las aristas de $G$ y dos
% vértices son adyacentes si las correspondientes aristas son
% ajenas. Denotemos por $K_{n}$ a la gráfica completa con $n$
% vértices. En la gráfica $G_{n}=M(K_{n})$, el grupo $S_{n}$ actúa de
% manera natural.

% Un complejo simplicial $\Delta$ está formado por un conjunto de
% vértices y una colección de subconjuntos de vértices, llamados
% \emph{simplejos}, tal que la colección es cerrada bajo tomar
% subconjuntos. A cada complejo simplicial se le puede asociar un
% espacio topológico $|\Delta|$, llamado su \emph{realización
%   geométrica}. Los complejos simpliciales son una herramienta
% fundamental para introducir conceptos topológicos dentro de la
% combinatoria. 

% Consideraremos complejos simpliciales obtenidos a partir de
% gráficas. Es decir, a cada gráfica $G$ simple finita se le asocia un
% espacio topológico por medio del complejo simplicial $\Delta(G)$, cuyos
% simplejos son las subgráficas completas. Por lo tanto, a la gráfica
% $G$ se le asocian conceptos topológicos por medio de
% $|\Delta(G)|$. Por ejemplo, si $G_{1}$ y $G_{2}$ son gráficas finitas,
% diremos que $G_{1}$ y $G_{2}$ son homeomorfas si $|\Delta(G_{1})|$
% y $|\Delta(G_{2})|$ lo son.

% Se define entonces el complejo simplicial $M_{n}$ como
% $M_{n}=\Delta(G_{n})$. Puesto que $S_{n}$ actúa en $M_{n}$, las
% homologías de $M_{n}$ con coeficientes en el campo de los
% números complejos definen representaciones de $S_{n}$.

% En la tesis reproducimos el resultado de Bouc para $n\leq 6$ por medio de
% argumentos elementales de representaciones de grupos y
% topología. Además se calcula la descomposición en módulos irreducibles de las homologías
% reducidas de los complejos simpliciales obtenidos a partir de la
% gráfica de clanes $K(G_{n})$ en donde también actúa el grupo simétrico
% $S_{n}$, con $n\leq 6$.

% Un clan es una subgráfica completa maximal. La gráfica de clanes
% $K(G)$ de una gráfica $G$ es la gráfica cuyos vértices son los clanes
% de $G$ y dos vértices son adyacentes si los correspondientes clanes se
% intersecan.

En el primer capítulo se estudian conceptos de álgebra lineal, grupos,
anillos, módulos, representaciones de grupos y gráficas, los cuales se
usarán en los capítulos posteriores. Siendo los conceptos más
importantes el de representación de un grupo $G$ en un espacio
vectorial $V$, $G$-módulos, $FG$-módulos. Destacando el teorema de
Maschke. Se definen las gráficas de emparejamientos, clanes y
bipartita clánica.

En el segundo capítulo, estudiamos la relación entre tableros de Young
y representaciones del grupo simétrico. Describimos la construcción de
módulos de Specht los cuales dan una lista completa de las
representaciones irreducibles de $S_{n}$.

Los espacios de homología de complejos simpliciales son el tema del capítulo \ref{cha:hom-com-sim}.
En la sección \ref{com-sim-abs} se desarrollan conceptos como:
complejo simplicial abstracto $\Delta$, espacio de cadenas $C_{p}(\Delta)$, el
operador frontera $\partial_{p}$. En la sección \ref{hom-simp}, se
define espacio de homología de un complejo simplicial. Se
presentan los complejos de cadenas abstractos en la sección \ref{com-cad} y por
último, en la sección \ref{hom-ind}, se definen espacios topológicos
homotópicos, se menciona que los espacios de homología de espacios homotópicos son
isomorfos y a las gráficas finitas se les asocia complejos simpliciales.

Finalmente, en el capítulo \ref{cha:hom-com-emp} se utilizará la teoría presentada en los
capítulos anteriores. Se define el complejo de emparejamientos y se
enuncia el teorema de Bouc.  Además, calculamos la descomposición de
los módulos de homología reducida de complejos de emparejamientos $M_{n}$ en
submódulos irreducibles, para $n=4,5,6$. Este trabajo amplía los resultados de \cite{robles08:_repres}, pues de
forma análoga se calculan los módulos de homología reducida de los complejos
simpliciales obtenidos de las gráficas de clanes $K(G_{n})$, para
$n=5,6$.

\tableofcontents

\newpage \thispagestyle{empty}

\chapter{Preliminares}
\label{preliminares}

El objetivo de este capítulo es estudiar conceptos de álgebra
lineal, álgebra abstracta (grupos, anillos y módulos), representaciones
de grupos y gráficas, los cuales usaremos en los próximos capítulos. 

En la sección \ref{esp-vec} se revisan algunas definiciones y teoremas
de espacios vectoriales y transformaciones lineales que usaremos en el
desarrollo de este trabajo. Entre las definiciones que utilizamos
frecuentemente son la suma directa de espacios vectoriales y espacio
cociente. Se construye el espacio vectorial $FX$. Las pruebas que se
omiten se puede consultar en \cite{friedberg1982algebra} y
\cite{herstein1990algebra}. 

De la sección \ref{grupos}, los conceptos importantes son: grupo, homomorfismo e isomorfismo de grupos, grupo
simétrico, acción de un grupo~$G$ en un conjunto no vacío y acción lineal
de un grupo $G$ en un espacio vectorial $V$. Las demostraciones de
grupos que no incluiremos explícitamente se pueden encontrar en
\cite{fraleigh}. En la sección \ref{anillos} y \ref{modulos} se
definen anillos y módulos respectivamente.

Se desarrollan conceptos de representaciones de grupos de la sección
\ref{rep-grup} a la sección \ref{carac-induc}. Más específicamente, en
la sección \ref{rep-grup} observamos que si $F$ es un campo y $G$ un
grupo, se define la multiplicación en el espacio vectorial~$FG$
construido en la sección \ref{esp-vec} y se obtiene un anillo. Estudiar
$FG$-módulos, es equivalente a estudiar $G$-módulos, que a su vez, es
equivalente a estudiar representaciones de $G$ en $V$, donde $V$ es un
espacio vectorial sobre $F$. Se discute en la sección
\ref{reducibilidad} el teorema de Maschke, el cual
enuncia que cualquier $G$-módulo $V$ se puede descomponer como suma
directa de $G$-submódulos irreducibles. Consideramos el lema de Schur, pues su
demostración es similar a cierta proposición posterior que encontramos
al final de la sección \ref{schur}. Por otra parte, los caracteres de
una representación nos proporcionan información acerca de esta, lo
cual se describe con detalle en la sección \ref{caracteres}. Además,
por medio del producto interno de caracteres podemos determinar si una
representación es irreducible, cuándo dos representaciones son
equivalentes, la multiplicidad de cierto submódulo irreducible en la
descomposición de un $G$-módulo, entre otros teoremas que se muestran en la
sección~\ref{producto-interno}. En la sección \ref{carac-induc} de
caracteres y módulos inducidos, se exponen la reciprocidad de
Frobenius y la fórmula para obtener el carácter de una representación
inducida.

El propósito de la sección \ref{graficas} es definir gráficas,
gráfica de emparejamientos, clanes y bipartita clánica para lo cual
necesitamos algunas definiciones previas. Estas definiciones se
encuentran en \cite{harary} y \cite{LPV08a}.

% En éste trabajo todos los espacios vectoriales considerados serán de
% dimensión finita y sobre el campo de los números complejos.

\section{Álgebra lineal}
\label{esp-vec}

% \begin{theorem}{[\cite{friedberg1982algebra}, teorema 1.9, p.42]}.
%   \label{clunica}
%   Sea $V$ un espacio vectorial y $\beta=\{v_{1},\dots,v_{n}\}$ un
%   subconjunto de $V$. Entonces $\beta$ es una base de $V$, si y solo si
%   cada vector $v\in V$ puede ser expresado de manera única como una
%   combinación lineal de vectores de~$\beta$, es decir, puede ser
%   expresado de la forma
%   $$v=\lambda_{1}v_{1}+\cdots+\lambda_{n}v_{n}$$
%   para escalares únicos $\lambda_{1},\ldots,\lambda_{n}.$
% \end{theorem}

% \begin{theorem}{[\cite{friedberg1982algebra}, teorema 1.12, p.50]}.
%   \label{esp-iguales}
%   Sea $W$ un subespacio de un espacio vectorial $V$ de dimensión
%   $n$. Entonces, $W$ es de dimensión finita y $\dim(W)\leq
%   n$. Además, si~$\dim(W)=n$, entonces $W=V$.
% \end{theorem}
\begin{definition}
  Si $W_{1}$ y $W_{2}$ son dos subconjuntos no vacíos de un espacio
  vectorial $V$, entonces la \textbf{suma} de $W_{1}$ y $W_{2}$, que se
  expresa como $W_{1}+W_{2}$, es el conjunto $\{x+y:x\in W_{1}$ y $y\in
  W_{_2}\}$. La suma de cualquier número finito de subconjuntos no
  vacíos de $V$: $W_{1},\ldots,W_{n}$, se define análogamente como el
  conjunto
  $$W_{1}+\cdots+W_{n}=\{x_{1}+\cdots+x_{n}: x_{i}\in W_{i} \mbox{ para }i=1,2,\ldots,n\}$$
\end{definition}

Se tiene en general que si $W_{1}$ y $W_{2}$ son subespacios de $V$,
entonces $W_{1}+W_{2}$ es subespacio de $V$.

\begin{definition}
  \label{suma-directa}
  Un espacio vectorial $V$ es la \textbf{suma directa} de $W_{1}$ y
  $W_{2}$, denotada como $V=W_{1}\oplus W_{2}$, si $W_{1}$ y $W_{2}$
  son subespacios de $V$ tales que $W_{1}\cap W_{2}=\{0\}$ y
  $W_{1}+W_{2}=V.$ 
\end{definition}

\begin{theorem}{[\cite{friedberg1982algebra}, corolario 1.13, p.53]}.
  Sean $W_{1}$ y $W_{2}$ subespacios de $V$ tales que
  $V=W_{1}+W_{2}$. Luego, $V=W_{1}\oplus W_{2}$ si y solo si 
  $$\dim(V)=\dim(W_{1})+\dim(W_{2}).$$
\end{theorem}

\begin{definition}
  Sea $W$ un subespacio de un espacio vectorial $V$. Para toda $v\in V$ el conjunto $\{v\}+W=\{v+w:w\in W\}$ se
  llama \textbf{clase lateral de $W$ que contiene a $v$}. Es frecuente
  expresar a esta clase lateral como~$v+W$ en vez de $\{v\}+W$. 
\end{definition}

Se puede demostrar lo siguiente:
\begin{itemize}
\item $v_{1}+W=v_{2}+W$ si y solo si $v_{1}-v_{2}\in W.$
\end{itemize}
La suma y el producto puede definirse en el conjunto $S=\{v+W:v\in
V\}$ de todas las clases laterales de $W$ como: 
\begin{itemize}
\item $(v_{1}+W)+(v_{2}+W)=(v_{1}+v_{2})+W$ para todos $v_{1}$, $v_{2}\in V$.
\item $a(v+W)=av+W$ para todo $v\in V$ y $a\in \mathbb{C}$.
\end{itemize}
Se puede demostrar que las operaciones anteriores están bien definidas.

\begin{definition}
  El conjunto $S$ es un espacio vectorial bajo las operaciones
  definidas anteriormente y se llama \textbf{espacio cociente de $V$ módulo $W$} y se denota mediante $V/W$. 
\end{definition}

\begin{theorem}{[\cite{herstein1990algebra}, lema 4.8, p.168]}.
  \label{dim-esp-coc}
  Sea $(V/W)$ el espacio cociente de $V$ módulo $W$, se tiene:
  $$\dim(V/W)=\dim(V)-\dim(W)$$
\end{theorem}

% \begin{theorem}
%   \label{dim-esp-vec}
%   Sean $W_{1}$ y $W_{2}$ subespacios de un espacio vectorial
%   $V$. Entonces, 
%   $$\dim(W_{1}+W_{2})=\dim(W_{1})+\dim(W_{2})-\dim(W_{1}\cap W_{2}).$$ 
% \end{theorem}

% \section{Transformaciones lineales}
% \label{trasn-lin}
\begin{definition}
  \label{t-l}
  Sean $V$ y $W$ espacios vectoriales sobre el campo $F$. Una función
  $T:V\rightarrow W$ se llama \textbf{transformación lineal} de $V$ en
  $W$ si para todo $x,y\in V$ y $c\in F$ tenemos que
  \begin{enumerate}
  \item $T(x+y)=T(x)+T(y)$.
  \item $T(cx)=cT(x)$.
  \end{enumerate}
\end{definition}

\begin{definition}
  \label{ker-im}
  Sean $V$ y $W$ espacios vectoriales y sea $T:V\rightarrow W$ una
  transformación lineal. El \textbf{kernel o espacio nulo de $T$},
  es denotado como $\ker(T)$ y está dado por el conjunto:
  \begin{equation*}
    \ker(T)=\{v\in V\mid T(v)=0\}.
  \end{equation*}
  Definimos a la \textbf{imagen o rango de $T$}, denotado como
  $\im(T)$, dado por el conjunto:
  \begin{equation*}
    \im (T)=\{T(v)\mid v\in V\}.
  \end{equation*}
\end{definition}

\begin{theorem}{[\cite{friedberg1982algebra}, corolario 2.3, p.69]}.
  \label{imT}
  Sean $V$ y $W$ espacios vectoriales y $T:V \rightarrow W$
  una transformación lineal. Sea $\beta=\{v_{1},\ldots,v_{n}\}$ una base de $V$,
  entonces $\im(T)=\langle T(v_{1}),\ldots,T(v_{n})\rangle.$  
\end{theorem}

\begin{theorem}{[\cite{friedberg1982algebra}, teorema 2.22, p.98]}.
  \label{esp-isomorfos}
  Sean $V$ y $W$ espacios vectoriales de dimensión finita sobre el
  mismo campo $F$. Entonces $V$ es isomorfo a $W$
  si y solo si $\dim (V)=\dim(W).$
\end{theorem}

\begin{theorem}[Teorema de isomorfismo]
  \label{teorema-isomorfismo-esp}
  Si $f:V\rightarrow W$ es una transformación lineal, el espacio
  cociente $V/\ker(f)$ es isomorfo a $\im(f)$. Un isomorfismo entre estos dos
  espacios es el siguiente:
  $$\phi:V/\ker(f)\rightarrow \im(f)$$
  definido por $\phi(v+\ker(f))=f(v).$
\end{theorem}
\begin{proof}[Demostración.]
  Hay que demostrar que $\phi$ está bien definida, es decir, si
  tenemos que
  $v_1+\ker(f)=v_2+\ker(f)$, entonces
  $f(v_1)=f(v_{2}).$ Sabemos que 
  $v_1+\ker(f)=v_2+\ker(f)$ si y solo si
  $v_{1}-v_{2}\in \ker(f)$, es decir,~$f(v_{1}-v_{2})=f(v_1)-f(v_2)=0$,
  como se quería. Queda por demostrar de forma directa que $\phi$ es
  una transformación lineal; y para demostrar que es isomorfismo falta
  demostrar que es inyectiva y sobreyectiva, lo cual también se hace
  de forma directa.
\end{proof}

\begin{definition}
  \label{vec-coor}
  Sea $\beta=\{v_{1},\ldots,v_{n}\}$ una base para un espacio
  vectorial~$V$. Dado $v\in V$ definimos al \textbf{vector coordenado de $v$
  relativo a $\beta$}, denotado por $[v]_{\beta}$, mediante
  \begin{center}
    $[v]_{\beta}=\begin{pmatrix}
      \lambda_{1}  \\
      \vdots  \\
      \lambda_{n}  
    \end{pmatrix}$
  \end{center}
donde
\begin{equation*}
  v=\sum^{n}_{i=1}\lambda_{i}v_{i}.
\end{equation*}
\end{definition}

% \begin{definition}
%   \label{matr-rep-T}
%   Sean $\beta=\{v_{1},\ldots,v_{n}\}$ y
%   $\gamma=\{w_{1},\ldots,w_{m}\}$ bases de los espacios
%   vectoriales $V$ y $W$ respectivamente. Sea $T:V\rightarrow W$ una
%   transformación lineal. La \textbf{matriz que representa a $T$}
%   en las bases $\beta$ y $\gamma$, denotada como
%   $[T]^{\gamma}_{\beta}$, se define como la matriz $m\times n$:
%   \begin{equation*}
%     [T]^{\gamma}_{\beta}=([T(v_{1})]_{\gamma},[T(v_{2})]_{\gamma},\cdots,[T(v_{n})]_{\gamma})
%   \end{equation*}
%   En otras palabras, si
%   $T(v_{j})=\lambda_{1j}w_{1}+\lambda_{2j}w_{2}+\cdots+\lambda_{mj}w_{m}$,
%   entonces
% \begin{center}
%     $[T(v_{j})]_{\gamma}=\begin{pmatrix}
%       \lambda_{1j}\\
%       \lambda_{2j}\\
%       \vdots\\
%       \lambda_{mj}
%     \end{pmatrix}$,\quad y \quad $[T]^{\gamma}_{\beta}=\begin{pmatrix}
%       \lambda_{11} & \lambda_{12} & \cdots & \lambda_{1n} \\
%       \lambda_{21} & \lambda_{22} & \cdots & \lambda_{2n} \\
%       \vdots & \vdots & \ddots & \vdots \\ 
%       \lambda_{m1} & \lambda_{m2} & \cdots & \lambda_{mn} \\
%     \end{pmatrix}$.
%   \end{center}
% Si $V=W$ y $\beta=\gamma$, escribiremos $[T]_{\beta}$.
% \end{definition}
\begin{definition}
  \label{matr-rep-T}
  Denote $F$ un campo. Sean $V$ y $W$ espacios vectoriales de dimensión finita con bases $\beta=\{v_{1},\ldots,v_{n}\}$ y
  $\gamma=\{w_{1},\ldots,w_{m}\}$, respectivamente. Sea $T:V\rightarrow~W$ una
  transformación lineal. Entonces existen escalares únicos $a_{ij}\in F$
  con $i=1,\ldots,m$ y $j=1,\ldots,n$, tales que
  \begin{equation*}
    T(v_{j})=\sum_{i=1}^{m}a_{ij}w_{i} \mbox{ para } 1\leq j\leq n. 
  \end{equation*}
  Utilizando la notación anterior, llamaremos a la matriz $A$ de
  $m\times n$, definida mediante $A_{ij}=a_{ij}$, la \textbf{matriz que representa a $T$
  en las bases $\beta$ y $\gamma$} y la escribiremos de acuerdo a la
  notación de \cite{friedberg1982algebra} como
  $A=[T]^{\gamma}_{\beta}$. Si $V=W$ y $\beta=\gamma$, escribiremos
  $A=[T]_{\beta}$.
\end{definition}

A continuación, construiremos un espacio vectorial muy importante, ya
que lo usaremos más adelante en la sección \ref{rep-grup}
de representaciones de grupos.

Sea $F$ un campo y $X$ un conjunto no vacío. Se define $FX$ como:
\begin{equation*}
  FX=\{f:X\rightarrow F\mid \{x\in X\mid f(x)\neq 0\} \mbox{ es finito}\}.
\end{equation*}

\begin{lemma}
  \label{FX}
  $FX$ es un espacio vectorial con las siguientes operaciones:
  \begin{itemize}
  \item $(f+g)(x)=f(x)+g(x)$, donde $f,g\in FX$,
  \item $(rf)(x)=rf(x)$, con $r\in F$.
  \end{itemize}
\end{lemma}

\begin{proof}[Demostración]
  Para checar que la suma está bien definida, verifiquemos que si
  $f,g\in FX$, entonces $(f+g) \in FX$. Sean $f,g\in FX$ y $A,B$
  subconjuntos finitos de $X$, tales que $f(x)=0$ para $x\not\in A$ y
  $g(y)=0$ para $y\not\in B$. Notemos que $A\cup B$ es un
  subconjunto finito de $X$ y $(f+g)(z)=0$ para $z\not\in A\cup
  B$, por tanto, $(f+g) \in FX$. De forma similar, comprobemos que el
  producto por escalar está bien definido. Si $r\in F$, $f\in FX$ y
  $C$ es un subconjunto finito de $X$, tal que $f(x)=0$ para $x\not\in
  C$, entonces $(rf)(x)=0$ para $x\not\in C$, por tanto~$(rf)\in
  FX$. Queda por demostrar que se cumplan las otras condiciones de
  espacio vectorial.
\end{proof}

\begin{definition}
  \label{fun-caracteristica}
  Definimos a la \textbf{función característica} $\varphi:X\rightarrow
  FX$, tal que
  \[
  \varphi(x)(y)=
  \begin{cases}
    1, & \text{si } x=y \\
    0, & \text{si } x\neq y \\
  \end{cases}
  \]
\end{definition}
Notemos que $\varphi$ es inyectiva, puesto que, si $x,y\in X$ y $x\neq y$,
entonces $\varphi(x)(y)\neq\varphi(y)(y)$, pues
$\varphi(x)(y)=0$ y $\varphi(y)(y)=1$, así
$\varphi(x)\neq\varphi(y)$. $\varphi(X)$ es un subconjunto de $FX$ que
está en biyección con $X$.
\begin{theorem}
  \label{base-FX}
  $\varphi(X)$ es base de el espacio vectorial $FX$.
\end{theorem}
\begin{proof}[Demostración]
  Sea $D=\{x_{1},x_{2},\dots,x_{n}\}$ un subconjunto finito de $X$. Si
  $\sum_{x_{i}\in D} \lambda_{x_{i}}\varphi(x_{i})=0$, con
  $\lambda_{x_{i}}\in F$. Sea $y\in D$, entonces
  $\sum_{x_{i}\in D} \lambda_{x_{i}}\varphi(x_{i})(y)=~\lambda_{x_{i}}$,
  si $y=x_{i}$, entonces $\lambda_{x_{i}}=0$. De igual manera, para todo $y\in D$, se
  tiene $\lambda_{x_{i}}=0$. Por tanto, $\{\varphi(x_{i})\mid x_{i}\in D\}$ es linealmente independiente.

  Por otra parte, sea $E$ un subconjunto finito de $X$. Sea $f\in FX$, tal que~$f(x)\neq 0$ para $x\in E$, tenemos:
  $$\sum_{x\in X}f(x)\varphi(x)=\sum_{x\in
    E}f(x)\varphi(x)+\sum_{x\not\in E}f(x)\varphi(x)=\sum_{x\in E}f(x)\varphi(x)$$
  Ahora, tomemos $y\in E$, entonces $\sum_{x\in
    E}f(x)\varphi(x)(y)=f(x)$ si $y=x$. Así, para todo $y\in E$,
  tenemos $\sum_{x\in E}f(x)\varphi(x)(y)=f(x)$ cuando $y=x$. Por tanto, se puede escribir a $f$ como $f=\sum_{x\in X}f(x)\varphi(x)$. 
\end{proof}

Por abuso de notación y ya que $\varphi$ es inyectiva se
``identifica'' cada elemento $x\in X$ con su función característica
$\varphi(x)$. Con esta notación, cada elemento $f$ de el espacio vectorial
$FX$ se escribe como combinación lineal de los elementos de $X$:
\begin{equation}
  \label{sumas-formales}
  f=\sum_{x\in X} f(x)x.
\end{equation}

\section{Grupos}
\label{grupos}

\begin{definition}
  Una \textbf{operación binaria} en un conjunto $G$ es una función
  de la forma $*:G  \times G \rightarrow G$. Para cada $(a,b)\in G
  \times G$, denotaremos al elemento~$*((a,b))\in G$ por $ab$. 
\end{definition} 

\begin{definition} 
  Un \textbf{grupo} es un conjunto $G$, junto con una
  \textbf{operación binaria} $*$ que satisface las siguientes condiciones:
    \begin{enumerate}
    \item La operación es asociativa, es decir, $$a(bc)=(ab)c$$ para todo $a,b,c \in G$.
    \item Existe un elemento neutro $1 \in G$ que
      satisface: $$a1=1a=a$$ para todo $a \in G$.
    \item Para cada elemento $a \in G$ existe otro elemento $a^{-1} \in G$
      tal que $$aa^{-1}=a^{-1}a=1$$ Al elemento $a^{-1}$ se le llama inverso del elemento $a$.
    \end{enumerate}

    Decimos que un grupo $G$ es \textbf{abeliano} si para cualquier
    $a,b\in G$ se tiene~$ab=ba$.
\end{definition}
El número de elementos de un grupo $G$, es el \textbf{orden} de $G$, y
lo denotamos por $|G|$. Cuando $|G|$ es finito, decimos que $G$ es un
\textbf{grupo finito}.

Enseguida se muestran ejemplos de grupos que usaremos posteriormente.
\begin{example}
  \label{GL(nF)}
  El conjunto de matrices invertibles $n \times n$ con
  entradas en~$F$, denotado como $GL(n,F)$, junto con el producto de matrices es un grupo.  
\end{example}

\begin{example}
  \label{GL(V)}
  El conjunto de operadores lineales invertibles en un espacio vectorial $V$ de dimensión finita
  $n$ sobre el campo $F$, denotado por $GL(V)$, junto con la
  operación composición de operadores es un grupo.
\end{example}

\begin{example}
  \label{grupo-ciclico}
  Sea $n\in \mathbb{Z}$, {\setlength{\fboxsep}{0pt}\colorbox{green}{Naturales?}}
  construimos un grupo denotado por $C_{n}$ como sigue: consistirá
  en todos los símbolos $a^{i}$, con $i=0,1,2,\dots,n-1$ en donde $a^{0}=a^{n}$, $a^{i}a^{j}=a^{i+j}$ si $i+j\leq n$ y
  $a^{i}a^{j}=a^{i+j-n}$ si $i+j>n$. A $C_{n}$ se le conoce como \textbf{grupo
    cíclico}.
\end{example}

\begin{definition}
  Si $G$ y $H$ son grupos, entonces un \textbf{homomorfismo} de~$G$
  en $H$ es una función $\rho:G\rightarrow H$ la cual
  satisface $$\rho(ab)=\rho(a)\rho(b),$$
  para todo $a,b \in G$. Si $\rho$ es biyectivo diremos que $\rho$ es
  un \textbf{isomorfismo}.
\end{definition}

\begin{theorem}{\normalfont (\cite{fraleigh}, teorema 13.2, p.128)}.
  Sea $\rho:G\rightarrow G^{'}$ un homomorfismo de grupos. Entonces
  \begin{enumerate}
    \item $\rho(1)=1^{'}$, donde $1\in G$ y  $1^{'}\in G^{'}$ son los
    neutros respectivos. 
    \item Si $g\in G$, entonces $\rho(g^{-1})=\rho (g)^{-1}$.
  \end{enumerate}
\end{theorem}
%La demostración aparece en el teorema 13.12 de \cite{fraleigh}.
% \begin{definition}
%   Cuando un homomorfismo de grupos $\phi:G\rightarrow H$ es biyectivo,
%   diremos que $\phi$ es un  \textbf{isomorfismo}. También diremos que
%   $G$ y $H$ son grupos  \textbf{isomorfos} cuando exista un
%     isomorfismo entre ellos, usaremos la notación $G\cong H$.
% \end{definition}

% \begin{theorem}
%   Supongamos que $G$ y $H$ son grupos y sea $\phi:G\rightarrow H$ un
%   homomorfismo. Entonces $$G/\ker \phi\cong \im \phi$$ 
% \end{theorem}.

\begin{definition}
  Una \textbf{permutación de un conjunto} $Y$ es una función biyectiva
  $\pi:Y\rightarrow Y$.
\end{definition}

\begin{theorem}{\normalfont (\cite{fraleigh}, teorema 8.5, p.77)}.
  Sea $Y$ un conjunto no vacío y sea $S_{Y}$ el conjunto de todas las
  permutaciones de $Y$. Entonces $S_{Y}$ es un grupo con la operación dada por la composición de
  permutaciones.
\end{theorem}

\begin{definition}
  Sea $Y$ el conjunto finito $\{1,2,\ldots,n\}$. El grupo de todas las
  permutaciones de $Y$ es el\textbf{ grupo simétrico} y es denotado
  por $S_{n}$. Dado $\pi \in S_{n}$, usaremos la siguiente notación
  para designar a $\pi$
  \[ \pi=\left(
    \begin{array}{ccccc}
      1 & 2 & 3 & \cdots & n\\
      \pi(1) & \pi(2) & \pi(3) & \cdots & \pi(n) 
    \end{array} 
  \right)\] 
  donde debajo de cada $x\in Y$ se encuentra su valor o imagen
  $\pi(x)\in Y$.
\end{definition}

\begin{definition}
  \label{trans}
  Sea $Y$ el conjunto finito $\{1,2,\ldots,n\}$. Si $\pi\in S_{n}$ es una permutación y
  $x\in Y$ es un elemento, diremos que $\pi$ fija a $x$ si
  $\pi(x)=x$. En caso contrario, diremos que $\pi$ mueve a $x$. Ahora,
  si $i_{1},i_{2},\ldots, i_{k}$ son enteros distintos que pertenecen a
  $Y$ y si $\pi\in S_{n}$ es tal que
  $$\pi(i_{1})=i_{2},\pi(i_{2})=i_{3},\ldots,\pi(i_{k-1})=i_{k},\pi(i_{k})=i_{1}$$
  y $\pi$ fija a los otros elementos de $Y$, si los hay, entonces,
  diremos que $\pi$ es un \textbf{ciclo de longitud $k$} o un \textbf{$k$-ciclo}. Un ciclo de longitud 2 es una \textbf{transposición}.
\end{definition}

\begin{example}
  Sea $\pi\in S_{5}$, 
  \[\pi=\left(
    \begin{array}{ccccc}
      1 & 2 & 3 & 4 & 5\\
      2 & 4 & 3 & 1 & 5  
    \end{array} 
  \right)=(124)(3)(5)=(124),\]
  entonces $\pi$ es un 3-ciclo.
\end{example}

\begin{theorem}{\normalfont (\cite{fraleigh}, corolario 9.12, p.90)}.
  \label{per-prod-trans}
  Toda permutación $\pi\in S_{n}$ es producto de transposiciones.
\end{theorem}

\begin{theorem}{\normalfont (\cite{fraleigh}, teorema 9.15, p.91)}.
  \label{trans-par-impar}
  Sea $\pi\in S_{n}$. Si expresamos a~$\pi$ como $\pi=\tau_{1}\tau_{2}\cdots\tau_{k}=\psi_{1}\psi_{2}\cdots\psi_{s}$,
  donde $\tau_{i}$ y $\psi_{j}$ son transposiciones para
  $i=1,\ldots,k$, $j=1,\ldots,s$, entonces $k$ y $s$
  son ambos pares o bien ambos impares.
\end{theorem}

\begin{definition}
  \label{accion-grupo}
  Sea $G$ un grupo y $X$ un conjunto no vacío. Una  \textbf{acción de $G$
  en $X$} es una función $*:G \times X \rightarrow X$ tal que
\begin{enumerate}
\item $1*x=x$ para todo $x\in X.$
\item $(g_{1}g_{2})*x=g_{1}*(g_{2}*x)$ para todo $x\in X$ y $g_{1},g_{2}\in G.$
\end{enumerate}
   %Bajo estas condiciones, también decimos que $X$ es un \textbf{$G$-conjunto}.
   En ocasiones, por abuso de notación, en lugar de escribir $g*x$ usaremos la notación $gx$.
\end{definition}

\begin{definition}
  \label{accion-lineal}
  Sea $G$ un grupo y $V$ un
  espacio vectorial de dimensión
  finita sobre el campo $F$. Una
  \textbf{acción lineal de $G$ en
    $V$} es una función
 $$*:G\times V \rightarrow V $$
que satisface los axiomas:
\begin{enumerate}
\item $1v=v$, para todo $v\in V$ (donde $1$ es el neutro de $G$).
\item $(g_{1}g_{2})v=g_{1}(g_{2}v)$ para todos $v\in V$ y
  $g_{1},g_{2}\in G$.
\item $g_{1}(v+w)=g_{1}v+g_{1}w$, para $g_{1}\in G$ y $v,w \in V .$
\item $g_{1}(\lambda v)=\lambda(g_{1}v)$, para $\lambda \in F$,
  $v\in V$ y $g_{1}\in G.$
\end{enumerate}
Diremos que $V$ es un \textbf{$G$-módulo.}

{\setlength{\fboxsep}{0pt}\colorbox{green}{esta definición aquí o en
    sección de módulos???}}
\end{definition} 

\section{Anillos}
\label{anillos}

\begin{definition}
  Un conjunto no vacío $R$ se dice que es un \textbf{anillo} si en $R$
  están definidas dos operaciones, denotadas por $+$ y $\cdot$ tales
  que para cualquier $a,b,c \in R$:
  \begin{enumerate}
  \item $(R,+)$ es un grupo abeliano.
  \item $a\cdot (b\cdot c)=(a\cdot b)\cdot c$.
  \item $a\cdot (b+c)=a\cdot b+a\cdot c$ y $(b+c)\cdot a=b\cdot a+c\cdot a$
  \end{enumerate}

  Si existe un elemento $1\in R$ tal que $a\cdot 1=1\cdot a=a$ para todo
  $a\in R$, diremos que $R$ es un \textbf{anillo con elemento
    unitario}.
  
  Si la multiplicación de $R$ es tal que $a\cdot b=b\cdot a$ para todo
  $a,b\in R$ entonces llamamos a $R$ \textbf{anillo conmutativo}.

  Un anillo se dice que es un \textbf{anillo con división} si sus
  elementos distintos de cero forman un grupo bajo la multiplicación.

  Usaremos la notación $ab$ en lugar de $a\cdot b$, para $a,b\in R$.
\end{definition}
A continuación se muestran ejemplos de anillos.
\begin{example}
  Un campo es un anillo conmutativo con división.
\end{example}

\begin{example}
  El conjunto de los números enteros $\mathbb{C}$, los números reales~$\mathbb{R}$ y los números racionales $\mathbb{Q}$ son todos campos
  y por tanto también son anillos.
\end{example}

\begin{example}
  El conjunto de los enteros $\mathbb{Z}$ junto con la suma y la
  multiplicación usual forman un anillo conmutativo con elemento unitario.
\end{example}

\begin{example}
  El conjunto de matrices $M_{n}(\mathbb{Z})$ de $n\times n$ con
  entradas enteras es un anillo no conmutativo con elemento unitario.
\end{example}

\begin{example}
  Sea $A$ un grupo abeliano, denotemos a los endomorfismos de $A$ como
  el siguiente conjunto: 
  \begin{equation*}
    \End(A)=\{f:A\rightarrow A\mid f\mbox{ es un homomorfismo de grupos}\}.
    \label{endomorfismos}
  \end{equation*}
  Sea $f,g\in\End(A)$, definimos el producto $fg$ por $(fg)(x)=f(g(x))$ y
  la suma~$f+g$ por $(f+g)(x)=f(x)+g(x)$ . Entonces $\End(A)$ es un anillo.
\end{example}

En un anillo $R$, para $a\in R$, $n\in\mathbb{N}$, denotamos $na=a+\cdots+a$, donde hay $n$ sumandos.
\begin{definition}
  La \textbf{característica} de un anillo $R$ es el menor entero positivo $n$
  tal que $nx=0$ para toda $x\in R$. Si no existe tal entero, decimos
  que $R$ tiene característica cero.
\end{definition}

\section{Módulos}
\label{modulos}

\begin{definition}
  Sea $R$ un anillo cualquiera; un conjunto no vacío $M$ se dice que
  es un \textbf{$R$-módulo} o (módulo sobre $R$) si $M$ es un grupo abeliano
  bajo una operación $+$, tal que para cada $r\in R$ y $m\in M$ existe
  un elemento $rm\in M$ de tal modo que:
  \begin{enumerate}
  \item $r(a+b)=ra+rb$
  \item $r(sa)=(rs)a$
  \item $(r+s)a=ra+sa$
  \end{enumerate}
  para cualquier $a,b \in M$ y $r,s\in R$.

  Si $R$ tiene un elemento unitario, 1, y si $1m=m$ para todo elemento
  $m\in M$, entonces a $M$ lo llamaremos un \textbf{$R$-módulo
  unitario}.
\end{definition}

\begin{example}
  Todo grupo abeliano $G$ es un módulo sobre el anillo de los enteros
  $\mathbb{Z}$. Con la operación de $G$ como la suma y la
  multiplicación se define como: si $n>0$, $na=a+a+\cdots+a$, donde
  hay $n$ sumandos, si  $n<0$, $(-n)a=-(na)$ y $0a=0$, para todo $a\in G$ y $n\in \mathbb{Z}$.
\end{example}

\begin{example}
  Un grupo abeliano $A$ es un módulo sobre $\End(A)$, donde la suma
  $f+g$ está dada por $(f+g)(x)=f(x)+g(x)$ y la multiplicación $fa$
  por $f(a)$, para $f,g\in\End(A)$ y $a\in A$. 
\end{example}

\section{Representaciones de grupos}
\label{rep-grup}

La idea básica de teoría de representaciones es estudiar grupos considerándolos como
transformaciones lineales. Esto se hace como sigue: dado un grupo $G$
y un espacio vectorial $V$ de dimensión finita sobre el campo $F$,
se asocia a cada $g\in G$ una transformación lineal $\rho(g)\in GL(V)$ de manera que hace a la
función $\rho:G\rightarrow GL(V)$ un homomorfismo de grupos. $\rho$ es
llamada la \textbf{representación de
  $G$ en $V$}. 
%A dicho espacio $V$ se le llamará \emph{espacio de representación} y a
%su dimensión el \emph{grado} de la representación. 

Dada cierta representación $\rho$, se sigue que $\rho(G)$ es un
subgrupo de $GL(V)$, al cual se le puede aplicar las ideas de álgebra
lineal. Si $\rho$ es inyectiva, entonces $G\cong \rho(G)$.

Usualmente se estudia teoría de representaciones mediante la
representaciones en sí mismas o a través de sus correspondientes
$G$-módulos, donde $G$ es un grupo. Esto se explica mejor por analogía con otro tipo de representación.

Sea $S_{A}$ denote el grupo de permutaciones de un conjunto $A$. El
teorema de Cayley nos dice que cualquier grupo $G$ es isomorfo a un
subgrupo de $S_{A}$ para un conjunto $A$ suficientemente
grande. Formalmente, esto significa que se puede construir un
homomorfismo inyectivo $f:G\rightarrow S_{A}$, asi que $G\cong
f(G)$. Cada permutación $f(g)\in S_{A}$ ``representa'' al elemento de el grupo $g\in
G$. 
% Además, si tenemos algún otro homomorfismo $\pi:G\rightarrow
% S_{A}$, no necesariamente inyectivo, la imagen de el grupo $G$ bajo
% $\pi$ es un grupo de permutaciones isomorfo a algún grupo cociente de
% $G$. Tales funciones son  llamadas \textbf{representaciones
%   permutación} de $G$.

Si tenemos otro homomorfismo $\phi:G\rightarrow S_{A}$, no
necesariamente inyectivo. El grupo $G$ permuta el conjunto $A$ como sigue: dado
$\phi:G\rightarrow S_{A}$ se define $*:G\times A\rightarrow A$ como
$g*a=\phi(g)a$. Por propiedades de homomorfismo esta función tiene las
siguientes propiedades:
\begin{itemize}
\item $1*a=a$
\item $(gh)*a=g*(h*a)$
\end{itemize}

Estos son exactamente los axiomas de
acción de grupo (ver definición \ref{accion-grupo}). En adelante,
usaremos la notación $ga$ en lugar de $g*a$. De igual modo se puede
definir un homomorfismo $\phi:G\rightarrow S_{A}$ dada una acción de grupo. Esto establece una
correspondencia biyectiva entre el homomorfismo $\phi:G\rightarrow
S_{A}$ y acciones de grupo.

Análogamente, si dada una representación $\rho:G\rightarrow GL(V)$ e igualmente se define
$gv=\rho(g)v$ conseguimos nuevamente los axiomas de grupo y dos
propiedades más, las cuales establecen que $*$ es una acción lineal.
\begin{itemize}
\item $1v=v$
\item $(gh)v=g(hv)$
\item $g(v+w)=gv+gw$
\item $g(\lambda v)=\lambda(gv)$
\end{itemize}

Estos cuatro axiomas definen un $G$-módulo (ver definición
\ref{accion-lineal}). Inversamente se puede definir una representación
$\rho$ a partir de un $G$-módulo como sigue:
Dado un $G$-módulo, sea $\rho(g)(v)=gv$. Enseguida se
enuncia y demuestra justamente lo
discutido sobre representaciones de
$G$ en $V$ y $G$-módulos.

\begin{theorem}
  \label{rep-mod}
  Sea $G$ un grupo y $V$ un espacio vectorial de dimensión finita
  sobre el campo $F$. Existe una correspondencia biyectiva entre el conjunto de acciones
  lineales $G$ en $V$ y el conjunto de representaciones de $G$ en $GL(V)$.
\end{theorem}

\begin{proof}[Demostración]
  Supongamos primero que se tiene una acción lineal
  $$*:G\times V \rightarrow V$$
  para cada $g\in G$, usando esta acción definamos la función $\rho
  g:V \rightarrow V$ dada por $(\rho g)v=gv$ para todo $v\in V$.

  Enseguida demostraremos que $\rho g$ es una transformación lineal
  invertible. Sean $v,w\in V$, $\lambda \in F$, entonces:
  \begin{flalign*}
    &(\rho g)(v+w)=g(v+w)=gv+gw=(\rho g)v+(\rho g)w,\\
    &(\rho g)(\lambda v)=g(\lambda v)=\lambda(gv)=\lambda((\rho g)v).
  \end{flalign*}

  Así que $\rho g$ es lineal, ahora veamos que es invertible. Como $G$
  es un grupo, existe $g^{-1}\in G$.
  \begin{footnotesize}
    \begin{eqnarray*}
      ((\rho g)(\rho g)^{-1})v=(\rho g)((\rho g)^{-1}v)=(\rho g)((\rho g^{-1})v)=(\rho g)(g^{-1}v)=g(g^{-1}v)=(gg^{-1})v=v,\\
      ((\rho g)^{-1}(\rho g))v=(\rho g)^{-1}((\rho g)v)=(\rho g^{-1})((\rho g)v)=(\rho g^{-1})(gv)=g^{-1}(gv)=(g^{-1}g)v=v,
    \end{eqnarray*}
  \end{footnotesize}
  por lo que $\rho g$ tiene como inversa a $\rho g^{-1}$.

  Ahora, observemos que la función $\rho:G\rightarrow GL(V)$ es un
  homomorfismo de grupos. Si $g,h\in G$, entonces para todo
  $v\in V$:

  $$(\rho(gh))v=(gh)v=g(hv)=g((\rho h)v)=(\rho g)((\rho h)v)=(\rho g \circ \rho h)v$$
  por lo que $\rho(gh)=\rho g \circ \rho h$.

  Suponga ahora que se tiene un homomorfismo de grupos
  $\rho:G\rightarrow GL(V)$. Demostraremos que $gv:=(\rho g)v$ define
  un acción lineal de $G$ en $V$.

  \begin{enumerate}
  \item $1$ es el neutro de $G$, entonces $$1v=(\rho 1)v=1_{V}(v)=v$$
    para todo $v\in V$.
  \item Si $g,h\in G$ y $v\in V$, entonces $$(gh)v=(\rho g h)v=(\rho g
    \circ \rho h)v=\rho g((\rho h)v)=g((\rho h)v)=g(hv)$$
  \item Para $g\in G$ y $v,w\in V$, $$g(v+w)=(\rho g)(v+w)=(\rho
    g)v+(\rho g)w=gv+gw$$
  \item Para $\lambda\in F$, $v\in V$ y $g\in G$,
    $$g(\lambda v)=(\rho g)(\lambda v)=\lambda((\rho g)v)=\lambda(gv)$$
  \end{enumerate}
\end{proof}

% \begin{definition}
%   \label{representacion}
%   Si $G$ es un grupo y $V$ es un $\mathbb{C}$-espacio vectorial, una
%   \textbf{representación} de $G$ en $V$ es un homomorfismo 
%      $$\rho:G\rightarrow GL(V).$$
% \end{definition}

Así, el teorema \ref{rep-mod} dice que estudiar las representaciones de $G$
en $V$, es equivalente a estudiar $G$-módulos $V$. 

Ahora seguimos la construcción de el anillo de grupo dada en
[\cite{fulton1991representation}, sección 3.4, p.36]. Sea un grupo $G$, cualquier representación de $G$ asocia a cada
elemento de el grupo un operador lineal. Por tanto, en el contexto de
teoría de representaciones, los grupos pueden ser considerados como
conjuntos abstractos de operadores lineales. Consideremos al espacio
vectorial $FG$ construido como en el lema \ref{FX}, en donde $X$ es el grupo
$G$. Daremos al espacio vectorial $FG$ una operación multiplicación.

Sea el espacio vectorial $FG$. Si consideramos a los elementos de el
grupo~$G$ como operadores lineales invertibles podemos multiplicar
elementos de $FG$ exactamente como multiplicaríamos sumas de
transformaciones lineales:
\begin{equation*}
\label{mult-alg-grupo}
  (\sum_{g\in G}\lambda_{g}g)(\sum_{h\in G}\mu_{h}h)=\sum_{g\in
    G}\sum_{h \in G}\lambda_{g}\mu_{h}gh.
\end{equation*}
Esta definición de multiplicación convierte a el espacio vectorial
$FG$ en un anillo con elemento unitario, al cual llamamos
\textbf{anillo de grupo de $G$}.

Sea $V$ un $G$-módulo, es decir, se tiene la acción lineal de $G$ en $V$ 
$$*:G\times V\rightarrow V,$$ 
se define la función $*:FG\times V\rightarrow V$ como: 
\begin{equation*}
  (\sum_{g\in G}\lambda_{g}g)v=\sum_{g\in G}\lambda_{g}(gv),
\end{equation*}
para cada $\sum_{g\in G}\lambda_{g}g\in FG$ y $v\in V$. Así, $V$ es un
$FG$-módulo.

Ahora supongamos que se tiene un $FG$-módulo, definimos la función
$*:G\times V\rightarrow V$ dada por:
\begin{equation*}
  gv=(\sum_{g\in G}\lambda_{g}g)v
\end{equation*}
para cada $g\in G$ y $v\in V$. Por tanto, $V$ es un $G$-módulo.

Es equivalente estudiar $FG$-módulos y $G$-módulos, ya que tenemos una
correspondencia biyectiva entre estas estructuras.

Recordemos que $GL(n,F)$ es el grupo de matrices invertibles $n\times
n$ con entradas en el campo $F$ (ver ejemplo\ref{GL(nF)}). Y $GL(V)$ son los operadores lineales
invertibles en un espacio vectorial $V$ de dimensión finita $n$ sobre~$F$(ver ejemplo \ref{GL(V)}). Existe un isomorfismo entre los grupos
$GL(n,\mathbb{C})$ y $GL(V)$.

\begin{definition}
  Una \textbf{representación matricial de un grupo} $G$ es el
  homomorfismo de grupos 
  $$X:G\rightarrow GL(n,F),$$
  notemos que esta es la definición de represetación de $G$ en $V$, simplemente resaltamos que para cada $g\in G$, $X(g)$  es la
  matriz invertible $n \times n$ con entradas en $F$.
\end{definition}

% Se denotará $(\rho,V)$ para enfatizar el hecho de que la
% representación $\rho$ de $G$  es sobre el espacio $V$.

\begin{example}[Representación trivial]
  La representación trivial de un grupo $G$ es el homomorfismo
  $\rho:G\rightarrow GL(\mathbb{C})$ dado por
  $\rho(g)=1_{\mathbb{C}}$, para todo $g\in G$, donde $1_{\mathbb{C}}$
  es la función identidad $1_{\mathbb{C}}:\mathbb{C}\rightarrow\mathbb{C}$. Esta es una representación de grado 1, pues 1 es la
  dimensión de $\mathbb{C}$.
\end{example}

\begin{example}[Representación signo]
  Si $\pi\in S_{n}$ y $\pi=\tau_{1}\tau_{2}\cdots\tau_{k}$, donde $\tau_{i}$ son
  transposiciones (ver definición \ref{trans} y teorema \ref{per-prod-trans}), definimos la función
  signo $\sgn:S_{n} \rightarrow GL(\mathbb{C})$
  mediante: $$\sgn(\pi):=(-1)^{k}1_{\mathbb{C}},$$
  debido al teorema \ref{trans-par-impar}, la función signo está bien definida.
  % así que $\mathbb{C}$ recibe una estructura de $S_{n}$-módulo, con acción:
  % \[
  % \pi v=
  % \begin{cases}
  %   v, & \text{ si } \pi \text{ es par,}\\
  %   -v, & \text{ si } \pi \text{ es impar,}
  % \end{cases}
  % \]
  % para $\pi\in S_{n}$, $v\in \mathbb{C}$. Esta representación tiene grado $1$.
\end{example}

% \begin{example}[Representación defining]
% \label{rep-defining}
% Sea $S_{n}$ el grupo simétrico. Tomemos $V=\mathbb{C}^{n}=\{(x_{1},x_{2},\ldots,x_{n})\mid x_{i}\in
% \mathbb{C}, i=1,2,\ldots,n\}$ y la acción de $S_{n}$ en $V$ está dada
% por 
% \begin{equation*}
%   \pi(x_{1},x_{2},\ldots,x_{n})=(x_{\pi(1)},x_{\pi(2)},\ldots,x_{\pi(n)}),
% \end{equation*}
% donde $\pi\in S_{n}$, lo cual define un homomorfismo de grupos
% $S_{n}\rightarrow GL(\mathbb{C}^{n})$ conocida como \emph{representación
% defining}, que convierte a
% $V=\mathbb{C}^{n}$ en un $S_{n}$-módulo. Un submódulo importante de
% dimensión $n-1$ \setlength{\fboxsep}{0pt}\colorbox{green}{dim n-1???}de $\mathbb{C}^{n}$ es:
% \begin{equation*}
%   E=\Big\{(x_{1},x_{2},\ldots,x_{n})\in \mathbb{C}^{n}\mid\sum^{n}_{i=1}x_{i}=0\Big\}.
% \end{equation*}
% La representación asociada a éste submódulo se llama
% \textbf{representación estándar}.

% Sea $\mathbb{C}$ la representación trivial. Notemos que
% \begin{equation*}
%   \mathbb{C}^{n}=E\oplus \mathbb{C}
% \end{equation*}

% Ahora daremos un ejemplo en particular de la representación defining,
% consideremos $S_{3}$, sea $\mathbb{C}^{3}=\langle x_{1},x_{2},x_{3}\rangle$, donde

% \begin{center}
%   $x_{1}=\begin{pmatrix}
%     1 \\
%     0 \\
%     0
%   \end{pmatrix}$, $x_{2}=\begin{pmatrix}
%     0 \\
%     1 \\
%     0
%   \end{pmatrix}$, $x_{3}=\begin{pmatrix}
%     0 \\
%     0 \\
%     1
%   \end{pmatrix}$.
% \end{center}
% Las matrices \setlength{\fboxsep}{0pt}\colorbox{green}{\textbf{??}} de la representación regular son:
% \begin{center}
%   $X(1)=\begin{pmatrix}
%     1 & 0 & 0 \\
%     0 & 1 & 0 \\
%     0 & 0 & 1 \\
%   \end{pmatrix}$,\quad
%   $X((12))=\begin{pmatrix}
%     0 & 1 & 0 \\
%     1 & 0 & 0 \\
%     0 & 0 & 1 \\
%   \end{pmatrix}$,
% \end{center}

% \begin{center}
%   $X((13))=\begin{pmatrix}
%     0 & 0 & 1 \\
%     0 & 1 & 0 \\
%     1 & 0 & 0 \\
%   \end{pmatrix}$,\quad 
%   $X((23))=\begin{pmatrix}
%     1 & 0 & 0 \\
%     0 & 0 & 1 \\
%     0 & 1 & 0 \\
%   \end{pmatrix}$,
% \end{center}

% \begin{center}
%   $X((123))=\begin{pmatrix}
%     0 & 0 & 1 \\
%     1 & 0 & 0 \\
%     0 & 1 & 0 \\
%   \end{pmatrix}$,\quad $X((132))=\begin{pmatrix}
%     0 & 1 & 0 \\
%     0 & 0 & 1 \\
%     1 & 0 & 0 \\
%   \end{pmatrix}$.
% \end{center}
%\end{example}

% \begin{example}[Representación permutación]
%   %\label{CS}
%   Sea $G$ un grupo. Es posible tomar un conjunto en el cual actúe $G$
%   y resulte un $G$-módulo, como lo hacemos a continuación. Sea
%   $S=\{s_{1},s_{2},\ldots,s_{n}\}$ y $\mathbb{C}\boldsymbol{S}=\mathbb{C}\{s_{1},s_{2},\ldots,s_{n}\}$ denote el
%   espacio vectorial generado por $S$ sobre $\mathbb{C}$. Es decir,
%   $\mathbb{C}\boldsymbol{S}$ consiste de todas las combinaciones lineales
%   \begin{equation*}
%     \lambda_{1}s_{1}+\lambda_{2}s_{2}+\cdots +\lambda_{n}s_{n}
%   \end{equation*}
%   donde $\lambda_{i}\in\mathbb{C}$. La adición y multiplicación por escalar
%   en $\mathbb{C}\boldsymbol{S}$ está definida por
%   \begin{eqnarray*}
%     (\lambda_{1}s_{1}+\lambda_{2}s_{2}+\cdots
%     +\lambda_{n}s_{n})+(\lambda^{'}_{1}s_{1}+\lambda^{'}_{2}s_{2}+\cdots +\lambda^{'}_{n}s_{n})\\
%     =(\lambda_{1}+\lambda^{'}_{1})s_{1}+(\lambda_{2}+\lambda^{'}_{2})s_{2}+\cdots
%     +(\lambda_{n}+\lambda^{'}_{n})s_{n}
%   \end{eqnarray*}
%   y
%   \begin{eqnarray*}
%     c(\lambda_{1}s_{1}+\lambda_{2}s_{2}+\cdots +\lambda_{n}s_{n})=(c\lambda_{1})s_{1}+(c\lambda_{2})s_{2}+\cdots +(c\lambda_{n})s_{n},
%   \end{eqnarray*}
%   respectivamente. Ahora la acción de $G$ en $S$ se puede extender
%   linealmente a la acción en $\mathbb{C}\boldsymbol{S}$:
%   \begin{equation*}
%     g(\lambda_{1}s_{1}+\lambda_{2}s_{2}+\cdots +\lambda_{n}s_{n})=\lambda_{1}(gs_{1})+\lambda_{2}(gs_{2})+\cdots +\lambda_{n}(gs_{n})
%   \end{equation*}
%   para todo $g\in G$. Así que $\mathbb{C}\boldsymbol{S}$ resulta ser un $G$-módulo
%   de dimensión $|S|$. A $\mathbb{C}\boldsymbol{S}$ le llamamos
%   \emph{representación permutación} asociada a $S$. Los elementos de
%   $S$ forman una base para $\mathbb{C}\boldsymbol{S}$ llamada la
%   \emph{base estándar}.

%   Consideremos el grupo simétrico $S_{n}$ actuando en
%   $S=\{1,2,\ldots,n\}$. Ahora
%   \begin{equation*}
%     \mathbb{C}\boldsymbol{S}=\{\lambda_{1}1+\lambda_{2}2+\cdots+\lambda_{n}n\mid \lambda_{i}\in\mathbb{C}\}
%   \end{equation*}
%   con la acción
%   \begin{equation*}
%     \pi(\lambda_{1}1+\lambda_{2}2+\cdots+\lambda_{n}n)=\lambda_{1}\pi(1)+\lambda_{2}\pi(2)+\cdots+\lambda_{n}\pi(n)
%   \end{equation*}
% Consideremos un caso específico. Sea $S_{3}$, determinemos las
% matrices asociadas respecto a la base estándar $\{1,2,3\}$. Para
% $\pi=(12)$, calculamos
% \begin{center}
%   (12)1=2; \qquad (12)2=1; \qquad (12)3=3
% \end{center}
% así que
% \begin{center}
%   $X((12))=\begin{pmatrix}
%     0 & 1 & 0 \\
%     1 & 0 & 0 \\
%     0 & 0 & 1 \\
%   \end{pmatrix}$,
% \end{center}
% Notemos que las matrices restantes son las mismas que las de la
% representación regular del ejemplo \ref{rep-regular}. Se puede
% demostrar que esto es cierto para cualquier $n$. \setlength{\fboxsep}{0pt}\colorbox{green}{\textbf{??}}
% \end{example}
\begin{example}[Módulo de permutaciones]
  \label{CS}
  Sea $G$ un grupo, $F$ un campo, $X$ el conjunto $\{x_{1},x_{2},\ldots,x_{n}\}$ y $FX$ el
  espacio vectorial construido como en el lema \ref{FX}, es decir,
  si $x\in FX$, podemos escribir a $x$ como hicimos en la expresión
\ref{sumas-formales}, es decir,
$x=\lambda_{1}x_{1}+\lambda_{2}x_{2}+\cdots+\lambda_{n}x_{n}$.
  \begin{equation*}
    V=FX=\{\lambda_{1}x_{1}+\lambda_{2}x_{2}+\cdots
    +\lambda_{n}x_{n}\mid\lambda_{1},\ldots,\lambda_{n}\in F\}.
  \end{equation*}
  La adición y multiplicación por escalar
  en $FX$ está definida por
  \begin{eqnarray*}
    (\lambda_{1}x_{1}+\lambda_{2}x_{2}+\cdots
    +\lambda_{n}x_{n})+(\lambda^{'}_{1}x_{1}+\lambda^{'}_{2}x_{2}+\cdots +\lambda^{'}_{n}x_{n})\\
    =(\lambda_{1}+\lambda^{'}_{1})x_{1}+(\lambda_{2}+\lambda^{'}_{2})x_{2}+\cdots
    +(\lambda_{n}+\lambda^{'}_{n})x_{n}
  \end{eqnarray*}
  y
  \begin{eqnarray*}
    c(\lambda_{1}x_{1}+\lambda_{2}x_{2}+\cdots +\lambda_{n}x_{n})=(c\lambda_{1})x_{1}+(c\lambda_{2})x_{2}+\cdots +(c\lambda_{n})x_{n},
  \end{eqnarray*}
  respectivamente. Sean $v\in V$, con
  $v=\lambda_{1}x_{1}+\lambda_{2}x_{2}+\cdots+\lambda_{n}x_{n}$ y
  $g\in G$ entonces la acción de $G$ en $V$ está dada por
  \begin{equation*}
    gv=g(\lambda_{1}x_{1}+\lambda_{2}x_{2}+\cdots +\lambda_{n}x_{n})=\lambda_{1}(gx_{1})+\lambda_{2}(gx_{2})+\cdots +\lambda_{n}(gx_{n}).
  \end{equation*}
  Así, $FX$ resulta ser un $G$-módulo de dimensión $|X|$. A~$FX$ le llamamos
  \textbf{módulo de permutaciones}. Los elementos de $X$ forman una base para~$FX$.

  Por ejemplo, el grupo simétrico $S_{n}$ actuando en $\mathbb{I}_{n}=\{1,2,\ldots,n\}$.
  \begin{equation*}
    F\mathbb{I}_{n}=\{\lambda_{1}1+\lambda_{2}2+\cdots+\lambda_{n}n\mid
    \lambda_{i}\in F\},
  \end{equation*}
  sea $\pi\in S_{n}$, la acción es
  \begin{equation*}
    \pi(\lambda_{1}1+\lambda_{2}2+\cdots+\lambda_{n}n)=\lambda_{1}\pi(1)+\lambda_{2}\pi(2)+\cdots+\lambda_{n}\pi(n).
  \end{equation*}

  Consideremos un caso específico, el de $S_{2}=\{(1),(12)\}$ actuando en
  $\mathbb{I}_{2}$. Determinemos las
  matrices asociadas respecto a la base $\mathbb{I}_{2}$. Para
  $\pi=(12)$, calculamos (12)1=2 y (12)2=1. Así que
  \begin{center}
    $X((12))=\begin{pmatrix}
      0 & 1  \\
      1 & 0  
    \end{pmatrix}$,
  \end{center}
  la otra representación matricial $X((1))$ es:
  \begin{center}
    $X((1))=\begin{pmatrix}
      1 & 0  \\
      0 & 1  
    \end{pmatrix}$,
  \end{center}
Entonces $F\mathbb{I}_{2}$ es un $S_{2}$-módulo de permutaciones.

% Consideremos un caso específico, el de $S_{3}$ actuando en
%   $\mathbb{I}_{3}$. Determinemos las
%   matrices asociadas respecto a la base $\mathbb{I}_{3}$. Para
%   $\pi=(12)$, calculamos
%   \begin{center}
%     (12)1=2; \qquad (12)2=1; \qquad (12)3=3
%   \end{center}
%   así que
%   \begin{center}
%     $X((12))=\begin{pmatrix}
%       0 & 1 & 0 \\
%       1 & 0 & 0 \\
%       0 & 0 & 1 \\
%     \end{pmatrix}$,
%   \end{center}
%   las otras representaciones matriciales $X(\pi)$, con $\pi\in S_{3}$ son:
%   \begin{center}
%     $X(1)=\begin{pmatrix}
%       1 & 0 & 0 \\
%       0 & 1 & 0 \\
%       0 & 0 & 1 \\
%     \end{pmatrix}$,\quad  
%     $X((13))=\begin{pmatrix}
%       0 & 0 & 1 \\
%       0 & 1 & 0 \\
%       1 & 0 & 0 \\
%     \end{pmatrix}$,\quad 
%     $X((23))=\begin{pmatrix}
%       1 & 0 & 0 \\
%       0 & 0 & 1 \\
%       0 & 1 & 0 \\
%     \end{pmatrix}$,
%   \end{center}
  
%   \begin{center}
%     $X((123))=\begin{pmatrix}
%       0 & 0 & 1 \\
%       1 & 0 & 0 \\
%       0 & 1 & 0 \\
%     \end{pmatrix}$,\quad $X((132))=\begin{pmatrix}
%       0 & 1 & 0 \\
%       0 & 0 & 1 \\
%       1 & 0 & 0 \\
%     \end{pmatrix}$.
%   \end{center}
% $F\mathbb{I}_{3}$ es un $S_{3}$-módulo de permutaciones.
\end{example}

\begin{example}[Módulo regular]
  Sea $G$ un grupo y $F$ un campo. Procedemos como en el ejemplo \ref{CS}, pero ahora $G$
  actúa sobre el espacio vectorial~$FG$ con el producto usual de
  grupo. Sean $X=G=\{g_{1},g_{2},\ldots,g_{n}\}$ y 
  \begin{equation*}
    V=FG=\{\lambda_{1}g_{1}+\lambda_{2}g_{2}+\cdots
    +\lambda_{n}g_{n}\mid\lambda_{1},\ldots,\lambda_{n}\in F\}.
  \end{equation*}
 Sean $v\in V$ con $v=\lambda_{1}g_{1}+\lambda_{2}g_{2}+\cdots
 +\lambda_{n}g_{n}$ y $g\in G$, entonces la acción de~$G$ en $V$ está dada por
  \begin{equation*}
    gv=g(\lambda_{1}g_{1}+\lambda_{2}g_{2}+\cdots +\lambda_{n}g_{n})=\lambda_{1}(gg_{1})+\lambda_{2}(gg_{2})+\cdots +\lambda_{n}(gg_{n}).
  \end{equation*}
Con esta acción, $FG$ es un $G$-módulo, al cual llamamos \textbf{módulo regular}.  
\end{example}

\section{Reducibilidad}
\label{reducibilidad}

En esta sección, $G$ denota un grupo {\setlength{\fboxsep}{0pt}\colorbox{green}{finito???}} y $V$ un $\mathbb{C}$-espacio vectorial.

\begin{definition}
  Sea $V$ un $G$-módulo. Un $\boldsymbol{G}$\textbf{-submódulo} o
  \textbf{submódulo} de $V$ es un subespacio vectorial
  $W$ que es invariante bajo la acción de $G$, es decir, $ gw\in W$
  para todos $w\in W$, $g\in G$, tal que $W$ junto con la acción de
  $G$ es en sí mismo un $G$-módulo. Escribiremos $W\leq V$ si $W$ es
  submódulo de~$V$. 
\end{definition}

\begin{example}
  \label{sub-estandar}
  De el ejemplo \ref{CS}, un submódulo importante de dimensión $n-1$ de
  $F\mathbb{I}_{n}$ es:
  \begin{equation*}
    E=\Big\{(\lambda_{1}1+\lambda_{2}2+\cdots+\lambda_{n}n)\in F\mathbb{I}_{n}\mid\sum^{n}_{i=1}\lambda_{i}=0\Big\}.
  \end{equation*}
  Este submódulo se llama \textbf{módulo estándar}. 
  
  Sea $\mathbb{C}\cong
  \{\lambda_{1}1+\lambda_{2}2+\cdots+\lambda_{n}n\mid
  \lambda_{1}=\lambda_{2}=\cdots=\lambda_{n}\}$ submódulo de~$F\mathbb{I}_{n}$. Se
  obtiene que es isomorfo a la representación trivial, y que:

  \begin{equation*}
    F\mathbb{I}_{n}=E\oplus \mathbb{C}.
  \end{equation*}  
\end{example}

\begin{definition}
  Sea $V$ un $G$-módulo no trivial. Decimos que $V$ es \textbf{irreducible} si
  los únicos submódulos de $V$ son $0$ y $V$.
\end{definition}

% De la definición \ref{suma-directa} (suma directa) si $V$ es un
% $G$-módulo y $W_{1}$, $W_{2}$ son $G$-submódulos, entonces diremos
% que $W_{1}$ es \textbf{complemento directo} de $W_{2}$. 

\begin{theorem}[Teorema de Maschke][\cite{sagan2001symmetric}, teorema
  1.5.3, p.16], [\cite{fulton1991representation}, proposición 1.8, p.7].
  \label{maschke}
  Sea $G$ un grupo finito, $F$ un campo tal que su característica no
  divide al orden de $G$ y $V$ un $G$-módulo. Entonces $V$ es
  \textbf{completamente reducible}, es decir
  \begin{equation*}
    V\cong m_{1}W^{(1)}\oplus m_{2}W^{(2)}\oplus\cdots \oplus m_{k}W^{(k)},
  \end{equation*}
donde $W^{(i)}$ y $W^{(j)}$ con $i\neq j$ son $G$-submódulos
irreducibles de $V$ no isomorfos y $m_{i}$ es la multiplicidad, es
decir la cantidad de $W^{(i)}$.
\end{theorem}
En el siguiente ejemplo, no se cumple la hipótesis de el teorema de
Maschke, pues la característica de el campo divide al orden de el
grupo. Y en este ejemplo $V$ no es completamente reducible.
\begin{example}
  \label{ej-no-reducibilidad}
  Sea el campo $F=\{0,1\}$ y $G=C_{2}=\{g^{0},g^{1}\}$ el grupo
  cíclico de dos elementos. Sea el espacio vectorial
  $V=F^{2}=\{v_{1},v_{2},v_{3},v_{4}\}$, donde  
  $v_{1}=
  \begin{pmatrix}
    0\\
    0  
  \end{pmatrix},
  v_{2}=
  \begin{pmatrix}
    0\\
    1  
  \end{pmatrix},
  v_{3}=
  \begin{pmatrix}
    1\\
    0  
  \end{pmatrix},
  v_{4}=
  \begin{pmatrix}
    1\\
    1  
  \end{pmatrix}.
  $
  Definimos la función $\phi:G\rightarrow GL(V)$ como:
  \begin{center}
    $\phi(g^{r})=
    \begin{pmatrix}
      1 & 0 \\
      r & 1 
    \end{pmatrix}$,
  \end{center}
  la cual está bien definida y determina un homomorfismo de grupos.

Queremos determinar si $V$ tiene algún submódulo propio. Como $V$
tiene dimensión 2, un submódulo propio $W$ deberá tener dimensión 1,
es decir, sería generado por un elemento diferente de cero.

%Primero veamos que $\langle v_{3}\rangle$ y $\langle v_{4}\rangle$ son
%subespacios no invariantes.

Primero notemos que los subespacios $\langle v_{3}\rangle$ y $\langle
v_{4}\rangle$ no son invariantes.
\begin{center}
  $g^{1}v_{3}=\phi(g^{1})v_{3}=
  \begin{pmatrix}
      1 & 0 \\
      1 & 1 
    \end{pmatrix}
    \begin{pmatrix}
    1\\
    0  
  \end{pmatrix}=
  \begin{pmatrix}
    1\\
    1  
  \end{pmatrix}=v_{4},
  $
\end{center}
como $v_{4}\not\in\langle v_{3}\rangle$, por tanto el subespacio $\langle
v_{3}\rangle$ no es invariante, es decir, $\langle v_{3} \rangle$ no
es submódulo propio de $V$.
\begin{center}
  $g^{1}v_{4}=\phi(g^{1})v_{4}=
  \begin{pmatrix}
      1 & 0 \\
      1 & 1 
    \end{pmatrix}
    \begin{pmatrix}
    1\\
    1  
  \end{pmatrix}=
  \begin{pmatrix}
    1\\
    0  
  \end{pmatrix}=v_{3},
  $
\end{center}
similarmente, $v_{3}\not\in\langle v_{4}\rangle$, el subespacio $\langle
v_{4}\rangle$ no es submódulo propio de $V$.

A continuación se verifica que $\langle v_{2}\rangle=\{v_{1},v_{2}\}$ es el
único submódulo propio de $V$, pues $\langle v_{2}\rangle$ es
invariante bajo la acción de $G$, ya que $g^{0}v_{1}=v_{1}$,
$g^{1}v_{1}=v_{1}$, $g^{0}v_{2}=v_{2}$ y
\begin{center}
  $g^{1}v_{2}=\phi(g^{1})v_{2}=
  \begin{pmatrix}
      1 & 0 \\
      1 & 1 
    \end{pmatrix}
    \begin{pmatrix}
    0\\
    1  
  \end{pmatrix}=
  \begin{pmatrix}
    0\\
    1  
  \end{pmatrix}=v_{2}.
  $
\end{center}
Por tanto, $V$ no es completamente reducible, es decir, no podemos
escribir a $V$ como $V\cong\langle v_{2}\rangle\oplus W$, pues el
submódulo $W$ debería tener dimensión 1, y el único submódulo con
dimensión 1 es $\langle v_{2}\rangle$.
\end{example}

\begin{theorem}
  \label{interseccion-submodulos}
  Sea $V$ un $G$-módulo. Entonces la intersección de cualquier
  colección de submódulos de $V$ es un submódulo de $V$.
\end{theorem}

\begin{definition}
  Sea $V$ un $G$-módulo. Al submódulo de $V$ generado por las combinaciones lineales de los
  elementos de un subespacio $W$ de $V$ lo llamaremos \textbf{submódulo generado} por $W$ y lo denotaremos como $\langle W\rangle$.
\end{definition}

Sean $U$ y $V$ dos espacios vectoriales. Llamaremos \textbf{suma
  directa externa} de
$U$ y $V$ al conjunto $U\times V=\{(u,v):u\in U,v\in V\}$ con las operaciones
$$(u,v)+(u_{1},v_{1})=(u+u_{1},v+v_{1}),$$
$$k(u,v)=(ku,kv),$$
donde $u,u_{1}\in U$, $v,v_{1}\in V$ y $k\in \mathbb{C}$. Con estas
operaciones $U\times V$ es un espacio vectorial, que designaremos por
$$U\oplus V.$$

Así, dados $V$ y $W$ $G$-módulos, podemos formar un tercer $G$-módulo a partir
de la suma directa externa $V\oplus W$, definiendo la acción como
$a(v,w)=~(av,aw)$, con $a\in G$. Podemos extender esta definición a cualquier
cantidad finita de $G$-módulos.

Por otro lado, si $V$ es un $G$-módulo, se llama la \textbf{suma} de $W_{1}$ y $W_{2}$ y se denota con
$W_{1}+W_{2}$ al submódulo de $V$ generado por $W_{1}\cup W_{2}$, es
decir, $W_{1}+W_{2}=\langle W_{1}\cup W_{2}\rangle$. Esta definición se puede extender a una
colección arbitraria $\{W_{i}\}_{i\in I}$, es
decir,~$\sum_{i\in I}W_{i}$ es el submódulo generado por $\bigcup_{i\in I}W_{i}$.

\textbf{\emph{Observación:}} Si $W_{1}$ y $W_{2}$ son
submódulos de $V$ con $W_{1}\cap W_{2}=0$ tenemos que
$W_{1}+W_{2}\cong W_{1}\oplus W_{2}$.  Por tanto, si $V=W_{1}+W_{2}$,
es decir, si $V=\langle W_{1}\cup W_{2}\rangle$ y $W_{1}\cap W_{2}=0$, escribiremos $V=W_{1}\oplus W_{2}$
aunque estrictamente hablando es $V\cong W_{1}\oplus W_{2}$. En adelante no distinguiremos
entre suma directa y suma directa externa.

% dado por el siguiente isomorfismo:
% \begin{align*}
%   \phi:W_{1}+W_{2}&\rightarrow W_{1}\oplus W_{2}\\
%   w & \mapsto  (w_{1},w_{2})
% \end{align*}
% sean
% $w^{1}_{1},w^{2}_{1},\ldots,w^{n}_{1},w^{1}_{2},w^{2}_{2},\ldots,w^{m}_{2}$
% los elementos de $W_{1}\cup W_{2}$ donde $w^{i}_{1}\in W_{1}$ y
% $w^{i}_{2}\in W_{2}$, tomemos $w\in W_{1}+W_{2}$, es decir,
% \begin{align*}
%   w&=a_{1}w^{1}_{1}+a_{2}w^{2}_{1}+\ldots+a_{n}w^{n}_{1}+a_{n+1}w^{1}_{2}+a_{n+2}w^{2}_{2}+\ldots+a_{m}w^{m}_{2}\\
%   &=w_{1}+w_{2}
% \end{align*}
% donde $w_{1}=a_{1}w^{1}_{1}+a_{2}w^{2}_{1}+\ldots+a_{n}w^{n}_{1}$ y
% $w_{2}=a_{n+1}w^{1}_{2}+a_{n+2}w^{2}_{2}+\ldots+a_{m}w^{m}_{2}$,
% $w_{1}\in W_{1}$ puesto que $w_{1}$ es combinación lineal de los
% elementos de $W_{1}$, de forma análoga concluimos que $w_{2}\in
% W_{2}$. Ahora veamos que $\phi$ está bien definida, supongamos que
% $w=w_{1}+w_{2}=w^{'}_{1}+w^{'}_{2}$, así que
% $w_{1}-w^{'}_{1}=w^{'}_{2}-w_{2}$ y notamos que $w_{1}-w^{'}_{1}\in W_{1}$ y
% $w^{'}_{2}-w_{2}\in W_{2}$, así que
% $w_{1}-w^{'}_{1}=w^{'}_{2}-w_{2}\in W_{1}\cap W_{2}$, entonces
% $w_{1}-w^{'}_{1}=w^{'}_{2}-w_{2}=0$ por hipótesis, por lo tanto
% $w_{1}=w^{'}_{1}$ y $w^{'}_{2}=w_{2}$.
\begin{proposition}
  \label{modulos-iguales}
  Sea $M$ un $G$-módulo. Supongamos que $S\leq M$, donde $S$ es irreducible. Supongamos
  además que $S$ aparece con multiplicidad 1 en la descomposición en
  irreducibles de $M$, y sea $S^{'}\leq M$, tal que $S^{'}\cong S$. Entonces~$S=S^{'}$.
\end{proposition}

\begin{proof}[Demostración.]
  Consideremos $S\cap S^{'}\leq S$, así que $S\cap S^{'}=0$ o $S\cap
  S^{'}=S$, pues $S$ es irreducible.

  Si $S\cap S^{'}=0$, consideremos $U$ el submódulo generado por $S\cup S'$ al cual
  denotamos como $S+S^{'}$. Puesto que $S\cap S^{'}=0$, entonces
  $S+S^{'}\cong S\oplus S^{'}$ como notamos en la observación anterior,
  así que $M$ tendría un submódulo isomorfo a dos copias de $S$, lo cual
  no es posible por hipótesis. 

  Por otro lado, si $S\cap S^{'}=S$, entonces $\dim (S\cap S^{'})=\dim(S)$, además por hipótesis $S\cong S^{'}$, así
  $\dim(S)=\dim(S^{'})$. Por tanto $\dim(S\cap S^{'})=\dim(S^{'})$ y
  dado que $S\cap S^{'}\leq S^{'}$, %por el teorema \ref{esp-iguales}, 
  tenemos que $S\cap S^{'}=S^{'}$, con lo que~$S=S^{'}$ como se quería
  demostrar.
\end{proof}

\section{$G$-homomorfismo y lema de Schur} 
\label{schur}

\begin{definition}
  Dados dos $G$-módulos $V$ y $W$. Un $G$\textbf{-homomorfismo} (o simplemente
  un \textbf{homomorfismo}) es una transformación lineal $\phi:V\rightarrow
  W$ tal que
  \begin{equation*}
    \phi(gv)=g\phi(v),
  \end{equation*}
para todo $g\in G$ y $v\in V$. Decimos entonces que $\phi$ \emph{preserva} la acción
de $G$.
\end{definition}

\begin{definition}
  Dados dos $G$-módulos $V$ y $W$. Un $G$\textbf{-isomorfismo} es un
  $G$-homomorfismo $\phi:V\rightarrow W$ el cual es biyectivo. En
  este caso decimos que $V$ y $W$ son $G$\textbf{-isomorfos} o
  $G$\textbf{-equivalentes}, y lo denotaremos como~$V\cong W$.
\end{definition}

\begin{proposition}{[\cite{sagan2001symmetric}, proposición 1.7.9, p.21]}.
  \label{ker-im-sub}
  Sea $\phi:V\rightarrow W$ sea un $G$-homomorfismo. Entonces
  \begin{enumerate}
  \item $\ker\phi$ es un $G$-submódulo de $V$, y
  \item $\im\phi$ es un $G$-submódulo de $W$.
  \end{enumerate}
\end{proposition}  
Si $W$ es submódulo de $V$, entonces en el espacio cociente
$V/W$ puede darse una acción de $G$ como $g(v+W)=gv+W$, la cual está
bien definida (pues $W$ es invariante bajo la acción de $G$), y
satisface los axiomas de módulo sobre $G$, por lo que obtenemos una
estructura de $G$-módulo en~$V/W$ llamada \textbf{módulo cociente}.

\begin{theorem}[Teorema de isomorfismo de módulos]
  \label{teorema-isomorfismo-mod}
  Sea $\phi:V\rightarrow W$ un $G$-homomorfismo. Entonces el
  módulo cociente $V/\ker\phi$ es isomorfo a $\im\phi$ por medio de
  $v+\ker\phi\mapsto\phi(v)$.
\end{theorem}

\begin{theorem}[Lema de Schur]
  \label{lema-schur}
  Si $V$ y $W$ son $G$-módulos irreducibles, y~$\phi:V\rightarrow W$
  es un $G$-homomorfismo no trivial, entonces $\phi$ es un isomorfismo.
\end{theorem}

\begin{proof}[Demostración.]
  Como $\ker \phi\leq V$, y $V$ es irreducible, entonces $\ker \phi=0$
  o~$\ker\phi=V$, pero  $\ker\phi\neq V$ pues $\phi$ es no trivial,
  así que $\ker \phi=0$ y por lo tanto, $\phi$ es inyectiva. 

  Análogamente, $\im\phi\leq W$ e $\im \phi\neq 0$, así que $\im
  \phi=W$ y entonces $\phi$ es sobreyectiva.
\end{proof}

\begin{proposition}
  \label{im-mod-irreducible}
  Si $V$, $W$ son $G$-módulos y $f:V\rightarrow W$ un
  $G$-homomorfismo, $S\leq V$ con $S$ irreducible, entonces $f(S)\cong
  S$ o $f(S)=0$.
\end{proposition}

\begin{proof}[Demostración.]
  Tomemos el siguiente $G$-homomorfismo
  $S\stackrel{i}{\hookrightarrow} V\stackrel{f}{\rightarrow}W$
  donde~$i$ es la función inclusión. Así, por el teorema \ref{teorema-isomorfismo-mod}, $S/\ker(f\circ
  i)\cong\im(f\circ i)$. Como $\ker(f\circ
  i)\leq S$ entonces $\ker(f\circ i)=0$ o $\ker(f\circ i)=S$, pues $S$
  es irreducible.

  Si $\ker(f\circ i)=0$, se sigue que
  $$S\cong S/0\cong\im(f\circ i)=f(S).$$

  Por otro lado, si $\ker(f\circ i)=S$, tenemos
  $$0=S/S\cong\im(f\circ i)=f(S).$$
\end{proof}
% \begin{theorem}
%   Sea $V$ un $G$ módulo tal que
%   \begin{equation*}
%     V\cong m_{1}V^{(1)}\oplus m_{2}V^{(2)}\oplus\cdots\oplus
%     m_{k}V^{(k)},
%   \end{equation*}
%   donde para cada $V^{(i)}$ y $V^{(j)}$ con $i\neq j$ son submódulos irreducibles no isomorfos y $\dim
%   V^{(i)}=d_{i}$. Entonces $\dim V=m_{1}d_{1}+m_{2}d_{2}+\cdots+m_{k}d_{k}.$
% \end{theorem}
\section{Caracteres de representaciones}
\label{caracteres}

Mucha información contenida en una representación $X:G\rightarrow
GL(n,\mathbb{C})$ puede ser obtenida de la traza de sus
correspondientes representaciones matriciales $X(g)$ con $g\in G$. De
esto se ocupa la teoría de caracteres.

\begin{definition}
  Sea $X(g)$ una representación matricial con $g\in G$. Entonces el \textbf{carácter} de $X$ es
  \begin{equation*}
    \chi(g)=\tr X(g),
  \end{equation*}
  donde $\tr$ denota la traza de una matriz dada. Dicho de otro modo,
  $\chi$ es la función
\begin{equation*}
  G\stackrel{\tr X}{\rightarrow}\mathbb{C}.
\end{equation*}
Si $V$ es un $G$-módulo, su carácter es el carácter de alguna
representación matricial $X$ correspondiente a $V$ (por propiedades de
la traza, el carácter no depende de la base de $V$ escogida). En ocasiones, para resaltar
que el carácter corresponde al $G$-módulo $V$ lo denotaremos como $\chi_{V}$.
\end{definition}

Parte de la terminología utilizada para representaciones se
usará para los correspondientes caracteres. Por ejemplo, si $X$ tiene
carácter $\chi$, diremos que~$\chi$ es irreducible siempre que $X$ lo
sea.

Se muestra al final de esta sección, un ejemplo de cómo calcular caracteres de la representación
trivial, signo y estándar con la acción de $S_{3}$.
\begin{definition}
  Dos elementos $g,g_{1}\in G$ son \textbf{conjugados} si
  \begin{equation*}
  g=hg_{1}h^{-1}
\end{equation*}
para algún $h\in G$. El conjunto de todos los elementos conjugados
dado $g$ es llamado \textbf{clase de conjugación} de $g$ y es
denotado por $K_{g}$.
\end{definition}

\begin{proposition}{[\cite{sagan2001symmetric}, proposición 1.8.5, p.32]}.
  Sea $G$ un grupo finito y sea $\chi$ el carácter de una
  representación $\rho:G\rightarrow GL(V)$ de grado $n$. Entonces,
  \begin{enumerate}
  \item $\chi(1)=n$.
  \item Si $K$ es una clase de conjugación de $G$, entonces $\chi(g)=\chi(h)$ para $g,h\in K$. 
  \end{enumerate}
\label{carac-const}
\end{proposition}
%Se expone la demostración en la proposición 1.8.5 de \cite{sagan2001symmetric}.
\begin{definition}
  Una \textbf{función de clase} es una función
  $\theta:G\rightarrow\mathbb{C}$ que es constante en clases de
  conjugación de $G$, es decir, si $K$ es una clase de conjugación de
  $G$, entonces $\theta(g)=\theta(h)$ para $g,h\in K$. El conjunto de
  todas las funciones de clase en $G$ es denotado por $R(G)$.
\end{definition} 
Los caracteres son funciones de clase (ver inciso 2 de la proposición
\ref{carac-const}).
\begin{theorem}{[\cite{sagan2001symmetric}, proposición 1.10.1, p.40]}, {[\cite{sagan2001symmetric}, proposición 1.10.2, p.42]}.
  \label{no-rep-irr-no-cla-conj}
  Sea $G$ un grupo finito. Entonces,
  \begin{enumerate}
  \item Los caracteres irreducibles
    $\chi_{1},\chi_{2},\ldots,\chi_{r}$ de $G$ forman una base del
    espacio de funciones de clase $R(G)$.
    \item El número de representaciones irreducibles de $G$, salvo
      isomorfismos, es igual al número de clases de conjugación de $G$.
  \end{enumerate}
\end{theorem}
%Se puede consultar la demostración en 1.10.2 de \cite{sagan2001symmetric}.
En general no se logra una correspondencia biyectiva natural entre las
representaciones de $G$ y las clases de conjugación. Sin embargo, en
el siguiente capítulo observamos que en el caso de el grupo simétrico sí se cuenta con
esta correspondencia.

Si $K$ es una clase de conjugación y $\chi$ es una carácter, se define
$\chi_{K}$ el valor del carácter en la clase dada.
\begin{equation*}
  \chi_{K}=\chi(g)
\end{equation*}
para cualquier $g\in K$. Esto sugiere la definición de tabla de caracteres de
un grupo. 
\begin{definition}
  Sea $G$ un grupo finito. La \textbf{tabla de caracteres} de $G$ es
  una tabla cuyos renglones están indexados por los caracteres
  irreducibles de~$G$ y las columnas indexadas por las clases de
  conjugación. La entrada de la tabla en el renglón $\chi$ y la columna $K$ es
\end{definition}

\begin{center}
  \begin{tabular}{c |c c c}
    & $\cdots$ & $K$ & $\cdots$\\
    \hline
    $\vdots$ &  & $\vdots$  \\
    $\chi$   & $\cdots$ & $\chi_{K}$  \\
    $\vdots$ &          &   
  \end{tabular}
\end{center}
Por convención, el primer renglón corresponde al carácter trivial, y la
primera columna corresponde a la clase identidad $K=(1)$.

\begin{example}
  \label{caracteres-S3}
  Sea $G=S_{3}$. A continuación calcularemos el
  carácter $\chi_{V}(g)$ de tres $S_{3}$-módulos $V$ y elementos
  $g\in S_{3}$. Basta calcular $\chi_{V}:S_{3}\rightarrow\mathbb{C}$
  en representantes de cada clase de conjugación de $S_{3}$, como $(1)$,
  $(12)$ y $(123)$.
  
  Primero consideremos $V=\mathbb{C}$, la representación trivial. El
  espacio $V$ es de dimensión $1$ (por lo que $\chi_{V}$ es la traza de
  la matriz $1\times 1$) y tiene a $1\in \mathbb{C}$ como
  base. Tenemos que $(1)1=1$, por lo que $\chi_{V}((1))=1$, así también
  $(12)1=1$ y $(123)1=1$, se sigue que $\chi_{V}((12))=1$ y
  $\chi_{V}((123))=1$.

  Ahora denotemos al espacio de representación de la representación
  signo como $V=\mathbb{\hat C}$. Nuevamente
  $\dim V=1$, pero $(1)1=1$, $(12)1=-1$ y $(123)1=1$, por lo que
  $\chi_{V}((1))=1$,  $\chi_{V}((12))=-1$ y  $\chi_{V}((123))=1$.

  Por último sea $V=E$, la representación estándar. Es decir,
  \begin{equation*}
    E\cong\{(x_{1},x_{2},x_{3})\in \mathbb{C}^{3}\mid x_{1}+x_{2}+x_{3}=0\}.
  \end{equation*}
  En este caso, $\dim V=2$, y una base de $E$ es
  $\beta=\{(1,-1,0),(1,0,-1)\}$. Si $g=(1)$, entonces la matriz
  asociada a $g_{V}$ en la base dada es la matriz identidad $2\times
  2$, por lo que $\chi_{V}((1))=2$.
  
  Por otro lado, si $g=(12)$, tenemos:
  \begin{eqnarray*}
    (12)(1,-1,0)&=&(-1,1,0)=-1(1,-1,0)+0(1,0,-1),\\
    (12)(1,0,-1)&=&(0,1,-1)=-1(1,-1,0)+1(1,0,-1)
  \end{eqnarray*} 
  por lo que 
  \begin{center}
  $[g_{V}]_{\beta}=
  \begin{pmatrix}
    -1 & -1 \\
    0 & 1 
  \end{pmatrix}$,
\end{center}
así que $\chi_{V}((12))=0$.

Finalmente, si $g=(123)$, tenemos:
\begin{eqnarray*}
  (123)(1,-1,0)&=&(0,1,-1)=-1(1,-1,0)+1(1,0,-1),\\
  (123)(1,0,-1)&=&(-1,1,0)=-1(1,-1,0)+0(1,0,-1)
\end{eqnarray*} 
por lo que 
\begin{center}
  $[g_{V}]_{\beta}=
  \begin{pmatrix}
    -1 & -1 \\
    1 & 0 
  \end{pmatrix}$,
\end{center}
así que $\chi_{V}((123))=-1$.

Con los cálculos realizados obtenemos la tabla de caracteres de
$S_{3}$ (ver tabla \ref{tabla-car-S3}). El primer renglón indica el
número de elementos en cada clase de conjugación, y en el segundo
renglón se coloca un elemento representativo de cada clase de
conjugación.
\begin{table}[htpb]
  \centering
  \begin{tabular}{ c| c c c}
      & 1 & 3 & 2 \\
      $S_{3}$ & (1) & (12) & (123) \\
      \hline
      $\chi_{\mathbb{C}}$ & 1 & 1 & 1 \\
      $\chi_{\mathbb{\hat C}}$ & 1 & -1 & 1 \\
      $\chi_{E}$ & 2 & 0 & -1 
    \end{tabular}
    
  \caption{Tabla de caracteres de $S_{3}$}
  \label{tabla-car-S3}
\end{table}

Las tablas de caracteres que utilizaremos en el capítulo
\ref{cha:hom-com-emp} se encuentran en~\cite{liebeck}.
\end{example}

\section{Producto interno de caracteres}
\label{producto-interno}

Mediante el producto interno de caracteres  podemos determinar si una
representación es irreducible o cuándo dos representaciones son
equivalentes. Además de otros resultados que veremos a continuación.
\begin{definition}
  % Sean $\chi$, $\psi$ los caracteres de dos representaciones de un grupo
  % $G$, entonces definimos el producto interno de $\chi$ y $\psi$ sobre
  % el espacio de funciones de $G$ en $\mathbb{C}$ como:
  Si $\chi$, $\psi$ son funciones de $G$ en $\mathbb{C}$, definimos el
  producto interno de $\chi$ y $\psi$ como:
  \begin{equation*}
    \langle\chi,\psi\rangle=\frac{1}{|G|}\sum_{g\in G}\chi(g)\overline{\psi(g)}.
  \end{equation*}
\end{definition}

\begin{theorem}{[\cite{sagan2001symmetric}, teorema 1.9.3, p.35]}.
  \label{prod-car-irr}
  Sean $V$ y $W$ representaciones irreducibles de un grupo $G$ con
  caracteres $\chi$, $\psi$ respectivamente. Entonces
  \[
  \langle\chi,\psi\rangle=\delta_{\chi\psi}:=
  \begin{cases}
    1, & \text{ si }  V\cong W \\
    0, & \text{ en otro caso. } 
  \end{cases}
  \]
\end{theorem}
\begin{corollary}{[\cite{sagan2001symmetric}, corolario 1.9.4, p.37]}.
  \label{multiplicidad}
  Sea $V$ un $G$-módulo con carácter $\chi_{V}$ y supongamos que 
  \begin{equation*}
    V\cong m_{1}V^{(1)}\oplus m_{2}V^{(2)}\oplus\cdots\oplus
    m_{k}V^{(k)},
  \end{equation*}
donde $V^{(1)},\ldots,V^{(k)}$ son $G$-módulos
irreducibles no isomorfos entre sí y $m_{i}$ denota la multiplicidad
de $V^{(i)}$. Entonces
\begin{enumerate}
\item $\chi_{V}=m_{1}\chi_{V^{(1)}}+m_{2}\chi_{V^{(2)}}+\cdots+m_{k}\chi_{V^{(k)}}$.
\item $\langle\chi_{V},\chi_{V^{(j)}}\rangle=m_{j}$ para cada $j$.
\item $\langle\chi_{V},\chi_{V}\rangle=m_{1}^{2}+m_{2}^{2}+\cdots+m_{k}^{2}$.
\item $V$ es un $G$-módulo irreducible si y solo si $\langle\chi_{V},\chi_{V}\rangle=1$.
\item Sea $W$ otro $G$-módulo con carácter $\chi_{W}$ . Entonces 
  $$V\cong W \mbox{ si y solo si } \chi_{V}(g)=\chi_{W}(g)$$
  para cada $g\in G$.
\end{enumerate}
\end{corollary}

En el siguiente ejemplo se calcula el producto interno entre caracteres.
\begin{example}
  \label{rep-triv-sig-est}
  De el ejemplo \ref{caracteres-S3}, calculamos los siguientes
  productos internos basados en los valores de la tabla \ref{tabla-car-S3}:
  \begin{align*}
    \langle\chi_{\mathbb{C}},\chi_{\mathbb{C}}\rangle&=\frac{1}{6}(1(\chi_{\mathbb{C}}(1))(\chi_{\mathbb{C}}(1))+3(\chi_{\mathbb{C}}(12))(\chi_{\mathbb{C}}(12))+2(\chi_{\mathbb{C}}(123))(\chi_{\mathbb{C}}(123)))\\
    &=\frac{1}{6}(1(1)(1)+3(1)(1)+2(1)(1))=1,  \\
    \langle\chi_{\widehat{\mathbb{C}}},\chi_{\widehat{\mathbb{C}}}\rangle&=\frac{1}{6}(1(1)(1)+3(-1)(-1)+2(1)(1))=1,  \\
    \langle\chi_{E},\chi_{E}\rangle&=\frac{1}{6}(1(2)(2)+3(0)(0)+2(-1)(-1))=1.
  \end{align*}
 Por el inciso 4 de el corolario \ref{multiplicidad}, las representaciones trivial,
 signo y estándar son representaciones irreducibles de $S_{3}$.
\end{example}

\begin{corollary}
  \label{prod-car-pos}
  Sean $V$ y $W$ $G$-módulos, entonces $\langle\chi_{V},\chi_{W}\rangle$ es un número entero no negativo.
\end{corollary}
\begin{proof}[Demostración]
  Sea $\{V^{(1)},V^{(2)},\ldots,V^{(k)}\}$ una lista completa de $G$-submódulos
  irreducibles. Escribimos a $V$ como $V\cong
  m_{1}V^{(1)}\oplus m_{2}V^{(2)}\oplus\cdots\oplus m_{k}V^{(k)}$ y a $W$ como $W\cong
  r_{1}V^{(1)}\oplus r_{2}V^{(2)}\oplus\cdots\oplus r_{k}V^{(k)}$ por el
  teorema de Maschke (ver teorema \ref{maschke}), donde $m_{i}$ y $r_{i}$ es la multiplicidad de el módulo
  irreducible $V^{(i)}$, se sigue que
  \begin{align*}
    %\langle\chi_{V},\chi_{W}\rangle=\sum m_{i}r_{j}\langle\chi_{V_{i}},\chi_{V_{j}}\rangle\\
    \langle\chi_{V},\chi_{W}\rangle&=\langle\chi_{V},r_{1}\chi_{V^{(1)}}+r_{2}\chi_{V^{(2)}}+\cdots+r_{k}\chi_{V^{(k)}}\rangle\\
    &=r_{1}\langle\chi_{V},\chi_{V^{(1)}}\rangle+r_{2}\langle\chi_{V},\chi_{V^{(2)}}\rangle+\cdots+r_{k}\langle\chi_{V},\chi_{V^{(k)}}\rangle\\
    &=r_{1}m_{1}+r_{2}m_{2}+\cdots+r_{k}m_{k},
  \end{align*}
  así que  $\langle\chi_{V},\chi_{W}\rangle$ es un entero no
  negativo. 
\end{proof}

\section{Caracteres y módulos inducidos}
\label{carac-induc}

Dados un grupo $G$ con subgrupo $H\leq G$, hay una forma de obtener
representaciones de $G$ a partir de representaciones de $H$ y
viceversa, usando operaciones de
restricción e inducción.

\begin{definition}
  Dada una función $f:A\rightarrow B$ y un subconjunto $C\subseteq A$,
  la \textbf{restricción} de $f$ al conjunto $C$ es la función
  $f_{|C}:C\rightarrow B$, dada por: 
  \begin{equation*}
    \label{eq:restriccion}
    f_{|C}(x)=f(x), \quad\mbox{para cada } x \in C.
  \end{equation*}
\end{definition}

Sea $V$ un $G$-módulo y $f:H\rightarrow G$ un homomorfismo de
grupos. Tenemos una composición:
\begin{equation*}
  H\stackrel{f}{\rightarrow}G \rightarrow GL(V),
\end{equation*}
que induce una estructura de $H$-módulo en $V$. La acción de $H$ en
$V$ está determinada por:
\begin{equation*}
  hv=f(h)v,
\end{equation*}
para $h\in H$, $v\in V$, donde en el lado derecho de la ecuación hemos
utilizado la acción ya definida de $G$ en $V$. 

\begin{definition}
  Sea $H$ un subgrupo de $G$ y $f:H\rightarrow G$ la función
  inclusión. Sea $V$ un $G$-módulo. El $H$-modulo obtenido por el
  procedimiento anterior se denota con $V\downarrow^{G}_{H}$ y se
  le llama \textbf{restricción} de $V$ de $G$ a $H$.
\end{definition}

Si el carácter de el $G$-módulo $V$ es $\chi_{V}$, entonces
denotaremos al carácter de $V\downarrow^{G}_{H}$ como
$\chi_{V}\downarrow^{G}_{H}$. Además, notemos que
$\chi_{V}\downarrow^{G}_{H}=\chi_{V\downarrow^{G}_{H}}$.

% \begin{definition}
%   Considere $H\leq G$ y $X:G\rightarrow GL(n,\mathbb{C})$ una
%   representación matricial de $G$. La \textbf{restricción} de $X$ a
%   $H$, $X\downarrow^{G}_{H}$, está dada por
%   \begin{equation*}
%     X\downarrow^{G}_{H}(h)=X(h)
%   \end{equation*}
%   para todo $h\in H$. Si el $X$ tiene carácter $\chi$, entonces denote
%   el carácter de~$X\downarrow^{G}_{H}$ por $\chi\downarrow^{G}_{H}$.
% \end{definition}

\begin{proposition}
\label{fun-cla-caracter}
Una función de clase $\theta:G\rightarrow\mathbb{C}$ es un carácter si
y solo si $\theta\neq 0$ y $\langle\theta,\chi\rangle$ es un entero no
negativo para todo carácter irreducible~$\chi$ de $G$. 
\end{proposition}
\begin{proof}[Demostración]
Si $\theta$ es un carácter, entonces $\theta(1)>0$, por lo tanto
$\theta\neq 0$. Además, por el inciso $2$ de el corolario
\ref{multiplicidad}, $\langle\theta,\chi\rangle$ es un entero no
negativo si $\chi$ es irreducible.

Supongamos ahora $\langle\theta,\chi_{j}\rangle=a_{j}$ para
$j=1,\ldots,k$, donde $\chi_{1},\chi_{2},\ldots,\chi_{k}$ forman una
colección completa de caracteres irreducibles y $a_{j}$ es un entero
no negativo. Entonces $\theta=\sum^{k}_{j=1}a_{j}\chi_{j}$, de donde
$\theta$ es el carácter del módulo~$a_{1}V^{(1)}\oplus
a_{2}V^{(2)}\oplus\cdots\oplus a_{k}V^{(k)}\neq 0$. 
\end{proof}

\begin{definition}
  Sea $H\leq G$ y $\theta:H\rightarrow \mathbb{C}$ una función. Definimos la
  función $\theta\uparrow^{G}_{H}:G\rightarrow \mathbb{C}$ como:
  \begin{equation*}
    \theta\uparrow^{G}_{H}(g)=\frac{1}{|H|}\sum_{\substack{x\in
        G\\x^{-1}gx\in H}}\theta(x^{-1}gx).
  \end{equation*}
  Decimos que la función $\theta\uparrow^{G}_{H}$ es \textbf{inducida}
  por la función $\theta$.
\end{definition}
Observemos que si $H\leq G$, y $x,g\in G$ son tales que $x^{-1}gx\in
H$, entonces para todo $h\in H$ se tiene que
$(xh)^{-1}g(xh)=h^{-1}(x^{-1}gx)h\in H$. Además, si $\theta$ es una función de
clase, $\theta(h^{-1}(x^{-1}gx)h)=\theta(x^{-1}gx)$. Sea $T$ una \textbf{transversal izquierda} de $H$ en $G$, es decir, una
colección de representantes de las clases laterales izquierdas de $H$,
de modo que si $T=\{t_{1},\ldots,t_{s}\}$ entonces
$G=t_{1}H\cup\cdots\cup t_{s}H$. Por lo tanto, si
$\theta:H\rightarrow\mathbb{C}$ es una función de clase, podemos
calcular a $\theta\uparrow^{G}_{H}$ por medio de la fórmula:
\begin{equation}
  \label{fun-ind-trans}
  \theta\uparrow^{G}_{H}(g)=\sum_{\substack{x\in T,\\x^{-1}gx\in H}}\theta(x^{-1}gx).
\end{equation}
Por otro lado, notemos:
\begin{equation*}
  \theta\uparrow^{G}_{H}(1)=\sum_{\substack{x\in T,\\x^{-1}1x\in
      H}}\theta(x^{-1}1x)=\sum_{\substack{x\in T,\\1\in
      H}}\theta(1)=|T|\theta(1)=[G:H]\theta(1).\\
\end{equation*}
donde $[G:H]$ es el número de clases laterales de $H$ en $G$, llamado
el \emph{índice} de $H$ en $G$.
\begin{proposition}
  Si $H\leq G$ y $\theta:H\rightarrow \mathbb{C}$ es una función de
  clase, entonces $\theta\uparrow^{G}_{H}:G\rightarrow \mathbb{C}$ es
  una función de clase. 
\end{proposition}
\begin{proof}[Demostración]
  Para $g,y\in G$ tenemos que
  \begin{align*}
    \theta\uparrow^{G}_{H}(ygy^{-1})&=\frac{1}{|H|}\sum_{\substack{x\in
        G\\x^{-1}(ygy^{-1})x\in H}}\theta(x^{-1}ygy^{-1}x)\\
    &=\frac{1}{|H|}\sum_{\substack{x\in
        G\\(y^{-1}x)^{-1}g(y^{-1}x)\in H}}\theta((y^{-1}x)^{-1}g(y^{-1}x))\\
    &=\frac{1}{|H|}\sum_{\substack{yu\in
        G\\u^{-1}gu\in H}}\theta(u^{-1}gu)\\
     &=\frac{1}{|H|}\sum_{\substack{u\in
        G\\u^{-1}gu\in H}}\theta(u^{-1}gu)\\
    &=\theta\uparrow^{G}_{H}(g)
  \end{align*}
en donde se hizo el cambio de variable $u=y^{-1}x$ (esto es $x=yu$), y notamos
que $yu\in G$ si y solo si $u\in G$.
\end{proof}

La suma de dos funciones de clase es una función de clase y el
producto por escalar por una función de clase es una función de clase
y la función constante cero $0:G\rightarrow \mathbb{C}$ está en el conjunto de todas las funciones de clase en $G$
que anteriormente denotamos como $R(G)$, se sigue que $R(G)$ es un $\mathbb{C}$-subespacio vectorial
del espacio de todas las funciones  de $G$ en $\mathbb{C}$ y por lo
tanto, podemos considerar el producto de funciones de clase, es decir,
si $\psi$, $\theta$ son funciones de clase, podemos definir
$\langle\psi,\theta\rangle$ como $\frac{1}{|G|}\sum_{g\in
  G}\psi(g)\overline{\theta(g)}$, el cual se puede demostrar que es un producto
interno en el espacio de funciones de clase.
\begin{theorem}[Reciprocidad de Frobenius]
  \label{frobenius}
  Sea $H\leq G$ y $\theta:H\rightarrow\mathbb{C}$,
  $\phi:G\rightarrow \mathbb{C}$ funciones de
  clase. Entonces
  \begin{equation*}
    \langle\theta\uparrow^{G}_{H},\phi\rangle=\langle\theta,\phi\downarrow^{G}_{H}\rangle.
  \end{equation*}
\end{theorem}
\begin{proof}[Demostración]
  Tenemos que 
  \begin{align*}
    \langle\theta\uparrow^{G}_{H},\phi\rangle&=\frac{1}{|G|}\sum_{g\in
      G}\theta\uparrow^{G}_{H}(g)\overline{\phi(g)}\\
    &=\frac{1}{|G|}\sum_{g\in G}\Bigg(\frac{1}{|H|}\sum_{\substack{x\in
        G\\x^{-1}gx\in H}}\theta(x^{-1}gx)\Bigg)\overline{\phi(g)}\\
    &=\frac{1}{|G|}\frac{1}{|H|}\sum_{\{(g,x)\mid g,x\in G,x^{-1}gx\in
      H\}}\theta(x^{-1}gx)\overline{\phi(g)},\\
    \intertext{hacemos el cambio $u=x^{-1}gx$, entonces $g=xux^{-1}$,}
    &=\frac{1}{|G|}\frac{1}{|H|}\sum_{\{(xux^{-1},x)\mid xux^{-1},x\in
      G,u\in H\}}\theta(u)\overline{\phi(xux^{-1})}\\
    &=\frac{1}{|G|}\frac{1}{|H|}\sum_{\{(xux^{-1},x)\mid xux^{-1},x\in G,u\in H\}}\theta(u)\overline{\phi(u)}\\
    &=\frac{1}{|G|}\frac{1}{|H|}|G|\sum_{u\in
      H}\theta(u)\overline{\phi(u)}\\
    \intertext{pues $u\in H$, $xux^{-1}\in G$ para todo $x\in G$,
      además $\phi(xux^{-1})=\phi(u)$}
    &=\langle\theta,\phi\downarrow^{G}_{H}\rangle.
 \end{align*} 
\end{proof}

\begin{corollary}
  \label{car-inducido}
  Si $H\leq G$ y $\theta:H\rightarrow \mathbb{C}$ un carácter
  entonces $\theta\uparrow^{G}_{H}:~G\rightarrow~\mathbb{C}$ es un carácter.
\end{corollary}
\begin{proof}[Demostración]
  Ya sabemos que $\theta\uparrow^{G}_{H}$ es una función de
  clase. Aplicamos el criterio de la proposición
  \ref{fun-cla-caracter}. Tenemos que
  $\theta\uparrow^{G}_{H}(1)=[G:H]\theta(1)>0$ y si $\chi$ es un
  carácter irreducible de $G$, entonces
  \begin{equation*}
    \langle\theta\uparrow^{G}_{H},\chi\rangle=\langle\theta,\chi\downarrow^{G}_{H}\rangle,
  \end{equation*}
  el cual es un entero no negativo, pues es el producto interno de dos
  caracteres (ver corolario \ref{prod-car-pos}). El resultado se sigue entonces de la proposición
  \ref{fun-cla-caracter}.
\end{proof}
Supondremos en lo que resta de esta sección que $G$ es un grupo, $H$ un subgrupo de
$G$  y $T=\{t_{1},t_{2},\ldots,t_{s}\}$ una transversal izquierda de
$H$ en $G$ fija, donde $t_{1}=1$.

Si $W$ es un $H$-módulo, el corolario \ref{car-inducido} implica que
$(\chi_{W})\uparrow^{G}_{H}$ es un carácter. Queremos encontrar un
$G$-módulo $V$ tal que $\chi_{V}=(\chi_{W})\uparrow^{G}_{H}$.
\begin{definition}
  Sea $V$ un $G$-módulo, $W$ un subespacio de $V$ tal que $\{g\in
  G\mid gW=W\}=H$ (de modo que $W$ es un $H$-módulo). Decimos que el
  $G$-módulo $V$ es \textbf{inducido} por el $H$-módulo $W$ si
  $V=\bigoplus^{s}_{i=1}t_{i}W$ como espacio vectorial.
\end{definition}
Notemos que la definición no depende de la transversal escogida, pues
si~$t=t^{'}h$ con $h\in H$, entonces $tW=t^{'}hW=t^{'}W$.
\begin{proposition}
  \label{car-ind-W-V}
  Si el $G$-módulo $V$ es inducido por el $H$-módulo $W$, entonces
  $\chi_{V}=(\chi_{W})\uparrow^{G}_{H}$.
\end{proposition}
\begin{proof}[Demostración]
  % Si $\beta=\{w_{1},w_{2},\ldots,w_{r}\}$ es base de $W$, entonces $\gamma=\bigcup^{s}_{j=1}t_{j}\beta$ es
  % base de $V$, por lo que dado $g\in G$ calcularemos $\chi_{V}(g)$
  % usando esa base. Tenemos entonces que:
  % \begin{equation*}
  %   \chi_{V}(g)=\tr([g]_{\gamma})=\sum_{\{j\mid gt_{j}W\leq t_{j}W\}}\tr(g\mid t_{j}W),
  % \end{equation*}
  % y $gt_{j}W\leq t_{j}W$ si y solo si $t^{-1}_{j}gt_{j}W\leq W$, si y
  % solo si $t^{-1}_{j}gt_{j}\in H$. Es inmediato que
  % $[g]_{t_{j}\beta}=[t^{-1}_{j}gt_{j}]_{\beta}$, por lo que $\tr(g\mid
  % t_{j}W)=\chi_{V}(t^{-1}_{j}gt_{j})$, y por la ecuación \ref{fun-ind-trans}, queda
  % demostrado.
  Si $\beta=\{w_{1},w_{2},\ldots,w_{r}\}$ es base de $W$, entonces $\gamma=\bigcup^{s}_{j=1}t_{j}\beta$ es
  base de $V$. Dado $g\in G$, calcularemos $\chi_{V}(g)$
  usando esa base. Así,
  \begin{equation}
    \label{car-base-tjb}    
    \chi_{V}(g)=\tr([g]_{\gamma})=\sum_{\{j\mid gt_{j}W\leq t_{j}W\}}\tr([g]_{t_{j}\beta})
  \end{equation}
  Veamos que $[g]_{t_{j}\beta}=[t^{-1}_{j}gt_{j}]_{\beta}$, si
  $t^{-1}_{j}gt_{j}\in H$.  Tomemos $w_{k}\in \beta$, escribimos a~
  $gt_{j}w_{k}$ como:
  \begin{equation*}
    gt_{j}w_{k}=\lambda_{1}t_{j}w_{1}+\cdots+\lambda_{r}t_{j}w_{r}
  \end{equation*}
  de modo que,
  \begin{equation*}
    t^{-1}_{j}gt_{j}w_{k}=\lambda_{1}w_{1}+\cdots+\lambda_{r}w_{r}\\
  \end{equation*}
  por tanto,
   \begin{center}
    $[g(t_{j}w_{k})]_{t_{j}\beta}=\begin{pmatrix}
      \lambda_{1}  \\
      \vdots  \\
      \lambda_{r}  
    \end{pmatrix}=[t^{-1}_{j}gt_{j}(w_{k})]_{\beta}$.
  \end{center}
  De donde
  \begin{equation*}
    [g]_{t_{j}\beta}=[t^{-1}_{j}gt_{j}]_{\beta}.
  \end{equation*}
  Por lo anterior, tenemos que
  \begin{equation*}
    \tr([g]_{t_{j}\beta})=\tr([t^{-1}_{j}gt_{j}]_{\beta})=\chi_{W}(t^{-1}_{j}gt_{j}).\\
  \end{equation*}
  Continuando la ecuación \ref{car-base-tjb}, 
  \begin{align*}
    \chi_{V}(g)&=\tr([g]_{\gamma})=\sum_{\{j\mid gt_{j}W\leq t_{j}W\}}\tr([g]_{t_{j}\beta})\\
    %\intertext{se sigue de la ecuación \ref{fun-ind-trans}, lo siguiente}
    &=\sum_{\substack{t_{j}\in T,\\t^{-1}_{j}gt_{j}\in H}}\chi_{W}(t^{-1}_{j}gt_{j})=(\chi_{W})\uparrow^{G}_{H}(g).
  \end{align*}
\end{proof}

\section{Gráficas}
\label{graficas}

\begin{definition}
  Una \textbf{gráfica} $G$ consiste de un conjunto finito no vacío $V=V(G)$
  de $p$ \textbf{vértices} junto con un conjunto $E=E(G)$ de $q$ pares no
  ordenados de vértices distintos pertenecientes al conjunto $V(G)$. Cada par $e=\{u,v\}$ de vértices
  en $E$ es una \textbf{arista} de $G$, y se dice que $e$ une a $u$ y
  $v$. Escribimos a $e=uv$ y decimos que $u$ y $v$ son \textbf{adyacentes};
  %un vértice $u$ y una arista $e$ son \emph{incidentes} si $u\in e$. 
  Dos aristas $e$ y $f$ son \textbf{ajenas} si $e\cap f=\emptyset$.
  % Una gráfica con $p$ vértices y $q$
  % aristas es llamada una $(p,q)$ gráfica.
\end{definition}

Una manera natural de ver una gráfica es poner puntos para los
vértices y una línea entre dos puntos si los vértices correspondientes
son adyacentes.
\begin{definition}
  Decimos que $G$ es una gráfica \textbf{completa} si para
  todos~$u,v\in V(G)$ con $u\neq v$ se tiene que $uv\in E(G)$. Una gráfica completa con
  $n$ vértices se denota como $K_{n}$.
\end{definition}
\begin{example}
  Gráfica completa de 5 vértices.
  \bigskip

  \begin{minipage}{1.0\linewidth}
    \centering
    \begin{tikzpicture}[rotate=90,scale=.7]
      \GraphInit[vstyle=Classic] 
      \SetUpVertex[MinSize=1pt]
      \SetVertexNoLabel 
      \grComplete[RA=2.3]{5}
    \end{tikzpicture}
  
    $K_{5}$
  \end{minipage}
  \label{fig:K5}
\end{example}

\begin{definition}
  \label{graf-emparejamientos}
  Consideremos la gráfica completa de $n$ vértices $K_{n}$ cuyos
  vértices están etiquetados como $1,2,\ldots,n$. Además
  $\overline{ij}$ denotará la arista que une al vértice $i$ con el
  vértice $j$ (donde $\overline{ij}=\overline{ji}$).
  
  Llamaremos \textbf{gráfica de emparejamientos} de orden $n$ a la
  gráfica $G_{n}$ tal que:
  
  \begin{enumerate}
  \item Su conjunto de vértices $V(G_{n})$ consta de las aristas de la gráfica
    $K_{n}$. 
  \item Si $v_{i}=\overline{pq}$ y $v_{j}=\overline{rs}$ están en
    $V(G_{n})$, $\{v_{i},v_{j}\}$ es una arista en $E(G_{n})$ si $v_{i}$
    y $v_{j}$ son ajenas.
  \end{enumerate}
\end{definition} 

\begin{example}
  A continuación se muestran ejemplos de gráficas de emparejamientos,
  obtenidas a partir de su respectiva gráfica completa.
  \begin{center}
    \begin{minipage}{0.26\linewidth}
      \centering
      \begin{tikzpicture}[x=0.8 cm,y=0.8 cm]
        \draw[help lines] (-2,0);% grid (0,2);
        \GraphInit[vstyle=Classic] \SetUpVertex[MinSize=1pt]
        \Vertex[x=-2,y=0,Math,LabelOut,Lpos=180]{1}
        \Vertex[x=0,y=0,Math]{2}
        \Vertex[x=-1,y=2,Math,LabelOut,Lpos=90]{3} \Edge(1)(2)
        \Edge(1)(3) \Edge(2)(3)
      \end{tikzpicture}
  
      $K_{3}$
    \end{minipage}
    \begin{minipage}{0.26\linewidth}
      \centering
      \begin{tikzpicture}[x=0.8 cm,y=0.8 cm]
        \draw[help lines] (-2,0);% grid (0,2);
        \GraphInit[vstyle=Classic] \SetUpVertex[MinSize=1pt]
        \Vertex[x=-2,y=0,Math,LabelOut,Lpos=90,L=\overline{12}]{12}
        \Vertex[x=-1,y=0,Math,LabelOut,Lpos=90,L=\overline{13}]{13}
        \Vertex[x=0,y=0,Math,LabelOut,Lpos=90,L=\overline{23}]{23}
      \end{tikzpicture}
  
      $G_{3}$
    \end{minipage}
  \end{center}

  \begin{center}
    \begin{minipage}{0.3\linewidth}
      \centering
      \begin{tikzpicture}[x=0.8 cm,y=0.8 cm]
        \draw[help lines] (-2,0);% grid (0,2);
        \GraphInit[vstyle=Classic] \SetUpVertex[MinSize=1pt]
        \Vertex[x=-2,y=0,Math,LabelOut,Lpos=180]{2}
        \Vertex[x=0,y=0,Math]{3}
        \Vertex[x=-2,y=2,Math,LabelOut,Lpos=180]{1}
        \Vertex[x=0,y=2,Math]{4} 
        \Edge(1)(2) \Edge(1)(3) \Edge(1)(4)
        \Edge(2)(3) \Edge(2)(4) \Edge(3)(4)
      \end{tikzpicture}
  
      $K_{4}$
    \end{minipage}
    \begin{minipage}{0.26\linewidth}
      \centering
      \begin{tikzpicture}[x=0.8 cm,y=0.8 cm]
        \draw[help lines] (-2,0);% grid (0,2);
        \GraphInit[vstyle=Classic] \SetUpVertex[MinSize=1pt]
        \Vertex[x=-2,y=2,Math,LabelOut,Lpos=90,L=\overline{12}]{12}
        \Vertex[x=-2,y=0,Math,LabelOut,Lpos=-90,L=\overline{34}]{34}
        \Vertex[x=-1,y=0,Math,LabelOut,Lpos=-90,L=\overline{24}]{24}
        \Vertex[x=-1,y=2,Math,LabelOut,Lpos=90,L=\overline{13}]{13}
        \Vertex[x=0,y=2,Math,LabelOut,Lpos=90,L=\overline{14}]{14}
        \Vertex[x=0,y=0,Math,LabelOut,Lpos=-90,L=\overline{23}]{23}
        \Edge(12)(34) \Edge(24)(13) \Edge(14)(23)
      \end{tikzpicture}
  
      $G_{4}$
    \end{minipage}
  \end{center}
\end{example}
\begin{definition}
  Sea $S$ un conjunto y $F=\{S_{1},\cdots,S_{p}\}$ una familia de
  subconjuntos distintos no vacíos de $S$ cuya unión es $S$. La
  \textbf{gráfica de intersección} de $F$ es denotada por $\Omega(F)$
  y definida por $V(\Omega(F))=F$, con~$S_{i}$ y $S_{j}$ adyacentes
  siempre que $i\neq j$ y $S_{i}\cap S_{j}\neq\emptyset$.
\end{definition}

\begin{definition}
  Una gráfica $H$ se llama \textbf{subgráfica} de $G$ si tiene todos
  sus vértices y aristas en $G$, es decir, $V(H)\subseteq V (G)$ y
  $E(H)\subseteq E(G)$. 
\end{definition}

\begin{definition}
  Un \textbf{clan} de una gráfica es una subgráfica completa
  maximal, es decir, no está contenida en ninguna otra subgráfica completa.
\end{definition}

\begin{definition}
  La \textbf{gráfica de clanes} de una gráfica dada $G$ es la gráfica
  de intersección de la familia de clanes de $G$. Denotemos a la
  gráfica de clanes de $G$ como $K(G)$. 
\end{definition}
\begin{example}
  La gráfica $K_{4}$ de la figura \ref{KG} es la gráfica de clanes de $G$ de la figura \ref{G}.

  \begin{center}
    \begin{figure}[h]
      \begin{minipage}[h]{0.45\linewidth}
        \centering
        \begin{tikzpicture}[scale=.8]
          \GraphInit[vstyle=Classic] \SetUpVertex[MinSize=1pt]
          \SetVertexNoLabel
          \grTriangularGrid[prefix=G,Math,RA=1.5]{3}%
        \end{tikzpicture}

        \caption{$G$}
        \label{G}
      \end{minipage}
      \begin{minipage}[h]{0.45\linewidth}
        \centering
        \begin{tikzpicture}[scale=.8]
          \GraphInit[vstyle=Classic] \SetUpVertex[MinSize=1pt]
          \SetVertexNoLabel \grTetrahedral[RA=1.9]
        \end{tikzpicture}
    
        \caption{$K(G)$}
        \label{KG}
      \end{minipage}
    \end{figure}
  \end{center}
\end{example}

\begin{definition}
  La \textbf{gráfica bipartita clánica} de $G$ es definida como la gráfica
  $BK(G)$ con $V(BK(G))=V(G)\cup V(K(G))$, donde $x\in V(G)$,
  $c\in (K(G))$ son adyacentes si $x\in c$.
\end{definition}
\begin{example}
  \label{BKG}
  Sea $G$ la gráfica de la figura \ref{G}. La gráfica bipartita
  clánica de $G$ se muestra enseguida, donde los vértices en color
  gris corresponden a los clanes de $G$ y los vértices restantes son
  los vértices de $G$.
  \bigskip

  \begin{minipage}{1.0\linewidth}
    \centering
    \begin{tikzpicture}[scale=.45]
      \SetVertexNoLabel 
      \GraphInit[vstyle=Classic]
      \SetUpVertex[MinSize=1pt] 
      \grStar[RA=2,Math,rotation=270]{4}
      \grEmptyCycle[RA=2,Math,rotation=90,prefix=b]{3}
      \grEmptyCycle[RA=4,Math,rotation=90,prefix=c]{3}
      \EdgeFromOneToSeq{b}{a}{0}{1}{2} \EdgeFromOneToSeq{a}{b}{2}{1}{1}
      \EdgeFromOneToSeq{a}{b}{0}{1}{1} \EdgeFromOneToSeq{b}{a}{2}{0}{1}
      \EdgeIdentity{b}{c}{3} 
      \AddVertexColor{gray}{a3,b0,b1,b2}
    \end{tikzpicture}
    
    $BK(G)$
  \end{minipage}
\end{example}

\chapter[Representaciones del grupo simétrico]{Tableros de Young y representaciones del grupo simétrico}
\label{repr-grupo-simetrico}

En este capítulo, estudiamos la relación entre representaciones del
grupo simétrico $S_{n}$ y objetos combinatorios llamados tableros de
Young. Definiremos los tableros de Young en la sección \ref{tablero},
pero por
ahora es suficiente decir que es el llenado de cierta configuración de cajas
con elementos del conjunto $\{1,2,\ldots,n\}$. Un ejemplo se muestra a
continuación:

\begin{center}
  \ytableausetup{mathmode,notabloids,boxsize=1.3em}
  \begin{ytableau}
    1 & 2 & 4 \\
    3 & 5 & 6 \\
    7 & 8 \\
    9
  \end{ytableau} 
\end{center}

Hay una elegante descripción de representaciones irreducibles de
$S_{n}$ mediante los tableros de Young. Recordemos que hay tres representaciones irreducibles de
$S_{3}$ (ver ejemplo \ref{rep-triv-sig-est}), resulta que pueden ser descritas usando el conjunto de diagramas de
Young con tres cajas. La correspondencia es ilustrada abajo.

\begin{center}
  \ytableausetup{mathmode,notabloids,boxsize=1.2em}
  \begin{minipage}[h]{0.3\linewidth}
    \centering
    \ydiagram{3} 

    representación trivial
  \end{minipage}
  \begin{minipage}[h]{0.3\linewidth}
    \centering
    \ydiagram{1,1,1}

    representación signo
  \end{minipage}
  \begin{minipage}[h]{0.3\linewidth}
    \centering
    \ydiagram{2,1}

    representación estándar
  \end{minipage}
\end{center}

En general, las representaciones irreducibles de $S_{n}$ pueden ser
descritas usando diagramas de Young de $n$ cajas. Además, podemos dar
una base para cada representación irreducible usando tableros
Young estándar, los cuales consisten en numerar las cajas de un diagrama de
Young con $1,2,\ldots,n$ tal que los renglones y las columnas sean
crecientes. Por ejemplo, las base de la representación estándar de
$S_{3}$ corresponde a los siguientes dos tableros estándar:

\begin{center}
  \ytableausetup{mathmode,notabloids,boxsize=1.2em}
  \begin{ytableau}
    1 & 2\\
    3
  \end{ytableau} \qquad
  \begin{ytableau}
    1 & 3\\
    2
  \end{ytableau}
\end{center}

En este capítulo, describimos la relación entre los tableros de Young
y representaciones de $S_{n}$. Las pruebas son omitidas, pero pueden
encontrarse en el segundo capítulo de \cite{sagan2001symmetric}.

En la sección \ref{tablero}, introducimos diagramas de Young y tableros
de Young. En la sección \ref{modulo-permutacion}, introducimos los
tabloides y los usamos para construir una representación de $S_{n}$
conocida como módulo de permutación $M^{\lambda}$. Sin embargo, los
módulos de permutación son generalmente reducibles. En la sección
\ref{modulo-specht} construimos representaciones irreducibles de
$S_{n}$ conocidos como módulos de Specht $S^{\lambda}$, los cuales son
subespacios del correspondiente $M^{\lambda}$. Los módulos de
Specht $S^{\lambda}$ corresponden biyectivamente a los diagramas de
Young de forma $\lambda$ y forman una lista completa de
representaciones irreducibles de $S_{n}$.

\section{Tablero de Young}
\label{tablero}

Primero establecemos algunas definiciones y notaciones
considerando particiones y diagramas de Young.

\begin{definition}
  Una \textbf{partición} de un entero positivo $n$ es una secuencia de
  enteros positivos
  $\lambda=(\lambda_{1},\lambda_{2},\ldots,\lambda_{l})$ que satisface
  $\lambda_{1}\geq\lambda_{2}\geq\cdots\geq\lambda_{l}>0$ y
  $n=\lambda_{1}+\lambda_{2}+\cdots+\lambda_{l}$. Escribimos
  $\lambda\vdash n$ para denotar que $\lambda$ es una partición de $n$.
\end{definition}

Por ejemplo, el número $4$ tiene cinco particiones: $(4)$, $(3,1)$,
$(2,2)$, $(2,1,1)$, $(1,1,1,1)$. Podemos también representar
particiones usando diagramas de Young como sigue.

\begin{definition}
  Si $\lambda=(\lambda_{1},\lambda_{2},\ldots,\lambda_{l})$ es una
  partición de $n$, entonces el \textbf{diagrama de Young} de $\lambda$
  consiste de $n$ cajas colocadas en $l$ renglones alineados a la
  izquierda, donde el $i$-ésimo
  renglón tiene  $\lambda_{i}$ cajas.
\end{definition}

Por ejemplo, los diagramas de Young correspondientes a las particiones
del $4$ son: 

\begin{center}
  %\ytableausetup{smalltableaux}
  \ytableausetup{mathmode, boxsize=1em}
  \begin{minipage}[h]{0.2\linewidth}
    \centering \ydiagram{4}\bigskip

    (4)
  \end{minipage}
  \begin{minipage}[h]{0.15\linewidth}
    \centering \ydiagram{3,1}\medskip

    (3,1)
  \end{minipage}
  \begin{minipage}[h]{0.15\linewidth}
    \centering \ydiagram{2,2}\medskip

    (2,2)
  \end{minipage}
  \begin{minipage}[h]{0.15\linewidth}
    \centering \ydiagram{2,1,1}\smallskip

    (2,1,1)
  \end{minipage}
  \begin{minipage}[h]{0.15\linewidth}
    \centering \ydiagram{1,1,1,1}\smallskip

    (1,1,1,1)
  \end{minipage}
\end{center}

Existe una correspondencia biyectiva entre particiones y diagramas
de Young, y nosotros usaremos los dos términos indistintamente.

Un tablero Young es obtenido llenando las cajas de un diagrama de
Young con números naturales.

\begin{definition}
  Sea $\lambda\vdash n$. Un \textbf{tablero $\boldsymbol{t}$ (de Young) de forma
    $\boldsymbol{\lambda}$}, se obtiene llenando las cajas de un
  diagrama de Young para $\lambda$ con $1,2,\ldots,n$, los cuales
  aparecen exactamente una vez. En este caso, decimos que $t$ es un
  $\lambda$-tablero.
\end{definition}

Por ejemplo, aquí están todos los tableros correspondientes a la
partición~$(2,1)$:

\begin{center}
  \ytableausetup{mathmode,notabloids,boxsize=1.2em}
  \begin{ytableau}
    1 & 2\\
    3
  \end{ytableau} \quad
  \begin{ytableau}
    2 & 1\\
    3
  \end{ytableau}\quad
  \begin{ytableau}
    1 & 3\\
    2
  \end{ytableau}\quad
  \begin{ytableau}
    3 & 1\\
    2
  \end{ytableau}\quad
  \begin{ytableau}
    2 & 3\\
    1
  \end{ytableau}\quad
  \begin{ytableau}
    3 & 2\\
    1
  \end{ytableau}\quad.
\end{center}

\begin{definition}
  Un \textbf{tablero (de Young) estándar} es un tablero de Young cuyas
  entradas son crecientes en cada renglón y columna.
\end{definition}
Los únicos tableros estándar para $(2,1)$ son:

\begin{center}
  \ytableausetup{mathmode,notabloids,boxsize=1.2em}
  \begin{ytableau}
    1 & 2\\
    3
  \end{ytableau}\quad y \quad
  \begin{ytableau}
    1 & 3\\
    2
  \end{ytableau}\quad .
\end{center}
Otro ejemplo de un tablero estándar es:
\begin{center}
  \ytableausetup{mathmode,notabloids,boxsize=1.2em}
  \begin{ytableau}
    1 & 2 & 4\\
    3 & 5 & 6\\
    7 & 8\\
    9
  \end{ytableau}
\end{center}

Antes de seguir adelante, recordaremos algunos hechos básicos
sobre permutaciones. Cada permutación $\pi \in S_{n}$ tiene una
descomposición en ciclos disjuntos. Por ejemplo, $(123)(45)$ denota la
permutación que envía $1\rightarrow 2 \rightarrow 3 \rightarrow 1$  e
intercambia al $4$ y $5$ (si $n>5$, entonces por convención los otros
elementos permanecen fijos por $\pi$). La \textbf{estructura cíclica} de $\pi\in S_{n}$ es la
partición de $n$ formada por las longitudes de los ciclos en la
descomposición. Así, $(123)(45)\in S_{5}$ tiene la estructura
cíclica (3,2). Dos elementos de $S_{n}$ son conjugados si y solo si
tienen la misma estructura cíclica. % La forma mas fácil de ver esto es
% considerar la conjugación simplemente como una reetiquetación de los
% elementos cuando la permutación es escrita en notación cíclica. En
% efecto, si $$\pi=(a_{1},a_{2},\ldots,a_{k})(b_{1},b_{2},\ldots,b_{l})\cdots,$$
% y $\sigma$ manda $x$ a $x^{'}$, entonces
% $$\sigma\pi\sigma^{-1}=(a^{'}_{1},a^{'}_{2},\ldots,a^{'}_{k})(b^{'}_{1},b^{'}_{2},\ldots,b^{'}_{l})\cdots.$$
Esto significa que las clases de conjugación de $S_{n}$ están
caracterizadas por la estructura cíclica, y en consecuencia
corresponden a las particiones de $n$, los cuales son equivalentes a
los diagramas de Young de tamaño $n$. Recordemos de el teorema
\ref{no-rep-irr-no-cla-conj}, que el número de representaciones irreducibles de un
grupo finito es igual al número de sus clases de conjugación. Así,
nuestro objetivo en las próximas dos secciones es construir una
representación irreducible de $S_{n}$ correspondiente a cada diagrama
de Young.

\section{Tabloides y módulo de permutación $M^{\lambda}$}
\label{modulo-permutacion}

En esta sección, construimos representaciones de $S_{n}$ usando clases
de equivalencia de los tableros, conocidas como tabloides.

\begin{definition}
  Dos $\lambda$-tableros $t_{1}$ y $t_{2}$ son \textbf{equivalentes
    por renglones}, denotado como $t_{1}\sim t_{2}$, si los
  correspondientes renglones de los dos tableros contienen los mismos
  elementos. Un \textbf{tabloide} de forma $\lambda$, o
  $\lambda$-tabloide es la clase de equivalencia
  $$\{t\}=\{t_{1}\mid t_{1}\sim t\}.$$
  El tabloide $\{t\}$ es dibujado como el tablero $t$ sin las barras
  verticales.
\end{definition}
Por ejemplo, si
\begin{center}
  \ytableausetup{mathmode,notabloids,boxsize=1.2em}
  $t=$
  \begin{ytableau}
    1 & 2  \\
    3
  \end{ytableau}
\end{center}
entonces $\{t\}$ es el tabloide dibujado como

\begin{center}
  \ytableausetup{mathmode,tabloids,boxsize=1.2em}
  \ytableaushort{12,3}
\end{center}

el cual representa la clase de equivalencia que contiene a los
siguientes tableros:
\begin{center}  
  \ytableausetup{mathmode,notabloids,boxsize=1.2em}
  \begin{ytableau}
    1 & 2  \\
    3
  \end{ytableau}\qquad
  \begin{ytableau}
    2 & 1  \\
    3
  \end{ytableau}\quad .
\end{center}

El orden de las entradas entre cada renglón es irrelevante. Por
ejemplo:

\begin{center}
  \ytableausetup{mathmode,tabloids,boxsize=1.2em}
  \ytableaushort{145,23}\quad$=$\quad
  \ytableaushort{451,32}\quad$\neq$\quad
  \ytableaushort{521,34}\quad .
\end{center}

Pretendemos definir una representación de $S_{n}$ en un espacio
vectorial cuya base es exactamente el conjunto de tabloides de una
forma dada. Necesitamos una forma para que los elementos de $S_{n}$
actúen en los tabloides, lo cual, se logra dejando que la permutaciones intercambien las entradas del
tabloide.

Por ejemplo, el ciclo $(123)\in S_{3}$ actúa en el tabloide
reemplazando el 1 por el 2, el 2 por el 3 y el 3 por el 1, como se
muestra abajo:
\begin{center}
  \ytableausetup{mathmode,tabloids,boxsize=1.2em}
  $(123)$
  \ytableaushort{12,3}\quad$=$\quad
  \ytableaushort{23,1}\quad .
\end{center}

Debemos comprobar que esta acción está bien definida, esto es, si $t_{1}$
y $t_{2}$ son equivalentes por renglones, es decir,
$\{t_{1}\}=\{t_{2}\}$, entonces el resultado de la permutación debe
ser el mismo, esto es, $\pi\{t_{1}\}=\pi\{t_{2}\}$. Esto es claro,
pues $\pi$ simplemente da la instrucción de mover algún número de un
renglón a otro.

Ahora, que hemos definido una forma para que $S_{n}$ actúe en los tabloides,
podemos definir una representación de $S_{n}$. Recordemos que una
representación de un grupo $G$ en un espacio vectorial $V$ sobre el
campo de los números complejos es equivalente a extender a $V$ a un $G$-módulo,
por lo que usaremos el término módulo para describir representaciones.

\begin{definition}
  Sea $\lambda\vdash n$. $M^{\lambda}$ denota el espacio vectorial
  cuya base es el conjunto de $\lambda$-tabloides. Entonces
  $M^{\lambda}$ es una representación de $S_{n}$ conocida como \textbf{módulo
  de permutación correspondiente a $\boldsymbol{\lambda}$.}
\end{definition}

Mostraremos algunos ejemplos de los módulos de
permutación $M^{\lambda}$. Los módulos obtenidos a partir de los siguientes diagramas de
Young son de hecho representaciones conocidas.

\begin{center}
  \ytableausetup{mathmode,notabloids,boxsize=1.2em}
  \ydiagram{4} \qquad
  \ydiagram{1,1,1,1}\qquad
  \ydiagram{3,1}
\end{center}

\begin{example}
  Consideremos $\lambda=(n)$. Así que $M^{\lambda}$ es el espacio
  vectorial generado por un solo tabloide 
  \begin{center}
    \ytableausetup{mathmode,tabloids}    
    \ytableaushort{1 2\cdots n} 
  \end{center}
Donde el tabloide está fijo por $S_{n}$, por lo que $M^{\lambda}$ es
una representación trivial de dimensión uno.
\end{example}

\begin{example}
  Consideremos $\lambda=(1^{n})=(1,1,\ldots,1)$. Entonces un $\lambda$-tabloide es
  simplemente una permutación $\{1,2,\ldots,n\}$ en $n$ renglones y $S_{n}$ actúa en los
  tabloides haciendo actuar la correspondiente permutación. Se sigue que $M^{(1,1,\ldots,1)}$ es
  isomorfo a la representación regular de $S_{n}$.
\end{example}

\begin{example}
  \label{tabloides-ejemplo}
  Consideremos
  $\lambda=(n-1,1)$. Sea $\{t_{i}\}$
  un $\lambda$-tabloide con $i$ en
  el segundo renglón. Entonces
  $M^{\lambda}$ tiene base
  $\{t_{1}\},\{t_{2}\},\ldots,\{t_{n}\}$. También,
  note que la acción de $\pi\in
  S_{n}$ envía $t_{i}$ a $t_{\pi(i)}$. Así,
  $M^{(n-1,1)}$ es isomorfo a la
  representación $\mathbb{C}\{1,2,\ldots,n\}$. 

  Por ejemplo, si $n=4$, la
  representación $M^{(3,1)}$ tiene
  la siguiente base:
  \begin{center}
    \ytableausetup{mathmode,tabloids,boxsize=1.2em}    
    $t_{1}=$\ytableaushort{234,1}\qquad 
    $t_{2}=$\ytableaushort{134,2} \qquad
    $t_{3}=$\ytableaushort{124,3} \qquad
    $t_{4}=$\ytableaushort{123,4} \qquad
  \end{center}
\end{example}

\section{Módulo de Specht}
\label{modulo-specht}
En la sección previa, construimos $S_{n}$-módulos $M^{\lambda}$
conocidos como módulos de permutación. En esta sección, consideramos
submódulos irreducibles de~$M^{\lambda}$ que corresponden únicamente a
$\lambda$. 

El grupo $S_{n}$ actúa en el conjunto de tableros de Young de manera
natural: para un tablero $t$ de tamaño $n$ y una permutación $\sigma\in
S_{n}$ el tablero $\sigma t$ es el tablero que coloca el número $\sigma(i)$
en la caja donde $t$ coloca a $i$. Por ejemplo, 

\begin{center}(123)(45)
  \begin{ytableau}
    1 & 2 & 4 & 5 \\
    3 & 6\\
    7
  \end{ytableau}
  =
  \begin{ytableau}
    2 & 3 & 5 & 4 \\
    1 & 6\\
    7
  \end{ytableau}
\end{center}

Observe que un tabloide está fijo por las permutaciones, las cuales
solo permuta las entradas de los renglones entre ellos mismos. Estas
permutaciones forman un subgrupo de $S_{n}$, al cual llamamos grupo
renglón. Similarmente, definimos al grupo columna.

\begin{definition}
  Para un tablero $t$ de tamaño $n$, el \textbf{grupo renglón} de $t$,
  denotado por $R_{t}$, es el subgrupo de $S_{n}$ que consiste de las
  permutaciones que solo mueven elementos entre cada renglón
  de $t$. Similarmente, el \textbf{grupo columna} $C_{t}$ es el
  subgrupo de $S_{n}$ que consiste de las permutaciones las cuales
  solo permutan los elementos entre cada columna de $t$.
\end{definition}

Por ejemplo, si
\begin{center}$t=$
  \ytableausetup{mathmode} %,boxsize=1 
  \begin{ytableau}
    4 & 1 & 2\\
    3 & 5
  \end{ytableau}\quad ,
\end{center}
entonces
$$R_{t}=S_{\{1,2,4\}}\times S_{\{3,5\}}, \quad \mbox{ y } \quad C_{t}= S_{\{3,4\}}\times
S_{\{1,5\}}\times S_{\{2\}}.$$
Seleccionaremos ciertos elementos del espacio $M^{\lambda}$
que usaremos para generar un subespacio.

\begin{definition}
  Si $t$ es un tablero, entonces el \textbf{politabloide} asociado es
  $$e_{t}=\sum_{\pi\in C_{t}}\sgn(\pi)\pi\{t\}.$$
\end{definition}
Entonces encontramos a $e_{t}$ sumando todos los tabloides que
resultan de las permutaciones columna de $t$, teniendo en cuenta el
signo de la permutación columna. Por ejemplo, si

\begin{center}$t=$
  \ytableausetup{mathmode} %, boxsize=1em
  \begin{ytableau}
    4 & 1 & 2\\
    3 & 5
  \end{ytableau}\quad ,
\end{center}
entonces
\begin{center}
  $e_{t}=$
  \ytableausetup{mathmode,tabloids} %boxsize=1em,
  \ytableaushort{412,35}
  \quad $-$ \quad \ytableaushort{312,45}
  \quad $-$ \quad \ytableaushort{452,31}
  \quad $+$ \quad \ytableaushort{352,41}\quad .
\end{center}

Ahora, mediante el siguiente lema, vemos que $S_{n}$ actúa en el
conjunto de politabloides.
\begin{lemma}{[\cite{sagan2001symmetric}, lema 2.3.3, p.61]}.
  \label{lema}
  Sea $t$ un tablero y $\pi$ una permutación. Entonces $e_{\pi t}=\pi e_{t}$.
\end{lemma}
Ahora podemos obtener un submódulo irreducible de
$M^{\lambda}$.
\begin{definition}
  Para cualquier partición $\lambda$, el correspondiente
  \textbf{módulo de Specht}, denotado como $S^{\lambda}$, es el submódulo
  de $M^{\lambda}$ generado por los todos los politabloides $e_{t}$ de
  forma $\lambda$.
\end{definition}
% Mostraremos algunos ejemplos. Los módulos
% de Specht correspondientes a los siguientes diagramas de Young resultan
% familiares a las representaciones irreducibles. 
% \begin{center}
%   \ytableausetup{notabloids}  
%   \begin{minipage}[h]{0.2\linewidth}
%     \ydiagram{4}
%   \end{minipage}%\quad
%   \begin{minipage}[h]{0.15\linewidth}
%     \centering
%     \ydiagram{1,1,1,1}
%   \end{minipage}%\quad
%   \begin{minipage}[h]{0.2\linewidth}
%     \ydiagram{3,1}
%   \end{minipage}
% \end{center}
\begin{example}
  \label{n}
  Considere $\lambda=(n)$, sólo hay un único politabloide, el cual es
  \begin{center}
    \ytableausetup{mathmode,tabloids}    
    \ytableaushort{1 2\cdots n} \quad .
  \end{center}
Donde el politabloide está fijo por $S_{n}$, así $S^{(n)}$ es la
representación trivial de dimensión uno. 
\end{example}

\begin{example}
  \label{1n} 
  Considere $\lambda=(1^{n})=(1,1,\ldots,1)$. Sea
  
  \begin{center}
    \begin{minipage}[h]{0.1\linewidth}
      $t\quad=$
    \end{minipage}
    \begin{minipage}[h]{0.05\linewidth}
      \ytableausetup{mathmode,notabloids}
      \begin{ytableau}
        1\\
        2\\
        \vdots\\
        n
      \end{ytableau} 
    \end{minipage}.
  \end{center}

  % Observe que $e_{t}$ es la suma de todos los $\lambda$-tabloides
  % multiplicados por el signo de permutación, desarrollar ésto nos
  % tomaría tiempo. En su lugar veamos el siguiente análisis.
  Para cualquier $\lambda$-tablero $t^{'}$,
  resulta que $e_{t}=e_{t^{'}}$ si $t^{'}$ es obtenido de $t$ mediante una
  permutación par, o $e_{t}=-e_{t^{'}}$ si $t^{'}$ es obtenido de $t$
  mediante una permutación impar. Así, $S^{\lambda}$ es una
  representación de dimensión uno. Del lema \ref{lema}, $\pi e_{t}=e_{\pi t}=\sgn(\pi)e_{t}$. De lo cual, observamos que
  $S^{(1,1,\ldots,1)}$ es la representación signo.
\end{example}

\begin{example}
  \label{n-1}
  Considere $\lambda=(n-1,1)$. Continuando con
  la notación del ejemplo \ref{tabloides-ejemplo}, donde usamos
  $\{t_{i}\}$ para denotar el $\lambda$-tabloide con $i$ en el segundo
  renglón, vemos que los politabloides tienen la forma  $\{t_{i}\}-
  \{t_{j}\}$. En efecto, el politabloide construido de el tablero

\begin{center}$t=$
   \ytableausetup{mathmode, boxsize=1.5em,notabloids}
    \begin{ytableau}
      j & a & b & \cdots\\
      i\\
    \end{ytableau}
  \end{center}
es igual a $\{t_{i}\}- \{t_{j}\}$. Usemos temporalmente
$\boldsymbol{e_{i}}$ para denotar el tabloide~$\{t_{i}\}$. Entonces
$S^{\lambda}$ es generado por los elementos de la forma
$\boldsymbol{e_{i}}-\boldsymbol{e_{j}}$, y se sigue que
$$S^{(n-1,1)}=\{c_{1}\boldsymbol{e_{1}}+c_{2}\boldsymbol{e_{2}}+\cdots+c_{n}\boldsymbol{e_{n}}\mid
c_{1}+c_{2}+\cdots+c_{n}=0\}.$$ 
Esta es la representación estándar y es irreducible. La suma directa de las representaciones estándar y la
representación trivial nos da la
representación~$\mathbb{C}\{1,2,\ldots,n\}$, es decir,
$S^{(n-1,1)}\oplus S^{(n)}=M^{(n-1,1)}$.
\end{example}

Sabemos que $S_{3}$ tiene tres representaciones irreducibles: trivial,
signo y estándar. Estas son exactamente las descritas anteriormente. Además,
hay exactamente tres particiones del $3$: $(3)$, $(1,1,1)$,
$(2,1)$ y en este caso, las representaciones irreducibles son
exactamente los módulos de Specht. Esto es cierto en general.

\begin{theorem}{[\cite{sagan2001symmetric}, teorema 2.4.6, p.66]}.
  Los módulos de Specht $S^{\lambda}$ para $\lambda\vdash n$ forman
  una lista completa de representaciones irreducibles de $S_{n}$ sobre~$\mathbb{C}$.
  \label{todas-repr-irre}
\end{theorem}
Recordemos que al final de la sección \ref{tablero} notamos que el
número de representaciones irreducibles de $S_{n}$ es igual al número
de diagramas de Young con $n$ cajas. Este teorema da una biyección
entre estos dos conjuntos.

Note que los politabloides son generalmente linealmente dependientes. Por
ejemplo, como vimos en el ejemplo \ref{1n}, cualquier par de
politabloides en $S^{(1,1,\ldots,1)}$ son de hecho linealmente
dependientes. Ya que sabemos que $S^{\lambda}$ es generado por los
politabloides, podemos preguntarnos cómo seleccionar una base. El
siguiente teorema nos da la respuesta.

\begin{theorem}{[\cite{sagan2001symmetric}, teorema 2.5.2, p.67]}.
  \label{base-S}
  Sea $\lambda$ cualquier partición. El conjunto
  \begin{equation*}
    \{e_{t}\mid \mbox{t es un $\lambda$-tablero estándar}\}
  \end{equation*}
  es una base para $S^{\lambda}$ como espacio vectorial.
\end{theorem}
% Enseguida se enuncian algunos resultados como consecuencia de el
% teorema \ref{base-S}. Sea $f^{\lambda}$ el número de
% $\lambda$-tableros estándar.
% \begin{corollary}
%   \label{dimS}
%   Sea $\lambda\vdash n$, entonces $\dim S^{\lambda}=f^{\lambda}$.
% \end{corollary}
Enseguida se enuncian dos definiciones que usaremos en el capítulo \ref{cha:hom-com-emp}
relacionados con la fórmula de Bouc.

\begin{definition}
  Sea $\lambda\vdash n$. El \textbf{conjugado} de $\lambda$ es la
  partición que se obtiene listando el número de cajas en cada columna
  del diagrama de Young asociado a la $\lambda$. El conjugado de
  $\lambda$ se denota como $\lambda^{'}$. Si $\lambda=\lambda^{'}$ se
  dice que~$\lambda$ es \textbf{autoconjugada}.
\end{definition}

\begin{definition}
  Sea $\lambda\vdash n$. Llamemos \textbf{diagonal} de el
  $\lambda$-tablero, denotada por $d(\lambda)$ al número de
  cajas en la diagonal de el correspondiente tablero de forma $\lambda$.
\end{definition}

\begin{example}
  Sea $\lambda=(4,2,1)$, así que $\lambda^{'}=(3,2,1,1)$
  \begin{equation*}
    \ytableausetup{notabloids,boxsize=1.2em} 
    \lambda=\ydiagram{4,2,1} \qquad
    \lambda^{'}=\ydiagram{3,2,1,1}
  \end{equation*}
  En este caso $d(\lambda)=d(\lambda^{'})=2$.
\end{example}

\begin{example}Las particiones de el número 3 son:
  \begin{center}
    \ytableausetup{mathmode,notabloids,boxsize=1.2em}
    \begin{minipage}[h]{0.3\linewidth}
      \centering \ydiagram{3}

      (3)
    \end{minipage}
    \begin{minipage}[h]{0.3\linewidth}
      \centering \ydiagram{1,1,1}

      (1,1,1)
    \end{minipage}
    \begin{minipage}[h]{0.3\linewidth}
      \centering \ydiagram{2,1}
      
      (2,1)
    \end{minipage}
  \end{center}
 La única partición de 3 que es autoconjugada es $\lambda=(2,1)=\lambda^{'}$.
\end{example}

\begin{example}Las particiones de el número 4 son:
  \begin{center}
    \ytableausetup{mathmode,notabloids,boxsize=1.2em}
    \begin{minipage}[h]{0.2\linewidth}
      \centering \ydiagram{4}

      (4)
    \end{minipage}
    \begin{minipage}[h]{0.15\linewidth}
      \centering \ydiagram{1,1,1,1}

      (1,1,1,1)
    \end{minipage}
    \begin{minipage}[h]{0.2\linewidth}
      \centering \ydiagram{3,1}
      
      (3,1)
    \end{minipage}
    \begin{minipage}[h]{0.2\linewidth}
      \centering \ydiagram{2,1,1}
      
      (2,1,1)
    \end{minipage}
    \begin{minipage}[h]{0.2\linewidth}
      \centering \ydiagram{2,2}
      
      (2,2)
    \end{minipage}
  \end{center}
 La única partición de 4 que es autoconjugada es $\lambda=(2,2)=\lambda^{'}$.
\end{example}

\chapter{Homología de complejos simpliciales}
\label{cha:hom-com-sim}

En este capítulo se define homología  y homología reducida de un
complejo simplicial, para lo cual requerimos conceptos previos como:
complejo simplicial abstracto $\Delta$, espacio de cadenas $C_{p}(\Delta)$ y el
operador frontera $\partial_{p}$ descritos en la sección
\ref{com-sim-abs}. 

En la sección \ref{hom-simp} se introducen los conceptos de ciclos, fronteras y la
función aumento, para definir homología y homología reducida.

Se presentan los complejos de cadenas en la sección \ref{com-cad} y se
establece cierto diagrama conmutativo de los complejos de cadenas en donde
se define una trasformación lineal entre los espacios de tales
complejos de cadenas la cual induce una transformación lineal de las homologías
de los correspondientes espacios.

Por último en la sección \ref{hom-ind}, a los complejos simpliciales
$\Delta$ se les asocian conceptos topológicos por medio de su
realización geométrica $|\Delta|$. Por otra parte, si dos espacios son
homeomorfos, tienen homologías isomorfas. Una condición más débil que
homeomorfismo que implica el mismo resultado, es el de espacios
homotópicos.

Las demostraciones que se omiten se pueden encontrar en
\cite{munkres1984elements}.

\section{Complejos simpliciales abstractos}
\label{com-sim-abs}

\begin{definition}
Un \textbf{complejo simplicial abstracto} es una colección finita
$\Delta$ de conjuntos no vacíos, tal que si $A$ es un elemento de $\Delta$,
cada subconjunto no vacío de $A$ pertenece a $\Delta$.
\end{definition}

El elemento $A$ de $\Delta$ es llamado \textbf{simplejo} de
$\Delta$. Si $A$ tiene $p+1$ elementos, decimos que $A$ es un
\emph{p-simplejo} y su dimensión es $p$. La dimensión de $\Delta$
es el máximo de las dimensiones de los simplejos de $\Delta$. Cada
subconjunto no vacío de $A$ es llamado \textbf{cara} de $A$. El
conjunto de \textbf{vértices} $V(\Delta)$ de $\Delta$ es la unión de los
elementos de un punto de $\Delta$. No haremos distinción entre los
vértices~$v\in V(\Delta)$ y los $0$-simplejos $\{v\}\in \Delta$. Una
subcolección de $\Delta$, que es a su vez es un complejo, se llama
\textbf{subcomplejo} de $\Delta$.

\begin{definition}
  El subcomplejo de $\Delta$ que consiste de todos los simplejos de
  $\Delta$ de dimensión a lo más $p$, se llama \textbf{$\boldsymbol{p}$-esqueleto} de
  $\Delta$ y se denota por~$\Delta^{(p)}$. El $1$-esqueleto
  $\Delta^{(1)}$ de cualquier complejo simplicial de $\Delta$ es una
  gráfica.
\end{definition}

\begin{definition}
  Un complejo simplicial $\Delta$ es \textbf{conexo} si
  $\Delta^{(1)}$ es una gráfica conexa, es decir, si existe un camino en
  $\Delta^{(1)}$  para cualquiera dos vértices.
\end{definition}

Primero introducimos la idea de un \emph{n-simplejo orientado}. Un
\textbf{0-simplejo orientado} es un punto $v$. Un \textbf{1-simplejo
  orientado} es un segmento de línea dirigido $v_{0}v_{1}$ uniendo los
puntos $v_{0}$ y $v_{1}$ en dirección de $v_{0}$ a $v_{1}$ como se
ilustra en la figura \ref{fig:1-simplejo}, así que
$v_{0}v_{1}\neq v_{1}v_{0}$, pero estaremos de acuerdo en que~$v_{0}v_{1}=-v_{1}v_{0}$. Un \textbf{2-simplejo orientado} es una
región triangular $v_{0}v_{1}v_{2}$ con dirección de $v_{0}$ a $v_{1}$
a $v_{2}$ como se aprecia en la figura \ref{fig:2-simplejo}, claramente~$v_{0}v_{1}v_{2}$ tiene el mismo orden que
$v_{1}v_{2}v_{0}$ y $v_{2}v_{0}v_{1}$, pero con orientación opuesta a
$v_{0}v_{2}v_{1}$, $v_{2}v_{1}v_{0}$ y $v_{1}v_{0}v_{2}$, es decir, estaremos de acuerdo que:
$$v_{0}v_{1}v_{2}=v_{1}v_{2}v_{0}=v_{2}v_{0}v_{1}=-v_{0}v_{2}v_{1}=-v_{2}v_{1}v_{0}=-v_{1}v_{0}v_{2}.$$

Note que $v_{i}v_{j}v_{k}$ es igual a $v_{0}v_{1}v_{2}$, si
\[ \left(
  \begin{array}{ccc}
    0 & 1 & 2 \\
    i & j & k 
  \end{array} 
\right)\] 

es una permutación par y es igual a $-v_{0}v_{1}v_{2}$ si la
permutación es impar.

Un \textbf{3-simplejo orientado} está dado por una secuencia ordenada
$v_{0}v_{1}v_{2}v_{3}$ de cuatro vértices de un tetraedro sólido
como se observa en la figura \ref{fig:3-simplejo} y
acordaremos que $v_{0}v_{1}v_{2}v_{3}=\pm v_{i}v_{j}v_{r}v_{s}$,
dependiendo si la permutación es par o impar. Así, dos maneras
de enumerar los vértices de un $n$-simplejo se considerarán
equivalentes, o producen la misma orientación si difieren una de la otra
por una permutación par. 

\begin{figure}[h]
  \centering
  \begin{minipage}[h]{0.45\linewidth}
    \centering
    \begin{tikzpicture}[scale=1.2]
      \GraphInit[vstyle=Classic] \SetUpVertex[MinSize=1pt]
      \Vertex[x=0,y=0,Math]{v_{0}} \Vertex[x=1.5,y=1.5,Math]{v_{1}}
      \Edge[style={->}](v_{0})(v_{1})
      \AddVertexColor{black}{v_{0},v_{1}}
    \end{tikzpicture}
  
    \caption{1-simplejo}
    \label{fig:1-simplejo}
  \end{minipage}
  \begin{minipage}[h]{0.45\linewidth}
    \centering
    \begin{tikzpicture}[scale=0.8]
      \tikzstyle{circ} = [circle, minimum width=1.2mm, inner
      sep=0pt,draw,fill] \tikzstyle{num} = [yshift=4mm] \node[circ](a)
      at (0,0){}; \node[circ] (b) at (2,0){}; \node[circ] (c) at
      (1,2){};
      \begin{pgfonlayer}{background}
        \fill[draw,line width=0.3mm,top color=black!30,bottom
        color=gray!30] (a.center)node[num]{$v_{0}$} --
        (b.center)node[num]{$v_{1}$} -- (c.center) node[num]{$v_{2}$}
        --(a.center);
      \end{pgfonlayer}
      \draw[>=latex,->] (0,0) -- (2,0); \draw[>=latex,->] (2,0) --
      (1,2); \draw[>=latex,->] (1,2) -- (0,0);
    \end{tikzpicture}
  
    \caption{2-simplejo}
    \label{fig:2-simplejo}
  \end{minipage}
\end{figure}

\begin{figure}[h]
  \centering
  \begin{tikzpicture}[scale=1.1]%[x=0.5 cm,y=1cm]
    \tikzstyle{circ} = [circle, minimum width=1mm, inner
    sep=0pt,draw,fill] \tikzstyle{num} = [yshift=4mm] \node[circ]
    (a) at (0,0) {}; \node[circ] (b) at (2,0) {}; \node[circ] (c) at
    (1,1.7) {}; \node[circ] (d) at (1,0.5) {};
    \begin{pgfonlayer}{background}
      \fill[draw, line width=0.3mm,top color=black!30,bottom
      color=gray!30] (a.center)node[num]{} -- (b.center) node[num]{}
      -- (d.center) node[num]{} --(a.center);
    \end{pgfonlayer}
    \begin{pgfonlayer}{background}
      \fill[draw, line width=0.3mm,top color=black!30,bottom
      color=gray!30] (a.center)node[num]{} -- (d.center) node[num]{}
      -- (c.center) node[num]{} --(a.center);
    \end{pgfonlayer}
    \begin{pgfonlayer}{background}
      \fill[draw, line width=0.3mm,top color=black!30,bottom
      color=gray!30](b.center)node[num]{} -- (c.center) node[num]{}
      -- (d.center) node[num]{} --(b.center);
    \end{pgfonlayer}
  \end{tikzpicture}
  
  \caption{3-simplejo}
  \label{fig:3-simplejo}
\end{figure}

\begin{definition}
  Sea $\Delta$ un complejo simplicial. Una \textbf{\emph{p}-cadena} en
  $\Delta$ es una función $c$ de el conjunto de $p$-simplejos
  orientados de
  $\Delta$ al campo de los números complejos, tal que:
    $c(\sigma)=-c(\sigma^{'})$ si $\sigma$ y $\sigma^{'}$ son
      orientaciones opuestas del mismo simplejo.
\end{definition}

% Dado que en nuestro caso sólo estudiamos complejos simpliciales cuyo
% conjunto de vértices es finito, la segunda condición no se aplica.

Sumamos $p$-cadenas sumando sus valores y producto por escalar de una
$p$-cadena es la multiplicación de su valor por un elemento $\lambda\in\mathbb{C}$; el $\mathbb{C}$-espacio vectorial resultante es
denotado por $C_{p}(\Delta)$ y es llamado el \textbf{espacio de
  \emph{p}-cadenas (orientadas)} de $\Delta$. Si $p<0$ o $p>\dim \Delta$,
$C_{p}(\Delta)$ denota al espacio trivial.

\begin{definition}
  Si $\sigma$ es un simplejo orientado, la \textbf{cadena elemental} $c$
  correspondiente a $\sigma$ es la función definida como:
  \[ 
  \begin{array}{cl}
    c(\sigma)=1, & \\
    c(\sigma^{'})=-1 & \mbox{si $\sigma^{'}$ tiene orientación opuesta de $\sigma$}, \\
    c(\tau)=0 & \mbox{para todos los otros simplejos orientados $\tau$}. 
  \end{array}\] 
  \end{definition}

Por abuso de notación, muchas veces usamos el símbolo $\sigma$ para
denotar no solo a un simplejo, o a un simplejo orientado, sino también
a la $p$-cadena elemental $c$ correspondiente al simplejo orientado
$\sigma$. Con esta convención, si $\sigma$ y $\sigma^{'}$ tienen
orientaciones opuestas del mismo simplejo, entonces podemos escribir
$\sigma^{'}=-\sigma$, pues esta ecuación se mantiene cuando nos
referimos a $\sigma$ y $\sigma^{'}$ como cadenas elementales.

\begin{lemma}
   Una base para $C_{p}(\Delta)$ se puede obtener
   tomando una orientación por cada $p$-simplejo y usando las
   correspondientes cadenas elementales como elementos de la base.
\end{lemma}

\begin{proof}[Demostración]
  Orientando (arbitrariamente) a cada \emph{p}-simplejo de $\Delta$,
  toda \emph{p}-cadena se puede escribir de manera única como una
  combinación lineal finita
$$c=\sum n_{i}\sigma_{i},$$
de las correspondientes cadenas elementales $\sigma_{i}$. La cadena
$c$ asigna el valor~$n_{i}$ al \emph{p}-simplejo orientado
$\sigma_{i}$ , el valor $(-n_{i})$ a la orientación opuesta de
$\sigma_{i}$ y el valor $0$ a todo \emph{p}-simplejo orientado que no
aparece en la suma. 
%Entonces la afirmación se sigue del teorema $\ref{clunica}$.
\end{proof}

\begin{corollary}{[\cite{munkres1984elements}, corolario 5.2, p.36]}.
  Toda función $f$ de los p-simplejos orientados de $\Delta$ en un
  espacio vectorial $V$ , tal que $f(-\sigma)=-f(\sigma)$ para todo
  p-simplejo orientado~$\sigma$ puede extenderse de manera única a una
  trasformación lineal de $C_{p}(\Delta)\rightarrow V$.
\end{corollary}

\begin{definition}
  Sea $\sigma=(v_{0},\ldots ,v_{p})$ un simplejo orientado con $p>0$
  (sin embargo, también será denotado como $\sigma=v_{0}\ldots v_{p}$
  cuando no tengamos problemas de confusión),
  definimos la trasformación lineal
  $\partial_{p}:C_{p}(\Delta)\rightarrow~C_{p-1}(\Delta)$ por medio de:
  \begin{equation}
    \label{ofrontera}
    \partial_{p}(\sigma)=\partial_{p}(v_{0}\ldots
    v_{p})=\sum^{p}_{i=0}(-1)^{i}(v_{0}\ldots \widehat v_{i}\ldots v_{p}),
  \end{equation}
  al que llamamos el \textbf{\emph{p}-ésimo operador frontera}, donde
  $\widehat v_{i}$ indica que el vértice $v_{i}$ es borrado del arreglo.
\end{definition}

Puesto que $C_{p}(\Delta)$ es el espacio trivial para $p<0$, diremos
que $\partial_{p}$ es la \textit{trasformación cero} para $p\leq
0$. Mostremos ahora que $\partial_{p}$ está bien definido y que
$\partial_{p}(-\sigma)=-\partial_{p}(\sigma)$. Es suficiente
mostrar que la ecuación (\ref{ofrontera}) cambia de signo si intercambiamos dos
vértices adyacentes en el orden $v_{0}\ldots v_{p}$, y para esto
debemos comparar las expresiones:
$$\partial_{p}(v_{0}\ldots v_{j} v_{j+1} \ldots v_{p}) \mbox{ y } \partial_{p}(v_{0}\ldots v_{j+1} v_{j} \ldots v_{p}).$$

Para $i\neq j$, $j+1$, el $i$-ésimo término de estas dos expresiones
difieren precisamente por un signo; los términos son idénticos,
excepto que $v_{j}$ y~$v_{j+1}$ aparecen intercambiados. Veamos que
sucede sobre el $i$-ésimo término cuando $i=j$ y $i=j+1$. En la primera
expresión obtenemos:
$$(-1)^{j}(\ldots v_{j-1} \widehat v_{j}v_{j+1}v_{j+2}\ldots)+(-1)^{j+1}(\ldots v_{j-1}v_{j}\widehat v_{j+1}v_{j+2}\ldots).$$
mientras que en la segunda expresión tenemos:
$$(-1)^{j}(\ldots v_{j-1}\widehat v_{j+1}v_{j}v_{j+2}\ldots)+(-1)^{j+1}(\ldots v_{j-1}v_{j+1}\widehat v_{j}v_{j+2}\ldots).$$
Comparando estas dos expresiones observamos que solo difieren por un signo.

\begin{example}
  De acuerdo a lo anterior, tenemos que
  \begin{enumerate}
  \item para un \emph{1-simplejo}: $\partial_{1}(v_{0}v_{1})= v_{1}-v_{0}$,
  \item para un \emph{2-simplejo}: $\partial_{2}(v_{0}v_{1}v_{2})=v_{1}v_{2}-v_{0}v_{2}+v_{0}v_{1}$,
  \item para un \emph{3-simplejo}:
    $\partial_{3}(v_{0}v_{1}v_{2}v_{3})=v_{1}v_{2}v_{3}-v_{0}v_{2}v_{3}+v_{0}v_{1}v_{3}-v_{0}v_{1}v_{2}$. 
  \end{enumerate}
\end{example}

\section{Homología simplicial}
\label{hom-simp}

\begin{definition}
   El kernel de $\partial_{p}:C_{p}(\Delta)\rightarrow
   C_{p-1}(\Delta)$ es llamado el espacio de
   \textbf{\emph{p}-ciclos} y denotado por $Z_{p}(\Delta)$. La imagen
   de $\partial_{p+1}:C_{p+1}(\Delta)\rightarrow C_{p}(\Delta)$ es
   llamado el espacio de \textbf{\emph{p}-fronteras} y es denotado por $B_{p}(\Delta)$.
\end{definition}

\begin{definition}
  Sea $\varepsilon:C_{0}\rightarrow \mathbb{C}$ la transformación
  lineal sobreyectiva definida por $\varepsilon(v)=1$ para cada
  vértice $v\in \Delta$. Entonces si $c$ es una $0$-cadena,
  $\varepsilon(c)$ es igual a la suma de los valores de $c$ en los
  vértices de $\Delta$, es decir:
  $$\varepsilon(\sum \lambda_{i}v_{i})=\sum\lambda_{i}.$$
  Llamaremos a $\varepsilon$ la \textbf{función aumento} para
  $C_{0}(\Delta)$.
\end{definition}

\begin{theorem}
  $\partial_{p-1}\circ\partial_{p}=0$ para cualquier $p$ y también $\varepsilon\circ\partial_{1}=0$.
\end{theorem}

\begin{proof}[Demostración]
  Calculamos
  \begin{align*}
    \partial_{p-1}\partial_{p}(v_{0}\ldots
    v_{p})&=\sum_{i=0}^{p}(-1)^{i}\partial_{p-1}(v_{0}\ldots \widehat v_{i}\ldots v_{p})\\
    &=\sum_{j<i}(-1)^{i}(-1)^{j}(\ldots \widehat v_{j} \ldots \widehat v_{i} \ldots)\\
    &+\sum_{j>i}(-1)^{i}(-1)^{j-1}(\ldots\widehat v_{i}\ldots \widehat
    v_{j}\ldots).
  \end{align*}
  Los términos de estas dos sumas se cancelan a pares.
\end{proof}

\begin{corollary}
  $B_{p}(\Delta)$ es subespacio de $Z_{p}(\Delta)$.
\end{corollary}
\begin{proof}[Demostración]
  Primero veamos que $B_{n}(\Delta)$ es subconjunto de
  $Z_{n}(\Delta)$, tenemos que
  $B_{n}(\Delta)=\partial[C_{n+1}(\Delta)]$, si $b\in B_{n}(\Delta)$,
  podemos escribir a $b$ como~$b=\partial_{n+1}(c)$ para algún $c\in
  C_{n+1}(\Delta)$. Así,
$$\partial_{n}(b)=\partial_{n}(\partial_{n+1}(c))=0.$$
Por lo tanto, $b\in Z_{n}(\Delta)$.  Las otras condiciones se siguen
de que $\partial_{n+1}$ es una transformación lineal.
\end{proof}
De forma análoga se demuestra que $B_{0}(\Delta)$ es subespacio de $\ker(\varepsilon)$.
% \begin{enumerate}
%   \item Tomemos $0\in C_{n+1}$, así que $\partial_{n+1}(0)=\widehat 0$,
%     por lo tanto $\widehat 0\in B_{n}(\Delta)$
%   \item Sean $b_{1}\in B_{n}(\Delta)$ y $b_{2}\in B_{n}(\Delta)$,
%     donde $b_{1}=\partial_{n+1}(c_{1})$ y $b_{2}=\partial_{n+1}(c_{2})$ para
%     algún $c_{1},c_{2}\in C_{n+1}(\Delta)$, así que $b_{1}+b_{2}\in B_{n}$
%     pues $b_{1}+b_{2}=\partial_{n+1}(c_{1})+\partial_{n+1}(c_{2})=\partial_{n+1}(c_{1}+c_{2})$
%     ya que $c_{1}+c_{2}\in C_{n+1}(\Delta)$.
%   \item Por último sea $a\in \mathbb{C}$, y $b\in B_{n}(\Delta)$ con
%     $b=\partial_{n+1}(c)$ para algún $c\in C_{n+1}(\Delta)$, notemos
%     que $ac\in C_{n+1}(\Delta)$, así que $ab\in B_{n}(\Delta)$ pues $ab=a\partial_{n+1}(c)=\partial_{n+1}(ac)$.
% \end{enumerate}
\begin{definition}
   El espacio cociente
   $$H_{p}(\Delta)=Z_{p}(\Delta)/B_{p}(\Delta)$$
   se llama el \textbf{\emph{p}-ésimo espacio de homología de
     $\Delta$}. Si $H_{p}(\Delta)$ es un módulo cociente, se llama el
   \textbf{\emph{p}-ésimo módulo de homología de
     $\Delta$}.
\end{definition}

\begin{definition}
  Definimos el \textbf{espacio de homología reducida} de $\Delta$ en
  dimensión~$0$, denotado por $\widetilde H_{0}(\Delta)$, como
  \begin{equation*}
    \widetilde H_{0}(\Delta)=\ker\varepsilon/\im \partial_{1}.
  \end{equation*}
  (Si $p>0$, $\widetilde H_{p}(\Delta)$ denota el espacio usual
  $H_{p}(\Delta)$.)
\end{definition}

\begin{theorem}{[\cite{munkres1984elements}, teorema 7.2, p.43]}.
  Sea $\Delta$ un complejo simplicial no vacío, entonces
  $$H_{0}(\Delta)\cong \widetilde H_{0}(\Delta)\oplus
  \mathbb{C}.$$
  Además, $\widetilde H_{0}(\Delta)=0$ si $\Delta$ es conexo.
\end{theorem}

\begin{example}
  Calculemos para $n=0, 1, 2$ los espacios $Z_{n}(\Delta)$,
  $B_{n}(\Delta)$ y~$H_{n}(\Delta)$ para la superficie $\Delta$ del tetraedro.

  Como $C_{-1}(\Delta)=0$ por definición, se sigue que
  $$\boldsymbol{Z_{0}(\Delta)}=C_{0}(\Delta).$$
  Por otro lado, $C_{1}(\Delta)=\langle v_{0}v_{1},v_{0}v_{2},v_{0}v_{3},v_{1}v_{2},v_{1}v_{3},v_{2}v_{3}\rangle$,
  así que por el teorema $\ref{imT}$ tenemos, 
  \begin{align}
    \label{frontera0}
    \boldsymbol{B_{0}(\Delta)}=&\partial_{1}[C_{1}(\Delta)]\nonumber\\
    &=\langle \partial_{1}(v_{0}v_{1}),\partial_{1}(v_{0}v_{2}),\partial_{1}(v_{0}v_{3}),\partial_{1}(v_{1}v_{2}),\partial_{1}(v_{1}v_{3}),\partial_{1}(v_{2}v_{3})\rangle\nonumber\\
    &=\langle v_{1}-v_{0},v_{2}-v_{0},v_{3}-v_{0},v_{2}-v_{1},v_{3}-v_{1},v_{3}-v_{2}\rangle\nonumber\\
    &=\langle v_{1}-v_{0},v_{2}-v_{0},v_{3}-v_{0}\rangle,
  \end{align} 
  pues los vectores $v_{2}-v_{1}, v_{3}-v_{1}, v_{3}-v_{2}$ son
  combinación lineal de los vectores de $\ref{frontera0}$, ya que:
  $$v_{2}-v_{1}=(v_{2}-v_{0})-(v_{1}-v_{0})$$
  $$v_{3}-v_{1}=(v_{3}-v_{0})-(v_{1}-v_{0})$$
  $$v_{3}-v_{2}=(v_{3}-v_{0})-(v_{2}-v_{0}),$$
  Así, $\dim(B_{0}(\Delta))=3$, además $C_{0}(\Delta)=\langle
  v_{0},v_{1},v_{2},v_{3}\rangle$, por lo que
 $\dim(Z_{0}(\Delta))=\dim(C_{0}(\Delta))=4$. Luego de el teorema
 \ref{dim-esp-coc}, $\dim(Z_{0}(\Delta)/B_{0}(\Delta))=1$, así mismo
 la $\dim(\mathbb{C})=1$ y por el teorema \ref{esp-isomorfos}, se
 sigue que $Z_{0}(\Delta)/B_{0}(\Delta)\cong \mathbb{C}$.
 Por lo tanto:
 $$\boldsymbol{H_{0}(\Delta)}=Z_{0}(\Delta)/B_{0}(\Delta)\cong \mathbb{C}.$$
 Un razonamiento similar se sigue para calcular los espacios de homología restantes.
 Ahora calculemos $B_{1}(\Delta)$. Sabemos que:
 $$C_{2}(\Delta)=\langle
 v_{0}v_{1}v_{2},v_{0}v_{2}v_{3},v_{0}v_{1}v_{3},v_{1}v_{2}v_{3}\rangle.$$
 Nuevamente por el teorema $\ref{imT}$ tenemos:
 \begin{align}  
    \label{generadores-B1}
    &\boldsymbol{B_{1}(\Delta)}=\partial_{2}[C_{2}(\Delta)]=\langle\partial_{2}(v_{0}v_{1}v_{2}),\partial_{2}(v_{0}v_{2}v_{3}),\partial_{2}(v_{0}v_{1}v_{3}),\partial_{2}
    (v_{1}v_{2}v_{3})\rangle \nonumber\\
    &=\langle v_{1}v_{2}-v_{0}v_{2}+v_{0}v_{1},v_{2}v_{3}-v_{0}v_{3}+v_{0}v_{2},\nonumber\\
    &\phantom{{}=v_{1}v_{2}-v_{0}v_{2}+v_{0}v_{1},v_{2}v_{3}}v_{1}v_{3}-v_{0}v_{3}+v_{0}v_{1},v_{2}v_{3}-v_{1}v_{3}+v_{1}v_{2}\rangle\nonumber\\
    &=\langle v_{1}v_{2}-v_{0}v_{2}+v_{0}v_{1},v_{2}v_{3}-v_{0}v_{3}+v_{0}v_{2},v_{1}v_{3}-v_{0}v_{3}+v_{0}v_{1}\rangle
  \end{align} 
 
  A continuación veremos que el conjunto generador de $Z_{1}(\Delta)$
  es el mismo que $B_{1}(\Delta)$. Sea $c\in C_{1}(\Delta)$, es decir,
 $$c=n_{1}v_{0}v_{1}+n_{2}v_{0}v_{2}+n_{3}v_{0}v_{3}+n_{4}v_{1}v_{2}+n_{5}v_{1}v_{3}+n_{6}v_{2}v_{3}$$
 tal que:
 \begin{align*}
   \partial_{1}(c)&=n_{1}(v_{1}-v_{0})+n_{2}(v_{2}-v_{0})+n_{3}(v_{3}-v_{0})\\
   &\phantom{{}=n_{1}}+n_{4}(v_{2}-v_{1})+n_{5}(v_{3}-v_{1})+n_{6}(v_{3}-v_{2})\\
   % \nonumber \\
   &=(-n_{1}-n_{2}-n_{3})v_{0}+(n_{1}-n_{4}-n_{5})v_{1}\\
   &\phantom{{}=-n_{1}}+(n_{2}+n_{4}-n_{6})v_{2}+(n_{3}+n_{5}+n_{6})v_{3}\\
   &=0.
 \end{align*}
 Así que tenemos que resolver el siguiente sistema de ecuaciones:
 \[\begin{array}{rrrrr}
   -n_{1} & -n_{2} & -n_{3} & = & 0 \\
   n_{1} & -n_{4} & -n_{5} & = & 0 \\
   n_{2} & +n_{4} & -n_{6} & = & 0 \\
   n_{3} & +n_{5} & +n_{6} & = & 0. 
 \end{array}\]
 Lo representamos en forma matricial.
 \[ \left(
   \begin{array}{rrrrrr}
  % n_{1} & n_{2} & n_{3} & n_{4} & n_{5} & n_{6} \\
     -1  & -1    & -1   & 0    & 0     & 0 \\
     1   & 0     &    0 & -1   & -1    & 0 \\
     0   & 1     &    0 & 1   & 0    & -1 \\
     0   & 0     &    1 & 0   & 1    & 1 
   \end{array} 
 \right)\]
 y lo llevamos a su forma escalonada reducida
 \[ \left(
   \begin{array}{rrrrrr}
     % n_{1} & n_{2} & n_{3} & n_{4} & n_{5} & n_{6} \\
     1     &    0  & 0     & -1    & -1    & 0 \\
     0     &    1  & 0     &  1    & 0     & -1 \\
     0     &    0  & 1     & 0     & 1     & 1 \\
     0     &    0  & 0     & 0     & 0     & 0 
   \end{array} 
 \right)\]
 De lo cual concluimos:
 \[\begin{array}{rrrrr}
   % n_{1}& = & n_{4} & n_{5} & n_{6} \\  
   n_{1} & = & n_{4} & +n_{5} & \\
   n_{2} & = & -n_{4} &      &+n_{6} \\
   n_{3} & = &        &-n_{5}&-n_{6}. 
 \end{array}\]
 Entonces podemos escribir a $c\in Z_{1}(\Delta)$ como:
 \begin{align}
   \label{generadores-Z1}
   c&=(n_{4}+n_{5})v_{0}v_{1}+(-n_{4}+n_{6})v_{0}v_{2}+(-n_{5}-n_{6})v_{0}v_{3}\nonumber\\
   &\phantom{{}=n_{4}}+n_{4}v_{1}v_{2}+n_{5}v_{1}v_{3}+n_{6}v_{2}v_{3}\nonumber\\
   &=n_{4}(v_{0}v_{1}-v_{0}v_{2}+v_{1}v_{2})+n_{5}(v_{0}v_{1}-v_{0}v_{3}+v_{1}v_{3})\nonumber\\
   &\phantom{{}=n_{4}}+n_{6}(v_{0}v_{2}-v_{0}v_{3}+v_{2}v_{3}).
 \end{align}
 Por tanto, de la ecuación \ref{generadores-Z1} tenemos:
 $$\boldsymbol{Z_{1}(\Delta)}=\langle v_{0}v_{1}-v_{0}v_{2}+v_{1}v_{2},v_{0}v_{1}-v_{0}v_{3}+v_{1}v_{3},v_{0}v_{2}-v_{0}v_{3}+v_{2}v_{3}\rangle.$$
 De las ecuaciones \ref{generadores-B1} y \ref{generadores-Z1}, obtenemos que  $Z_{1}(\Delta)$ y $B_{1}(\Delta)$ tienen
 el mismo conjunto generador, por lo tanto
 $Z_{1}(\Delta)=B_{1}(\Delta)$, en consecuencia,
 $$\boldsymbol{H_{1}(\Delta)}=Z_{1}(\Delta)/B_{1}(\Delta)=0.$$
 Los simplejos de dimensión más alta son los 2-simplejos así
 que $C_{3}(\Delta)=0$ por lo que,
 $$\boldsymbol{B_{2}(\Delta)}=\partial_{3}[C_{3}(\Delta)]=0.$$ 
Determinaremos ahora $Z_{2}(\Delta)$. Si $c\in C_{2}(\Delta)$, es decir 
 $$c=n_{1}v_{0}v_{1}v_{2}+n_{2}v_{0}v_{2}v_{3}+n_{3}v_{0}v_{1}v_{3}+n_{4}v_{1}v_{2}v_{3}$$
 tal que
 \begin{align*}
   \partial_{2}(c)&=n_{1}(v_{1}v_{2}-v_{0}v_{2}+v_{0}v_{1})+n_{2}(v_{2}v_{3}-v_{0}v_{3}+v_{0}v_{2})\\
   &+n_{3}(v_{1}v_{3}-v_{0}v_{3}+v_{0}v_{1})+n_{4}(v_{2}v_{3}-v_{1}v_{3}+v_{1}v_{2})\\
   &=(n_{1}+n_{3})v_{0}v_{1}+(-n_{1}+n_{2})v_{0}v_{2}+(-n_{2}-n_{3})v_{0}v_{3}\\
   & +(n_{1}+n_{4})v_{1}v_{2}+(n_{3}-n_{4})v_{1}v_{3}+(n_{2}+n_{4})v_{2}v_{3}\\
   &=0
 \end{align*}
 entonces $n_{1}=n_{2}=-n_{3}=-n_{4}$, así que podemos escribir a $c$
 de la siguiente forma: 
 $$c=n_{1}(v_{0}v_{1}v_{2}+v_{0}v_{2}v_{3}-v_{0}v_{1}v_{3}-v_{1}v_{2}v_{3}),$$
 de donde vemos que
 $$Z_{2}(\Delta)=\langle v_{0}v_{1}v_{2}+v_{0}v_{2}v_{3}-v_{0}v_{1}v_{3}-v_{1}v_{2}v_{3}\rangle,$$
 es decir, $\boldsymbol{Z_{2}(\Delta)}\cong \mathbb{C}$. Luego  $\boldsymbol{H_{2}}(\Delta)=Z_{2}(\Delta)/B_{2}(\Delta)\cong\mathbb{C}$.
\end{example}

\section{Complejo de cadenas}
\label{com-cad}

\begin{definition}
  Un \textbf{complejo de cadenas} $(A,\partial)$ es una sucesión
  $$A=\{\cdots,A_{2},A_{1},A_{0},A_{-1},A_{-2},\cdots\}$$
  de espacios vectoriales $A_{k}$, junto con una colección
  $\partial=\{\partial_{k}\mid k \in \mathbb{Z}\}$ de transformaciones
  lineales tales que $\partial_{k}:A_{k}\rightarrow A_{k-1}$ y
  $\partial_{k-1}\partial_{k}=0.$
  % Al complejo de cadenas en donde $A_{-1}=\mathbb{C}$ y $\partial_{0}=\varepsilon$,
  % donde $\varepsilon$ es la función aumento, le llamamos \textbf{complejo de cadenas aumentado}.
\end{definition}

\begin{theorem}
  Si $(A,\partial)$ es un complejo de cadenas, entonces la imagen bajo
  $\partial_{k}$ es un subespacio de el kernel de $\partial_{k-1}$.
\end{theorem}
\begin{proof}[Demostración.]
  Considere
  \begin{small}
    \[
    \begin{array}{ccccc}
      A_{k} & \stackrel{\partial_{k}}{\longrightarrow} & A_{k-1} &
      \stackrel{\partial_{k-1}}{\longrightarrow} & A_{k-2}.
    \end{array} 
    \]
  \end{small}
  $\partial_{k-1}\partial_{k}=0,$ pues $(A,\partial)$ es un complejo
  de cadenas. Esto es $\partial_{k-1}[\partial_{k}[A_{k}]]=0.$ Así,
  $\partial_{k}[A_{k}]$ está contenido en el kernel de
  $\partial_{k-1}$, como se quería demostrar.
\end{proof}

\begin{definition}
  Si $(A,\partial)$ es un complejo de cadenas, entonces el
  kernel~$Z_{k}(A)$ de $\partial_{k}$ es el\textbf{ espacio de}
  $\boldsymbol{k}$\textbf{-ciclos,} y la imagen $B_{k}(A)=\partial_{k+1}[A_{k+1}]$ es el \textbf{espacio
    de} $\boldsymbol{k}$\textbf{-fronteras.} El espacio cociente $H_{k}(A)=Z_{k}(A)/B_{k}(A)$
  es la $\boldsymbol{k}$\textbf{-ésima homología de $A$.}
\end{definition}

\begin{theorem}
  Sean $(A,\partial)$ y $(A^{'},\partial^{'})$ complejos de cadenas, y
  supongamos que hay una colección $f$ de transformaciones lineales
  $f_{k}:A_{k}\rightarrow A^{'}_{k}$ como se indica en el diagrama  
  \[
  \begin{CD}
    \cdots @>{\partial_{k+2}}>> A_{k+1} @>{\partial_{k+1}}>> A_{k} @>{\partial_{k}}>> A_{k-1} @>{\partial_{k-1}}>> \cdots\\
    @.   @VVf_{k+1}V   @VVf_{k}V   @VVf_{k-1}V    \\
    \cdots @>{\partial^{'}_{k+2}}>> A^{'}_{k+1} @>{\partial^{'}_{k+1}}>> A^{'}_{k} @>{\partial^{'}_{k}}>> A^{'}_{k-1} @>{\partial^{'}_{k-1}}>> \cdots
  \end{CD}
  \]
  Además, supongamos que para todo $k$ se tiene
  $$f_{k-1}\partial_{k}=\partial^{'}_{k}f_{k}.$$
  Entonces $f_{k}$ induce una transformación lineal
  $f_{*k}:H_{k}(A)\rightarrow H_{k}(A^{'}).$
\end{theorem}
\begin{proof}[Demostración.]
  Sea $z\in Z_{k}(A)$. Ahora
  \begin{equation*}
    \partial^{'}_{k}(f_{k}(z))=f_{k-1}(\partial_{k}(z))=f_{k-1}(0)=0,
  \end{equation*}
  así, $f_{k}(z)\in Z_{k}(A^{'})$. Definamos
  $f_{*k}:H_{k}(A)\rightarrow H_{k}(A^{'})$ por
  \begin{equation}
    \label{trans-lin-homologias}
    f_{*k}(z+B_{k}(A))=f_{k}(z)+B_{k}(A^{'}).
  \end{equation}
  
  Primero debemos mostrar que $f_{*k}$ está bien definida, es decir,
  independientemente de la elección de un representante de
  $z+B_{k}(A).$ Supongamos que $z_{1}\in (z+B_{k}(A)).$ Entonces
  $(z_{1}-z)\in B_{k}(A)$, y existe $c\in A_{k+1}$ tal que
  $z_{1}-z=\partial_{k+1}(c).$ Pero,
  $$f_{k}(z_{1})-f_{k}(z)=f_{k}(z_{1}-z)=f_{k}(\partial_{k+1}(c))=\partial^{'}_{k+1}(f_{k+1}(c))$$
  y este último término es un elemento de
  $\partial^{'}_{k+1}[A^{'}_{k+1}]=B_{k}(A^{'}).$ Por lo tanto
  $$f_{k}(z_{1})\in (f_{k}(z)+B_{k}(A^{'})).$$
  Así dos representantes de la misma clase lateral en
  $H_{k}(A)=Z_{k}(A)/B_{k}(A)$ son enviados en representantes de la misma
  clase lateral en el subespacio $H_{k}(A^{'})=Z_{k}(A^{'})/B_{k}(A^{'})$. Esto muestra
  que $f_{*k}:H_{k}(A)\rightarrow H_{k}(A^{'})$ está bien definida por
  la ecuación~$(\ref{trans-lin-homologias})$.
  
  Demostrar que $f_{*k}$ es una transformación lineal se sigue de la
  linealidad de $f_{k}$.
\end{proof}

Notemos que si $f_{k}$ es un isomorfismo, entonces $f_{k}$ induce un
isomorfismo~$f_{*k}$, es decir, $H_{k}(A)\cong H_{k}(A^{'})$.

\section[Homomorfismos inducidos]{Homomorfismos inducidos por funciones homotópicas}
\label{hom-ind}

En esta sección sea $I$ el intervalo $[0,1]$. Es posible definir homología de un espacio topológico general por
medio del concepto de la \emph{homología singular}. Por otro lado, a cada complejo
simplicial $\Delta$ se le puede asociar un espacio topológico denotado $|\Delta|$
llamado la \textbf{realización geométrica} de $\Delta$, dado por:

$$|\Delta|=\{\alpha:V(\Delta)\rightarrow I\mid\alpha^{-1}(0,1]\in
\Delta, \sum _{v\in V(\Delta)}\alpha(v)=1\}.$$
Ahora a $|\Delta|$ le vamos a dar la topología generada por la métrica
\begin{equation*}
  \label{metrica}
  d:|\Delta|\times |\Delta|\rightarrow  \mathbb{R} \quad \mbox{tal
    que}\quad d(\alpha,\beta)=\sqrt{\sum _{v\in V(\Delta)}(\alpha(v)-\beta(v))^{2}}.
\end{equation*}

A los complejos simpliciales se les asocian conceptos topológicos por
medio de la realización geométrica. Por ejemplo, dos complejos se
dicen homeomorfos si las realizaciones geométricas los son.

\begin{definition}
  Sean $X$ y $Y$ espacios topológicos. Dos funciones continuas
  $h,k:X\rightarrow Y$ son \textbf{homotópicas} si existe una función continua
  $$F:X\times I\rightarrow Y,$$
  tal que $F(x,0)=h(x)$ y $F(x,1)=k(x)$ para todo $x\in X.$ Si $h$ y
  $k$ son homotópicas, lo denotamos como $h\simeq k$. Pensamos a $F$
  como una forma de ``deformar'' a $h$ continuamente en $k$, conforme
  $t$ varía de $0$ a $1$.
\end{definition}

% \begin{theorem}
%   Si $h,k:|K|\rightarrow |L|$ son homotópicas, entonces
%   $h_{*},k_{*}:H_{p}(K)\rightarrow H_{p}(L)$ son iguales. El mismo
%   resultado se preserva para homologías reducidas.
% \end{theorem}

Si dos espacios son homeomorfos, tienen homologías isomorfas. Una
condición más débil que homeomorfismo que implica el mismo resultado,
es el de homotópicamente equivalentes.
 
\begin{definition}
  Dos espacios topológicos $X$ y $Y$ se dicen \textbf{homotópicamente
    equivalentes} o bien \textbf{homotópicos}, si existen las funciones
  $$f:X\rightarrow Y \quad \mbox{ y }\quad g:Y\rightarrow X$$
  tales que $g\circ f\simeq i_{X}$ y $f\circ g\simeq i_{Y}$. Las
  funciones $f$ y $g$ son llamadas\textbf{ equivalencias homotópicas},
  y $g$ es la \textbf{inversa homotópica} de $f$. 
\end{definition}
Si $X$ es homotópico a un punto, se dice que $X$ es
\textbf{contraíble}.

\begin{theorem}{[\cite{munkres1984elements}, teorema 19.5, p.108]}.
  Sean $X$, $Y$ espacios topológicos, si $X$ y $Y$ son homotópicos, entonces $\widetilde H_{n}(X)\cong \widetilde
  H_{n}(Y)$ para $n>0$. En particular, si $X$ es contraíble, entonces $\widetilde
  H_{n}(X)=0$ para $n>0$.
  \label{esp-homotopicos-homologias-iso}
\end{theorem}

\begin{theorem}{[\cite{munkres1984elements}, teorema 34.3, p.194]}.
  Sea $\Delta$ un complejo simplicial, entonces la homología
  simplicial de $\Delta$ es isomorfa a la homología singular de su
  realización geométrica~$|\Delta|$.
  \label{homologia-realizacion}
\end{theorem}
 
A cada gráfica $G$ finita se le asocia un espacio topológico por
medio del complejo simplicial $\Delta(G)$ cuyos simplejos son
las subgráficas completas. Entonces a $G$ le podemos asociar conceptos topológicos por
medio de $\Delta(G)$. Por ejemplo, escribiremos $G_{1}\simeq G_{2}$ si
se tiene que $\Delta(G_{1})\simeq\Delta(G_{2})$.

Por otro lado si $G$ tiene una acción del grupo simétrico
$S_{n}$, entonces las homologías de $G$ son módulos
de $S_{n}$.

Ahora, se ejemplifican los conceptos antes expuestos.
\begin{example}
  Sea la gráfica $G$ de la figura \ref{G1}, entonces $\Delta(G)$ consta de los
  subconjuntos no vacíos de $\{1,2,4\}$, $\{2,3,5\}$, $\{2,4,5\}$,
  $\{4,5,6\}$. En la figura \ref{fig:realizacion-geometrica} se
  observa un dibujo de $|\Delta(G)|$. Así que $|\Delta(G)|\simeq D^{2}$, donde $D^{2}=\{x\in\mathbb{R}^{2}
\mid |x|\leq 1\}$.

\begin{center}
    \begin{figure}[h]
      \begin{minipage}[h]{0.45\linewidth}
        \centering
        \begin{tikzpicture}[scale=.8]
          \GraphInit[vstyle=Classic] \SetUpVertex[MinSize=1pt]
          \SetVertexNoLabel
          \grTriangularGrid[prefix=G,Math,RA=1.5]{3}%
        \end{tikzpicture}

        \caption{$G$}
        \label{G1}
      \end{minipage}  
      \begin{minipage}[h]{0.45\linewidth}
        \centering
        \begin{tikzpicture}[scale=0.6]%[x=0.5 cm,y=1cm]
          \tikzstyle{circ} = [circle, minimum width=1mm, inner
          sep=0pt,draw,fill] \tikzstyle{num} = [yshift=4mm]
          % QUAD
          \node[circ] (a) at (0,0) {}; \node[circ] (b) at (2,0) {};
          \node[circ] (c) at (1,2) {}; \node[circ] (d) at (3,2) {};
          \node[circ] (e) at (4,0) {}; \node[circ] (f) at (2,4) {};
          \begin{pgfonlayer}{background}
            \fill[draw, line width=0.3mm,top color=black!30,bottom
            color=gray!30] (a.center)node[num]{1} -- (b.center) node[num]{2}
            -- (c.center) node[num]{4} --
            (a.center);%node[num]{4} -- (a.center);
          \end{pgfonlayer}
          \begin{pgfonlayer}{background}
            \fill[draw, line width=0.3mm,top color=black!30,bottom
            color=gray!30] (b.center)node[num]{2} -- (e.center) node[num]{3}
            -- (d.center) node[num]{5} --
            (b.center);%node[num]{4} -- (a.center);
          \end{pgfonlayer}
          \begin{pgfonlayer}{background}
            \fill[draw, line width=0.3mm,top color=black!30,bottom
            color=gray!30](b.center)node[num]{2} -- (d.center) node[num]{5}
            -- (c.center) node[num]{4} --
            (b.center);%node[num]{4} -- (a.center);
          \end{pgfonlayer}
          \begin{pgfonlayer}{background}
            \fill[draw, line width=0.3mm,top color=black!30,bottom
            color=gray!30] (c.center)node[num]{4} -- (d.center) node[num]{5}
            -- (f.center) node[num]{6} --
            (c.center);%node[num]{4} -- (a.center);
          \end{pgfonlayer}
        \end{tikzpicture}
        
        \caption{$|\Delta(G)|$}
        \label{fig:realizacion-geometrica}
      \end{minipage}
    \end{figure}
  \end{center}
\end{example}

\chapter[Homologías de $M_{n}$ y $K(M_{n})$]{Homologías del complejo de emparejamientos}
\label{cha:hom-com-emp}

En este capítulo se utilizarán todos los conceptos y resultados
presentados en los capítulos anteriores. Se definirá el complejo de
emparejamientos y finalmente conoceremos la
fórmula de Bouc. Además, calcularemos los módulos de homología reducida de los
complejos de emparejamientos y de sus respectivas gráficas de clanes para
$n\leq 6$.

\section{Complejo de emparejamientos}
\label{complejo-emparejamientos}

Recordemos que denotamos como $G_{n}$ a la gráfica de
emparejamientos de orden $n$ (ver definición \ref{graf-emparejamientos}).

\begin{definition}
El \textbf{complejo de emparejamientos} de $K_{n}$ será denotado
entonces por $M_{n}$, y se define como $M_{n}=\Delta(G_{n})$
\end{definition}

Puesto que $S_{n}$ actúa de manera
natural en $G_{n}$, los espacios $C_{p}(M_{n})$, $Z_{p}(M_{n})$,
$B_{p}(M_{n})$ y por tanto $\widetilde H_{p}(M_{n})$ se convierten en
$S_{n}$-módulos sobre~$\mathbb{C}$. En este capítulo consideramos
la descomposición en módulos irreducibles de
$\widetilde H_{p}(M_{n})=\widetilde H_{p}(\Delta(G_{n}))$, para
$n=4,5,6$ y $\widetilde H_{p}(\Delta(K(G_{n})))$, para $n=5,6$.

\begin{example}
  \label{ejemploM4}
  Considérese la gráfica $K_{4}$; construiremos el complejo de
  emparejamientos $M_{4}$ dado por el conjunto de vértices
  $$V=\{\overline{12},\overline{13},\overline{14},\overline{23},\overline{24},\overline{34}\},$$
  que es el conjunto de aristas de la gráfica de $K_{4}$. La familia de
  $1$-simplejos estará dada por el siguiente conjunto:
  $$\{\{\overline{12},\overline{34}\},\{\overline{13},\overline{24}\},\{\overline{14},\overline{23}\}\}.$$ 
  Es decir,
\begin{equation*}
  M_{4}=\{\{\overline{12}\},\{\overline{13}\},\{\overline{14}\},\{\overline{23}\},\{\overline{24}\},\{\overline{34}\},\{\overline{12},\overline{34}\},\{\overline{13},\overline{24}\},\{\overline{14},\overline{23}\}\}.
\end{equation*}

\begin{center}
  \begin{minipage}{0.26\linewidth}
    \centering
    \begin{tikzpicture}[x=0.8 cm,y=0.8 cm]
      \draw[help lines] (-2,0);% grid (0,2);
      \GraphInit[vstyle=Classic] \SetUpVertex[MinSize=1pt]
      \Vertex[x=-2,y=0,Math,LabelOut,Lpos=180]{2}
      \Vertex[x=0,y=0,Math]{3}
      \Vertex[x=-2,y=2,Math,LabelOut,Lpos=180]{1}
      \Vertex[x=0,y=2,Math]{4} \Edge(1)(2) \Edge(1)(3) \Edge(1)(4)
      \Edge(2)(3) \Edge(2)(4) \Edge(3)(4)
    \end{tikzpicture}
  
    $K_{4}$
  \end{minipage}
  \begin{minipage}{0.26\linewidth}
    \centering
    \begin{tikzpicture}[x=0.8 cm,y=0.8 cm]
      \draw[help lines] (-2,0);% grid (0,2);
      \GraphInit[vstyle=Classic] \SetUpVertex[MinSize=1pt]
      \Vertex[x=-2,y=2,Math,LabelOut,Lpos=90,L=\overline{12}]{12}
      \Vertex[x=-2,y=0,Math,LabelOut,Lpos=-90,L=\overline{34}]{34}
      \Vertex[x=-1,y=0,Math,LabelOut,Lpos=-90,L=\overline{24}]{24}
      \Vertex[x=-1,y=2,Math,LabelOut,Lpos=90,L=\overline{13}]{13}
      \Vertex[x=0,y=2,Math,LabelOut,Lpos=90,L=\overline{14}]{14}
      \Vertex[x=0,y=0,Math,LabelOut,Lpos=-90,L=\overline{23}]{23} \Edge(12)(34)
      \Edge(24)(13) \Edge(14)(23)
    \end{tikzpicture}
  
    $G_{4}$
  \end{minipage}
  \begin{minipage}{0.38\linewidth}
    \centering
    \newcommand{\arista}[2]{$\overline{#1},\overline{#2}$}

    \begin{tikzpicture}[x=1 cm,y=0.8 cm]
      \draw[help lines] (-2,0);% grid (2,0);
      \GraphInit[vstyle=Classic] \SetUpVertex[MinSize=1pt]
      \SetVertexNoLabel
      \Vertex[x=-2,y=0]{x}
      \Vertex[x=0,y=0]{y} 
      \Vertex[x=2,y=0]{z}
      \draw (x) node[below=3pt]{\arista{12}{34}};
      \draw (y) node[below=3pt]{\arista{13}{24}};
      \draw (z) node[below=3pt]{\arista{14}{24}};
    \end{tikzpicture}

    $K(G_{4})$
  \end{minipage}
\end{center}
\end{example}
El teorema \ref{bouc} que se enuncia a continuación, nos muestra la descomposición en módulos de
Specht de los módulos de homología reducida de complejos de emparejamientos $\widetilde H_{p}(M_{n})$. 

\begin{theorem}[Teorema de Bouc, \cite{MR756517}]
Para todos $n,k\geq1$:
\begin{equation*}
  % \label{eq:6}
  \widetilde H_{k-1}(M_{n})\cong_{S_{n}}\bigoplus_{\substack{\lambda:\lambda\vdash n\\
      \lambda=\lambda^{'}\\d(\lambda)=n-2k}} S^{\lambda}.
\end{equation*}
\label{bouc}
\end{theorem}

\begin{example}
  \quad

  Sea $n=3$ y $k=1$, \ytableausetup{boxsize=0.2em}
  \begin{equation*}
    \widetilde H_{0}(G_{3})\cong_{S_{3}}\bigoplus_{\substack{\lambda:\lambda\vdash 3\\
        \lambda=\lambda^{'}\\d(\lambda)=3-2(1)=1}} S^{\lambda}=S^{(2,1)}=S^{\,\ydiagram{2,1}}.
  \end{equation*}
          
  Sea $n=4$ y $k=1$,
  \begin{equation*}
    \widetilde H_{0}(G_{4})\cong_{S_{4}}\bigoplus_{\substack{\lambda:\lambda\vdash 4\\
        \lambda=\lambda^{'}\\d(\lambda)=4-2(1)=2}} S^{\lambda}=S^{(2,2)}=S^{\,\ydiagram{2,2}}.
  \end{equation*}

  Sea $n=5$ y $k=2$,
  \begin{equation*}
    \widetilde H_{1}(G_{5})\cong_{S_{5}}\bigoplus_{\substack{\lambda:\lambda\vdash 5\\
        \lambda=\lambda^{'}\\d(\lambda)=5-2(2)=1}} S^{\lambda}=S^{(3,1,1)}=S^{\,\ydiagram{3,1,1}}.
  \end{equation*}
        
  Sea $n=6$ y $k=2$,
  \begin{equation*}
    \widetilde H_{1}(G_{6})\cong_{S_{6}}\bigoplus_{\substack{\lambda:\lambda\vdash 6\\
        \lambda=\lambda^{'}\\d(\lambda)=6-2(2)=2}} S^{\lambda}=S^{(3,2,1)}=S^{\,\ydiagram{3,2,1}}.
  \end{equation*}
\end{example}

En los ejemplos de las próximas secciones se calculan explícitamente
por métodos elementales los siguientes módulos de homología reducida: $\widetilde
H_{p}(M_{})$, para $n=4,5,6$ y $\widetilde H_{p}(\Delta(K(G_{n})))$, para $n=5,6$.

\section{Módulos de homología reducida de $M_{4}$}
\label{hom-red-M4}

Enseguida, obtendremos los módulos de homología reducida de $M_{4}$. Tomando la
notación de los $p$-simplejos orientados, los $S_{4}$-módulos de cadenas son:
\begin{equation*}
  C_{0}(M_{4})=\langle(\overline{12}),(\overline{13}),(\overline{14}),(\overline{23}),(\overline{24}),(\overline{34})\rangle.
\end{equation*}
\begin{equation*}
 C_{1}(M_{4})=\langle(\overline{12},\overline{34}),(\overline{13},\overline{24}),(\overline{14},\overline{23})\rangle.
\end{equation*}
Sean
\begin{center}
  \begin{tabular}{ccc}
    $a_{1}=\overline{12}$ & $a_{4}=\overline{23}$ & $b_{1}=(\overline{12},\overline{34})$\\
    $a_{2}=\overline{13}$ & $a_{5}=\overline{24}$ & $b_{2}=(\overline{13},\overline{24})$\\
    $a_{3}=\overline{14}$ & $a_{6}=\overline{34}$ & $b_{3}=(\overline{14},\overline{23})$.\\
  \end{tabular}
\end{center}
Consideremos a $\beta_{0}=\{a_{1},a_{2},a_{3},a_{4},a_{5},a_{6}\}$ y
$\beta_{1}=\{b_{1},b_{2},b_{3}\}$ como las bases de $C_{0}(M_{4})$ y
$C_{1}(M_{4})$ respectivamente. Notemos que $\dim C_{0}(M_{4})=6$ y
$\dim C_{1}(M_{4})=3$. A continuación hacemos actuar un representante de cada
clase de conjugación del grupo simétrico $S_{4}$ sobre los elementos
de la base de $C_{0}(M_{4})$ y $C_{1}(M_{4})$.
\begin{center}
  \begin{tabular}{llll}
    $(12)a_{1}=a_{1}$ & $(123)a_{1}=a_{4}$ & $(1234)a_{1}=a_{4}$ & $(12)(34)a_{1}=a_{1}$ \\
    $(12)a_{2}=a_{4}$ & $(123)a_{2}=a_{1}$ & $(1234)a_{2}=a_{5}$ & $(12)(34)a_{2}=a_{5}$ \\
    $(12)a_{3}=a_{5}$ & $(123)a_{3}=a_{5}$ & $(1234)a_{3}=a_{1}$ & $(12)(34)a_{3}=a_{4}$ \\
    $(12)a_{4}=a_{2}$ & $(123)a_{4}=a_{2}$ & $(1234)a_{4}=a_{6}$ & $(12)(34)a_{4}=a_{3}$ \\
    $(12)a_{5}=a_{3}$ & $(123)a_{5}=a_{6}$ & $(1234)a_{5}=a_{2}$ & $(12)(34)a_{5}=a_{2}$ \\
    $(12)a_{6}=a_{6}$ & $(123)a_{6}=a_{3}$ & $(1234)a_{6}=a_{3}$ & $(12)(34)a_{6}=a_{6}$. \\
  \end{tabular}
\end{center}

\begin{center}
  \begin{tabular}{llll}
    $(12)b_{1}=b_{1}$  & $(123)b_{1}=-b_{3}$ & $(1234)b_{1}=-b_{3}$ & $(12)(34)b_{1}=b_{1}$ \\
    $(12)b_{2}=-b_{3}$ & $(123)b_{2}=b_{1}$  & $(1234)b_{2}=-b_{2}$ & $(12)(34)b_{2}=-b_{2}$ \\
    $(12)b_{3}=-b_{2}$ & $(123)b_{3}=-b_{2}$ & $(1234)b_{3}=b_{1}$  & $(12)(34)b_{3}=-b_{3}$. \\
  \end{tabular}
\end{center}
Entonces tenemos las representaciones $\theta_{1}$ y $\theta_{2}$ de
$S_{4}$ en $C_{0}(M_{4})$ y $C_{1}(M_{4})$ definidas respectivamente como se
muestra enseguida:

\begin{center}
  $\theta_{1}(12)= \left(
    \begin{array}{rrrrrr}
      1 & 0 & 0 & 0 & 0 & 0\\
      0 & 0 & 0 & 1 & 0 & 0\\
      0 & 0 & 0 & 0 & 1 & 0\\
      0 & 1 & 0 & 0 & 0 & 0\\
      0 & 0 & 1 & 0 & 0 & 0\\
      0 & 0 & 0 & 0 & 0 & 1\\
    \end{array} 
  \right)$,\quad 
  $\theta_{1}(123)= \left(
    \begin{array}{rrrrrr}
      0 & 1 & 0 & 0 & 0 & 0\\
      0 & 0 & 0 & 1 & 0 & 0\\
      0 & 0 & 0 & 0 & 0 & 1\\
      1 & 0 & 0 & 0 & 0 & 0\\
      0 & 0 & 1 & 0 & 0 & 0\\
      0 & 0 & 0 & 0 & 1 & 0\\
    \end{array} 
  \right)$,
\end{center}

\begin{center}
  $\theta_{1}(1234)= \left(
    \begin{array}{rrrrrr}
      0 & 0 & 1 & 0 & 0 & 0\\
      0 & 0 & 0 & 0 & 1 & 0\\
      0 & 0 & 0 & 0 & 0 & 1\\
      1 & 0 & 0 & 0 & 0 & 0\\
      0 & 1 & 0 & 0 & 0 & 0\\
      0 & 0 & 0 & 1 & 0 & 0\\
    \end{array} 
  \right)$, \quad
  $\theta_{1}(12)(34)= \left(
    \begin{array}{rrrrrr}
      1 & 0 & 0 & 0 & 0 & 0\\
      0 & 0 & 0 & 0 & 1 & 0\\
      0 & 0 & 0 & 1 & 0 & 0\\
      0 & 0 & 1 & 0 & 0 & 0\\
      0 & 1 & 0 & 0 & 0 & 0\\
      0 & 0 & 0 & 0 & 0 & 1\\
    \end{array} 
  \right)$, 
\end{center}

\begin{center}
  $\theta_{2}(12)= \left(
    \begin{array}{rrr}
      1 & 0 & 0 \\
      0 & 0 & -1 \\
      0 & -1 & 0 \\
    \end{array} 
  \right)$, \quad
  $\theta_{2}(123)= \left(
    \begin{array}{rrr}
      0 & 1 & 0 \\
      0 & 0 & -1 \\
      -1 & 0 & 0 \\
    \end{array} 
  \right)$,
\end{center}

\begin{center}
  $\theta_{2}(1234)= \left(
    \begin{array}{rrr}
      0 & 0 & 1 \\
      0 & -1 & 0 \\
      -1 & 0 & 0 \\
    \end{array} 
  \right)$, \quad
  $\theta_{2}(12)(34)= \left(
    \begin{array}{rrr}
      1 & 0 & 0 \\
      0 & -1& 0 \\
      0 & 0 & -1 \\
    \end{array} 
  \right)$,
\end{center}
de tal forma que:

\begin{tabular}{r r r}
  $\chi_{C_{0}(M_{4})}((1))=6$, & $\chi_{C_{0}(M_{4})}((12))=2$, & $\chi_{C_{0}(M_{4})}((123))=0$, \\
  $\chi_{C_{0}(M_{4})}((1234))=0$, & $\chi_{C_{0}(M_{4})}((12)(34))=2$, & \\
\end{tabular}
\bigskip	

\begin{tabular}{r r r}
  $\chi_{C_{1}(M_{4})}((1))=3$, & $\chi_{C_{1}(M_{4})}((12))=1$, & $\chi_{C_{1}(M_{4})}((123))=0$, \\
  $\chi_{C_{1}(M_{4})}((1234))=-1$, & $\chi_{C_{1}(M_{4})}((12)(34))=-1$. & \\
\end{tabular}
\medskip

Añadiendo estos dos últimos caracteres a la tabla de caracteres de
$S_{4}$ tenemos la tabla~\ref{tab:S_4}.

\begin{table}[htpb]
  \centering
  \begin{tabular}{c|r r r r r}
    No. Elementos & 1 & 6 & 8 & 6 & 3 \\
    Clase & (1) & (12) & (123) & (1234) &(12)(34)\\
    \hline
    $\chi_{\mathbb{C}}$ & 1 & 1 & 1 & 1 & 1 \\
    $\chi_{S^{(1,1,1,1)}}$ & 1 & -1 & 1 & -1 & 1\\
    $\chi_{S^{(3,1)}}$ & 3 & 1 & 0 & -1 & -1\\
    $\chi_{S^{(2,1,1)}}$ & 3 & -1 & 0 & 1 & -1 \\
    $\chi_{S^{(2,2)}}$ & 2 & 0 & -1 & 0 & 2 \\
    \hline
    $\chi_{C_{0}(M_{4})}$ & 6 & 2 & 0 & 0 & 2 \\
    $\chi_{C_{1}(M_{4})}$ & 3 & 1 & 0 & -1 & -1
  \end{tabular}

\caption{Tabla de caracteres de $S_{4}$, $C_{0}(M_{4})$ y $C_{1}(M_{4})$.}
\label{tab:S_4}
\end{table}

Queremos escribir a $C_{0}(M_{4})$ como suma directa de módulos
irreducibles de $S_{4}$, así que calculemos el producto interno de $\chi_{C_{0}(M_{4})}$ con los
caracteres de los módulos irreducibles de $S_{4}$ para conocer la
multiplicidad con la que aparecen estos últimos (corolario \ref{multiplicidad}).
\begin{align*}
  \langle\chi_{C_{0}(M_{4})},\chi_{\mathbb{C}}\rangle &=\frac{1}{24}((1)(1\cdot6)+(6)(1\cdot2)+(8)(1\cdot0)+(6)(1\cdot0)+(3)(1\cdot2))\\
  &=\frac{1}{24}(6+12+6)=1,\\
  \langle\chi_{C_{0}(M_{4})},\chi_{S^{(1,1,1,1)}}\rangle &=\frac{1}{24}((1)(1\cdot6)+(6)(-1\cdot2)+(8)(1\cdot0)+(6)(-1\cdot0)+(3)(1\cdot2))\\
  &=\frac{1}{24}(6-12+6)=0,\\
  \langle\chi_{C_{0}(M_{4})},\chi_{S^{(3,1)}}\rangle &=\frac{1}{24}((1)(3\cdot6)+(6)(1\cdot2)+(8)(0\cdot0)+(6)(-1\cdot0)+(3)(-1\cdot2))\\
  &=\frac{1}{24}(18+12-6)=1,\\
  \langle\chi_{C_{0}(M_{4})},\chi_{S^{(2,1,1)}}\rangle &=\frac{1}{24}((1)(3\cdot6)+(6)(-1\cdot2)+(8)(0\cdot0)+(6)(1\cdot0)+(3)(-1\cdot2))\\
  &=\frac{1}{24}(18-12-6)=0,\\
  \langle\chi_{C_{0}(M_{4})},\chi_{S^{(2,2)}}\rangle &=\frac{1}{24}((1)(2\cdot6)+(6)(0\cdot2)+(8)(-1\cdot0)+(6)(0\cdot0)+(3)(2\cdot2))\\
  &=\frac{1}{24}(12+12)=1.
\end{align*}
De lo anterior se sigue:
\begin{equation}
  \label{eq:C_0(M_4)}
  C_{0}(M_{4})\cong \mathbb{C}\oplus S^{(3,1)}\oplus S^{(2,2)}.
\end{equation}
De la tabla \ref{tab:S_4},
se observa que $\chi_{C_{1}(M_{4})}=\chi_{S^{(3,1)}}$, así que
por el corolario \ref{multiplicidad} inciso 5 tenemos: 
\begin{equation}
  \label{eq:C_1(M_4)}
  C_{1}(M_{4})\cong S^{(3,1)}.
\end{equation}
Con lo cual tenemos el siguiente complejo de cadenas aumentado
donde $\partial_{1},\partial_{2},\varepsilon$ son los correspondientes
operadores frontera y la función aumento:

\begin{small}
  \[
    \begin{array}{ccccccccccccc}
      \dots 0 & \rightarrow & 0 &
      \stackrel{\partial_{2}}{\rightarrow} & C_{1}(M_{4}) &
      \stackrel{\partial_{1}}{\rightarrow} & C_{0}(M_{4}) & \stackrel{\varepsilon}{\rightarrow} &
      \mathbb{C} & \rightarrow  & 0 & \rightarrow & 0 \dots
    \end{array} 
    \]
  \end{small}
Sean $f_{0}$ y $f_{1}$ los isomorfismos obtenidos de las expresiones
\ref{eq:C_0(M_4)}, \ref{eq:C_1(M_4)} respectivamente, y definimos en
general al homomorfismo $\widehat\partial_{k}$ como
$\widehat\partial_{k}=f_{k-1}\circ \partial_{k}\circ f^{-1}_{k}$, con
lo cual tenemos el complejo de cadenas aumentado: 

\begin{small}
    \[
    \begin{array}{ccccccccccccc}
      \dots 0 & \rightarrow & 0 &
      \stackrel{\widehat\partial_{2}}{\rightarrow} &  S^{(3,1)} &
      \stackrel{\widehat\partial_{1}}{\rightarrow} & \mathbb{C} \oplus
      S^{(3,1)}\oplus S^{(2,2)} & \stackrel{\widehat\varepsilon}{\rightarrow} &
      \mathbb{C} & \rightarrow  & 0 & \rightarrow & 0 \dots
    \end{array} 
    \]
  \end{small}

En la figura \ref{fig:diagrama-conmutativo4} se muestra diagrama
conmutativo de los complejos de cadenas anteriores.
\begin{figure}[!hbtp]
  \centering
  \[
  \begin{CD}
    0 @>{\partial_{2}}>> C_{1}(M_{4}) @>{\partial_{1}}>> C_{0}(M_{4}) @>{\varepsilon}>> \mathbb{C}\\
    @VVV   @V{f_{1}}VV   @V{f_{0}}VV   @VVV    \\
    0 @>{\widehat\partial_{2}}>> S^{(3,1)} @>{\widehat\partial_{1}}>>
    \mathbb{C} \oplus S^{(3,1)}\oplus S^{(2,2)} @>{\widehat
      \varepsilon}>> \mathbb{C}
  \end{CD}
  \]
  
  \caption{Diagrama conmutativo de los complejos de cadenas de $M_{4}$}
\label{fig:diagrama-conmutativo4}
\end{figure}

Ahora calculemos los módulos de homología reducida $\widetilde H_{0}(M_{4})$ y
$\widetilde H_{1}(M_{4})$.

Por el teorema \ref{teorema-isomorfismo-mod} y como
$\widehat\varepsilon$ es sobreyectiva tenemos:
$$(\mathbb{C} \oplus S^{(3,1)}\oplus S^{(2,2)})/\ker\widehat\varepsilon\cong\im\widehat\varepsilon=\mathbb{C},$$
entonces
\begin{equation}
\label{ker-0-4}
\ker\widehat\varepsilon\cong S^{(3,1)}\oplus S^{(2,2)}.
\end{equation}
Por otra parte, de la proposición \ref{im-mod-irreducible}, $\im\widehat\partial_{1}\cong S^{(3,1)}$ o $\im\widehat\partial_{1}=0$.

Como $\partial_{1}((\overline{12},\overline{34}))=\overline{34}-\overline{12}$,
tenemos que $\im\partial_{1}\neq 0$ (en general $\partial_{k}\neq 0$
por definición del operador frontera), se sigue que
$\im\widehat\partial_{1}\neq 0$ puesto que el diagrama es
conmutativo. Por lo tanto, 

% Si $\partial_{1}(C_{1}(M_{4}))=0$ entonces $f_{0}(\partial_{1}(C_{1}(M_{4})))=0$,
% pues $f_{0}$ es morfismo, como el diagrama es conmutativo se tiene
% $f_{0}(\partial_{1}(C_{1}(M_{4})))=\widehat\partial_{1}(f_{1}(C_{1}(M_{4})))$,
% además $\widehat\partial_{1}(f_{1}(C_{1}(M_{4})))=\widehat\partial_{1}(S^{(3,1)})$,
% pues $f_{1}$ es suprayectiva, por lo tanto $\widehat\partial_{1}(S^{(3,1)})=0$.
\begin{equation}
  \label{im-1-4}
  \im\widehat\partial_{1}\cong S^{(3,1)}.
\end{equation}

De las expresiones \ref{ker-0-4}, \ref{im-1-4} y de la
proposición \ref{modulos-iguales} concluimos:
\begin{equation}
  \label{ker-0-4=}
  \ker\widehat\varepsilon=S^{(3,1)}\oplus S^{(2,2)}.
\end{equation}
\begin{equation}
\label{im-1-4=}
\im\widehat\partial_{1}=S^{(3,1)}.
\end{equation}

Nuevamente por el teorema \ref{teorema-isomorfismo-mod},
$$S^{(3,1)}/\ker\widehat\partial_{1}\cong\im\widehat\partial_{1}= S^{(3,1)},$$
entonces
\begin{equation}
  \ker\widehat\partial_{1}=0.
  \label{ker-1-4}
\end{equation}

Como $\widehat\partial_{2}$ es un homomorfismo de módulos, tenemos
\begin{equation}
  \im\widehat\partial_{2}=\widehat\partial_{2}(0)=0.
  \label{im-2-4}
\end{equation}
Por lo tanto, de la ecuaciones \ref{ker-0-4=}, \ref{im-1-4=},
\ref{ker-1-4} y \ref{im-2-4} obtenemos:
\begin{align*}
\widetilde H_{0}(M_{4})&=\ker \widehat\varepsilon/\im
\widehat\partial_{1}=(S^{(3,1)}\oplus S^{(2,2)})/S^{(3,1)}=S^{(2,2)},\\
\widetilde H_{1}(M_{4})&=\ker \widehat\partial_{1}/\im \widehat\partial_{2}=0/0=0.
\end{align*}

\section{Módulos de homología reducida de $M_{5}$}
\label{hom-red-M5}

Consideremos ahora la gráfica $G_{5}$ (ver figura~\ref{fig:G_5}) y el complejo de emparejamientos $M_{5}$, que
está dado por el conjunto de vértices (aristas de la gráfica de $K_{5}$):
$$V=\{a_{1},a_{2},a_{3},a_{4},a_{5},a_{6},a_{7},a_{8},a_{9},a_{10}\},$$
donde
\begin{table}[!hbtp]
  \centering
  \begin{tabular}{lllll}
    $a_{1}=\overline{12}$ & $a_{2}=\overline{13}$ & $a_{3}=\overline{14}$ & $a_{4}=\overline{15}$ & $a_{5}=\overline{23}$ \\
    $a_{6}=\overline{24}$ & $a_{7}=\overline{25}$ & $a_{8}=\overline{34}$ & $a_{9}=\overline{35}$ & $a_{10}=\overline{45}$.
  \end{tabular}
\end{table}

Entonces la familia de $1$-simplejos orientados estará dada por:
\begin{center}
  \begin{tabular}[h]{lll}
    $b_{1}=(a_{1},a_{8})$ & $b_{6}=(a_{2},a_{10})$ & $b_{11}=(a_{4},a_{6})$  \\
    $b_{2}=(a_{1},a_{9})$ & $b_{7}=(a_{3},a_{5})$ & $b_{12}=(a_{4},a_{8})$  \\
    $b_{3}=(a_{1},a_{10})$ & $b_{8}=(a_{3},a_{7})$ & $b_{13}=(a_{5},a_{10})$  \\
    $b_{4}=(a_{2},a_{6})$ & $b_{9}=(a_{3},a_{9})$ & $b_{14}=(a_{6},a_{9})$  \\
    $b_{5}=(a_{2},a_{7})$ & $b_{10}=(a_{4},a_{5})$ & $b_{15}=(a_{7},a_{8})$.  
  \end{tabular}
\end{center}
\begin{figure}[!hbtp]
  \centering
  \begin{tikzpicture}[rotate=90,scale=0.75]
    \newcommand{\aset}[2]{$\{#1,#2\}$} \GraphInit[vstyle=Classic]
    \SetUpVertex[MinSize=17pt] \SetVertexNoLabel \SetVertexMath
    \grPetersen[RA=3,RB=1.5]
    \AssignVertexLabel{a}{\textsl{$\overline{12}$},\textsl{$\overline{34}$},\textsl{$\overline{15}$},\textsl{$\overline{23}$},\textsl{$\overline{45}$}}
    \AssignVertexLabel{b}{\textsl{$\overline{35}$},\textsl{$\overline{25}$},\textsl{$\overline{24}$},\textsl{$\overline{14}$},\textsl{$\overline{13}$}}
  \end{tikzpicture}
  
  \caption{Gráfica de emparejamientos $G_{5}$}
  \label{fig:G_5}
\end{figure}
\begin{table}[!hbtp]
  \centering
  \begin{small}
    \begin{tabular}{c |r r r r r r r}
      No. Elementos& 1 & 10 & 20 & 30 & 24 & 15 & 20  \\
      Clase & (1) & (12) & (123) & (1234) & (12345) & (12)(34) & (123)(45) \\
      \hline
      $\chi_{S^{(5)}}$       & 1 & 1 & 1 & 1 & 1 & 1 & 1 \\
      $\chi_{S^{(1,1,1,1,1)}}$ & 1 & -1 & 1 & -1 & 1 & 1 & -1\\
      $\chi_{S^{(4,1)}}$      & 4 & 2 & 1 & 0 & -1 & 0 & -1\\
      $\chi_{S^{(2,1,1,1)}}$   & 4 & -2 & 1 & 0 & -1 & 0 & 1 \\
      $\chi_{S^{(3,1,1)}}$    & 6 & 0 & 0 & 0 & 1 & -2 & 0 \\
      $\chi_{S^{(3,2)}}$     & 5 & 1 & -1 & -1 & 0 & 1 & 1 \\
      $\chi_{S^{(2,2,1)}}$   & 5 & -1 & -1 & 1 & 0 & 1 & -1 
    \end{tabular}
  \end{small}

  \caption{Tabla de caracteres de $S_{5}$}
  \label{tab:S_5}
\end{table}

Con el teorema de la reciprocidad de Frobenius (teorema
\ref{frobenius}) obtendremos la descomposición de los $S_{5}$-módulos de
cadenas $C_{0}(M_{5})$ y $C_{1}(M_{5})$ en módulos de Specht con un
menor número de cálculos. 

Consideremos:
\begin{eqnarray*}
  V_{0}&=&\langle\overline{45}\rangle\\
  H_{0}&=&\{g\in S_{5}\mid gV_{0}=V_{0}\}=\{\langle(1),(12),(123),(45),(12)(45),(123)(45)\rangle\}\\
\end{eqnarray*}

Entonces $C_{0}(M_{5})=V_{0}\uparrow^{S_{5}}_{H_0}$, por lo que:

\begin{table}[!hbtp]
  \centering
    \begin{tabular}{c |r r r r r r}
      No. Elementos& 1 & 1 & 3 & 3 & 2 & 2 \\
      Clase & (1) & (45) & (12) & (12)(45) & (123) & (123)(45) \\
      \hline
      $\chi_{S^{(5)}\downarrow_{H_{0}}}$       & 1 & 1 & 1 & 1 & 1 & 1 \\
      $\chi_{S^{(1,1,1,1,1)}\downarrow_{H_{0}}}$ & 1 & -1 & -1 & 1 & 1 & -1 \\
      $\chi_{S^{(4,1)}\downarrow_{H_{0}}}$      & 4 & 2 & 2 & 0 & 1 & -1 \\
      $\chi_{S^{(2,1,1,1)}\downarrow_{H_{0}}}$   & 4 & -2 & -2 & 0 & 1 & 1 \\
      $\chi_{S^{(3,1,1)}\downarrow_{H_{0}}}$     & 6 & 0 & 0 & -2 & 0 & 0 \\
      $\chi_{S^{(3,2)}\downarrow_{H_{0}}}$      & 5 & 1 & 1 & 1 & -1 & 1 \\
      $\chi_{S^{(2,2,1)}\downarrow_{H_{0}}}$    & 5 & -1 & -1 & 1 & -1 & -1 \\
      \hline
      $\chi_{V_{0}}$ & 1 & 1 & 1 & 1 & 1 & 1 \\
    \end{tabular}

\caption{Caracteres de $S_{5}$ restringidos a $H_{0}$ y carácter de $V_{0}$}
\label{tab:restriccion-H_0}
\end{table}

%{\scriptsize
  \begin{align*}
    \langle\chi_{C_{0}(M_{5})},\chi_{S^{(5)}}\rangle_{S_{5}}&=\langle\chi_{V_{0}\uparrow^{S_{5}}_{H_0}},\chi_{S^{(5)}}\rangle_{S_{5}}=\langle\chi_{V_{0}},\chi_{S^{(5)}\downarrow_{H_{0}}}\rangle_{H_{0}}\\ 
    &=\frac{1}{12}(1+1+3+3+2+2)=1,\\ 
    \langle\chi_{C_{0}(M_{5})},\chi_{S^{(1,1,1,1,1)}}\rangle_{S_{5}}&=\langle\chi_{V_{0}\uparrow^{S_{5}}_{H_0}},\chi_{S^{(1,1,1,1,1)}}\rangle_{S_{5}}=\langle\chi_{V_{0}},\chi_{S^{(1,1,1,1,1)}\downarrow_{H_{0}}}\rangle_{H_{0}}\\
    &=\frac{1}{12}(1-1-3+3+2-2)=0, \\
    \langle\chi_{C_{0}(M_{5})},\chi_{S^{(4,1)}}\rangle_{S_{5}}&=\langle\chi_{V_{0}\uparrow^{S_{5}}_{H_0}},\chi_{S^{(4,1)}}\rangle_{S_{5}}=\langle\chi_{V_{0}},\chi_{S^{(4,1)}\downarrow_{H_{0}}}\rangle_{H_{0}}\\
    &=\frac{1}{12}(4+2+6+0+2-2)=1, \\
    \langle\chi_{C_{0}(M_{5})},\chi_{S^{(2,1,1,1)}}\rangle_{S_{5}}&=\langle\chi_{V_{0}\uparrow^{S_{5}}_{H_0}},\chi_{S^{(2,1,1,1)}}\rangle_{S_{5}}=\langle\chi_{V_{0}},\chi_{S^{(2,1,1,1)}\downarrow_{H_{0}}}\rangle_{H_{0}}\\
    &=\frac{1}{12}(4-2-6+0+2+2)=0, \\
    \langle\chi_{C_{0}(M_{5})},\chi_{S^{(3,1,1)}}\rangle_{S_{5}}&=\langle\chi_{V_{0}\uparrow^{S_{5}}_{H_0}},\chi_{S^{(3,1,1)}}\rangle_{S_{5}}=\langle\chi_{V_{0}},\chi_{S^{(3,1,1)}\downarrow_{H_{0}}}\rangle_{H_{0}}\\
    &=\frac{1}{12}(6+0+0-6+0+0)=0, \\
    \langle\chi_{C_{0}(M_{5})},\chi_{S^{(3,2)}}\rangle_{S_{5}}&=\langle\chi_{V_{0}\uparrow^{S_{5}}_{H_0}},\chi_{S^{(3,2)}}\rangle_{S_{5}}=\langle\chi_{V_{0}},\chi_{S^{(3,2)}\downarrow_{H_{0}}}\rangle_{H_{0}}\\
    &=\frac{1}{12}(5+1+3+3-2+2)=1, \\
    \langle\chi_{C_{0}(M_{5})},\chi_{S^{(2,2,1)}}\rangle_{S_{5}}&=\langle\chi_{V_{0}\uparrow^{S_{5}}_{H_0}},\chi_{S^{(2,2,1)}}\rangle_{S_{5}}=\langle\chi_{V_{0}},\chi_{S^{(2,2,1)}\downarrow_{H_{0}}}\rangle_{H_{0}}\\
    &=\frac{1}{12}(5-1-3+3-2-2)=0. 
  \end{align*}
%}

De los productos internos calculados anteriormente obtenemos:
\begin{equation}
  \label{eq:C0-M5}
  C_{0}(M_{5})\cong \mathbb{C}\oplus S^{(4,1)} \oplus S^{(3,2)}. 
\end{equation}
Ahora consideremos los siguientes conjuntos, nuevamente para usar
reciprocidad de Frobenius.
\begin{eqnarray*}
  V_{1}&=&\langle(\overline{13},\overline{24})\rangle,\\
  H_{1}&=&\{g\in S_{5}\mid gV_{1}=V_{1}\}.
\end{eqnarray*}

  \begin{table}[!hbtp]
    \centering
    \begin{small}
      \begin{tabular}{c |r r r r r}
        & & (24) & (1432) & (14)(23) & \\
        Elementos & (1) & (13) & (1234) & (12)(34) & (13)(24) \\
        \hline
        $\chi_{S^{(5)}\downarrow_{H_{1}}}$       & 1 & 1 & 1 & 1 & 1 \\
        $\chi_{S^{(1,1,1,1,1)}\downarrow_{H_{1}}}$ & 1 & -1 & -1 & 1 & 1 \\
        $\chi_{S^{(4,1)}\downarrow_{H_{1}}}$      & 4 & 2 & 0 & 0 & 0 \\
        $\chi_{S^{(2,1,1,1)}\downarrow_{H_{1}}}$   & 4 & -2 & 0 & 0 & 0 \\
        $\chi_{S^{(3,1,1)}\downarrow_{H_{1}}}$     & 6 & 0 & 0 & -2 & -2 \\
        $\chi_{S^{(3,2)}\downarrow_{H_{1}}}$      & 5 & 1 & -1 & 1 & 1 \\
        $\chi_{S^{(2,2,1)}\downarrow_{H_{1}}}$    & 5 & -1 & 1 & 1 & 1 \\
        \hline
        $\chi_{V_{1}}$ & 1 & 1 & -1 & -1 & 1 \\
      \end{tabular}

    \end{small}
    \caption{Caracteres de $S_5$ restringidos a $H_{1}$ y carácter de $V_{1}$}
    \label{tab:restriccion-H_1}
  \end{table}

%{\scriptsize
  \begin{align*}
    \langle\chi_{C_{1}(M_{5})},\chi_{S^{(5)}}\rangle_{S_{5}}&=\langle\chi_{V_{1}\uparrow^{S_{5}}_{H_1}},\chi_{S^{(5)}}\rangle_{S_{5}}=\langle\chi_{V_{1}},\chi_{S^{(5)}\downarrow_{H_{1}}}\rangle_{H_{1}}\\
    &=\frac{1}{8}(1+2-2-2+1)=0,\\
    \langle\chi_{C_{1}(M_{5})},\chi_{S^{(1,1,1,1,1)}}\rangle_{S_{5}}&=\langle\chi_{V_{1}\uparrow^{S_{5}}_{H_1}},\chi_{S^{(1,1,1,1,1)}}\rangle_{S_{5}}=\langle\chi_{V_{1}},\chi_{S^{(1,1,1,1,1)}\downarrow_{H_{1}}}\rangle_{H_{1}}\\
    &=\frac{1}{8}(1-2+2-2+1)=0, \\
    \langle\chi_{C_{1}(M_{5})},\chi_{S^{(4,1)}}\rangle_{S_{5}}&=\langle\chi_{V_{1}\uparrow^{S_{5}}_{H_1}},\chi_{S^{(4,1)}}\rangle_{S_{5}}=\langle\chi_{V_{1}},\chi_{S^{(4,1)}\downarrow_{H_{1}}}\rangle_{H_{1}}\\
    &=\frac{1}{8}(4+4+0+0+0)=1, \\
    \langle\chi_{C_{1}(M_{5})},\chi_{S^{(2,1,1,1)}}\rangle_{S_{5}}&=\langle\chi_{V_{1}\uparrow^{S_{5}}_{H_1}},\chi_{S^{(2,1,1,1)}}\rangle_{S_{5}}=\langle\chi_{V_{1}},\chi_{S^{(2,1,1,1)}\downarrow_{H_{1}}}\rangle_{H_{1}}\\
    &=\frac{1}{8}(4-4+0+0+0)=0, \\
    \langle\chi_{C_{1}(M_{5})},\chi_{S^{(3,1,1)}}\rangle_{S_{5}}&=\langle\chi_{V_{1}\uparrow^{S_{5}}_{H_1}},\chi_{S^{(3,1,1)}}\rangle_{S_{5}}=\langle\chi_{V_{1}},\chi_{S^{(3,1,1)}\downarrow_{H_{1}}}\rangle_{H_{1}}\\
    &=\frac{1}{8}(6+0+0+4-2)=1, \\
    \langle\chi_{C_{1}(M_{5})},\chi_{S^{(3,2)}}\rangle_{S_{5}}&=\langle\chi_{V_{1}\uparrow^{S_{5}}_{H_1}},\chi_{S^{(3,2)}}\rangle_{S_{5}}=\langle\chi_{V_{1}},\chi_{S^{(3,2)}\downarrow_{H_{1}}}\rangle_{H_{1}}\\
    &=\frac{1}{8}(5+2+2-2+1)=1, \\
    \langle\chi_{C_{1}(M_{5})},\chi_{S^{(2,2,1)}}\rangle_{S_{5}}&=\langle\chi_{V_{1}\uparrow^{S_{5}}_{H_1}},\chi_{S^{(2,2,1)}}\rangle_{S_{5}}=\langle\chi_{V_{1}},\chi_{S^{(2,2,1)}\downarrow_{H_{1}}}\rangle_{H_{1}}\\
    &=\frac{1}{8}(5-2-2-2+1)=0.
  \end{align*}
%}

De donde obtenemos:
\begin{equation}
  \label{eq:C1-M5}
  C_{1}(M_{5})\cong S^{(4,1)}\oplus S^{(3,1,1)}\oplus S^{(3,2)}.
\end{equation}
En la figura \ref{fig:diagrama-conmutativo5} se muestra el diagrama
conmutativo de los complejos de cadenas de $M_{5}$, donde $f_{0}$ y
$f_{1}$ son los isomorfismos obtenidos de las expresiones~\ref{eq:C0-M5} y \ref{eq:C1-M5}.

\begin{figure}[h]
  \centering
    \[
    \begin{CD}
      0 @>{\partial_{2}}>> C_{1}(M_{5}) @>{\partial_{1}}>> C_{0}(M_{5}) @>{\varepsilon}>> \mathbb{C}\\
      @VVV   @Vf_{1} VV   @Vf_{0} VV   @VVV    \\
      0 @>{\widehat\partial_{2}}>> S^{(4,1)}\oplus S^{(3,1,1)}\oplus
      S^{(3,2)} @>{\widehat\partial_{1}}>> \mathbb{C}\oplus S^{(4,1)}
      \oplus S^{(3,2)} @>{\widehat \varepsilon}>> \mathbb{C}
    \end{CD}
    \]

    \caption{Diagrama conmutativo de los complejos de cadenas de
      $M_{5}$}
\label{fig:diagrama-conmutativo5}
\end{figure}

Calculemos los módulos de homología reducida $\widetilde H_{0}(M_{4})$ y
$\widetilde H_{1}(M_{4})$.

Como $\widehat\varepsilon$ es sobreyectiva y por el teorema \ref{teorema-isomorfismo-mod} tenemos:
$$(\mathbb{C}\oplus S^{(4,1)} \oplus
S^{(3,2)})/\ker\widehat\varepsilon\cong\im\widehat\varepsilon=\mathbb{C}$$
así que
\begin{equation*}
  \label{ker0-5}
  \ker\widehat\varepsilon\cong S^{(4,1)} \oplus S^{(3,2)}.
\end{equation*}
Sabemos que $\widetilde H_{0}(M_{5})=\ker \widehat\varepsilon/\im
\widehat\partial_{1}=0$, pues $M_{5}$  es
conexo. Se sigue entonces que~$\ker \widehat\varepsilon\cong
\im\widehat\partial_{1}$, con lo cual:
\begin{equation}
  \label{im1-5}
  \im \widehat\partial_{1}\cong S^{(4,1)} \oplus S^{(3,2)}.
\end{equation}
Además de el teorema \ref{teorema-isomorfismo-mod} y la expresión \ref{im1-5}, se tiene:
$$(S^{(4,1)}\oplus S^{(3,1,1)}\oplus S^{(3,2)})/\ker
\widehat\partial_{1}\cong \im \widehat\partial_{1}.$$
De lo anterior y la proposición \ref{modulos-iguales} tenemos:
\begin{equation}
  \label{ker1-5}
  \ker \widehat\partial_{1}= S^{(3,1,1)}.
\end{equation}

Por otro lado,
\begin{equation}
  \im\widehat\partial_{2}=\widehat\partial_{2}(0)=0.
  \label{im-2-5}
\end{equation}
pues $\widehat\partial_{2}$ es un homomorfismo de módulos.

De las ecuaciones \ref{ker1-5} y \ref{im-2-5} concluimos:
\begin{equation*}
  \widetilde H_{1}(M_{5})=\ker \widehat\partial_{1}/\im \widehat\partial_{2}=S^{(3,1,1)}/0=S^{(3,1,1)}.
\end{equation*}
 
Anteriormente usamos el hecho de que $M_{5}$ es conexo para concluir
por medio de un teorema de topología algebraica que $\widetilde
H_{0}(M_{5})=0$. Veamos ahora, sin embargo, que es posible
calcular $\im \widehat\partial_{1}$ por métodos elementales.

Verifiquemos primero que $\dim(\im \partial_{1})=9$. De el teorema
\ref{imT}, sabemos que
$\im \partial_{1}=\langle\partial_{1}(b_{1}),\ldots,\partial_{1}(b_{15})\rangle$,
veamos que el número de vectores linealmente independientes de la
$\im \partial_{1}$ es 9, es
decir, si
$$\lambda_{1}\partial_{1}(b_{1})+\lambda_{2}\partial_{1}(b_{2})+\ldots+\lambda_{15}\partial_{1}(b_{15})=0$$
entonces:
\begin{align*}
  &\lambda_{1}(a_{8}-a_{1})+\lambda_{2}(a_{9}-a_{1})+\lambda_{3}(a_{10}-a_{1})+\lambda_{4}(a_{6}-a_{2})+\lambda_{5}(a_{7}-a_{2})\\
  &\qquad
  {}+\lambda_{6}(a_{10}-a_{2})+\lambda_{7}(a_{5}-a_{3})+\lambda_{8}(a_{7}-a_{3})+\lambda_{9}(a_{9}-a_{3})+\lambda_{10}(a_{5}-a_{4})\\
  &\qquad{}+\lambda_{11}(a_{6}-a_{4})+\lambda_{12}(a_{8}-a_{4})+\lambda_{13}(a_{10}-a_{5})+\lambda_{14}(a_{9}-a_{6})+\lambda_{15}(a_{8}-a_{7})\\
  &=(-\lambda_{1}-\lambda_{2}-\lambda_{3})a_{1}+(-\lambda_{4}-\lambda_{5}-\lambda_{6})a_{2}+(-\lambda_{7}-\lambda_{8}-\lambda_{9})a_{3}\\
  &\qquad{}+(-\lambda_{10}-\lambda_{11}-\lambda_{12})a_{4}+(\lambda_{7}+\lambda_{10}-\lambda_{13})a_{5}+(\lambda_{4}+\lambda_{11}-\lambda_{14})a_{6}\\
  &\qquad{}+(\lambda_{5}+\lambda_{8}-\lambda_{15})a_{7}+(\lambda_{1}+\lambda_{12}+\lambda_{15})a_{8}+(\lambda_{2}+\lambda_{9}+\lambda_{14})a_{9}\\
  &\qquad{}+(\lambda_{3}+\lambda_{6}+\lambda_{13})a_{10}=0,
\end{align*}
con lo cual obtenemos el siguiente sistema de ecuaciones:
 \[\begin{array}{rrrrr}
   -\lambda_{1} & -\lambda_{2} & -\lambda_{3} & = & 0 \\
   -\lambda_{4} & -\lambda_{5} & -\lambda_{6} & = & 0 \\
   -\lambda_{7} & -\lambda_{8} & -\lambda_{9} & = & 0 \\
   -\lambda_{10} & -\lambda_{11} & -\lambda_{12} & = & 0 \\
   \lambda_{7} & +\lambda_{10} & -\lambda_{13} & = & 0 \\
   \lambda_{4} & +\lambda_{11} & -\lambda_{14} & = & 0 \\
   \lambda_{5} & +\lambda_{8} & -\lambda_{15} & = & 0 \\
   \lambda_{1} & +\lambda_{12} & +\lambda_{15} & = & 0 \\
   \lambda_{2} & +\lambda_{9} & +\lambda_{14} & = & 0 \\
   \lambda_{3} & +\lambda_{6} & +\lambda_{13} & = & 0 
 \end{array}\]
al que representamos en su forma matricial:
\[ \left(
  \begin{tabular}{rrrrrrrrrrrrrrr}
    -1 & -1 & -1 & 0  & 0  & 0  & 0  & 0  & 0  & 0  & 0  & 0  & 0  & 0  & 0  \\
    0  & 0  & 0  & -1 & -1 & -1 & 0  & 0  & 0  & 0  & 0  & 0  & 0  & 0  & 0  \\
    0  & 0  & 0  & 0  & 0  & 0  & -1 & -1 & -1 & 0  & 0  & 0  & 0  & 0  & 0  \\
    0  & 0  & 0  & 0  & 0  & 0  & 0  & 0  & 0  & -1 & -1 & -1 & 0  & 0  & 0  \\
    0  & 0  & 0  & 0  & 0  & 0  & 1  & 0  & 0  & 1  & 0  & 0  & -1 & 0  & 0  \\
    0  & 0  & 0  & 1  & 0  & 0  & 0  & 0  & 0  & 0  & 1  & 0  & 0  & -1 & 0  \\
    0  & 0  & 0  & 0  & 1  & 0  & 0  & 1  & 0  & 0  & 0  & 0  & 0  & 0  & -1 \\
    1  & 0  & 0  & 0  & 0  & 0  & 0  & 0  & 0  & 0  & 0  & 1  & 0  & 0  & 1  \\
    0  & 1  & 0  & 0  & 0  & 0  & 0  & 0  & 1  & 0  & 0  & 0  & 0  & 1  & 0  \\
    0  & 0  & 1  & 0  & 0  & 1  & 0  & 0  & 0  & 0  & 0  & 0  & 1  & 0  & 0 
  \end{tabular}
\right),\] 
lo llevamos a su forma escalonada:
\[ \left(
  \begin{tabular}{rrrrrrrrrrrrrrr}
    1 & 1 & 1 & 0 & 0 & 0 & 0 & 0  & 0  & 0  & 0  & 0  & 0  & 0  & 0  \\
    0 & 1 & 1 & 0 & 0 & 0 & 0 & 0  & 0  & 0  & 0  & -1 & 0  & 0  & -1 \\
    0 & 0 & 1 & 0 & 0 & 0 & 0 & 0  & -1 & 0  & 0  & -1 & 0  & -1 & -1 \\
    0 & 0 & 0 & 1 & 0 & 0 & 0 & 0  & 0  & 0  & 1  & 0  & 0  & -1 & 0  \\
    0 & 0 & 0 & 0 & 1 & 0 & 0 & 1  & 0  & 0  & 0  & 0  & 0  & 0  & -1 \\
    0 & 0 & 0 & 0 & 0 & 1 & 0 & -1 & 0  & 0  & -1 & 0  & 0  & 1  & 1  \\
    0 & 0 & 0 & 0 & 0 & 0 & 1 & 0  & 0  & 1  & 0  & 0  & -1 & 0  & 0  \\
    0 & 0 & 0 & 0 & 0 & 0 & 0 & 1  & 1  & -1 & 0  & 0  & 1  & 0  & 0  \\
    0 & 0 & 0 & 0 & 0 & 0 & 0 & 0  & 0  & 1  & 1  & 1  & 0  & 0  & 0  \\
    0 & 0 & 0 & 0 & 0 & 0 & 0 & 0  & 0  & 0  & 0  & 0  & 0  & 0  & 0 
  \end{tabular}
\right).\]
Como el rango de la matriz es 9, obtenemos $\dim(\im\partial_{1})=9$, lo cual implica que
$\dim(\im\widehat\partial_{1})=9$ (se sigue de la definición de
$\widehat\partial_{1}$).

Por el teorema \ref{im-mod-irreducible} tenemos:
  $$\widehat\partial_{1}(S^{(4,1)})=S^{(4,1)} \quad \mbox{o }\quad \widehat\partial_{1}(S^{(4,1)})=0$$
  $$\widehat\partial_{1}(S^{(3,1,1)})=0$$
  $$\widehat\partial_{1}(S^{(3,2)})=S^{(4,1)} \quad \mbox{o }\quad \widehat\partial_{1}(S^{(3,2)})=0$$
además $\dim(S^{(4,1)})=4$ y $\dim(S^{(3,2)})=5$, luego
$$\im\widehat\partial_{1}=\widehat\partial_{1}(S^{(4,1)}\oplus S^{(3,1,1)}\oplus S^{(3,2)})=S^{(4,1)}\oplus S^{(3,2)}.$$

\section{Homologías reducidas de $K(M_{5})$}
\label{hom-red-KM5}

La gráfica de clanes de $G_{5}$, es decir $K(G_{5})$, se muestra en la figura~\ref{fig:KG_5}.
Consideremos ahora su complejo de subgráficas completas $\Delta(K(G_{5}))$
al que denotaremos simplemente como~$K(M_{5})$. Este complejo tiene como
vértices (aristas de la gráfica de~$M_{5}$):
\begin{center}
  \begin{tabular}[h]{lll}
    $a_{1}=(\overline{12},\overline{34})$ & $a_{6}=(\overline{13},\overline{45})$ & $a_{11}=(\overline{15},\overline{24})$  \\
    $a_{2}=(\overline{12},\overline{35})$ & $a_{7}=(\overline{14},\overline{23})$ & $a_{12}=(\overline{15},\overline{34})$  \\
    $a_{3}=(\overline{12},\overline{45})$ & $a_{8}=(\overline{14},\overline{25})$ & $a_{13}=(\overline{23},\overline{45})$  \\
    $a_{4}=(\overline{13},\overline{24})$ & $a_{9}=(\overline{14},\overline{35})$ & $a_{14}=(\overline{24},\overline{35})$  \\
    $a_{5}=(\overline{13},\overline{25})$ & $a_{10}=(\overline{15},\overline{23})$ & $a_{15}=(\overline{25},\overline{34})$  
  \end{tabular}
\end{center}

La familia de $1$-simplejos orientados estará dada por:
\begin{center}
  \begin{tabular}[h]{llll}
    $b_{1}=(a_{1},a_{2})$ & $b_{9}=(a_{3},a_{13})$ & $b_{17}=(a_{6},a_{13})$ & $b_{25}=(a_{10},a_{11})$ \\
    $b_{2}=(a_{1},a_{3})$ & $b_{10}=(a_{4},a_{5})$ & $b_{18}=(a_{7},a_{8})$ & $b_{26}=(a_{10},a_{12})$ \\
    $b_{3}=(a_{1},a_{12})$ & $b_{11}=(a_{4},a_{6})$ & $b_{19}=(a_{7},a_{9})$ & $b_{27}=(a_{10},a_{13})$ \\
    $b_{4}=(a_{1},a_{15})$ & $b_{12}=(a_{4},a_{11})$ & $b_{20}=(a_{7},a_{10})$ & $b_{28}=(a_{11},a_{12})$ \\
    $b_{5}=(a_{2},a_{3})$ & $b_{13}=(a_{4},a_{14})$ & $b_{21}=(a_{7},a_{13})$ & $b_{29}=(a_{11},a_{14})$ \\
    $b_{6}=(a_{2},a_{9})$ & $b_{14}=(a_{5},a_{6})$ & $b_{22}=(a_{8},a_{9})$ & $b_{30}=(a_{12},a_{15})$. \\
    $b_{7}=(a_{2},a_{14})$ & $b_{15}=(a_{5},a_{8})$ & $b_{23}=(a_{8},a_{15})$ &  \\
    $b_{8}=(a_{3},a_{6})$ & $b_{16}=(a_{5},a_{15})$ & $b_{24}=(a_{9},a_{14})$ &
  \end{tabular}
\end{center}
Los $2$-simplejos orientados son:

\begin{tabular}[h]{llll}
  $c_{1}=(a_{1},a_{2},a_{3})$ & $c_{4}=(a_{3},a_{6},a_{13})$ &$c_{7}=(a_{5},a_{8},a_{15})$ &$c_{9}=(a_{7},a_{10},a_{13})$ \\
  $c_{2}=(a_{1},a_{12},a_{15})$ & $c_{5}=(a_{4},a_{5},a_{6})$ &$c_{8}=(a_{7},a_{8},a_{9})$ &$c_{10}=(a_{10},a_{11},a_{12})$. \\
  $c_{3}=(a_{2},a_{9},a_{14})$ & $c_{6}=(a_{4},a_{11},a_{14})$ &  &
\end{tabular}

\begin{figure}[!hbtp]
  \centering
  \begin{tikzpicture}[scale=.85]%[rotate=90,scale=1]
    \newcommand{\aset}[2]{$\{#1,#2\}$} \GraphInit[vstyle=Classic]
    % \tikzset{VertexStyle/.style={draw,circle}}
    \SetVertexNoLabel \SetVertexMath \SetUpVertex[MinSize=17pt]
    \grEmptyCycle[RA=1,rotation=-90]{5}
    \grEmptyCycle[RA=2.5,prefix=w,rotation=-90]{5}
    \grCycle[RA=3.8,prefix=z,rotation=90]{5} \EdgeInGraphMod{a}{5}{2}
    \EdgeMod{a}{w}{5}{1} \EdgeMod{a}{w}{5}{-1} \EdgeMod{w}{z}{5}{2}
    \EdgeMod{w}{z}{5}{-2}
    \AssignVertexLabel{z}{\textsl{$a_{1}$},\textsl{$a_{3}$},\textsl{$a_{6}$},\textsl{$a_{5}$},\textsl{$a_{15}$}}
    \AssignVertexLabel{w}{\textsl{$a_{4}$},\textsl{$a_{8}$},\textsl{$a_{12}$},\textsl{$a_{2}$},\textsl{$a_{13}$}}
    \AssignVertexLabel{a}{\textsl{$a_{7}$},\textsl{$a_{11}$},\textsl{$a_{9}$},\textsl{$a_{10}$},\textsl{$a_{14}$}}
  \end{tikzpicture}
 
  \caption{Gráfica de clanes $K(G_{5})$}
  \label{fig:KG_5}
\end{figure}

A continuación obtendremos la descomposición de los $S_{5}$-módulos de
cadenas $C_{k}(K(M_{5}))$, para $k=0,1,2$, en
módulos de Specht por medio de el teorema \ref{frobenius} de la reciprocidad de Frobenius.

Consideremos:
\begin{eqnarray*}
V_{0}=\langle a_{1}\rangle=\langle
(\overline{12},\overline{34})\rangle,\\
H_{0}=\{g\in S_{5}\mid gV_{0}=V_{0}\}.
\end{eqnarray*}

Entonces $C_{0}(K(M_{5}))=V_{0}\uparrow^{S_{5}}_{H_0}$.

  \begin{table}[!hbtp]
    \centering
    \resizebox*{!}{5.2cm}{
    \begin{tabular}{c |r r r r}
      &     &      &        & (12)(34) \\
      &     & (12) & (1324) & (13)(24) \\
      Elementos & (1) & (34) & (1423) & (14)(23) \\
      \hline
      $\chi_{S^{(5)}\downarrow_{H_{0}}}$       & 1 & 1  & 1  & 1  \\
      $\chi_{S^{(1,1,1,1,1)}\downarrow_{H_{0}}}$ & 1 & -1 & -1 & 1  \\
      $\chi_{S^{(4,1)}\downarrow_{H_{0}}}$      & 4 & 2  & 0  & 0  \\
      $\chi_{S^{(2,1,1,1)}\downarrow_{H_{0}}}$   & 4 & -2 & 0  & 0  \\
      $\chi_{S^{(3,1,1)}\downarrow_{H_{0}}}$    & 6 & 0  & 0  & -2 \\
      $\chi_{S^{(3,2)}\downarrow_{H_{0}}}$     & 5 & 1  & -1  & 1  \\
      $\chi_{S^{(2,2,1)}\downarrow_{H_{0}}}$   & 5 & -1  & 1  & 1  \\
      \hline
      $\chi_{V_{0}}$ & 1 & 1 & 1 & 1 
    \end{tabular}}
    
    \caption{Caracteres de $S_{5}$ restringidos a $H_{0}$ y carácter de $V_{0}$}
    \label{tab:clanes-H_0-5}
  \end{table}

\begin{eqnarray*}
  \langle\chi_{C_{0}(K(M_{5}))},\chi_{S^{(5)}}\rangle_{S_{5}}=\langle\chi_{V_{0}\uparrow^{S_{5}}_{H_0}},\chi_{S^{(5)}}\rangle_{S_{5}}=\langle\chi_{V_{0}},\chi_{S^{(5)}\downarrow_{H_{0}}}\rangle_{H_{0}}=1,\\
 \langle\chi_{C_{0}(K(M_{5}))},\chi_{S^{(1,1,1,1,1)}}\rangle_{S_{5}}=\langle\chi_{V_{0}\uparrow^{S_{5}}_{H_0}},\chi_{S^{(1,1,1,1,1)}}\rangle_{S_{5}}=\langle\chi_{V_{0}},\chi_{S^{(1,1,1,1,1)}\downarrow_{H_{0}}}\rangle_{H_{0}}=0,\\
\langle\chi_{C_{0}(K(M_{5}))},\chi_{S^{(4,1)}}\rangle_{S_{5}}=\langle\chi_{V_{0}\uparrow^{S_{5}}_{H_0}},\chi_{S^{(4,1)}}\rangle_{S_{5}}=\langle\chi_{V_{0}},\chi_{S^{(4,1)}\downarrow_{H_{0}}}\rangle_{H_{0}}=1,\\
\langle\chi_{C_{0}(K(M_{5}))},\chi_{S^{(2,1,1,1)}}\rangle_{S_{5}}=\langle\chi_{V_{0}\uparrow^{S_{5}}_{H_0}},\chi_{S^{(2,1,1,1)}}\rangle_{S_{5}}=\langle\chi_{V_{0}},\chi_{S^{(2,1,1,1)}\downarrow_{H_{0}}}\rangle_{H_{0}}=0,\\
\langle\chi_{C_{0}(K(M_{5}))},\chi_{S^{(3,1,1)}}\rangle_{S_{5}}=\langle\chi_{V_{0}\uparrow^{S_{5}}_{H_0}},\chi_{S^{(3,1,1)}}\rangle_{S_{5}}=\langle\chi_{V_{0}},\chi_{S^{(3,1,1)}\downarrow_{H_{0}}}\rangle_{H_{0}}=0,\\
\langle\chi_{C_{0}(K(M_{5}))},\chi_{S^{(3,2)}}\rangle_{S_{5}}=\langle\chi_{V_{0}\uparrow^{S_{5}}_{H_0}},\chi_{S^{(3,2)}}\rangle_{S_{5}}=\langle\chi_{V_{0}},\chi_{S^{(3,2)}\downarrow_{H_{0}}}\rangle_{H_{0}}=1,\\
\langle\chi_{C_{0}(K(M_{5}))},\chi_{S^{(2,2,1)}}\rangle_{S_{5}}=\langle\chi_{V_{0}\uparrow^{S_{5}}_{H_0}},\chi_{S^{(2,2,1)}}\rangle_{S_{5}}=\langle\chi_{V_{0}},\chi_{S^{(2,2,1)}\downarrow_{H_{0}}}\rangle_{H_{0}}=1.\\
\end{eqnarray*}
Con los productos internos calculados obtenemos:
\begin{equation}
  C_{0}(K(M_{5}))\cong \mathbb{C}\oplus S^{(4,1)}\oplus
  S^{(3,2)}\oplus S^{(2,2,1)}.
  \label{C0-KM5}
\end{equation}
Consideremos ahora los siguientes conjuntos:
\begin{eqnarray*}
V_{1}=\langle b_{1}\rangle=\langle (a_{1},a_{2})\rangle=\langle((\overline{12},\overline{34}),(\overline{12},\overline{35}))\rangle,\\
H_{1}=\{g\in S_{5}\mid gV_{1}=V_{1}\}=\{(1),(12),(45),(12)(45)\}.
\end{eqnarray*}
% \begin{align*}
%   (1)b_{1}&=b_{1}\\
%   (12)b_{1}&=b_{1}\\
%   (45)b_{1}&=-b_{1}\\
%   (12)(45)b_{1}&=-b_{1}
% \end{align*}
\begin{table}[!hbtp]
  \centering
  \begin{tabular}{c |r r r r}
    Elementos & (1) & (12) & (45) & (12)(45) \\
    \hline
    $\chi_{S^{(5)}\downarrow_{H_{1}}}$ & 1 & 1  & 1  & 1 \\
    $\chi_{S^{(1,1,1,1,1)}\downarrow_{H_{1}}}$ & 1 & -1 & -1 & 1  \\
    $\chi_{S^{(4,1)}\downarrow_{H_{1}}}$ & 4 & 2  & 2  & 0  \\
    $\chi_{S^{(2,1,1,1)}\downarrow_{H_{1}}}$ & 4 & -2 & -2 & 0  \\
    $\chi_{S^{(3,1,1)}\downarrow_{H_{1}}}$ & 6 & 0  & 0  & -2 \\
    $\chi_{S^{(3,2)}\downarrow_{H_{1}}}$ & 5 & 1  & 1  & 1  \\
    $\chi_{S^{(2,2,1)}\downarrow_{H_{1}}}$ & 5 & -1 & -1 & 1  \\
    \hline
    $\chi_{V_{1}}$ & 1 & 1 & -1 & -1 \\
  \end{tabular}

  \caption{Caracteres de $S_{5}$ restringidos a $H_{1}$ y carácter de $V_{1}$}
  \label{tab:clanes-H_1-5}
\end{table}

De la tabla \ref{tab:clanes-H_1-5} calculamos los productos internos:
\begin{eqnarray*}
  \langle\chi_{C_{1}(K(M_{5}))},\chi_{S^{(5)}}\rangle_{S_{5}}=\langle\chi_{V_{1}\uparrow^{S_{5}}_{H_1}},\chi_{S^{(5)}}\rangle_{S_{5}}=\langle\chi_{V_{1}},\chi_{S^{(5)}\downarrow_{H_{1}}}\rangle_{H_{1}}=0,\\
  \langle\chi_{C_{1}(K(M_{5}))},\chi_{S^{(1,1,1,1,1)}}\rangle_{S_{5}}=\langle\chi_{V_{1}\uparrow^{S_{5}}_{H_1}},\chi_{S^{(1,1,1,1,1)}}\rangle_{S_{5}}=\langle\chi_{V_{1}},\chi_{S^{(1,1,1,1,1)}\downarrow_{H_{1}}}\rangle_{H_{1}}=0,\\
  \langle\chi_{C_{1}(K(M_{5}))},\chi_{S^{(4,1)}}\rangle_{S_{5}}=\langle\chi_{V_{1}\uparrow^{S_{5}}_{H_1}},\chi_{S^{(4,1)}}\rangle_{S_{5}}=\langle\chi_{V_{1}},\chi_{S^{(4,1)}\downarrow_{H_{1}}}\rangle_{H_{1}}=1,\\
  \langle\chi_{C_{1}(K(M_{5}))},\chi_{S^{(2,1,1,1)}}\rangle_{S_{5}}=\langle\chi_{V_{1}\uparrow^{S_{5}}_{H_1}},\chi_{S^{(2,1,1,1)}}\rangle_{S_{5}}=\langle\chi_{V_{1}},\chi_{S^{(2,1,1,1)}\downarrow_{H_{1}}}\rangle_{H_{1}}=1,\\
  \langle\chi_{C_{1}(K(M_{5}))},\chi_{S^{(3,1,1)}}\rangle_{S_{5}}=\langle\chi_{V_{1}\uparrow^{S_{5}}_{H_1}},\chi_{S^{(3,1,1)}}\rangle_{S_{5}}=\langle\chi_{V_{1}},\chi_{S^{(3,1,1)}\downarrow_{H_{1}}}\rangle_{H_{1}}=2,\\
  \langle\chi_{C_{1}(K(M_{5}))},\chi_{S^{(3,2)}}\rangle_{S_{5}}=\langle\chi_{V_{1}\uparrow^{S_{5}}_{H_1}},\chi_{S^{(3,2)}}\rangle_{S_{5}}=\langle\chi_{V_{1}},\chi_{S^{(3,2)}\downarrow_{H_{1}}}\rangle_{H_{1}}=1,\\
  \langle\chi_{C_{1}(K(M_{5}))},\chi_{S^{(2,2,1)}}\rangle_{S_{5}}=\langle\chi_{V_{1}\uparrow^{S_{5}}_{H_1}},\chi_{S^{(2,2,1)}}\rangle_{S_{5}}=\langle\chi_{V_{1}},\chi_{S^{(2,2,1)}\downarrow_{H_{1}}}\rangle_{H_{1}}=1.
\end{eqnarray*}
De donde resulta:
\begin{equation}
  C_{1}(K(M_{5}))\cong S^{(4,1)}\oplus S^{(2,1,1,1)}\oplus
  2S^{(3,1,1)}\oplus S^{(3,2)} \oplus S^{(2,2,1)}.
  \label{C1-KM5}
\end{equation}
Ahora tomemos los siguientes conjuntos:
\begin{eqnarray*}
V_{2}=\langle c_{1}\rangle=\langle
(a_{1},a_{2},a_{3})\rangle&=&\langle((\overline{12},\overline{34}),(\overline{12},\overline{35}),(\overline{12},\overline{45}))\rangle,\\
H_{2}&=&\{g\in S_{5}\mid gV_{2}=V_{2}\}.
\end{eqnarray*}
Con lo cual conseguimos la tabla \ref{tab:clanes-H_2-5}.
\begin{table}[!hbtp]
  \centering
  \begin{tabular}{c |r r r r r r}
    &     &      & (34) & (12)(34) &       &  \\
    &     &      & (35) & (12)(35) & (345) & (12)(345) \\
    Elementos & (1) & (12) & (45) & (12)(45) & (354) & (12)(354) \\
    \hline
    $\chi_{S^{(5)}\downarrow_{H_{2}}}$ & 1 & 1  & 1  & 1 & 1 & 1 \\
    $\chi_{S^{(1,1,1,1,1)}\downarrow_{H_{2}}}$ & 1 & -1 & -1 & 1 & 1 & -1\\
    $\chi_{S^{(4,1)}\downarrow_{H_{2}}}$ & 4 & 2  & 2  & 0 & 1 & -1\\
    $\chi_{S^{(2,1,1,1)}\downarrow_{H_{2}}}$ & 4 & -2 & -2 & 0 & 1 & 1\\
    $\chi_{S^{(3,1,1)}\downarrow_{H_{2}}}$ & 6 & 0  & 0  & -2& 0 & 0\\
    $\chi_{S^{(3,2)}\downarrow_{H_{2}}}$ & 5 & 1  & 1  & 1 & -1& 1\\
    $\chi_{S^{(2,2,1)}\downarrow_{H_{2}}}$ & 5 & -1 & -1 & 1 & -1& -1\\
    \hline
    $\chi_{V_{2}}$ & 1 & 1 & -1 & -1& 1 & 1\\
  \end{tabular}

  \caption{Caracteres de $S_{5}$ restringidos a $H_{2}$ y carácter de $V_{2}$}
  \label{tab:clanes-H_2-5}
\end{table}

La cual nos ayuda a calcular los siguientes productos internos:
\begin{eqnarray*}
  \langle\chi_{C_{2}(K(M_{5}))},\chi_{S^{(5)}}\rangle_{S_{5}}=\langle\chi_{V_{2}\uparrow^{S_{5}}_{H_{2}}},\chi_{S^{(5)}}\rangle_{S_{5}}=\langle\chi_{V_{2}},\chi_{S^{(5)}\downarrow_{H_{2}}}\rangle_{H_{2}}=0,\\
  \langle\chi_{C_{2}(K(M_{5}))},\chi_{S^{(1,1,1,1,1)}}\rangle_{S_{5}}=\langle\chi_{V_{2}\uparrow^{S_{5}}_{H_{2}}},\chi_{S^{(1,1,1,1,1)}}\rangle_{S_{5}}=\langle\chi_{V_{2}},\chi_{S^{(1,1,1,1,1)}\downarrow_{H_{2}}}\rangle_{H_{2}}=0,\\
  \langle\chi_{C_{2}(K(M_{5}))},\chi_{S^{(4,1)}}\rangle_{S_{5}}=\langle\chi_{V_{2}\uparrow^{S_{5}}_{H_{2}}},\chi_{S^{(4,1)}}\rangle_{S_{5}}=\langle\chi_{V_{2}},\chi_{S^{(4,1)}\downarrow_{H_{2}}}\rangle_{H_{2}}=0,\\
  \langle\chi_{C_{2}(K(M_{5}))},\chi_{S^{(2,1,1,1)}}\rangle_{S_{5}}=\langle\chi_{V_{2}\uparrow^{S_{5}}_{H_{2}}},\chi_{S^{(2,1,1,1)}}\rangle_{S_{5}}=\langle\chi_{V_{2}},\chi_{S^{(2,1,1,1)}\downarrow_{H_{2}}}\rangle_{H_{2}}=1,\\
  \langle\chi_{C_{2}(K(M_{5}))},\chi_{S^{(3,1,1)}}\rangle_{S_{5}}=\langle\chi_{V_{2}\uparrow^{S_{5}}_{H_{2}}},\chi_{S^{(3,1,1)}}\rangle_{S_{5}}=\langle\chi_{V_{2}},\chi_{S^{(3,1,1)}\downarrow_{H_{2}}}\rangle_{H_{2}}=1,\\
  \langle\chi_{C_{2}(K(M_{5}))},\chi_{S^{(3,2)}}\rangle_{S_{5}}=\langle\chi_{V_{2}\uparrow^{S_{5}}_{H_{2}}},\chi_{S^{(3,2)}}\rangle_{S_{5}}=\langle\chi_{V_{2}},\chi_{S^{(3,2)}\downarrow_{H_{2}}}\rangle_{H_{2}}=0,\\
  \langle\chi_{C_{2}(K(M_{5}))},\chi_{S^{(2,2,1)}}\rangle_{S_{5}}=\langle\chi_{V_{2}\uparrow^{S_{5}}_{H_{2}}},\chi_{S^{(2,2,1)}}\rangle_{S_{5}}=\langle\chi_{V_{2}},\chi_{S^{(2,2,1)}\downarrow_{H_{2}}}\rangle_{H_{2}}=0.\\
\end{eqnarray*}

De donde obtenemos:
\begin{equation}
  \label{C2-KM5}
  C_{2}(K(M_{5}))\cong S^{(2,1,1,1)}\oplus S^{(3,1,1)}.
\end{equation}

En la figura \ref{fig:diagrama-conmutativo-clanes5} se muestra el diagrama
conmutativo de los complejos de cadenas de $K(M_{5})$, donde $f_{0}$,
$f_{1}$ y $f_{2}$ son los isomorfismos obtenidos de las expresiones
\ref{C0-KM5}, \ref{C1-KM5} y \ref{C2-KM5}.

Denotemos a $\widehat C_{k}(K(M_{5}))$, para $k=0,1,2$, como sigue:
\begin{eqnarray*}
  \widehat C_{0}(K(M_{5}))&=&\mathbb{C}\oplus S^{(4,1)}\oplus
  S^{(3,2)}\oplus S^{(2,2,1)}.\\
  \widehat C_{1}(K(M_{5}))&=&S^{(4,1)}\oplus S^{(2,1,1,1)}\oplus 2S^{(3,1,1)}\oplus S^{(3,2)} \oplus S^{(2,2,1)}.\\
  \widehat C_{2}(K(M_{5}))&=&S^{(2,1,1,1)}\oplus S^{(3,1,1)}.\\
\end{eqnarray*}
\begin{figure}[h]
  \centering
    \[
  \begin{CD}
    0 @>{\partial_{3}}>> C_{2}(K(M_{5})) @>{\partial_{2}}>> C_{1}(K(M_{5})) @>{\partial_{1}}>> C_{0}(K(M_{5})) @>{\varepsilon}>> \mathbb{C}\\
    @VVV   @Vf_{2}VV   @Vf_{1}VV  @Vf_{0}VV  @VVV    \\
    0 @>{\widehat\partial_{3}}>> \widehat C_{2}(K(M_{5}))
    @>{\widehat\partial_{2}}>> \widehat C_{1}(K(M_{5}))
    @>{\widehat\partial_{1}}>> \widehat C_{0}(K(M_{5})) 
    @>{\widehat \varepsilon}>> \mathbb{C}
  \end{CD}
  \]
  
  \caption{Diagrama conmutativo de los complejos de cadenas de $K(M_{5})$}
  \label{fig:diagrama-conmutativo-clanes5}
\end{figure}

% \begin{sidewaysfigure}%[h]
%   %\centering
%   {\small
%   \[
%   \begin{CD}
%     0 @>{\partial_{3}}>> C_{2}(K(M_{5})) @>{\partial_{2}}>> C_{1}(K(M_{5})) @>{\partial_{1}}>> C_{0}(K(M_{5})) @>{\varepsilon}>> \mathbb{C}\\
%     @VVV   @Vf_{2}VV   @Vf_{1}VV  @Vf_{0}VV  @VVV    \\
%     0 @>{\widehat\partial_{3}}>> S^{(2,1,1,1)}\oplus S^{(3,1,1)}
%     @>{\widehat\partial_{2}}>> S^{(4,1)}\oplus S^{(2,1,1,1)}\oplus
%     2S^{(3,1,1)}\oplus S^{(3,2)} \oplus S^{(2,2,1)}
%     @>{\widehat\partial_{1}}>> \mathbb{C}\oplus S^{(4,1)}\oplus
%     S^{(3,2)}\oplus S^{(2,2,1)}@>{\widehat \varepsilon}>> \mathbb{C}
%   \end{CD}
%   \]
%    }
  
%   \caption{Diagrama conmutativo de los complejos de cadenas de $K(M_{5})$}
%   \label{fig:diagrama-conmutativo-clanes5}
% \end{sidewaysfigure}

Calculemos los módulos de homología reducida $\widetilde
H_{k}(K(M_{5}))$, para $k=~0,1,2$.

Como $\widehat\varepsilon$ es sobreyectiva y por el teorema \ref{teorema-isomorfismo-mod} tenemos:
\begin{equation*}
  (\mathbb{C}\oplus S^{(4,1)}\oplus S^{(3,2)}\oplus
  S^{(2,2,1)})/\ker\widehat\varepsilon\cong \im \widehat\varepsilon=\mathbb{C},
\end{equation*}
así que
\begin{equation*}
  \label{ker0-KM5}
  \ker\widehat\varepsilon\cong S^{(4,1)} \oplus S^{(3,2)}\oplus S^{(2,2,1)}.
\end{equation*}
Sabemos que $\widetilde H_{0}(K(M_{5}))=\ker \widehat\varepsilon/\im
\widehat\partial_{1}=0$, pues $K(M_{5})$ es conexo. Se sigue que $\ker \widehat\varepsilon\cong
\im\widehat\partial_{1}$, con lo cual:
\begin{equation}
  \label{im1-KM5}
  \im \widehat\partial_{1}\cong S^{(4,1)} \oplus S^{(3,2)}\oplus S^{(2,2,1)}.
\end{equation}

Además de el teorema \ref{teorema-isomorfismo-mod} y la expresión \ref{im1-KM5} se tiene:
$$(S^{(4,1)}\oplus S^{(2,1,1,1)}\oplus 2S^{(3,1,1)}\oplus S^{(3,2)}
\oplus S^{(2,2,1)})/\ker \widehat\partial_{1}\cong \im \widehat\partial_{1}.$$
De lo anterior y la proposición \ref{modulos-iguales} tenemos:
\begin{equation}
  \label{ker1-KM5}
  \ker \widehat\partial_{1}=S^{(2,1,1,1)}\oplus 2S^{(3,1,1)}.
\end{equation}
Por otro lado,
\begin{equation*}
  \im\widehat\partial_{3}=\widehat\partial_{3}(0)=0,
  \label{im3-KM5}
\end{equation*}
pues $\widehat\partial_{3}$ es un homomorfismo de módulos. Ya que
$M_{5}\simeq K(M_{5})$ (resultado de \cite{larrion2009clique}), por el teorema
\ref{esp-homotopicos-homologias-iso}, tenemos que $\widetilde
H_{2}(M_{5})\cong\widetilde H_{2}(K(M_{5}))$ y  $\widetilde H_{2}(M_{5})=0$, así que:

\begin{equation*}
\widetilde H_{2}(K(M_{5}))=\ker \widehat\partial_{2}/\im \widehat\partial_{3}=0,
\end{equation*}
por lo que $\ker \widehat\partial_{2}=0$. De el teorema
\ref{teorema-isomorfismo-mod} tenemos:
$$(S^{(2,1,1,1)}\oplus S^{(3,1,1)})/\ker \widehat\partial_{2}\cong \im
\widehat\partial_{2}.$$
De lo anterior y la proposición \ref{modulos-iguales} tenemos:
\begin{equation}
  \im \widehat\partial_{2}=S^{(2,1,1,1)}\oplus S^{(3,1,1)}.
  \label{im2-KM5}
\end{equation}

De las ecuaciones \ref{ker1-KM5} y \ref{im2-KM5} tenemos:

\begin{eqnarray*}
  \widetilde H_{1}(K(M_{5}))&=&\ker \widehat\partial_{1}/\im
  \widehat\partial_{2}\\
  &=&(S^{(2,1,1,1)}\oplus 2S^{(3,1,1)})/(S^{(2,1,1,1)}\oplus S^{(3,1,1)})=S^{(3,1,1)}.
\end{eqnarray*}

Anteriormente usamos el hecho de que $M_{5}\simeq K(M_{5})$  para
concluir que $\widetilde H_{2}(K(M_{5}))=\widetilde
H_{2}(M_{5})=0$. Sin embargo veamos que 
es posible calcular $\im \widehat\partial_{2}$ por métodos elementales.

Verifiquemos primero que $\dim(\im \partial_{2})=10$. Del teorema
\ref{imT}, sabemos que
$\im \partial_{2}=\langle\partial_{2}(c_{1}),\ldots,\partial_{2}(c_{10})\rangle$,
veamos que el número de vectores linealmente independientes de la
$\im \partial_{2}$ es 10, es
decir, si
$$\lambda_{1}\partial_{2}(c_{1})+\lambda_{2}\partial_{2}(c_{2})+\ldots+\lambda_{10}\partial_{2}(c_{10})=0$$
así que:

\begin{footnotesize}
  \begin{align*}
    &\lambda_{1}(a_{2}a_{3}-a_{1}a_{3}+a_{1}a_{2})+\lambda_{2}(a_{12}a_{15}-a_{1}a_{15}+a_{1}a_{12})+\lambda_{3}(a_{9}a_{14}-a_{2}a_{14}+a_{2}a_{9})\\
    &+\lambda_{4}(a_{6}a_{13}-a_{3}a_{13}+a_{3}a_{6})+\lambda_{5}(a_{5}a_{6}-a_{4}a_{6}+a_{4}a_{5})+\lambda_{6}(a_{11}a_{14}-a_{4}a_{14}+a_{4}a_{11})\\
    &+\lambda_{7}(a_{8}a_{15}-a_{5}a_{15}+a_{5}a_{8})+\lambda_{8}(a_{8}a_{9}-a_{7}a_{9}+a_{7}a_{8})+\lambda_{9}(a_{10}a_{13}-a_{7}a_{13}+a_{7}a_{10})\\
    &+\lambda_{10}(a_{11}a_{12}-a_{10}a_{12}+a_{10}a_{11})\\
    &=0.
  \end{align*}
\end{footnotesize}
Claramente $\lambda_{1}=\lambda_{2}=\cdots=\lambda_{10}=0$,
intuitivamente esto se puede deducir de la gráfica \ref{fig:KG_5}.
Sea $b_{j}$ la frontera de algún $2$-simplejo $c_{i}$, las aristas
de $b_{j}$ no pertenece a ningún otro triángulo distinto de $c_{i}$, por
lo tanto $\partial_{2}(c_{j})$ no es combinación lineal de las fronteras
$\partial_{2}(c_{i})$ con $i\neq j$.

Así que $\dim(\im\partial_{2})=10$, lo cual implica que
$\dim(\im\widehat\partial_{2})=10$ (se sigue de la definición de
$\widehat\partial_{2}$).

Ahora, por el teorema \ref{im-mod-irreducible} tenemos:
\begin{align*}
\widehat\partial_{2}(S^{(2,1,1,1)})=S^{(2,1,1,1)} \quad & \mbox{o }\quad \widehat\partial_{2}(S^{(2,1,1,1)})=0\\
&\mbox{y}\\
\widehat\partial_{2}(S^{(3,1,1)})=S^{(3,1,1)} \quad & \mbox{o }\quad \widehat\partial_{2}(S^{(3,1,1)})=0.
\end{align*}
Además $\dim(S^{(2,1,1,1)})=4$ y $\dim(S^{(3,1,1)})=6$, entonces:
$$\im\widehat\partial_{2}=\widehat\partial_{2}(S^{(2,1,1,1)}\oplus S^{(3,1,1)})=S^{(2,1,1,1)}\oplus S^{(3,1,1)}.$$

\section{Módulos de homología reducida de $M_{6}$}
\label{hom-red-M6}

Consideremos el complejo de emparejamientos $M_{6}$, que se obtiene de
la gráfica $G_{6}$ que se muestra en~\ref{fig:emparejamiento6}. El
complejo $M_{6}$ tiene
el conjunto de vértices (aristas de la gráfica de $K_{6}$):
\begin{center}
  \begin{tabular}[h]{lllll}
    $a_{1}=\overline{12}$ & $a_{4}=\overline{15}$ & $a_{7}=\overline{24}$ & $a_{10}=\overline{34}$ & $a_{13}=\overline{45}$  \\
    $a_{2}=\overline{13}$ & $a_{5}=\overline{16}$ & $a_{8}=\overline{25}$ & $a_{11}=\overline{35}$ & $a_{14}=\overline{46}$  \\
    $a_{3}=\overline{14}$ & $a_{6}=\overline{23}$ & $a_{9}=\overline{26}$ & $a_{12}=\overline{36}$ & $a_{15}=\overline{56}$  
  \end{tabular}
\end{center}
los $1$-simplejos:
\begin{center}
  \begin{tabular}[h]{llll}
    $b_{1}=(a_{1},a_{10})$ & $b_{13}=(a_{3},a_{6})$ & $b_{25}=(a_{5},a_{6})$ & $b_{37}=(a_{8},a_{10})$ \\
    $b_{2}=(a_{1},a_{11})$ & $b_{14}=(a_{3},a_{8})$ & $b_{26}=(a_{5},a_{7})$ & $b_{38}=(a_{8},a_{12})$ \\
    $b_{3}=(a_{1},a_{12})$ & $b_{15}=(a_{3},a_{9})$ & $b_{27}=(a_{5},a_{8})$ & $b_{39}=(a_{8},a_{14})$ \\
    $b_{4}=(a_{1},a_{13})$ & $b_{16}=(a_{3},a_{11})$ & $b_{28}=(a_{5},a_{10})$ & $b_{40}=(a_{9},a_{10})$ \\
    $b_{5}=(a_{1},a_{14})$ & $b_{17}=(a_{3},a_{12})$ & $b_{29}=(a_{5},a_{11})$ & $b_{41}=(a_{9},a_{11})$ \\
    $b_{6}=(a_{1},a_{15})$ & $b_{18}=(a_{3},a_{15})$ & $b_{30}=(a_{5},a_{13})$ & $b_{42}=(a_{9},a_{13})$ \\
    $b_{7}=(a_{2},a_{7})$ & $b_{19}=(a_{4},a_{6})$ & $b_{31}=(a_{6},a_{13})$ & $b_{43}=(a_{10},a_{15})$ \\
    $b_{8}=(a_{2},a_{8})$ & $b_{20}=(a_{4},a_{7})$ & $b_{32}=(a_{6},a_{14})$ & $b_{44}=(a_{11},a_{14})$ \\
    $b_{9}=(a_{2},a_{9})$ & $b_{21}=(a_{4},a_{9})$ & $b_{33}=(a_{6},a_{15})$ & $b_{45}=(a_{12},a_{13})$. \\
    $b_{10}=(a_{2},a_{13})$ & $b_{22}=(a_{4},a_{10})$ & $b_{34}=(a_{7},a_{11})$ &  \\
    $b_{11}=(a_{2},a_{14})$ & $b_{23}=(a_{4},a_{12})$ & $b_{35}=(a_{7},a_{12})$ &  \\
    $b_{12}=(a_{2},a_{15})$ & $b_{24}=(a_{4},a_{14})$ & $b_{36}=(a_{7},a_{15})$ &  
  \end{tabular}
\end{center}
y los $2$-simplejos:
\begin{center}
  \begin{tabular}[h]{lll}
    $c_{1}=(a_{1},a_{10},a_{15})$ & $c_{6}=(a_{2},a_{9},a_{13})$ & $c_{11}=(a_{4},a_{7},a_{12})$  \\
    $c_{2}=(a_{1},a_{11},a_{14})$ & $c_{7}=(a_{3},a_{6},a_{15})$ & $c_{12}=(a_{4},a_{9},a_{10})$  \\
    $c_{3}=(a_{1},a_{12},a_{13})$ & $c_{8}=(a_{3},a_{8},a_{12})$ & $c_{13}=(a_{5},a_{6},a_{13})$  \\
    $c_{4}=(a_{2},a_{7},a_{15})$ & $c_{9}=(a_{3},a_{9},a_{11})$ & $c_{14}=(a_{5},a_{7},a_{11})$  \\
    $c_{5}=(a_{2},a_{8},a_{14})$ & $c_{10}=(a_{4},a_{6},a_{14})$ & $c_{15}=(a_{5},a_{8},a_{10})$.  
  \end{tabular}
\end{center}

\begin{figure}[h]
  \centering
  \begin{tikzpicture}[scale=.5]
    \GraphInit[vstyle=Normal] \SetVertexNoLabel
    \grStar[RA=3.7,Math]{7}%
    \grEmptyCycle[RA=8,prefix=c]{24} \EdgeInGraphMod*{a}{6}{1}{0}{2}
    \EdgeInGraphMod*{c}{24}{1}{0}{2} \EdgeFromOneToSeq{a}{c}{1}{2}{5}
    \EdgeFromOneToSeq{a}{c}{2}{6}{9}
    \EdgeFromOneToSeq{a}{c}{3}{10}{13}
    \EdgeFromOneToSeq{a}{c}{4}{14}{17}
    \EdgeFromOneToSeq{a}{c}{5}{18}{21}
    \EdgeFromOneToSeq{a}{c}{0}{22}{23}
    \EdgeFromOneToSeq{a}{c}{0}{0}{1}
    \AssignVertexLabel{a}{\textsl{$a_{15}$},\textsl{$a_{10}$},\textsl{$a_{13}$},\textsl{$a_{12}$},\textsl{$a_{14}$},\textsl{$a_{11}$},\textsl{$a_{1}$}}
    \AssignVertexLabel{c}{\textsl{$a_{7}$},\textsl{$a_{2}$},\textsl{$a_{8}$},\textsl{$a_{5}$},\textsl{$a_{9}$},\textsl{$a_{4}$},\textsl{$a_{9}$},\textsl{$a_{2}$},\textsl{$a_{6}$},\textsl{$a_{5}$},\textsl{$a_{7}$},\textsl{$a_{4}$},\textsl{$a_{8}$},\textsl{$a_{3}$},\textsl{$a_{6}$},\textsl{$a_{4}$},\textsl{$a_{8}$},\textsl{$a_{2}$},\textsl{$a_{7}$},\textsl{$a_{5}$},\textsl{$a_{9}$},\textsl{$a_{3}$},\textsl{$a_{6}$},\textsl{$a_{3}$}}
  \end{tikzpicture}
  
  \caption{Gráfica de emparejamientos $G_{6}$}
  \label{fig:emparejamiento6}
\end{figure}

\begin{table}[!hbtp]
  \resizebox*{!}{5.5cm}{
    \centering
    \begin{tabular}{c |r r r r r r r r r r r}
      No. de Elementos  & 720 & 48 & 18 & 16 & 8 & 6 & 5 & 48 & 18 & 8 &6\\
      Clase&(1)& (2) & (3) & (2,2) & (4)& (3,2) & (5) & (2,2,2) & (3,3)
      & (4,2)  & (6) \\
      \hline
      $\chi_{S^{{(6)}}}$         & 1 & 1  & 1  & 1 & 1 & 1 & 1 & 1 & 1 & 1&1 \\
      $\chi_{S^{{(1,1,1,1,1,1)}}}$ & 1 & -1 & 1  & 1 & -1&-1 & 1 &-1 & 1& 1&-1 \\
      $\chi_{S^{{(5,1)}}}$       & 5 & 3  & 2  & 1 & 1 & 0 & 0 &-1 &-1&-1&-1 \\
      $\chi_{S^{{(2,1,1,1,1)}}}$  & 5 & -3 &  2 & 1 &-1 & 0 & 0 & 1 &-1&-1&1  \\
      $\chi_{S^{{(4,1,1)}}}$     & 10& 2  & 1  & -2& 0 &-1 & 0 &-2 & 1& 0&1  \\
      $\chi_{S^{{(3,1,1,1)}}}$    & 10&-2  & 1  &-2 & 0 & 1 & 0 & 2 & 1& 0&-1 \\
      $\chi_{S^{{(4,2)}}}$       & 9 & 3  & 0  & 1 & -1& 0 &-1 & 3 & 0& 1&0 \\
      $\chi_{S^{{(2,2,1,1)}}}$    & 9 & -3 & 0  & 1 & 1 & 0 &-1 &-3 & 0& 1&0 \\
      $\chi_{S^{{(3,3)}}}$       & 5 & 1  & -1  & 1 &-1 &1 & 0 & -3& 2& -1&0 \\
      $\chi_{S^{{(2,2,2)}}}$     & 5  & -1& -1  & 1 & 1 & -1& 0& 3 & 2& -1&0 \\
      $\chi_{S^{{(3,2,1)}}}$     & 16 & 0 & -2  & 0 & 0 & 0 & 1& 0 &-2& 0 &0
    \end{tabular}}
    
    \caption{Tabla de caracteres de $S_{6}$}
    \label{tab:S_6}
  \end{table}
  La tabla \ref{tab:S_6} nos ayudará a construir las tablas de
  caracteres restringidas a los conjuntos que se establecen a
  continuación, esto para obtener la descomposición de los $S_{6}$-módulos de
  cadenas $C_{k}(M_{6})$, para $k=0,1,2$, en
  módulos de Specht de forma análoga como se hizo en los ejemplos anteriores.
  
  Consideremos:
  $$V_{0}=\langle a_{1}\rangle=\langle (\overline{12})\rangle,$$

  \begin{footnotesize}
    \begin{align*}
      H_{0}=&\{g\in S_{6}\mid
      gV_{0}=V_{0}\}=\{(1),(12),(34),(35),(36),(45),(46),(56),(345),(346),\\
      &(354),(356),(364),(365),(456),(465),(12)(34),(12)(35),(12)(36),(12)(45),\\
      &(12)(46),(12)(56),(34)(56),(35)(46),(36)(45),(3456),(3465),(3546),(3564),\\
      &(3645),(3654),(345)(12),(346)(12),(354)(12),(356)(12),(364)(12),(365)(12),\\
      &(456)(12),(465)(12),(12)(34)(56),(12)(35)(46),(12)(36)(45),(3456)(12),\\
      &(3465)(12),(3546)(12),(3564)(12),(3645)(12),(3654)(12)\}.
    \end{align*}
  \end{footnotesize}

 \begin{table}[!hbtp]
    \centering
    \begin{tabular}{c |r r r r r r r r}
      No. de Elementos  & 1 & 7 & 8 & 9 & 6 & 8 & 3 & 6  \\
      Clase&(1)& (2) & (3) & (2,2) & (4)& (3,2) & (2,2,2) & (4,2)\\
      \hline
      $\chi_{S^{{(6)}}\downarrow_{H_{0}}}$         & 1 & 1  & 1  & 1 & 1 & 1 & 1 & 1 \\
      $\chi_{S^{{(1,1,1,1,1,1)}}\downarrow_{H_{0}}}$ & 1 & -1 & 1  & 1 & -1&-1 &-1 & 1 \\
      $\chi_{S^{{(5,1)}}\downarrow_{H_{0}}}$       & 5 & 3  & 2  & 1 & 1 & 0 &-1 &-1 \\
      $\chi_{S^{{(2,1,1,1,1)}}\downarrow_{H_{0}}}$  & 5 & -3 &  2 & 1 &-1 & 0 & 1 &-1  \\
      $\chi_{S^{{(4,1,1)}}\downarrow_{H_{0}}}$     & 10& 2  & 1  & -2& 0 &-1 &-2 & 0  \\
      $\chi_{S^{{(3,1,1,1)}}\downarrow_{H_{0}}}$    & 10&-2  & 1  &-2 & 0 & 1 & 2 & 0 \\
      $\chi_{S^{{(4,2)}}\downarrow_{H_{0}}}$       & 9 & 3  & 0  & 1 & -1& 0 & 3 & 1 \\
      $\chi_{S^{{(2,2,1,1)}}\downarrow_{H_{0}}}$    & 9 & -3 & 0  & 1 & 1 & 0 &-3 & 1 \\
      $\chi_{S^{{(3,3)}}\downarrow_{H_{0}}}$       & 5 & 1  & -1  & 1 &-1 &1 & -3& -1 \\
      $\chi_{S^{{(2,2,2)}}\downarrow_{H_{0}}}$     & 5  & -1& -1  & 1 & 1 & -1& 3 & -1 \\
      $\chi_{S^{{(3,2,1)}}\downarrow_{H_{0}}}$     & 16 & 0 & -2  & 0 & 0 & 0 & 0 & 0 \\
      \hline
      $\chi_{V_{0}}$        & 1 & 1 & 1 & 1 & 1 & 1 &1 & 1  \\
    \end{tabular}

    \caption{Caracteres de $S_{6}$ restringidos a $H_{0}$ y carácter de $V_{0}$}
    \label{tab:restriccion-H_0-M-6}
  \end{table}

  De los productos internos calculados a partir de la la tabla
  \ref{tab:restriccion-H_0-M-6} obtenemos:
  \begin{equation}
    C_{0}(M_{6})\cong \mathbb{C}\oplus S^{(5,1)}\oplus S^{(4,2)}.
    \label{C0-M6}
  \end{equation}
  Ahora tomemos:
  \begin{equation*}
    V_{1}=\langle b_{1}\rangle =\langle
    (a_{1},a_{2})\rangle=\langle(\overline{12},\overline{34})\rangle,
  \end{equation*}
  \begin{small}
    \begin{align*}
      H_{1}=&\{g \in S_{6}\mid
      gV_{1}=V_{1}\}=\{(1),(12),(34),(56),(12)(34),(12)(56),(34)(56),\\
      &(13)(24),(14)(23),(12)(34)(56),(13)(24)(56),(14)(23)(56),(2314),\\
      &(2413),(2314)(56),(2413)(56)\}.
    \end{align*}
  \end{small}
  Con $V_{1}$ y $H_{1}$ construimos la tabla
  \ref{tab:restriccion-H-1-M-6} y calculamos los productos internos
  correspondientes con la reciprocidad de Frobenius de forma similar a los ejemplos anteriores.
  
 \begin{table}[!hbtp]
    \resizebox*{!}{5.8cm}{
      \centering
      \begin{tabular}{c |r r r r r r r r}
        &   & (12) & (12)(34) &          &             &              & &  \\        
        &   & (34) & (12)(56) &(13)(24)  &             & (13)(24)(56) & (2314)& (2314)(56)\\
        Elementos &(1)& (56) & (34)(56) & (14)(23) & (12)(34)(56)& (14)(23)(56) &(2413)&(2413)(56)\\
        \hline
        $\chi_{S^{{(6)}}\downarrow_{H_{1}}}$           & 1 & 1  & 1  & 1 & 1 & 1 & 1 & 1 \\
        $\chi_{S^{{(1,1,1,1,1,1)}}\downarrow_{H_{1}}}$ & 1 & -1 & 1  & 1 &-1 &-1 &-1 & 1  \\
        $\chi_{S^{{(5,1)}}\downarrow_{H_{1}}}$         & 5 & 3  & 1  & 1 &-1 &-1 & 1 &-1  \\
        $\chi_{S^{{(2,1,1,1,1)}}\downarrow_{H_{1}}}$   & 5 & -3 &  1 & 1 & 1 & 1 &-1 &-1  \\
        $\chi_{S^{{(4,1,1)}}\downarrow_{H_{1}}}$       & 10& 2  & -2 & -2&-2 &-2 & 0 & 0  \\
        $\chi_{S^{{(3,1,1,1)}}\downarrow_{H_{1}}}$     & 10&-2  & -2 &-2 & 2 & 2 & 0 & 0  \\
        $\chi_{S^{{(4,2)}}\downarrow_{H_{1}}}$         & 9 & 3  & 1  & 1 & 3 &3  &-1 & 1  \\
        $\chi_{S^{{(2,2,1,1)}}\downarrow_{H_{1}}}$     & 9 & -3 & 1  & 1 & -3&-3 & 1 & 1  \\
        $\chi_{S^{{(3,3)}}\downarrow_{H_{1}}}$         & 5 & 1  & 1  & 1 &-3 &-3 &-1 & -1 \\
        $\chi_{S^{{(2,2,2)}}\downarrow_{H_{1}}}$       & 5 & -1 & 1  & 1 & 3 & 3 & 1 & -1 \\
        $\chi_{S^{{(3,2,1)}}\downarrow_{H_{1}}}$       & 16& 0  & 0  & 0 & 0 & 0 & 0 &  0 \\
        \hline
        $\chi_{V_{1}}$            & 1 & 1  & 1  & -1& 1 & -1&-1 &-1  \\
      \end{tabular}}
    
    \caption{Caracteres de $S_{6}$ restringidos a $H_{1}$ y carácter de $V_{1}$}
    \label{tab:restriccion-H-1-M-6}
  \end{table}  
  %\begin{footnotesize}
    De donde obtenemos:
    \begin{equation}
      C_{1}(M_{6})=S^{(5,1)}\oplus S^{{(4,1,1)}} \oplus S^{{(4,2)}}
      \oplus S^{{(3,3)}} \oplus S^{{(3,2,1)}}.
      \label{C1-M6}
    \end{equation}
    Por último consideremos:
    \begin{equation*}
      V_{2}=\langle c_{1}\rangle=\langle(\overline{12},\overline{34},\overline{56})\rangle,
    \end{equation*}
  %\end{footnotesize}
  \begin{tiny}
    \begin{align*}
      H_{2}=&\{g\in S_{6}\mid
      gV_{2}=V_{2}\}=\{(1),(12),(34),(56),(12)(34),(12)(56),(34)(56),(13)(24),(14)(23),(15)(26),(16)(25),\\
      &(35)(46),(36)(45),(12)(34)(56),(13)(24)(56),(14)(23)(56),(15)(26)(34),(16)(25)(34),(35)(46)(12),\\
      &(36)(45)(12),(2314),(2413),(1625),(1526),(3546),(3645),(2314)(56),(2413)(56),(1625)(34),(1526)(34),\\
      &(3546)(12),(3645)(12),(135)(246),(136)(245),(145)(236),(146)(235),(153)(264),(154)(263),(163)(254),\\
      &(164)(253),(135246),(136245),(145236),(146235),(153264),(154263),(163254),(164253)\}.
    \end{align*}
  \end{tiny}

  \begin{table}[h]
    \centering
    \resizebox*{!}{5.8cm}{
      \begin{tabular}{c |r r r r r r r r r r}
        No. de Elementos  & 1 & 3 & 3 & 6 & 6 & 1 & 6 & 8 & 6 & 8 \\
        Clase &(1)& (2) & (2,2) & (2,2) & (4)& (2,2,2) & (2,2,2) & (3,3) & (4,2) & (6)\\
        \hline
        $\chi_{S^{{(6)}}\downarrow_{H_{2}}}$         & 1 & 1  & 1  & 1 & 1 &1  & 1 & 1 & 1 & 1 \\
        $\chi_{S^{{(1,1,1,1,1,1)}}\downarrow_{H_{2}}}$ & 1 & -1 & 1  & 1 & -1&-1 &-1 & 1 & 1 & -1 \\
        $\chi_{S^{{(5,1)}}\downarrow_{H_{2}}}$       & 5 & 3  & 1  & 1 & 1 &-1 &-1 &-1 &-1 & -1 \\
        $\chi_{S^{{(2,1,1,1,1)}}\downarrow_{H_{2}}}$  & 5 & -3 &  1 & 1 &-1 & 1 & 1 &-1  &-1 & 1 \\
        $\chi_{S^{{(4,1,1)}}\downarrow_{H_{2}}}$     & 10& 2  & -2 & -2& 0 &-2 &-2 & 1  & 0 & 1 \\
        $\chi_{S^{{(3,1,1,1)}}\downarrow_{H_{2}}}$    & 10&-2  & -2 &-2 & 0 & 2 & 2 & 1 & 0 & -1 \\
        $\chi_{S^{{(4,2)}}\downarrow_{H_{2}}}$       & 9 & 3  & 1  & 1 & -1& 3 & 3 & 0 & 1 &  0 \\
        $\chi_{S^{{(2,2,1,1)}}\downarrow_{H_{2}}}$    & 9 & -3 & 1  & 1 & 1 &-3 &-3 & 0 & 1 &  0 \\
        $\chi_{S^{{(3,3)}}\downarrow_{H_{2}}}$       & 5 & 1  & 1  & 1 &-1 &-3 & -3& 2 &-1 &  0 \\
        $\chi_{S^{{(2,2,2)}}\downarrow_{H_{2}}}$     & 5  & -1& 1  & 1 & 1 & 3 & 3 & 2 & -1 & 0 \\
        $\chi_{S^{{(3,2,1)}}\downarrow_{H_{2}}}$     & 16 & 0 & 0  & 0 & 0 & 0 & 0 & -2 & 0 & 0 \\
        \hline
        $\chi_{V_{2}}$ & 1 & 1 & 1 & -1 & -1 & 1 &-1 & 1 &- 1 & 1  \\
      \end{tabular}}

    \caption{Caracteres de $S_{6}$ restringidos a $H_2$ y carácter de $V_{2}$}
    \label{tab:restriccion-H_2-M-6}
  \end{table}
  De la tabla \ref{tab:restriccion-H_2-M-6} obtenemos:
  \begin{equation}
    C_{2}(M_{6})\cong S^{(4,1,1)}\oplus S^{(3,3)}.
    \label{C2-M6}
  \end{equation}

  En la figura \ref{fig:diagrama-conmutativo6} se muestra el diagrama
  conmutativo de los complejos de cadena de $M_{6}$, donde $f_{0}$,
  $f_{1}$ y $f_{2}$ son los isomorfismos obtenidos de las expresiones
  \ref{C0-M6}, \ref{C1-M6} y \ref{C2-M6}.

  \begin{figure}[h]
    \centering
    \begin{widepage}
      \scriptsize{
        \[
        \begin{CD}
          0 @>{\partial_{3}}>> C_{2}(M_{6}) @>{\partial_{2}}>> C_{1}(M_{6}) @>{\partial_{1}}>> C_{0}(M_{6}) @>{\varepsilon}>> \mathbb{C}\\
          @VVV @Vf_{2}VV   @Vf_{1}VV  @Vf_{0}VV  @VVV    \\
          0 @>{\widehat\partial_{3}}>> S^{(4,1,1)}\oplus S^{(3,3)}
          @>{\widehat\partial_{2}}>> S^{(5,1)}\oplus S^{(4,1,1)}\oplus
          S^{(4,2)}\oplus S^{(3,3)}\oplus S^{(3,2,1)}
          @>{\widehat\partial_{1}}>> \mathbb{C} \oplus S^{(5,1)}\oplus
          S^{(4,2)} @>{\widehat \varepsilon}>> \mathbb{C}
        \end{CD}
        \]
      }
    \end{widepage}
\caption{Diagrama conmutativo de los complejos de cadenas de $M_{6}$}
\label{fig:diagrama-conmutativo6}
\end{figure}

Por calcular los módulos de homología reducida $\widetilde
H_{k}(M_{6})$, para $k=0,1,2$.

Como $\widehat\varepsilon$ es sobreyectiva y por el teorema \ref{teorema-isomorfismo-mod} tenemos:
\begin{equation*}
  (\mathbb{C} \oplus S^{(5,1)}\oplus S^{(4,2)})/\ker\widehat\varepsilon\cong \im \widehat\varepsilon=\mathbb{C},
\end{equation*}

así que
\begin{equation*}
  \label{ker0-M6}
  \ker\widehat\varepsilon\cong  S^{(5,1)}\oplus S^{(4,2)}, 
\end{equation*}
Sabemos que $\widetilde H_{0}(M_{6})=\ker \widehat\varepsilon/\im
\widehat\partial_{1}=0$ pues $M_{6}$ es conexo, se sigue que~$\ker \widehat\varepsilon\cong
\im\widehat\partial_{1}$, con lo cual:
\begin{equation}
  \label{im1-M6}
  \im \widehat\partial_{1}\cong  S^{(5,1)}\oplus S^{(4,2)}. 
\end{equation}

Además de el teorema \ref{teorema-isomorfismo-mod} y la expresión \ref{im1-M6} se tiene:
$$(S^{(5,1)}\oplus S^{(4,1,1)}\oplus S^{(4,2)}\oplus S^{(3,3)}\oplus S^{(3,2,1)})/\ker \widehat\partial_{1}\cong \im \widehat\partial_{1}$$
De lo anterior y la proposición \ref{modulos-iguales} tenemos:
\begin{equation}
  \label{ker1-M6}
  \ker \widehat\partial_{1}=S^{(4,1,1)}\oplus S^{(3,3)}\oplus S^{(3,2,1)}.
\end{equation}
\begin{figure}[h]
  \centering
  \begin{center}
    \begin{minipage}{0.45\linewidth}
      % \centering
      % \begin{figure}[h]
      \centering
      \begin{tikzpicture}[scale=.3]
        \SetVertexNoLabel \GraphInit[vstyle=Classic]
        \SetUpVertex[MinSize=1pt] \grStar[RA=4,Math]{7}
        \grEmptyCycle[RA=8,prefix=c]{24}
        \EdgeInGraphMod*{a}{6}{1}{0}{2}
        \EdgeInGraphMod*{c}{24}{1}{0}{2}
        \EdgeFromOneToSeq{a}{c}{1}{2}{5}
        \EdgeFromOneToSeq{a}{c}{2}{6}{9}
        \EdgeFromOneToSeq{a}{c}{3}{10}{13}
        \EdgeFromOneToSeq{a}{c}{4}{14}{17}
        \EdgeFromOneToSeq{a}{c}{5}{18}{21}
        \EdgeFromOneToSeq{a}{c}{0}{22}{23}
        \EdgeFromOneToSeq{a}{c}{0}{0}{1}
      \end{tikzpicture}

      \caption{$G_{6}$}
      \label{fig:emparejamiento6-1}
      % \end{figure}
    \end{minipage}%\quad
    \begin{minipage}{0.45\linewidth}
      % \begin{figure}[h]
      \centering
      \begin{tikzpicture}[scale=.3]
        \SetVertexNoLabel \GraphInit[vstyle=Classic]
        \SetUpVertex[MinSize=1pt] \grStar[RA=2,Math,rotation=270]{4}
        \grEmptyCycle[RA=4,Math,prefix=b]{6}
        \grEmptyCycle[RA=6,Math,prefix=c,rotation=-15]{12}
        \grEmptyCycle[RA=8,Math,prefix=d,rotation=-22]{24}
        \EdgeFromOneToSeq{a}{b}{1}{0}{1}
        \EdgeFromOneToSeq{a}{b}{2}{2}{3}
        \EdgeFromOneToSeq{a}{b}{0}{4}{5}
        \EdgeFromOneToSeq{b}{c}{0}{0}{1}
        \EdgeFromOneToSeq{b}{c}{1}{2}{3}
        \EdgeFromOneToSeq{b}{c}{2}{4}{5}
        \EdgeFromOneToSeq{b}{c}{3}{6}{7}
        \EdgeFromOneToSeq{b}{c}{4}{8}{9}
        \EdgeFromOneToSeq{b}{c}{5}{10}{11}
        \EdgeFromOneToSeq{c}{d}{0}{0}{1}
        \EdgeFromOneToSeq{c}{d}{1}{2}{3}
        \EdgeFromOneToSeq{c}{d}{2}{4}{5}
        \EdgeFromOneToSeq{c}{d}{3}{6}{7}
        \EdgeFromOneToSeq{c}{d}{4}{8}{9}
        \EdgeFromOneToSeq{c}{d}{5}{10}{11}
        \EdgeFromOneToSeq{c}{d}{6}{12}{13}
        \EdgeFromOneToSeq{c}{d}{7}{14}{15}
        \EdgeFromOneToSeq{c}{d}{8}{16}{17}
        \EdgeFromOneToSeq{c}{d}{9}{18}{19}
        \EdgeFromOneToSeq{c}{d}{10}{20}{21}
        \EdgeFromOneToSeq{c}{d}{11}{22}{23}
      \end{tikzpicture}

      \caption{$BK(G_{6})$}
      \label{grafica-contraible-M6}
      % \end{figure};
    \end{minipage}
  \end{center}
\end{figure}

Por otro lado, $\widehat\partial_{3}$ es un homomorfismo de módulos, así que:
\begin{equation*}
  \im\widehat\partial_{3}=\widehat\partial_{3}(0)=0.
  \label{im3-KM5}
\end{equation*}
Considerando la figura \ref{fig:emparejamiento6-1}. Se observa que contrayendo los
triángulos, se encuentra que $G_{6}$ es homotópica a $BK(G_{6})$ esto
puede demostrarse formalmente ocupando los resultados que aparecen en
\cite{LPV08a} y \cite{Pri92}. Por el teorema~\ref{esp-homotopicos-homologias-iso}, $\widetilde
H_{2}(M_{6})\cong\widetilde H_{2}(\Delta(BK(G_{6})))$. Además
$\widetilde H_{2}(\Delta(BK(G_{6})))=0$, pues en~$BK(G_{6})$ no existen $2$-simplejos.
%$$G_{6}\simeq
%H(G_{6})=\varepsilon(BK(G_{6}))=L(BK(G_{6}))=K(BK(G_{6}))\simeq BK(G_{6})$$

Por lo que:
\begin{equation*}
  \widetilde H_{2}(M_{6})=\ker \widehat\partial_{2}/\im \widehat\partial_{3}=0,
\end{equation*}
por lo que $\im \widehat\partial_{3}=\ker \widehat\partial_{2}$. De el teorema
\ref{teorema-isomorfismo-mod} tenemos:
$$(S^{(4,1,1)}\oplus S^{(3,3)})/\ker \widehat\partial_{2}\cong \im \widehat\partial_{2}.$$
De lo anterior y la proposición \ref{modulos-iguales} tenemos:
\begin{equation}
\im \widehat\partial_{2}=S^{(4,1,1)}\oplus S^{(3,3)}.
\label{im2-M6}
\end{equation}

De las ecuaciones \ref{ker1-M6}, \ref{im2-M6} y la proposición
\ref{modulos-iguales} tenemos:
\begin{eqnarray*}
  \widetilde H_{1}(M_{6})&=&\ker \widehat\partial_{1}/\im
  \widehat\partial_{2}\\
  &=&(S^{(4,1,1)}\oplus S^{(3,3)}\oplus
  S^{(3,2,1)})/(S^{(4,1,1)}\oplus S^{(3,3)})\\
  &=&S^{(3,2,1)}.
\end{eqnarray*}

En lo anterior usamos el hecho de que $G_{6}\simeq BK(G_{6})$, de donde
se sigue que $\widetilde H_{2}(M_{6})\cong\widetilde
H_{2}(\Delta(BK(G_{6})))=0$. Veamos ahora
que podemos calcular $\im \widehat\partial_{2}$ por métodos elementales.

Verifiquemos primero que $\dim(\im \partial_{2})=15$. Del teorema
\ref{imT}, sabemos que $\im \partial_{2}=\langle\partial_{2}(c_{1}),\ldots,\partial_{2}(c_{15})\rangle$,
veamos que el número de vectores linealmente independientes de la
$\im \partial_{2}$ es 15, es decir, si
$$\lambda_{1}\partial_{2}(c_{1})+\lambda_{2}\partial_{2}(c_{2})+\ldots+\lambda_{15}\partial_{2}(c_{15})=0$$
entonces,
\begin{footnotesize}
  \begin{align*}
    &\lambda_{1}(a_{10}a_{15}-a_{1}a_{15}+a_{1}a_{10})+\lambda_{2}(a_{11}a_{14}-a_{1}a_{14}+a_{1}a_{11})+\lambda_{3}(a_{12}a_{13}-a_{1}a_{13}+a_{1}a_{12})\\
    &+\lambda_{4}(a_{7}a_{15}-a_{2}a_{15}+a_{2}a_{7})+\lambda_{5}(a_{8}a_{14}-a_{2}a_{14}+a_{2}a_{8})+\lambda_{6}(a_{9}a_{13}-a_{2}a_{13}+a_{2}a_{9})\\
    &+\lambda_{7}(a_{6}a_{15}-a_{3}a_{15}+a_{3}a_{6})+\lambda_{8}(a_{8}a_{12}-a_{3}a_{12}+a_{3}a_{8})+\lambda_{9}(a_{9}a_{11}-a_{3}a_{11}+a_{3}a_{9})\\
    &+\lambda_{10}(a_{6}a_{14}-a_{4}a_{14}+a_{4}a_{6})+\lambda_{11}(a_{7}a_{12}-a_{4}a_{12}+a_{4}a_{7})+\lambda_{12}(a_{9}a_{10}-a_{4}a_{10}+a_{4}a_{9})\\
    &+\lambda_{13}(a_{6}a_{13}-a_{5}a_{13}+a_{5}a_{6})+\lambda_{14}(a_{7}a_{11}-a_{5}a_{11}+a_{5}a_{7})+\lambda_{15}(a_{8}a_{10}-a_{5}a_{10}+a_{5}a_{8})\\
    &=0.
  \end{align*}
\end{footnotesize}
Claramente $\lambda_{1}=\lambda_{2}=\cdots=\lambda_{15}=0$,
intuitivamente esto se puede deducir de la gráfica
\ref{fig:emparejamiento6}. Sea $b_{j}$ la frontera de algún $2$-simplejo $c_{i}$, las aristas
de $b_{j}$ no pertenece a ningún otro triángulo distinto de $c_{i}$, por
lo tanto $\partial_{2}(c_{j})$ no es combinación lineal de las fronteras
$\partial_{2}(c_{i})$ con $i\neq j$.

Así que $\dim(\im\partial_{2})=15$, lo cual implica que
$\dim(\im\widehat\partial_{2})=15$ (se sigue de la definición de
$\widehat\partial_{2}$).

Ahora, por el teorema \ref{im-mod-irreducible} tenemos:
\begin{align*}
  \widehat\partial_{2}(S^{(4,1,1)})=S^{(4,1,1)} \quad &\mbox{o }\quad \widehat\partial_{2}(S^{(4,1,1)})=0\\
  &\mbox{y}\\
  \widehat\partial_{2}(S^{(3,3)})=S^{(3,3)} \quad &\mbox{o }\quad \widehat\partial_{2}(S^{(3,3)})=0
\end{align*}

  Además $\dim(S^{(4,1,1)})=10$ y $\dim(S^{(3,3)})=5$, entonces:
  $$\im\widehat\partial_{2}=\widehat\partial_{2}(S^{(4,1,1)}\oplus
  S^{(3,3)})=S^{(4,1,1)}\oplus S^{(3,3)}.$$

\section{Módulos de homología reducida de $K(M_{6})$}
\label{hom-red-KM6}

Consideremos el complejo de clanes de la gráfica de emparejamientos
$M_{6}$, al que denotamos como $K(M_{6})$, que está dado por el
conjunto de vértices (triángulos de la gráfica de
$M_{6}$):

\begin{center}
  \begin{tabular}[h]{lll}
    $a_{1}=(\overline{12},\overline{34},\overline{56})$&$a_{6}=(\overline{13},\overline{26},\overline{45})$&$a_{11}=(\overline{15},\overline{24},\overline{36})$  \\
    $a_{2}=(\overline{12},\overline{35},\overline{46})$&$a_{7}=(\overline{14},\overline{23},\overline{56})$&$a_{12}=(\overline{15},\overline{26},\overline{34})$  \\
    $a_{3}=(\overline{12},\overline{36},\overline{45})$&$a_{8}=(\overline{14},\overline{25},\overline{36})$&$a_{13}=(\overline{16},\overline{23},\overline{45})$  \\
    $a_{4}=(\overline{13},\overline{24},\overline{56})$&$a_{9}=(\overline{14},\overline{26},\overline{35})$&$a_{14}=(\overline{16},\overline{24},\overline{35})$ \\
    $a_{5}=(\overline{13},\overline{25},\overline{46})$&$a_{10}=(\overline{15},\overline{23},\overline{46})$&$a_{15}=(\overline{16},\overline{25},\overline{34}).$  
  \end{tabular}
\end{center}

el conjunto $1$-simplejos:
\begin{center}
  \begin{tabular}[h]{llll}
    $b_{1}=(a_{1},a_{2})$ & $b_{13}=(a_{3},a_{8})$ & $b_{25}=(a_{4},a_{14})$ & $b_{37}=(a_{9},a_{12})$ \\     
    $b_{2}=(a_{1},a_{3})$ & $b_{14}=(a_{3},a_{11})$ &$b_{26}=(a_{5},a_{8})$ & $b_{38}=(a_{10},a_{11})$\\     
    $b_{3}=(a_{1},a_{4})$ & $b_{15}=(a_{3},a_{13})$ & $b_{27}=(a_{5},a_{6})$&$b_{39}=(a_{10},a_{12})$\\     
    $b_{4}=(a_{1},a_{7})$ & $b_{16}=(a_{4},a_{7})$ & $b_{28}=(a_{5},a_{15})$& $b_{40}=(a_{10},a_{13})$ \\     
    $b_{5}=(a_{1},a_{12})$ & $b_{17}=(a_{5},a_{10})$ & $b_{29}=(a_{6},a_{9})$& $b_{41}=(a_{11},a_{12})$\\     
    $b_{6}=(a_{1},a_{15})$ & $b_{18}=(a_{6},a_{13})$ & $b_{30}=(a_{6},a_{12})$&$b_{42}=(a_{11},a_{14})$\\ 
    $b_{7}=(a_{2},a_{3})$ & $b_{19}=(a_{8},a_{11})$ & $b_{31}=(a_{7},a_{8})$&$b_{43}=(a_{13},a_{14})$\\
    $b_{8}=(a_{2},a_{5})$ & $b_{20}=(a_{9},a_{14})$ & $b_{32}=(a_{7},a_{9})$&$b_{44}=(a_{13},a_{15})$\\
    $b_{9}=(a_{2},a_{9})$ & $b_{21}=(a_{12},a_{15})$ & $b_{33}=(a_{7},a_{10})$&$b_{45}=(a_{14},a_{15})$.\\
    $b_{10}=(a_{2},a_{10})$ & $b_{22}=(a_{4},a_{5})$ & $b_{34}=(a_{7},a_{13})$&\\
    $b_{11}=(a_{2},a_{14})$ & $b_{23}=(a_{4},a_{6})$ & $b_{35}=(a_{8},a_{9})$&\\
    $b_{12}=(a_{3},a_{6})$ & $b_{24}=(a_{4},a_{11})$ & $b_{36}=(a_{8},a_{15})$&
  \end{tabular}
\end{center}

y los $2$-simplejos:
\begin{center}
  \begin{tabular}[h]{lll}
    $c_{1}=(a_{1},a_{2},a_{3})$ & $c_{6}=(a_{3},a_{6},a_{13})$ & $c_{11}=(a_{13},a_{14},a_{15})$   \\
    $c_{2}=(a_{1},a_{4},a_{7})$ & $c_{7}=(a_{3},a_{8},a_{11})$ & $c_{12}=(a_{7},a_{10},a_{13})$\\
    $c_{3}=(a_{1},a_{12},a_{15})$& $c_{8}=(a_{4},a_{5},a_{6})$ & $c_{13}=(a_{4},a_{11},a_{14})$\\
    $c_{4}=(a_{2},a_{5},a_{10})$& $c_{9}=(a_{7},a_{8},a_{9})$ & $c_{14}=(a_{5},a_{8},a_{15})$\\
    $c_{5}=(a_{2},a_{9},a_{14})$& $c_{10}=(a_{10},a_{11},a_{12})$ & $c_{15}=(a_{6},a_{9},a_{12})$.
  \end{tabular}
\end{center}

\begin{figure}[h]
  \centering
  \begin{tikzpicture}[scale=.8]
    \GraphInit[vstyle=Normal] \SetVertexNoLabel
    \grCycle[RA=1,prefix=a,rotation=-90]{3}
    \grEmptyCycle[RA=3,prefix=b,rotation=-10]{12}
    \EdgeInGraphMod*{b}{12}{1}{0}{2} \EdgeFromOneToSeq{a}{b}{1}{0}{3}
    \EdgeFromOneToSeq{a}{b}{2}{4}{7} \EdgeFromOneToSeq{a}{b}{0}{8}{11}
    \AssignVertexLabel{a}{\textsl{$a_{2}$},\textsl{$a_{1}$},\textsl{$a_{3}$}}
    \AssignVertexLabel{b}{\textsl{$a_{7}$},\textsl{$a_{4}$},\textsl{$a_{15}$},\textsl{$a_{12}$},\textsl{$a_{6}$},\textsl{$a_{13}$},\textsl{$a_{11}$},\textsl{$a_{8}$},\textsl{$a_{10}$},\textsl{$a_{5}$},\textsl{$a_{14}$},\textsl{$a_{9}$}}
  \end{tikzpicture}
  
  \caption{Gráfica de clanes $K(G_{6})$}
  \label{fig:KM6}
\end{figure}

A continuación obtendremos la descomposición de los $S_{6}$-módulos de
cadenas $C_{k}(K(M_{6}))$, para $k=0,1,2$, en
módulos de Specht por medio de el teorema \ref{frobenius} de la reciprocidad de Frobenius.

Consideremos:
  $$V_{0}=\langle a_{1}\rangle=\langle(\overline{12},\overline{34},\overline{56})\rangle,$$
  \begin{footnotesize}
    \begin{align*}
      H_{0}=&\{g\in S_{6}\mid
      gV_{0}=V_{0}\}=\{(1),(12),(34),(56),(12)(34),(12)(56),(13)(24),(14)(23),\\
      &(15)(26),(16)(25),(34)(56),(35)(46),(36)(45),(1423),(1324),(1625),(1526),\\
      &(3645),(3546),(12)(34)(56),(13)(24)(56),(14)(23)(56),(15)(26)(34),(16)(25)(34),\\
      &(35)(46)(12),(36)(45)(12),(153)(264),(135)(246),(145)(236),(154)(263),\\
      &(136)(245),(163)(254),(146)(235),(164)(253),(1423)(56),(1324)(56),(1625)(34),\\
      &(1526)(34),(3645)(12),(3546)(12),(135246),(136245),(145236),(146235),\\
      &(153264),(154263),(163254),(164253)\}.
    \end{align*}
  \end{footnotesize}

  
  \begin{table}[!hbtp]
    \centering
    \begin{tabular}{c |r r r r r r r r}
      No. de Elementos  & 1 & 3 & 9 & 6 & 7 & 8 & 6 & 8  \\
      Clase&(1)& (2) & (2,2) & (4) & (2,2,2)& (3,3) & (4,2) & (6)\\
      \hline
      $\chi_{S^{{(6)}}\downarrow_{H_{0}}}$         & 1 & 1  & 1 & 1 & 1 & 1 & 1 & 1\\
      $\chi_{S^{{(1,1,1,1,1,1)}}\downarrow_{H_{0}}}$ & 1 & -1 & 1 & -1&-1 & 1 & 1 &-1\\
      $\chi_{S^{{(5,1)}}\downarrow_{H_{0}}}$       & 5 & 3  & 1 & 1 &-1 &-1 &-1 &-1\\
      $\chi_{S^{{(2,1,1,1,1)}}\downarrow_{H_{0}}}$  & 5 & -3 & 1 &-1 & 1 &-1  &-1 & 1\\
      $\chi_{S^{{(4,1,1)}}\downarrow_{H_{0}}}$     & 10& 2  & -2& 0 &-2 & 1  & 0 & 1\\
      $\chi_{S^{{(3,1,1,1)}}\downarrow_{H_{0}}}$    & 10&-2  &-2 & 0 & 2 & 1 & 0 &-1\\
      $\chi_{S^{{(4,2)}}\downarrow_{H_{0}}}$       & 9 & 3  & 1 & -1& 3 & 0 & 1 & 0\\
      $\chi_{S^{{(2,2,1,1)}}\downarrow_{H_{0}}}$    & 9 & -3 & 1 & 1 &-3 & 0 & 1 & 0\\
      $\chi_{S^{{(3,3)}}\downarrow_{H_{0}}}$       & 5 & 1  & 1 &-1 & -3& 2 & -1& 0\\
      $\chi_{S^{{(2,2,2)}}\downarrow_{H_{0}}}$     & 5  & -1 & 1 & 1 & 3 & 2 &-1& 0\\
      $\chi_{S^{{(3,2,1)}}\downarrow_{H_{0}}}$     & 16 & 0  & 0 & 0 & 0 &-2 & 0& 0\\
      \hline
      $\chi_{V_{0}}$        & 1 & 1 & 1 & 1 & 1 & 1 &1 & 1  \\
    \end{tabular}

    \caption{Caracteres de $S_{6}$ restringidos a $H_{0}$ y carácter de $V_{0}$}
    \label{tab:clanes-H_0-6}
  \end{table}
  Con los productos internos calculados a partir de la tabla
  \ref{tab:clanes-H_0-6} obtenemos:
  \begin{equation}
    C_{0}(K(M_{6}))\cong \mathbb{C}\oplus S^{(4,2)}\oplus S^{(2,2,2)}.
    \label{C0-KM6}
  \end{equation}
Consideremos los siguientes conjuntos:
  $$V_{1}=\langle b_{1}\rangle=\langle (a_{1},a_{2})\rangle=\langle(\overline{12},\overline{34},\overline{56}),(\overline{12},\overline{35},\overline{46})\rangle,$$
  \begin{footnotesize}
    \begin{align*}
      H_{1}&=\{g\in S_{6}\mid gV_{1}=V_{1}\}\\
      &=\{(1),(12),(36),(45),(12)(36),(12)(45),(34)(56),(35)(46),(36)(45),(12)(34)(56),\\
      &\qquad{}(12)(35)(46),(12)(36)(45),(12)(3564),(12)(3465),(3564),(3465)\}.
    \end{align*}
  \end{footnotesize}
  Los conjuntos $V_{1}$ y $H_{1}$ nos ayudan a construir la tabla \ref{tab:clanes-H_1-6}.
  
  \begin{table}[!hbtp]
    \centering
    \begin{tabular}{c |r r r r r r r r}
      No. de Elementos  & 1 & 1 & 2 & 3 & 2 & 2 & 2 & 3  \\
      Clase&(1)& (2) & (2) & (2,2) & (2,2)& (4) & (4,2) & (2,2,2)\\
      \hline
      $\chi_{S^{{(6)}}\downarrow_{H_{1}}}$         & 1 & 1 &1  & 1 & 1 & 1 & 1 & 1\\
      $\chi_{S^{{(1,1,1,1,1,1)}}\downarrow_{H_{1}}}$ & 1 & -1&-1 & 1 & 1 &-1 & 1 &-1\\
      $\chi_{S^{{(5,1)}}\downarrow_{H_{1}}}$       & 5 & 3 & 3 & 1 & 1 & 1 &-1 &-1\\
      $\chi_{S^{{(2,1,1,1,1)}}\downarrow_{H_{1}}}$  & 5 & -3& -3 & 1 &1 &-1 &-1 & 1\\
      $\chi_{S^{{(4,1,1)}}\downarrow_{H_{1}}}$     & 10& 2  & 2 &-2 &-2& 0 & 0 & -2\\
      $\chi_{S^{{(3,1,1,1)}}\downarrow_{H_{1}}}$    & 10&-2 & -2&-2 & -2& 0 & 0 &2\\
      $\chi_{S^{{(4,2)}}\downarrow_{H_{1}}}$       & 9 & 3 & 3 & 1 & 1 & -1& 1 & 3\\
      $\chi_{S^{{(2,2,1,1)}}\downarrow_{H_{1}}}$    & 9 &-3 & -3&1  & 1 & 1 & 1 & -3\\
      $\chi_{S^{{(3,3)}}\downarrow_{H_{1}}}$       & 5 & 1 & 1 &1  &1  & -1& -1& -3\\
      $\chi_{S^{{(2,2,2)}}\downarrow_{H_{1}}}$     & 5 &-1 & -1&1  & 1 & 1 &-1& 3\\
      $\chi_{S^{{(3,2,1)}}\downarrow_{H_{1}}}$     & 16 &0 &0  & 0 & 0 & 0 & 0& 0\\
      \hline
      $\chi_{V_{1}}$        & 1 & 1 & -1 & 1 & -1 & -1 &-1 & 1  \\
    \end{tabular}
    
    \caption{Caracteres de $S_{6}$ restringidos a $H_{1}$ y carácter de $V_{1}$}
    \label{tab:clanes-H_1-6}
  \end{table}
  Después de calcular los productos internos por medio de la tabla
  anterior resulta:
  \begin{equation}
    C_{1}(K(M_{6}))\cong S^{(2,1,1,1,1)}\oplus S^{(3,1,1,1)}\oplus
    S^{(4,2)} \oplus S^{(2,2,2)}\oplus S^{(3,2,1)}.
    \label{C1-KM6}
  \end{equation}
  Ahora tomemos los siguientes conjuntos:
  $$V_{2}=\langle c_{1}\rangle=\langle
  (a_{1},a_{2},a_{3})\rangle=\langle(\overline{12},\overline{34},\overline{56}),(\overline{12},\overline{35},\overline{46}),(\overline{12},\overline{36},\overline{45})\rangle,$$

\begin{footnotesize}
  \begin{align*}
    H_{2}=&\{g\in S_{6}\mid
    gV_{2}=V_{2}\}=\{(1),(12),(34),(35),(36),(45),(46),(56),(345),(346),\\
    &(354),(356),(364),(365),(456),(465),(12)(34),(12)(35),(12)(36),(12)(45),\\
    &(12)(46),(12)(56),(34)(56),(35)(46),(36)(45),(3456),(3465),(3546),(3564),\\
    &(3645),(3654),(345)(12),(346)(12),(354)(12),(356)(12),(364)(12),(365)(12),\\
    &(456)(12),(465)(12),(12)(34)(56),(12)(35)(46),(12)(36)(45),(3456)(12),\\
    &(3465)(12),(3546)(12),(3564)(12),(3645)(12),(3654)(12)\}.
  \end{align*}
\end{footnotesize}

\begin{table}[!hbtp]
  \centering
  \resizebox*{!}{6cm}{
  \begin{tabular}{c |r r r r r r r r r r}
    No. de Elementos  & 1 & 1 & 6 & 8 & 3 & 6 & 6 & 8 & 3 & 6  \\
    Clase &(1)& (2) & (2) & (3) & (2,2)& (2,2) & (4) & (3,2) & (2,2,2) & (4,2)\\
    \hline
    $\chi_{S^{{(6)}}\downarrow_{H_{2}}}$         &  1 &  1 &  1 &  1 &  1 &  1 &  1 &  1 &  1 &  1 \\ 
    $\chi_{S^{{(1,1,1,1,1,1)}}\downarrow_{H_{2}}}$ &  1 & -1 & -1 &  1 &  1 &  1 & -1 & -1 & -1& 1\\ 
    $\chi_{S^{{(5,1)}}\downarrow_{H_{2}}}$       &  5 &  3 &  3 &  2 &  1 &  1 &  1 &  0 & -1&-1\\
    $\chi_{S^{{(2,1,1,1,1)}}\downarrow_{H_{2}}}$  &  5 & -3 & -3 &  2 &  1 &  1 & -1 &  0 &  1&-1\\  
    $\chi_{S^{{(4,1,1)}}\downarrow_{H_{2}}}$     & 10 &  2 &  2 &  1 & -2 & -2 &  0 & -1 & -2 &0\\
    $\chi_{S^{{(3,1,1,1)}}\downarrow_{H_{2}}}$    & 10 & -2 & -2 &  1 & -2 & -2 &  0 &  1 &  2 &0\\
    $\chi_{S^{{(4,2)}}\downarrow_{H_{2}}}$       &  9 &  3 &  3 &  0 &  1 &  1 & -1 &  0 &  3 & 1\\ 
    $\chi_{S^{{(2,2,1,1)}}\downarrow_{H_{2}}}$    &  9 & -3 & -3 &  0 &  1 &  1 &  1 &  0 & -3 & 1\\   
    $\chi_{S^{{(3,3)}}\downarrow_{H_{2}}}$       &  5 &  1 &  1 & -1 &  1 &  1 & -1 &  1 & -3 &-1\\
    $\chi_{S^{{(2,2,2)}}\downarrow_{H_{2}}}$      &  5 & -1 & -1 & -1 &  1 &  1 &  1 & -1 &  3 &-1\\
    $\chi_{S^{{(3,2,1)}}\downarrow_{H_{2}}}$      & 16 &  0 &  0 & -2 &  0 &  0 &  0 &  0 &  0 &0 \\
    \hline
    $\chi_{V_{2}}$        & 1 & 1 & -1 & 1 & 1 & -1 &-1 & 1 & 1 & -1 \\
  \end{tabular}}

\caption{Caracteres de $S_{6}$ restringidos a $H_{2}$ y carácter de $V_{2}$}
\label{tab:clanes-H_2-6}
\end{table}

De donde obtenemos:

\begin{equation}
  C_{2}(K(M_{6}))\cong S^{(2,1,1,1,1)}\oplus S^{(3,1,1,1)}.
  \label{C2-KM6}
\end{equation}

En la figura \ref{fig:diagrama-conmutativo-clanes6} se muestra el diagrama
conmutativo de los complejos de cadena de $K(M_{6})$, donde $f_{0}$,
$f_{1}$ y $f_{2}$ son los isomorfismos obtenidos de las expresiones
\ref{C0-KM6}, \ref{C1-KM6} y \ref{C2-KM6}.

\begin{figure}[h]
  \centering
    \[
    \begin{CD}
      0 @>{\partial_{3}}>> C_{2}(K(M_{6})) @>{\partial_{2}}>> C_{1}(K(M_{6})) @>{\partial_{1}}>> C_{0}(K(M_{6})) @>{\varepsilon}>> \mathbb{C}\\
      @VVV   @Vf_{2}VV   @Vf_{1}VV   @Vf_{0}VV   @VVV    \\
      0  @>{\widehat\partial_{3}}>> \widehat C_{2}(K(M_{6}))
      @>{\widehat\partial_{2}}>>  \widehat C_{1}(K(M_{6}))
      @>{\widehat\partial_{1}}>> \widehat C_{0}(K(M_{6}))
      @>{\widehat\varepsilon}>> \mathbb{C}
    \end{CD}
    \]
  
  \caption{Diagrama conmutativo de los complejos de cadenas de $K(M_{6})$}
  \label{fig:diagrama-conmutativo-clanes6}
\end{figure}
donde
\begin{eqnarray*}
  \widehat C_{0}(K(M_{6}))&=&\mathbb{C} \oplus S^{(4,2)}\oplus S^{(2,2,2)}.\\
  \widehat C_{1}(K(M_{6}))&=&S^{(2,1,1,1,1)}\oplus S^{(3,1,1,1)}\oplus S^{(4,2)}\oplus S^{(2,2,2)}\oplus S^{(3,2,1)}.\\
  \widehat C_{2}(K(M_{6}))&=&S^{(2,1,1,1,1)}\oplus S^{(3,1,1,1)}. 
\end{eqnarray*}
% \begin{sidewaysfigure}%[h]
%   {\small
%     \[
%     \begin{CD}
%       0 @>{\partial_{3}}>> C_{2}(K(M_{6})) @>{\partial_{2}}>> C_{1}(K(M_{6})) @>{\partial_{1}}>> C_{0}(K(M_{6})) @>{\varepsilon}>> \mathbb{C}\\
%       @VVV   @Vf_{2}VV   @Vf_{1}VV   @Vf_{0}VV   @VVV    \\
%       0  @>{\widehat\partial_{3}}>> S^{(2,1,1,1,1)}\oplus S^{(3,1,1,1)} @>{\widehat\partial_{2}}>>
%       S^{(2,1,1,1,1)}\oplus S^{(3,1,1,1)}\oplus S^{(4,2)}\oplus
%       S^{(2,2,2)}\oplus S^{(3,2,1)} @>{\widehat\partial_{1}}>>
%       \mathbb{C} \oplus S^{(4,2)}\oplus S^{(2,2,2)} @>{\widehat
%         \varepsilon}>> \mathbb{C}
%     \end{CD}
%     \]
%   }
  
%   \caption{Diagrama conmutativo de los complejos de cadenas de $K(M_{6})$}
%   \label{fig:diagrama-conmutativo-clanes6}
% \end{sidewaysfigure}

Calculemos los módulos de homología reducida $\widetilde
H_{k}(K(M_{6}))$, para $k=~0,1,2$.

Como $\widehat\varepsilon$ es sobreyectiva y por el teorema
\ref{teorema-isomorfismo-mod} tenemos:

\begin{equation*}
  (\mathbb{C} \oplus S^{(4,2)}\oplus S^{(2,2,2)})/\ker\widehat\varepsilon\cong \im \widehat\varepsilon=\mathbb{C}
\end{equation*}
así que
\begin{equation*}
  \label{ker0-KM6}
  \ker\widehat\varepsilon\cong S^{(4,2)}\oplus S^{(2,2,2)}.
\end{equation*}
Sabemos que $\widetilde H_{0}(K(M_{6}))=\ker \widehat\varepsilon/\im
\widehat\partial_{1}=0$ pues $K(M_{6})$ es conexo, se sigue que $\ker \widehat\varepsilon\cong
\im\widehat\partial_{1}$, con lo cual:
\begin{equation}
  \label{im1-KM6}
  \im \widehat\partial_{1}\cong S^{(4,2)}\oplus S^{(2,2,2)}.
\end{equation}

Además de el teorema \ref{teorema-isomorfismo-mod} y la expresión \ref{im1-KM6} se tiene:
$$(S^{(2,1,1,1,1)}\oplus S^{(3,1,1,1)}\oplus S^{(4,2)}\oplus
      S^{(2,2,2)}\oplus S^{(3,2,1)})/\ker \widehat\partial_{1}\cong \im \widehat\partial_{1}.$$
De lo anterior y la proposición \ref{modulos-iguales} tenemos:
\begin{equation}
  \label{ker1-KM6}
  \ker \widehat\partial_{1}=S^{(2,1,1,1,1)}\oplus S^{(3,1,1,1)}\oplus S^{(3,2,1)}.
\end{equation}
Por otro lado, $\widehat\partial_{3}$ es un homomorfismo de módulos, así que:
\begin{equation*}
\im\widehat\partial_{3}=\widehat\partial_{3}(0)=0
\label{im3-KM6}
\end{equation*}
De \cite{larrion2009clique} sabemos $M_{6}\simeq K(M_{6})$,
por el teorema \ref{esp-homotopicos-homologias-iso} se sigue $\widetilde H_{2}(M_{6})\cong~\widetilde H_{2}(K(M_{6}))$ y  $\widetilde
H_{2}(M_{6})=0$, así que:

\begin{equation*}
\widetilde H_{2}(K(M_{6}))=\ker \widehat\partial_{2}/\im \widehat\partial_{3}=0.
\end{equation*}
por lo que $\ker \widehat\partial_{2}=0$. De el teorema
\ref{teorema-isomorfismo-mod} tenemos:
$$(S^{(2,1,1,1,1)}\oplus S^{(3,1,1,1)})/\ker \widehat\partial_{2}\cong \im
\widehat\partial_{2}.$$
De lo anterior y la proposición \ref{modulos-iguales} tenemos:
\begin{equation}
  \im \widehat\partial_{2}=S^{(2,1,1,1,1)}\oplus S^{(3,1,1,1)}.
  \label{im2-KM6}
\end{equation}

De las ecuaciones \ref{ker1-KM6} y \ref{im2-KM6} tenemos:
\begin{eqnarray*}
  \widetilde H_{1}(K(M_{6}))&=&\ker \widehat\partial_{1}/\im
  \widehat\partial_{2}\\
  &=&(S^{(2,1,1,1,1)}\oplus S^{(3,1,1,1)}\oplus
  S^{(3,2,1)})/(S^{(2,1,1,1,1)}\oplus S^{(3,1,1,1)})\\
  &=&S^{(3,2,1)}.
\end{eqnarray*}

De igual modo que en $M_{5}$, usamos el hecho de que $M_{6}\simeq K(M_{6})$  para
concluir que $\widetilde H_{2}(K(M_{6}))=\widetilde H_{2}(M_{6})=0$, sin embargo observemos que 
es posible calcular $\im \widehat\partial_{2}$ por medio de cálculos elementales.

Verifiquemos primero que $\dim(\im \partial_{2})=15$. Del teorema
\ref{imT}, sabemos que
$\im \partial_{2}=\langle\partial_{2}(c_{1}),\ldots,\partial_{2}(c_{15})\rangle$,
veamos que el número de vectores linealmente independientes de la
$\im \partial_{2}$ es 15, es
decir, si
$$\lambda_{1}\partial_{2}(c_{1})+\lambda_{2}\partial_{2}(c_{2})+\ldots+\lambda_{15}\partial_{2}(c_{15})=0,$$
así que:

\begin{footnotesize}
  \begin{align*}
    &\lambda_{1}(a_{2}a_{3}-a_{1}a_{3}+a_{1}a_{2})+\lambda_{2}(a_{4}a_{7}-a_{1}a_{7}+a_{1}a_{4})+\lambda_{3}(a_{12}a_{15}-a_{1}a_{15}+a_{1}a_{12})\\
    &+\lambda_{4}(a_{5}a_{10}-a_{2}a_{10}+a_{2}a_{5})+\lambda_{5}(a_{9}a_{14}-a_{2}a_{14}+a_{2}a_{9})+\lambda_{6}(a_{6}a_{13}-a_{3}a_{13}+a_{3}a_{6})\\
    &+\lambda_{7}(a_{8}a_{11}-a_{3}a_{11}+a_{3}a_{8})+\lambda_{8}(a_{5}a_{6}-a_{4}a_{6}+a_{4}a_{5})+\lambda_{9}(a_{8}a_{9}-a_{7}a_{9}+a_{7}a_{8})\\
    &+\lambda_{10}(a_{11}a_{12}-a_{10}a_{12}+a_{10}a_{11})+\lambda_{11}(a_{14}a_{15}-a_{13}a_{15}+a_{13}a_{14})\\
    &+\lambda_{12}(a_{10}a_{13}-a_{7}a_{13}+a_{7}a_{10})+\lambda_{13}(a_{11}a_{14}-a_{4}a_{14}+a_{4}a_{11})\\
    &+\lambda_{14}(a_{8}a_{15}-a_{5}a_{15}+a_{5}a_{8})\lambda_{15}(a_{9}a_{12}-a_{6}a_{12}+a_{6}a_{9})\\
    &=0.
  \end{align*}
\end{footnotesize}
Claramente $\lambda_{1}=\lambda_{2}=\cdots=\lambda_{15}=0$,
intuitivamente esto se puede deducir de la gráfica \ref{fig:KG_5}.
Sea $b_{j}$ la frontera de algún $2$-simplejo $c_{i}$, las aristas
de $b_{j}$ no pertenece a ningún otro triángulo distinto de $c_{i}$, por
lo tanto $\partial_{2}(c_{j})$ no es combinación lineal de las fronteras
$\partial_{2}(c_{i})$ con $i\neq j$.

Así que $\dim(\im\partial_{2})=15$, lo cual implica que
$\dim(\im\widehat\partial_{2})=15$ (se sigue de la definición de
$\widehat\partial_{2}$).

Ahora, por el teorema \ref{im-mod-irreducible} tenemos:
\begin{align*}
  \widehat\partial_{2}(S^{(2,1,1,1,1)})=S^{(2,1,1,1,1)} \quad &\mbox{o
  }\quad \widehat\partial_{2}(S^{(2,1,1,1,1)})=0\\
  &\mbox{y}\\
  \widehat\partial_{2}(S^{(3,1,1,1)})=S^{(3,1,1,1)} \quad &\mbox{o }\quad \widehat\partial_{2}(S^{(3,1,1,1)})=0.
\end{align*}

Además $\dim(S^{(2,1,1,1,1)})=5$ y $\dim(S^{(3,1,1,1)})=10$, entonces:
$$\im\widehat\partial_{2}=\widehat\partial_{2}(S^{(2,1,1,1,1)}\oplus S^{(3,1,1,1)})=S^{(2,1,1,1,1)}\oplus S^{(3,1,1,1)}.$$

\chapter*{Conclusiones}
 
En este trabajo, describimos la descomposición 
del~$S_{n}$-módulo~$\widetilde H_{k}(M_{n})$ en submódulos irreducibles, donde
$\widetilde H_{k}(M_{n})$ es el módulo de homología reducida de los
complejos de emparejamientos, para $n=4,5,6$.

Por otro lado, también calculamos la descomposición en módulos
irreducibles de $\widetilde H_{k}(K(M_{n}))$, es decir, de los módulos de
homología reducida de los complejos simpliciales determinados por
la gráfica de clanes $K(G_{n})$, para $n=5,6$.

Encontramos que $\widetilde H_{k}(M_{n})\cong_{S_n} \widetilde H_{k}(K(M_{n}))$ para
$n=5,6$ y $k\geq0$. Enseguida se muestran los módulos de homología reducida que son distintos de cero.
\ytableausetup{boxsize=0.2em}
\begin{equation*}
  \widetilde H_{1}(M_{5})\cong\widetilde H_{1}(K(M_{5}))\cong_{S_{5}} S^{(3,1,1)}=S^{\,\ydiagram{3,1,1}}.
\end{equation*}
\begin{equation*}
  \widetilde H_{1}(M_{6})\cong\widetilde H_{1}(K(M_{6}))\cong_{S_{6}} S^{(3,2,1)}=S^{\,\ydiagram{3,2,1}}.
\end{equation*}

En el artículo \cite{larrion2009clique} se demuestra que $G_{n}\simeq
K(G_{n})$ para $n\leq 8$. Esto implica, por el teorema
\ref{esp-homotopicos-homologias-iso}, que
se tiene el siguiente isomorfismo como espacios vectoriales:
\begin{equation*}
  \label{equ:iso}
  \widetilde H_{k}(M_{n})\cong \widetilde H_{k}(K(M_{n})).
\end{equation*}

Surge la pregunta interesante si también existe el isomorfismo entre
$\widetilde H_{k}(M_{n})$ y $\widetilde H_{k}(K(M_{n}))$ como
$S_{n}$-módulos, para $n=7,8$. De los cálculos que se muestran en la
tesis, se observa que para tales casos las complicaciones aumentarían
significativamente, por lo que no es práctico efectuarlos
directamente. Un posible camino es utilizar software especializado
para tales cálculos. 

Si los $S_{n}$-módulos $\widetilde H_{k}(M_{n})$ y $\widetilde
H_{k}(K(M_{n}))$ no son $S_{n}$-isomorfos, para algún $n\geq 7$, otro
problema interesante es encontrar una fórmula análoga a la de Bouc que
muestre la descomposición de los módulos de homología de $K(M_{n})$ en
submódulos irreducibles.

% Se sabe, sin embargo, que la ecuación~\ref{equ:iso} no es válida para
% $n\geq 9$. Por lo que otro problema interesante es encontrar una
% fórmula análoga a la de Bouc que muestre la descomposición de las
% homologías de $K(M_{n})$ en submódulos irreducibles.


\bibliographystyle{plain}
\bibliography{labiblio}

\printindex


\end{document}
