
\documentclass[12pt]{book}

\usepackage[dvipsnames]{xcolor}
\usepackage{amssymb,latexsym}
\usepackage{graphicx}

\usepackage[spanish,mexico,es-nolayout]{babel}
\usepackage[utf8]{inputenc}
\usepackage{amsmath}
%\usepackage{amssymb}
\usepackage{amsthm}
%\usepackage{graphicx}
\usepackage{color}
\usepackage{tikz}
\usepackage{tkz-berge}
\usepackage{makeidx}
\usepackage{url}
\usepackage{xspace}
\usepackage{tocbibind}
% ver http://gilmation.com/articles/latex-margins-for-book-binding/
% y http://tex.stackexchange.com/questions/50258/margins-of-book-class
\usepackage[margin=3.5cm]{geometry}
\geometry{bindingoffset=1cm}

\usepackage{babelbib}

\usetikzlibrary{positioning,shapes,fit,arrows,decorations.pathmorphing}
\definecolor{myblue}{RGB}{56,94,141}


\newtheorem{theorem}{Teorema}[section]
\newtheorem{corollary}[theorem]{Corolario}
\newtheorem{proposition}[theorem]{Proposición}

\theoremstyle{definition}

\newtheorem{definition}[theorem]{Definición}
\newtheorem{notation}[theorem]{Notación}
\newtheorem{example}[theorem]{Ejemplo}
\newtheorem{lemma}[theorem]{Lema}

\DeclareMathOperator{\im}{im}
\DeclareMathOperator{\sgn}{sgn}

\newcounter{in}
\newcounter{ini}

\makeindex

\newcommand{\elespacio}{1.4cm}

\begin{document}
\mainmatter 
\begin{titlepage}
  \begin{center}
    \null
    \vspace*{\fill}

    \includegraphics[scale=1.2,bb=55 20 0 0]{escudouaeh.pdf}

    \vspace*{\elespacio}

    \textsc{Universidad Autónoma del Estado de Hidalgo}

    \textsc{Instituto de Ciencias Básicas e Ingeniería}

    \textsc{Área Académica de Matemáticas y Física}

    \vspace*{\elespacio}

    {\Huge\bfseries Representaciones del grupo simétrico en homologías\par}

    \vspace*{\elespacio}

    {\large Tesis que para obtener el título de}

    \vspace*{\elespacio}

    {\Large\textsc{Licenciada en Matemáticas Aplicadas}}

    \vspace*{\elespacio}

    {\large presenta}

    \vspace*{\elespacio}

    {\Huge Briseida Guadalupe Trejo Escamilla}

    \vspace*{\elespacio}

    {\large bajo la dirección de}

    \bigskip

    {\Large Dr.~Rafael Villarroel Flores}

    \bigskip

    {Pachuca, Hidalgo. Junio de 2013.}

    \vspace*{\fill}

  \end{center}
\end{titlepage}

\thispagestyle{empty}
\begin{flushleft}
  {\bfseries\Large Resumen}
\end{flushleft}

El presente trabajo tiene como objetivo analizar la estructura de
G-módulos de homología de complejos simpliciales en los cuales actúa
un grupo G finito. Con ello en mente se muestra en primera instancia
un procedimiento para conocer los G-módulos de dimensión finita,
dotándonos de una herramienta para obtener información y realizar así
una clasificación de ellos.

\vspace{2cm}

\begin{flushleft}
  {\bfseries\Large Abstract}
\end{flushleft}

In this thesis blah blah blah blah blah blah blah blah blah
blah blah blah blah blah blah blah blah blah.



 \newpage \thispagestyle{empty}

\chapter{Representaciones de Grupos}
\label{cha:primer-capitulo}

\section{Representaciones}

\begin{definition}
  Una \textbf{operación binaria} en un conjunto $G$ es una función
  de la forma $G  \times G \rightarrow G$. Para cada $(a,b)\in G
  \times G$, denotaremos al elemento $*((a,b))\in G$ por $a*b$.
\end{definition}

\begin{definition}
  Un \textbf{grupo} es un conjunto no vacío $G$, junto con una
  \textbf{operación binaria} $*$ que satisface las siguientes condiciones:
    \begin{enumerate}
    \item La operación es asociativa, es decir, $$a*(b*c)=(a*b)*c$$ para todo $a,b,c \in G$.
    \item Existe un elemento neutro $e \in G$ que
      satisface $$a*e=e*a=a$$ para todo $a \in G$.
    \item Para cada elemento $a \in G$ existe otro elemento $a' \in G$
      tal que $$a*a'=a'*a=e$$ Al elemento $a'$ se le llama inverso del
      elemento a.
    \end{enumerate}
\end{definition}

\begin{definition}
  Si $G$ y $H$ son grupos, entonces un \textbf{homomorfismo} de $G$
  en $H$ es una función $\phi:G\rightarrow H$ la cual
  satisface $$\phi(ab)=\phi(a)\phi(b)$$ para todo $a,b \in G$
\end{definition}

\begin{theorem}
  Sea $\rho:G\rightarrow G^{'}$ un homomorfismo de grupos. Entonces
  \begin{enumerate}
    \item $\rho(e)=e^{'}$, donde $e\in G$ y  $e^{'}\in G^{'}$ son los
    neutros respectivos. 
    \item Si $g\in G$, entonces $\rho(g^{-1})=\rho (g)^{-1}$.
  \end{enumerate}
\end{theorem}

\begin{definition}
  Cuando un homomorfismo de grupos $\phi:G\rightarrow H$ es biyectivo,
  diremos que $\phi$ es un  \textbf{isomorfismo}. También diremos que
  $G$ y $H$ son grupos  \textbf{isomorfos} cuando exista un
    isomorfismo entre ellos, usaremos la notación $G\cong H$.
\end{definition}

\begin{theorem}
  Supongamos que $G$ y $H$ son grupos y sea $\phi:G\rightarrow H$ un
  homomorfismo. Entonces $$G/\ker \phi\cong \im \phi$$ 
\end{theorem}

  Denotemos $GL(n,\mathbb{C})$ al grupo de matrices
  invertibles  $n \times n$ con entradas en $\mathbb{C}$.  
  Y a $GL(V)$ a los operadores lineales invertibles en un
  $\mathbb{C}$-espacio vectorial $V$ de dimensión finita $n$.
  Además $GL(n,\mathbb{C})\cong GL(V)$ es un isomorfismo de grupos.


\begin{definition}
  Sea $G$ un grupo y $Y$ un conjunto no vacío. Una  \textbf{acción de $G$
  en $Y$} es una función $*:G \times Y \rightarrow Y$ tal que
\begin{enumerate}
\item $e*x=x$ para todo $x\in Y.$
\item $(g_{1}g_{2})*x=g_{1}*(g_{2}*x)$ para todo $x\in Y$ y $g_{1},g_{2}\in G.$
\end{enumerate}
   Bajo estas condiciones, $Y$ es un $G$-conjunto. En ocasiones, por
   abuso de notación, en lugar de $g*x$ usaremos la notación $gx.$
\end{definition}

\begin{definition}\textbf{Acciones lineales.}
  Sea $G$ un grupo. Una acción de $G$ en un $K$-espacio
  vectorial $V$ de dimensión finita es una función
 $$*:G\times V \rightarrow V $$
que satisface los axiomas:
\begin{enumerate}
\item $e*v=v$, para todo $v\in V$ (donde $e$ es el neutro de $G$).
\item $(g_{1}g_{2})*v=g_{1}*(g_{2}*v)$ para todos $v\in V$ y
  $g_{1},g_{2}\in G$.
\item $g_{1}*(v+w)=g_{1}*v+g_{1}*w$, para $g_{1}\in G$ y $v,w \in V .$
\item $g_{1}*(\lambda v)=\lambda(g_{1}*v)$, para $\lambda \in K$,
  $v\in V$ y $g_{1}\in G.$
\end{enumerate}
Diremos que la acción es $G$-lineal y que $V$ es un $G$-espacio
vectorial.
\end{definition} 

\begin{theorem}
  Existe una correspondencia biyectiva entre el conjunto de acciones
  lineales de un grupo $G$ en un $K$-espacio vectorial $V$ y el conjunto
  de homomorfismos de $G$ en $GL(V)$
\end{theorem}

  \textit{Demostración.} Supongamos primero que se tiene una acción lineal 
  $$*:G\times V \rightarrow V$$
  para cada $g\in G$, usando esta acción definamos la función $\rho
  g:V \rightarrow V$ dada por $(\rho g)v=gv$ para todo $v\in V$.

  Primero observemos que la función $\rho:G\rightarrow GL(V)$ es un
  homomorfismo de grupos ya que si $g,h\in G$, entonces para todo
  $v\in V$:

  $$(\rho(gh))v=(gh)v=g(hv)=g((\rho h)v)=(\rho g)((\rho h)v)=(\rho g \circ \rho h)v$$
  por lo que $\rho(gh)=\rho g \circ \rho h$

  Ahora demostremos que $\rho g$ es una transformación lineal
  invertible. Sean $v,w \in V$, $\lambda \in K$.

  \begin{flalign*}
   &(\rho g)(v+w)=g(v+w)=gv+gw=(\rho g)v+(\rho g)w\\
   &(\rho g)(\lambda v)=g(\lambda v)=\lambda(gv)=\lambda((\rho g)v)       
  \end{flalign*}

  Así que $\rho g$ es lineal, ahora veamos que es invertible. Como $G$
  es un grupo, existe $g^{-1}\in G$.

  \begin{eqnarray*}
     ((\rho g)(\rho g)^{-1})v=(\rho g)((\rho g)^{-1}v)=(\rho g)((\rho g^{-1})v)=(\rho g)(g^{-1}v)=g(g^{-1}v)=(gg^{-1})v=v\\
     ((\rho g)^{-1}(\rho g))v=(\rho g)^{-1}((\rho g)v)=(\rho g^{-1})((\rho g)v)=(\rho g^{-1})(gv)=g^{-1}(gv)=(g^{-1}g)v=v
   \end{eqnarray*}

Suponga ahora que se tiene un homomorfismo de grupos $\rho:G\rightarrow GL(V)$, por demostrar que $gv:=(\rho g)v$, define un
acción lineal de $G$ en $V$. 

\begin{enumerate}
   \item $e$ es el neutro de $G$, entonces $$ev=(\rho e)v=id_{V}=v$$ para
     todo $v\in V$. 
   \item Si $g,h\in G$ y $v\in V$, entonces $$(gh)v=(\rho g h)v=(\rho
     g \circ \rho h)v=\rho g((\rho h)v)=\rho g(hv)=g(hv)$$
   \item Para $g\in G$ y $v,w\in V$, $$g(v+w)=(\rho g)(v+w)=(\rho g)v+(\rho g)w=gv+gw$$
   \item Para $\lambda\in K$, $v\in V$ y $g\in G$, 
    $$g(\lambda v)=(\rho g)(\lambda v)=\lambda((\rho g)v)=\lambda(gv)$$
\end{enumerate}
$\qed$

\begin{definition}
  Si $G$ es un grupo y $V$ es un $K$-espacio vectorial, una
  \textbf{representación lineal} de $G$ en $V$ es un homomorfismo 
     $$\rho:G\rightarrow GL(V).$$
  Se denotará $(\rho,V)$ para enfatizar el hecho de que la
  representación $\rho$ de $G$  es sobre el espacio lineal $V$. A
  dicho espacio lineal $V$ se le llamará \emph{espacio de representación} y a
  su dimensión el \emph{grado} de la representación 
\end{definition}

Estudiar las representaciones lineales de un grupo $G$ en un espacio
vectorial $V$, es equivalente a estudiar las acciones lineales de $G$
en $V$. 

En este trabajo estudiaremos representaciones de grupos finitos en
espacios vectoriales complejos y de dimensión finita. 

\begin{example}(\textbf{Representación trivial})
  La representación trivial de un grupo $G$ es el homomorfismo
  $\rho:G\rightarrow \mathbb{C^{*}}$ dado por $\rho(g)=1$ para todo
  $g\in G$.
\end{example}

\begin{example}(\textbf{Representación signo})
  El grupo simétrico sobre $\mathbb{I}_{n}=\{1,2,\ldots,n\}$, denotado
  por $\mathcal{S}_{n}$ es el grupo de todas las permutaciones de $\mathbb{I}_{n}$ 
  Si $\pi=\tau_{1}\tau_{2}\cdots\tau_{k}$, donde $\tau_{i}$ son
  transposiciones, definimos la función signo
  $\sgn:\mathcal{S}_{n} \rightarrow\{\pm1\}$ mediante $$\sgn(\pi):=(-1)^{k}.$$
\end{example}

\begin{example}(\textbf{Representación regular})
  Sea $G$ cualquier grupo de orden $n$ y sea V el espacio vectorial de
  dimensión $n$ con base $\mathcal{B}=\{v_{g}\}_{g\in G}$ indexada por
  los elementos del grupos. Para cada $\sigma\in G$ definamos la función
  $\rho(\sigma):\mathcal{B}\rightarrow \mathcal{B}$ mediante $\rho(\sigma)(v_{g})=v_{\sigma g}$.
  Mostremos que la función $$\rho:G\rightarrow GL(V)$$ dada por
  $\sigma\mapsto\rho(\sigma)$ es un homomorfismo. Si $\sigma,\tau \in G$
  y si $v_{g}\in \mathcal{B}$, entonces $$\rho(\sigma
  \tau)v_{g}=v_{(\sigma\tau)g}=v_{\sigma(\tau g)}=\rho(\sigma)v_{\tau
    g}=\rho(\sigma)(\rho(\tau)v_{g})=\rho(\sigma)\rho(\tau)v_{g} $$
  así que $\rho(\sigma\tau)=\rho(\sigma)\rho(\tau)$. Por definición el 
  grado de la representación es el orden de $G$.

\end{example}

\chapter{Homología de Complejos Simpliciales}
%\label{cha:primer-capitulo}

\section{Complejo de Apareamiento}

\begin{definition}
Consideremos la gráfica completa de $n$ vértices $K_{n}$, tales
vértices son etiquetados como $a_{1},a_{2},\ldots,a_{n}$ y
$\overline{a_{i}a_{j}}$ denotará la arista que une al vértice $a_{i}$ con el
vértice $a_{j}$. Llamaremos \textbf{complejo de apareamiento} de orden
$n$ al complejo simplicial $M_{n}$ de dimensión $n$ tal que:

\begin{enumerate}
  \item Su conjunto de vértices $\mathcal{V}$ consta de las aristas de la gráfica
  $K_{n}$. 
  \item Si $v_{i}=\overline{a_{p}a_{q}}$ y $v_{j}=\overline{a_{r}a_{s}}$ están en
  $\mathcal{V}$, $\{v_{i},v_{j}\}$ es  un 1-simplejo de $M_{n}$ si $v_{i}$
  y $v_{j}$ son ajenas
\end{enumerate} 
\end{definition}

\begin{example}
Considerese la gráfica $K_{4}$
\end{example}


\bigskip
\bigskip
\bigskip
\bigskip
\bigskip
\bigskip
\bigskip
\bigskip
\bigskip
\bigskip

\begin{center}
Tabla de S4

\begin{tabular}{c|r r r r r}
  No. Elementos& 1 & 6 & 8 & 6 & 3 \\
  Clase & (1) & (12) & (123) & (1234) &(12)(34)\\
    \hline
  $\chi_{{1}}$ & 1 & 1 & 1 & 1 & 1 \\
  $\chi_{{2}}$ & 1 & -1 & 1 & -1 & 1\\
  $\chi_{{3}}$ & 3 & 1 & 0 & -1 & -1\\
  $\chi_{{4}}$ & 3 & -1 & 0 & 1 & -1 \\
  $\chi_{{5}}$ & 2 & 0 & -1 & 0 & 2 \\
    \hline
  $\chi_{C_{0}(M_{4})}$ & 6 & 2 & 0 & 0 & 2 \\
  $\chi_{C_{1}(M_{4})}$ & 3 & 1 & 0 & -1 & -1
\end{tabular}
\end{center}

\bigskip

\begin{center}
\begin{small}
\begin{tabular}{c |r r r r r r r}
  No. Elementos& 1 & 10 & 20 & 30 & 24 & 15 & 20  \\
  Clase & (1) & (12) & (123) & (1234) & (12345) & (12)(34) & (123)(45) \\
    \hline
  $\chi_{\mathbb{C}}=\xi_{1}$ & 1 & 1 & 1 & 1 & 1 & 1 & 1 \\
  $\chi_{\mathbb{C}^{'}}=\xi_{2}$ & 1 & -1 & 1 & -1 & 1 & 1 & -1\\
  $\chi_{V_{5}}=\xi_{3}$ & 4 & 2 & 1 & 0 & -1 & 0 & -1\\
  $\chi_{V_{5}^{'}}=\xi_{4}$ & 4 & -2 & 1 & 0 & -1 & 0 & 1 \\
  $\chi_{W_{1}}=\xi_{5}$ & 6 & 0 & 0 & 0 & 1 & -2 & 0 \\
  $\chi_{W_{2}}=\xi_{6}$ & 5 & 1 & -1 & -1 & 0 & 1 & 1 \\
  $\chi_{W_{2}^{'}}=\xi_{7}$ & 5 & -1 & -1 & 1 & 0 & 1 & -1 \\
  \hline
  $\chi_{C_{0}(M_{5})}$ & 10 & 4 & 1 & 0 & 0 & 2 & 1 \\
  $\chi_{C_{1}(M_{5})}$ & 15 & 3 & 0 & -1 & 0 & -1 & 0
\end{tabular}
\end{small}
\end{center}

\bigskip

\begin{tabular}{c |r r r r r r}
  No. Elementos& 1 & 1 & 3 & 3 & 2 & 2 \\
  Clase & (1) & (45) & (12) & (12)(45) & (123) & (123)(45) \\
    \hline
  $\xi_{1}$ & 1 & 1 & 1 & 1 & 1 & 1 \\
  $\xi_{2}$ & 1 & -1 & -1 & 1 & 1 & -1 \\
  $\xi_{3}$ & 4 & 2 & 2 & 0 & 1 & -1 \\
  $\xi_{4}$ & 4 & -2 & -2 & 0 & 1 & 1 \\
  $\xi_{5}$ & 6 & 0 & 0 & -2 & 0 & 0 \\
  $\xi_{6}$ & 5 & 1 & 1 & 1 & -1 & 1 \\
  $\xi_{7}$ & 5 & -1 & -1 & 1 & -1 & -1 \\
  \hline
  $\phi_{1}$ & 1 & 1 & 1 & 1 & 1 & 1 \\
\end{tabular}

\bigskip

\begin{tabular}{c |r r r r r}
   & & (24) & (1432) & (14)(23) & \\
  Elementos & (1) & (13) & (1234) & (12)(34) & (13)(24) \\
    \hline
  $\xi_{1}$ & 1 & 1 & 1 & 1 & 1 \\
  $\xi_{2}$ & 1 & -1 & -1 & 1 & 1 \\
  $\xi_{3}$ & 4 & 2 & 0 & 0 & 0 \\
  $\xi_{4}$ & 4 & -2 & 0 & 0 & 0 \\
  $\xi_{5}$ & 6 & 0 & 0 & -2 & -2 \\
  $\xi_{6}$ & 5 & 1 & -1 & 1 & 1 \\
  $\xi_{7}$ & 5 & -1 & 1 & 1 & 1 \\
  \hline
  $\phi_{1}$ & 1 & 1 & -1 & -1 & 1 \\
\end{tabular}

\bigskip

\begin{tabular}{c |r r r r}
  Elementos & (1) & (12) & (45) & (12)(45) \\
    \hline
  $\xi_{1}$ & 1 & 1  & 1  & 1 \\
  $\xi_{2}$ & 1 & -1 & -1 & 1  \\
  $\xi_{3}$ & 4 & 2  & 2  & 0  \\
  $\xi_{4}$ & 4 & -2 & -2 & 0  \\
  $\xi_{5}$ & 6 & 0  & 0  & -2 \\
  $\xi_{6}$ & 5 & 1  & 1  & 1  \\
  $\xi_{7}$ & 5 & -1 & -1 & 1  \\
  \hline
  $\phi_{1}$ & 1 & 1 & -1 & -1 \\
\end{tabular}

\bigskip

\begin{tabular}{c |r r r r r r r r}
          &   & (12) & (12)(34) &          &             &              & &  \\        
          &   & (34) & (12)(56) &(13)(24)  &             & (13)(24)(56) & (2314)& (2314)(56)\\
Elementos &(1)& (56) & (34)(56) & (14)(23) & (12)(34)(56)& (14)(23)(56) &(2413)&(2413)(56)\\
    \hline
  $S^{{(6)}}$           & 1 & 1  & 1  & 1 & 1 & 1 & 1 & 1 \\
  $S^{{(1,1,1,1,1,1)}}$ & 1 & -1 & 1  & 1 &-1 &-1 &-1 & 1  \\
  $S^{{(5,1)}}$         & 5 & 3  & 1  & 1 &-1 &-1 & 1 &-1  \\
  $S^{{(2,1,1,1,1)}}$   & 5 & -3 &  1 & 1 & 1 & 1 &-1 &-1  \\
  $S^{{(4,1,1)}}$       & 10& 2  & -2 & -2&-2 &-2 & 0 & 0  \\
  $S^{{(3,1,1,1)}}$     & 10&-2  & -2 &-2 & 2 & 2 & 0 & 0  \\
  $S^{{(4,2)}}$         & 9 & 3  & 1  & 1 & 3 &3  &-1 & 1  \\
  $S^{{(2,2,1,1)}}$     & 9 & -3 & 1  & 1 & -3&-3 & 1 & 1  \\
  $S^{{(3,3)}}$         & 5 & 1  & 1  & 1 &-3 &-3 &-1 & -1 \\
  $S^{{(2,2,2)}}$       & 5 & -1 & 1  & 1 & 3 & 3 & 1 & -1 \\
  $S^{{(3,2,1)}}$       & 16& 0  & 0  & 0 & 0 & 0 & 0 &  0 \\
    \hline
  $\phi_{1}$            & 1 & 1  & 1  & -1& 1 & -1&-1 &-1  \\
\end{tabular}

\bigskip

\begin{tabular}{c |r r r r r r r r r r}
No. de Elementos  & 1 & 3 & 3 & 6 & 6 & 1 & 6 & 8 & 6 & 8 \\
 &(1)& (2) & (2,2) & (2,2) & (4)& (2,2,2) & (2,2,2) & (3,3) & (4,2) & (6)\\
    \hline
  $S^{{(6)}}$         & 1 & 1  & 1  & 1 & 1 &1  & 1 & 1 & 1 & 1 \\
  $S^{{(1,1,1,1,1,1)}}$ & 1 & -1 & 1  & 1 & -1&-1 &-1 & 1 & 1 & -1 \\
  $S^{{(5,1)}}$       & 5 & 3  & 1  & 1 & 1 &-1 &-1 &-1 &-1 & -1 \\
  $S^{{(2,1,1,1,1)}}$  & 5 & -3 &  1 & 1 &-1 & 1 & 1 &-1  &-1 & 1 \\
  $S^{{(4,1,1)}}$     & 10& 2  & -2 & -2& 0 &-2 &-2 & 1  & 0 & 1 \\
  $S^{{(3,1,1,1)}}$    & 10&-2  & -2 &-2 & 0 & 2 & 2 & 1 & 0 & -1 \\
  $S^{{(4,2)}}$       & 9 & 3  & 1  & 1 & -1& 3 & 3 & 0 & 1 &  0 \\
  $S^{{(2,2,1,1)}}$    & 9 & -3 & 1  & 1 & 1 &-3 &-3 & 0 & 1 &  0 \\
  $S^{{(3,3)}}$       & 5 & 1  & 1  & 1 &-1 &-3 & -3& 2 &-1 &  0 \\
  $S^{{(2,2,2)}}$     & 5  & -1& 1  & 1 & 1 & 3 & 3 & 2 & -1 & 0 \\
  $S^{{(3,2,1)}}$     & 16 & 0 & 0  & 0 & 0 & 0 & 0 & -2 & 0 & 0 \\
  \hline
  $\phi_{1}$ & 1 & 1 & 1 & -1 & -1 & 1 &-1 & 1 &- 1 & 1  \\
\end{tabular}

El presente trabajo tiene como objetivo analizar la estructura de
G-módulos de homología de complejos simpliciales en los cuales actúa
un grupo G finito. Con ello en mente se muestra en primera instancia
un procedimiento para conocer los G-módulos de dimensión finita,
dotándonos de una herramienta para obtener información y realizar así
una clasificación de ellos.

% \bibliographystyle{plain}
% \bibliography{labiblio}

\printindex


\end{document}
