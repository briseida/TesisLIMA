\documentclass[12pt]{book}

\usepackage[dvipsnames]{xcolor}
\usepackage{amssymb,latexsym}
\usepackage{graphicx}

\usepackage[spanish,mexico,es-nolayout]{babel}
\usepackage[utf8]{inputenc}
\usepackage{amsmath}
%\usepackage{amssymb}
\usepackage{amsthm}
%\usepackage{graphicx}
\usepackage{color}
\usepackage{tikz}
\usepackage{tkz-berge}
\usepackage{makeidx}
\usepackage{url}
\usepackage{xspace}
\usepackage{tocbibind}
% ver http://gilmation.com/articles/latex-margins-for-book-binding/
% y http://tex.stackexchange.com/questions/50258/margins-of-book-class
\usepackage[margin=3.5cm]{geometry}
\geometry{bindingoffset=1cm}

\usepackage{babelbib}

\usetikzlibrary{positioning,shapes,fit,arrows,decorations.pathmorphing}
\definecolor{myblue}{RGB}{56,94,141}


\newtheorem{theorem}{Teorema}[section]
\newtheorem{corollary}[theorem]{Corolario}
\newtheorem{proposition}[theorem]{Proposición}

\theoremstyle{definition}

\newtheorem{definition}[theorem]{Definición}
\newtheorem{notation}[theorem]{Notación}
\newtheorem{example}[theorem]{Ejemplo}
\newtheorem{lemma}[theorem]{Lema}

\newcounter{in}
\newcounter{ini}

\makeindex

\newcommand{\elespacio}{1.4cm}

\begin{document}
\mainmatter 
\begin{titlepage}
  \begin{center}
    \null
    \vspace*{\fill}

    \includegraphics[scale=1.2,bb=55 20 0 0]{escudouaeh.pdf}

    \vspace*{\elespacio}

    \textsc{Universidad Autónoma del Estado de Hidalgo}

    \textsc{Instituto de Ciencias Básicas e Ingeniería}

    \textsc{Área Académica de Matemáticas y Física}

    \vspace*{\elespacio}

    {\Huge\bfseries Representaciones del grupo simétrico en homologías\par}

    \vspace*{\elespacio}

    {\large Tesis que para obtener el título de}

    \vspace*{\elespacio}

    {\Large\textsc{Licenciada en Matemáticas Aplicadas}}

    \vspace*{\elespacio}

    {\large presenta}

    \vspace*{\elespacio}

    {\Huge Briseida Guadalupe Trejo Escamilla}

    \vspace*{\elespacio}

    {\large bajo la dirección de}

    \bigskip

    {\Large Dr.~Rafael Villarroel Flores}

    \bigskip

    {Pachuca, Hidalgo. Junio de 2013.}

    \vspace*{\fill}

  \end{center}
\end{titlepage}

\thispagestyle{empty}
\begin{flushleft}
  {\bfseries\Large Resumen}
\end{flushleft}

En esta tesis se hace blah blah blah blah blah blah blah blah blah
blah blah blah blah blah blah blah blah blah.

\vspace{2cm}

\begin{flushleft}
  {\bfseries\Large Abstract}
\end{flushleft}

In this thesis blah blah blah blah blah blah blah blah blah
blah blah blah blah blah blah blah blah blah.



 \newpage \thispagestyle{empty}

\chapter{Representaciones de Grupos}
\label{cha:primer-capitulo}

\section{Representaciones}

\begin{definition}
  Una operación binaria en un conjunto $G$ es una función
  de la forma $G  \times G \rightarrow G$. Para cada $(a,b)\in G
  \times G$, denotaremos al elemento $*((a,b))\in G$ por $a*b$.
\end{definition}

\begin{definition}
  Un $\mathbf{grupo}$ es un conjunto no vacío $G$, junto con una
  \textbf{operación binaria} $*$ que satisface las siguientes condiciones:
    \begin{enumerate}
    \item La operación es asociativa, es decir, $$a*(b*c)=(a*b)*c$$ para todo $a,b,c \in G$.
    \item Existe un elemento neutro $e \in G$ que
      satisface $$a*e=e*a=a$$ para todo $a \in G$.
    \item Para cada elemento $a \in G$ existe otro elemento $a' \in G$
      tal que $$a*a'=a'*a=e$$ Al elemento $a'$ se le llama inverso del
      elemento a.
    \end{enumerate}
\end{definition}

\begin{definition}
  Si $G$ y $H$ son grupos, entonces un $\mathbf{homomorfismo}$ de $G$
  en $H$ es una función $\phi:G\rightarrow H$ la cual
  satisface $$\phi(ab)=\phi(a)\phi(b)$$ para todo $a,b \in G$
\end{definition}

\begin{definition}
  Cuando un homomorfismo de grupos $\phi:G\rightarrow H$ es biyectivo,
  diremos que $\phi$ es un  $\mathbf{isomorfismo}$. También diremos que
  $G$ y $H$ son grupos  $\mathbf{isomorfos}$ cuando exista un
    isomorfismo entre ellos, usaremos la notación $G\cong H$.
\end{definition}

\begin{theorem}
  Supongamos que $G$ y $H$ son grupos y sea $\phi:G\rightarrow H$ un
  homomorfismo. Entonces $$G/Ker \phi\cong Im \phi$$ 
\end{theorem}

  Sea $G$ un grupo y $GL(n,\mathbb{C})$ denota el grupo de matrices
  invertibles  $n \times n$ con entradas en $\mathbb{C}$.  


\begin{definition}
  Una representación de $G$ sobre $F$, es un homomorfismo $\rho$ de $G$
  en $GL(n,F)$, para algún $n$. El grado de $\rho$ es el entero $n$.
\end{definition}

% \bibliographystyle{plain}
% \bibliography{labiblio}

\printindex


\end{document}
