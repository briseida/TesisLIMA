
\documentclass[12pt]{book}

\usepackage[dvipsnames]{xcolor}
\usepackage{amssymb,latexsym}
\usepackage{graphicx}

\usepackage[spanish,mexico,es-nolayout]{babel}
\usepackage[utf8]{inputenc}
\usepackage{amsmath}
%\usepackage{amssymb}
\usepackage{amsthm}
%\usepackage{graphicx}
\usepackage{color}
\usepackage{tikz}
\usepackage{tkz-berge}
\usepackage{makeidx}
\usepackage{url}
\usepackage{xspace}
\usepackage{tocbibind}
% ver http://gilmation.com/articles/latex-margins-for-book-binding/
% y http://tex.stackexchange.com/questions/50258/margins-of-book-class
\usepackage[margin=3.5cm]{geometry}
\geometry{bindingoffset=1cm}

\usepackage{babelbib}

\usetikzlibrary{positioning,shapes,fit,arrows,decorations.pathmorphing}
\definecolor{myblue}{RGB}{56,94,141}


\newtheorem{theorem}{Teorema}[section]
\newtheorem{corollary}[theorem]{Corolario}
\newtheorem{proposition}[theorem]{Proposición}

\theoremstyle{definition}

\newtheorem{definition}[theorem]{Definición}
\newtheorem{notation}[theorem]{Notación}
\newtheorem{example}[theorem]{Ejemplo}
\newtheorem{lemma}[theorem]{Lema}

\DeclareMathOperator{\im}{im}
\DeclareMathOperator{\sgn}{sgn}

\newcounter{in}
\newcounter{ini}

\makeindex

\newcommand{\elespacio}{1.4cm}

\begin{document}
\mainmatter 
\begin{titlepage}
  \begin{center}
    \null
    \vspace*{\fill}

    \includegraphics[scale=1.2,bb=55 20 0 0]{escudouaeh.pdf}

    \vspace*{\elespacio}

    \textsc{Universidad Autónoma del Estado de Hidalgo}

    \textsc{Instituto de Ciencias Básicas e Ingeniería}

    \textsc{Área Académica de Matemáticas y Física}

    \vspace*{\elespacio}

    {\Huge\bfseries Representaciones del grupo simétrico en homologías\par}

    \vspace*{\elespacio}

    {\large Tesis que para obtener el título de}

    \vspace*{\elespacio}

    {\Large\textsc{Licenciada en Matemáticas Aplicadas}}

    \vspace*{\elespacio}

    {\large presenta}

    \vspace*{\elespacio}

    {\Huge Briseida Guadalupe Trejo Escamilla}

    \vspace*{\elespacio}

    {\large bajo la dirección de}

    \bigskip

    {\Large Dr.~Rafael Villarroel Flores}

    \bigskip

    {Pachuca, Hidalgo. Junio de 2013.}

    \vspace*{\fill}

  \end{center}
\end{titlepage}

\thispagestyle{empty}
\begin{flushleft}
  {\bfseries\Large Resumen}
\end{flushleft}

El presente trabajo tiene como objetivo analizar la estructura de
la homología de complejos simpliciales construidos a partir de una
gráfica simple finita en donde actúa el grupo simétrico $S_{n}$.

Se define la gráfica $M(G)$ de emparejamientos de la gráfica simple
$G$ como la gráfica cuyos vértices son las aristas de $G$ y dos
vértices son adyacentes si las correspondientes aristas son ajenas. En
la gráfica $G_{n}=M(K_{n})$, el grupo $S_{n}$ actúa de manera natural, por
lo que sus homologías con coeficientes complejos definen
representaciones de $S_{n}$. 

La descomposición en irreducibles de tales homologías ha sido exhibida
por Bouc (1992). En el presente trabajo se muestra el resultado
correspondiente para la gráfica de clanes $K(G_{6})$.

\vspace{2cm}

\begin{flushleft}
  {\bfseries\Large Abstract}
\end{flushleft}

In this thesis blah blah blah blah blah blah blah blah blah
blah blah blah blah blah blah blah blah blah.



 \newpage \thispagestyle{empty}

\chapter{Repaso de álgebra lineal}
%\label{cha:repaso-algebra-line}

En este capítulo se revisan algunos resultados de álgebra lineal, las
pruebas se omiten pero se puede consultar algún texto de álgebra
lineal como \cite{friedberg1982algebra}
  para más  detalles. En este trabajo todos los espacios vectoriales considerados serán dimensionalmente finitos.

\begin{definition}
  Sea $W$ un subespacio de un espacio vectorial $V$ sobre un campo
  $F$. Para toda $v\in V$ el conjunto $\{v\}+W=\{v+w:w\in W\}$ se
  llama \textbf{co-conjunto de $W$ que contiene a $v$}. Es frecuente
  expresar este co-conjunto como $v+W$ en vez de $\{v\}+W$. 
\end{definition}

Se puede demostrar lo siguiente:
 
$v_{1}+W=v_{2}+W$ si y solo si $v_{1}-v_{2}\in W.$

La suma y el producto puede definirse en el conjunto $S=\{v+W:v\in V\}$ de todos los co-conjuntos de $W$ como: 

$(v_{1}+W)+(v_{2}+W)=(v_{1}+v_{2})+W$ para toda $v_{1}$, $v_{2}\in V$ y

$a(v+W)=av+W$ para toda $v\in V$ y $a\in F$.

Se puede demostrar que las operaciones anteriores están bien definidas.

\begin{definition}
  El conjunto $S$ es un espacio vectorial bajo las operaciones
  definidas anteriormente y se llama \textbf{espacio cociente de $V$ módulo $W$} y se denota mediante $V/W$. 
\end{definition}

\begin{theorem}
  \label{dim-esp-coc}
  Sea $(V/W)$ el espacio cociente de $V$ módulo $W$, se tiene:
  $$\dim(V/W)=\dim(V)-\dim(W)$$
\end{theorem}

\begin{definition}
  Si $W_{1}$ y $W_{2}$ son dos subconjuntos no vacíos de un espacio
  vectorial $V$, entonces la \textbf{suma de $W_{1}$ y $W_{2}$}, que se
  expresa como $W_{1}+W_{2}$, es el conjunto $\{x+y:x\in W_{1}$ y $y\in
  W_{_2}\}$. La suma de cualquier número finito de subconjuntos no
  vacíos de $V$, $W_{1},\cdots,W_{n}$, se define análogamente como el
  conjunto
  $$W_{1}+\ldots+W_{n}=\{x_{1}+\ldots+x_{n}: x_{i}\in W_{i} \mbox{ para }i=1,2,\ldots,n\}$$
\end{definition}

\begin{definition}
  Un espacio vectorial $V$ es la \textbf{suma directa de $W_{1}$ y
  $W_{2}$}, denotada como $V=W_{1}\oplus W_{2}$, si $W_{1}$ y $W_{2}$
son subespacios de $V$ tales que $W_{1}\cap W_{2}=\{0\}$ y $W_{1}+W_{2}=V.$ 
\end{definition}

\begin{theorem}
  \label{esp-isomorfos}
  Sean $W_{1}$ y $W_{2}$ subespacios de $V$ tales que
  $V=W_{1}+W_{2}$. Luego, $V=W_{1}\oplus W_{2}$ si y solo si 
  $$\dim(V)=\dim(W_{1})+\dim(W_{2}).$$
\end{theorem}

\begin{theorem}
  \label{clunica}
  Sea $V$ un espacio vectorial y $\beta=\{x_{1},\dots,x_{n}\}$ un
  subconjunto de $V$. Luego $\beta$ es una base de $V$ si y solo si
  cada vector $y\in V$ puede ser expresado de manera única como una
  combinación lineal de vectores de $\beta$, es decir, puede ser
  expresado de la forma
  $$y+a_{1}x_{1}+\ldots,a_{n}x_{n}$$
  para escalares únicos $a_{1},\ldots,a_{n}.$
\end{theorem}

\begin{theorem}
  \label{imT}
  Sean $V$ y $W$ espacios vectoriales y sea $T:V \rightarrow W$
  lineal. Si $V$ tiene una base $\beta=\{v_{1},\ldots,v_{n}\}$,
  entonces $Im(T)=\langle T(v_{1}),\ldots,T(v_{n})\rangle.$  
\end{theorem}

\begin{theorem}
  Sean $V$ y $W$ espacios vectoriales (sobre el mismo campo
  $F$). Entonces $V$ es isomorfo a $W$ si y solo si $dim (V)=dim(W).$ 
\end{theorem}

\chapter{Representaciones de grupos}
\label{cha:primer-capitulo}

\section{Grupos}

\begin{definition}
  Una \textbf{operación binaria} en un conjunto $G$ es una función
  de la forma $G  \times G \rightarrow G$. Para cada $(a,b)\in G
  \times G$, denotaremos al elemento $*((a,b))\in G$ por $ab$. 
\end{definition} 

\begin{definition} 
  Un \textbf{grupo} es un conjunto no vacío $G$, junto con una
  \textbf{operación binaria} $*$ que satisface las siguientes condiciones:
    \begin{enumerate}
    \item La operación es asociativa, es decir, $$a(bc)=(ab)c$$ para todo $a,b,c \in G$.
    \item Existe un elemento neutro $1 \in G$ que
      satisface $$a1=1a=a$$ para todo $a \in G$.
    \item Para cada elemento $a \in G$ existe otro elemento $a' \in G$
      tal que $$aa'=a'a=1$$ Al elemento $a'$ se le llama inverso del elemento $a$.
    \end{enumerate}
\end{definition}

\begin{definition}
  Si $G$ y $H$ son grupos, entonces un \textbf{homomorfismo} de $G$
  en $H$ es una función $\phi:G\rightarrow H$ la cual
  satisface $$\phi(ab)=\phi(a)\phi(b)$$ para todo $a,b \in G$
\end{definition}

\begin{theorem}
  Sea $\rho:G\rightarrow G^{'}$ un homomorfismo de grupos. Entonces
  \begin{enumerate}
    \item $\rho(1)=1^{'}$, donde $1\in G$ y  $1^{'}\in G^{'}$ son los
    neutros respectivos. 
    \item Si $g\in G$, entonces $\rho(g^{-1})=\rho (g)^{-1}$.
  \end{enumerate}
\end{theorem}

La demostración del teorema anterior es elemental, por lo tanto la omitiremos.

\begin{definition}
  Cuando un homomorfismo de grupos $\phi:G\rightarrow H$ es biyectivo,
  diremos que $\phi$ es un  \textbf{isomorfismo}. También diremos que
  $G$ y $H$ son grupos  \textbf{isomorfos} cuando exista un
    isomorfismo entre ellos, usaremos la notación $G\cong H$.
\end{definition}

\begin{theorem}
  Supongamos que $G$ y $H$ son grupos y sea $\phi:G\rightarrow H$ un
  homomorfismo. Entonces $$G/\ker \phi\cong \im \phi$$ 
\end{theorem}

  Denotemos $GL(n,\mathbb{C})$ al grupo de matrices
  invertibles  $n \times n$ con entradas en $\mathbb{C}$.  
  Y a $GL(V)$ a los operadores lineales invertibles en un
  $\mathbb{C}$-espacio vectorial $V$ de dimensión finita $n$.
  Además $GL(n,\mathbb{C})\cong GL(V)$ es un isomorfismo de grupos.

\section{Acciones de grupos}

\begin{definition}
  Sea $G$ un grupo y $Y$ un conjunto no vacío. Una  \textbf{acción de $G$
  en $Y$} es una función $*:G \times Y \rightarrow Y$ tal que
\begin{enumerate}
\item $1*x=x$ para todo $x\in Y.$
\item $(g_{1}g_{2})*x=g_{1}*(g_{2}*x)$ para todo $x\in Y$ y $g_{1},g_{2}\in G.$
\end{enumerate}
   Bajo estas condiciones, $Y$ es un $G$-conjunto. En ocasiones, por
   abuso de notación, en lugar de $g*x$ usaremos la notación $gx.$
\end{definition}

\begin{definition}\textbf{Acciones lineales.}
  Sea $G$ un grupo. Una acción de $G$ en un $K$-espacio
  vectorial $V$ de dimensión finita es una función
 $$*:G\times V \rightarrow V $$
que satisface los axiomas:
\begin{enumerate}
\item $1v=v$, para todo $v\in V$ (donde $1$ es el neutro de $G$).
\item $(g_{1}g_{2})v=g_{1}(g_{2}v)$ para todos $v\in V$ y
  $g_{1},g_{2}\in G$.
\item $g_{1}(v+w)=g_{1}v+g_{1}w$, para $g_{1}\in G$ y $v,w \in V .$
\item $g_{1}(\lambda v)=\lambda(g_{1}v)$, para $\lambda \in K$,
  $v\in V$ y $g_{1}\in G.$
\end{enumerate}
Diremos que la acción es $G$-lineal y que $V$ es un $G$-espacio
vectorial.
\end{definition} 

\begin{theorem}
  Existe una correspondencia biyectiva entre el conjunto de acciones
  lineales de un grupo $G$ en un $K$-espacio vectorial $V$ y el conjunto
  de homomorfismos de $G$ en $GL(V)$
\end{theorem}

  \textit{Demostración.} Supongamos primero que se tiene una acción lineal 
  $$*:G\times V \rightarrow V$$
  para cada $g\in G$, usando esta acción definamos la función $\rho
  g:V \rightarrow V$ dada por $(\rho g)v=gv$ para todo $v\in V$.

  Primero observemos que la función $\rho:G\rightarrow GL(V)$ es un
  homomorfismo de grupos ya que si $g,h\in G$, entonces para todo
  $v\in V$:

  $$(\rho(gh))v=(gh)v=g(hv)=g((\rho h)v)=(\rho g)((\rho h)v)=(\rho g \circ \rho h)v$$
  por lo que $\rho(gh)=\rho g \circ \rho h$

  Ahora demostremos que $\rho g$ es una transformación lineal
  invertible. Sean $v,w \in V$, $\lambda \in K$.

  \begin{flalign*}
   &(\rho g)(v+w)=g(v+w)=gv+gw=(\rho g)v+(\rho g)w\\
   &(\rho g)(\lambda v)=g(\lambda v)=\lambda(gv)=\lambda((\rho g)v)       
  \end{flalign*}

  Así que $\rho g$ es lineal, ahora veamos que es invertible. Como $G$
  es un grupo, existe $g^{-1}\in G$.

  \begin{eqnarray*}
     ((\rho g)(\rho g)^{-1})v=(\rho g)((\rho g)^{-1}v)=(\rho g)((\rho g^{-1})v)=(\rho g)(g^{-1}v)=g(g^{-1}v)=(gg^{-1})v=v\\
     ((\rho g)^{-1}(\rho g))v=(\rho g)^{-1}((\rho g)v)=(\rho g^{-1})((\rho g)v)=(\rho g^{-1})(gv)=g^{-1}(gv)=(g^{-1}g)v=v
   \end{eqnarray*}

Suponga ahora que se tiene un homomorfismo de grupos $\rho:G\rightarrow GL(V)$, por demostrar que $gv:=(\rho g)v$, define un
acción lineal de $G$ en $V$. 

\begin{enumerate} 
   \item $1$ es el neutro de $G$, entonces $$1v=(\rho 1)v=id_{V}=v$$ para
     todo $v\in V$. 
   \item Si $g,h\in G$ y $v\in V$, entonces $$(gh)v=(\rho g h)v=(\rho
     g \circ \rho h)v=\rho g((\rho h)v)=\rho g(hv)=g(hv)$$
   \item Para $g\in G$ y $v,w\in V$, $$g(v+w)=(\rho g)(v+w)=(\rho g)v+(\rho g)w=gv+gw$$
   \item Para $\lambda\in K$, $v\in V$ y $g\in G$, 
    $$g(\lambda v)=(\rho g)(\lambda v)=\lambda((\rho g)v)=\lambda(gv)$$
\end{enumerate}

$\qed$

\section{Representaciones de grupos}

\begin{definition}
  Si $G$ es un grupo y $V$ es un $K$-espacio vectorial, una
  \textbf{representación lineal} de $G$ en $V$ es un homomorfismo 
     $$\rho:G\rightarrow GL(V).$$
  Se denotará $(\rho,V)$ para enfatizar el hecho de que la
  representación $\rho$ de $G$  es sobre el espacio lineal $V$. A
  dicho espacio lineal $V$ se le llamará \emph{espacio de representación} y a
  su dimensión el \emph{grado} de la representación 
\end{definition}

Estudiar las representaciones lineales de un grupo $G$ en un espacio
vectorial $V$, es equivalente a estudiar las acciones lineales de $G$
en $V$. 

En este trabajo estudiaremos representaciones de grupos finitos en
espacios vectoriales complejos y de dimensión finita. 

\begin{example}(\textbf{Representación trivial})
  La representación trivial de un grupo $G$ es el homomorfismo
  $\rho:G\rightarrow \mathbb{C^{*}}$ dado por $\rho(g)=1$ para todo
  $g\in G$.
\end{example}

\begin{example}(\textbf{Representación signo})
  El grupo simétrico sobre $\mathbb{I}_{n}=\{1,2,\ldots,n\}$, denotado
  por $\mathcal{S}_{n}$ es el grupo de todas las permutaciones de $\mathbb{I}_{n}$ 
  Si $\pi=\tau_{1}\tau_{2}\cdots\tau_{k}$, donde $\tau_{i}$ son
  transposiciones, definimos la función signo
  $\sgn:\mathcal{S}_{n} \rightarrow\{\pm1\}$ mediante $$\sgn(\pi):=(-1)^{k}.$$
\end{example}

\begin{example}(\textbf{Representación regular})
  Sea $G$ cualquier grupo de orden $n$ y sea V el espacio vectorial de
  dimensión $n$ con base $\mathcal{B}=\{v_{g}\}_{g\in G}$ indexada por
  los elementos del grupos. Para cada $\sigma\in G$ definamos la función
  $\rho(\sigma):\mathcal{B}\rightarrow \mathcal{B}$ mediante $\rho(\sigma)(v_{g})=v_{\sigma g}$.
  Mostremos que la función $$\rho:G\rightarrow GL(V)$$ dada por
  $\sigma\mapsto\rho(\sigma)$ es un homomorfismo. Si $\sigma,\tau \in G$
  y si $v_{g}\in \mathcal{B}$, entonces $$\rho(\sigma
  \tau)v_{g}=v_{(\sigma\tau)g}=v_{\sigma(\tau g)}=\rho(\sigma)v_{\tau
    g}=\rho(\sigma)(\rho(\tau)v_{g})=\rho(\sigma)\rho(\tau)v_{g} $$
  así que $\rho(\sigma\tau)=\rho(\sigma)\rho(\tau)$. Por definición el 
  grado de la representación es el orden de $G$.

\end{example}

\chapter{Homología de complejos simpliciales}
%\label{cha:primer-capitulo}

\section{Complejos simpliciales abstractos}

\begin{definition}
Un \textbf{complejo simplicial abstracto} es una colección finita
$\Delta$ de conjuntos no vacíos, tal que si $A$ es un elemento de $\Delta$,
cada subconjunto no vacío de $A$ pertenece a $\Delta$.
\end{definition}

El elemento $A$ de $\Delta$ es llamado \textbf{simplejo} de
$\Delta$; Si $A$ tiene $p+1$ elementos, decimos que $A$ es un
\emph{p-simplejo} y su dimensión es $p$. La dimensión de $\Delta$
es el máximo de las dimensiones de los simplejos de $\Delta$. El
conjunto \textbf{vértices} $V$ de $\Delta$ es la unión de los
elementos de un punto de $\Delta$; no hacemos distinción entre los
vértices $v\in V$ y los $0$-simplejos $\{v\}\in \Delta$. 

Primero introducimos la idea de un \emph{n-simplejo orientado}. Un
\textbf{0-simplejo orientado} es un punto $v$. Un \textbf{1-simplejo
  orientado} es un segmento de línea dirigido $v_{1}v_{2}$ uniendo los
puntos $v_{1}$ y $v_{2}$ en dirección de $v_{1}$ a $v_{2}$ ver \setlength{\fboxsep}{0pt}\colorbox{green}{FIGURA(referencia)}, así que
$v_{1}v_{2}\neq v_{2}v_{1}$, pero estaremos de acuerdo en que
$v_{1}v_{2}=-v_{2}v_{1}$. Un \textbf{2-simplejo orientado} es una
región triangular $v_{1}v_{2}v_{3}$ con dirección de $v_{1}$ a $v_{2}$
a $v_{3}$ ver \setlength{\fboxsep}{0pt}\colorbox{green}{FIGURA(referencia)}, claramente $v_{1}v_{2}v_{3}$ tiene el mismo orden que
$v_{2}v_{3}v_{1}$ y $v_{3}v_{1}v_{2}$, pero con orientación opuesta a
$v_{1}v_{3}v_{2}$, $v_{3}v_{2}v_{1}$ y $v_{2}v_{1}v_{3}$, es decir, estaremos de acuerdo que:
$$v_{1}v_{2}v_{3}=v_{2}v_{3}v_{1}=v_{3}v_{1}v_{2}=-v_{1}v_{3}v_{2}=-v_{3}v_{2}v_{1}=-v_{2}v_{1}v_{3}$$

Note que $v_{i}v_{j}v_{k}$ es igual a $v_{1}v_{2}v_{3}$ si

\[ \left(
  \begin{array}{ccc}
    1 & 2 & 3 \\
    i & j & k 
  \end{array} 
\right)\] 

es una permutación par y es igual a $-v_{1}v_{2}v_{3}$ si la
permutación es impar.

Un \textbf{3-simplejo orientado} está dado por una secuencia ordenada
$v_{1}v_{2}v_{3}v_{4}$ de cuatro vértices de un tetraedro sólido
ver \setlength{\fboxsep}{0pt}\colorbox{green}{FIGURA(referencia)} y
acordaremos que $v_{1}v_{2}v_{3}v_{4}=\pm v_{i}v_{j}v_{r}v_{s}$,
dependiendo si la permutación es par o impar. Así que dos
\emph{n-simplejos}, son equivalentes si difieren el uno del otro
por una permutación par. 

\begin{center}
  \begin{tikzpicture}
    \GraphInit[vstyle=Classic][scale=.2]
    \Vertex[x=0,y=0,Math]{v_{0}}
    \Vertex[x=1.5,y=1.5,Math]{v_{1}}
    \Edge[style={->}](v_{0})(v_{1})
  \end{tikzpicture}\quad
  \begin{tikzpicture}[scale=.6]
    \draw[help lines] (-2,0);% grid (2,3);
    \SetGraphUnit{2}
    \GraphInit[vstyle=Classic]
    \Vertex[x=-2,y=0,Math,LabelOut,Lpos=180,]{v_{0}}
    \Vertex[x=2,y=0,Math]{v_{1}}
    \Vertex[x=0,y=2.5,Math]{v_{2}}
    \Edge[style={->}](v_{0})(v_{1})
    \Edge[style={->}](v_{1})(v_{2})
    \Edge[style={->}](v_{2})(v_{0})
 \end{tikzpicture}\qquad
\begin{tikzpicture}[scale=.25]
  \SetVertexNoLabel
  \GraphInit[vstyle=Classic]
  \grTetrahedral[RA=4]
\end{tikzpicture}
\end{center}

\begin{definition}
  Sea $\Delta$ un complejo simplicial. Una \textbf{\emph{p}-cadena} en
  $\Delta$ es una función $c$ de el conjunto de $p$-simplejos de
  $\Delta$ a los complejos, tal que:
  \begin{enumerate}
    \item $c(\sigma)=-c(\sigma^{'})$ si $\sigma$ y $\sigma^{'}$ tienen
      direcciones opuestas del mismo simplejo.
    \item $c(\sigma)=0$ para todo $p$-simplejo orientado $\sigma$,
      excepto en un número finito de ellos.
  \end{enumerate} 
\end{definition}

Dado que en nuestro caso sólo estudiamos complejos simpliciales cuyo
conjunto de vértices es finito, la segunda condición no se aplica.

Sumamos $p$-cadenas sumando sus valores; el $\mathbb{C}$-espacio vectorial resultante es
denotado por $C_{p}(\Delta)$ y es llamado el \textbf{espacio de
  \emph{p}-cadenas (orientadas)} de $\Delta$. Si $p<0$ o $p>dim \Delta$,
$C_{p}(\Delta)$ denota al espacio trivial.

\begin{definition}
  Si $\sigma$ es un simplejo orientado, la \textbf{cadena elemental} $c$
  correspondiente a $\sigma$ es la función definida como:
  \[ 
  \begin{array}{cl}
    c(\sigma)=1, & \\
    c(\sigma^{'})=-1 & \mbox{si $\sigma^{'}$ tiene orientación opuesta de $\sigma$}, \\
    c(\tau)=0 & \mbox{para todos los otros simplejos orientados $\tau$}, 
  \end{array}\] 
  \end{definition}

Por abuso de notación, muchas veces usamos el símbolo $\sigma$ para
denotar no solo a un simplejo, o a un simplejo orientado, también
denotamos a la $p$-cadena elemental $c$ correspondiente al simplejo
orientado $\sigma$. Con esta convención, si $\sigma$ y $\sigma^{'}$
tienen orientaciones opuestas del mismo simplejo, entonces podemos
escribir $\sigma^{'}=-\sigma$, pues ésta ecuación se mantiene cuando
nos referimos a $\sigma$ y $\sigma^{'}$ como cadenas elementales.

\begin{lemma}
   Una base para $C_{p}(\Delta)$ se puede obtener
   tomando una orientación por cada $p$-simplejo y usando las
   correspondientes cadenas elementales como elementos de la base.
\end{lemma}

\textit{Demostración.} Orientando (arbitrariamente) a cada
\emph{p}-simplejo de $\Delta$, toda \emph{p}-cadena se puede escribir
de manera única como una combinación lineal finita
$$c=\sum n_{i}\sigma_{i},$$
de las correspondientes cadenas elementales $\sigma_{i}$. La
cadena $c$ asigna el valor $(n_{i})$ al \emph{p}-simplejo orientado
$\sigma_{i}$ , el valor $(-n_{i})$ a la orientación opuesta de
$\sigma_{i}$ y el valor $0$ a todo \emph{p}-simplejo orientado que no
aparece en la suma, se sigue del teorema $\ref{clunica}$. $\qed$

\begin{corollary}
  Toda función $f$ de los p-simplejos orientados de $\Delta$ en
  un espacio vectorial $V$ puede extenderse de manera única a una
  trasformación lineal de $C_{p}(\Delta)\rightarrow V$, tal que $f(-
  \sigma)=-f(\sigma)$ para todo p-simplejo orientado $\sigma$.
\end{corollary}

\begin{definition}
  Sea $\sigma=v_{0}\ldots v_{p}$ un simplejo orientado con $p>0$,
  definimos la trasformación lineal $\partial_{p}:C_{p}(\Delta)\rightarrow
  C_{p-1}(\Delta)$ como:

  \begin{equation}
    \label{ofrontera}
    \partial_{p}(\sigma)=\partial_{p}(v_{0}\ldots
    v_{p})=\sum^{p}_{i=0}(-1)^{i}(v_{0}\ldots \widehat v_{i}\ldots v_{p}),
  \end{equation}
  al que llamamos el \textbf{\emph{p}-ésimo operador frontera}, donde
  $\widehat v_{i}$ indica que el vértice $v_{i}$ es borrado del arreglo.
\end{definition}

Puesto que $C_{p}(\Delta)$ es el espacio trivial para $p<0$, diremos
que $\partial_{p}$ es la \textit{trasformación cero} para $p\leq
0$. Mostremos ahora que $\partial_{p}$ está bien definido y que
$\partial_{p}(-\sigma)=-\partial_{p}(\sigma)$. Para esto es suficiente
mostrar que la ecuación $(\ref{ofrontera})$ cambia de signo si intercambiamos dos
vértices adyacentes en el orden $v_{0}\ldots v_{p}$, entonces,
debemos comparar las expresiones:
$$\partial_{p}(v_{0}\ldots v_{j} v_{j+1} \ldots v_{p}) \mbox{ y } \partial_{p}(v_{0}\ldots v_{j+1} v_{j} \ldots v_{p}).$$

Para $i\neq j$, $j+1$, el $i$-ésimo término de estas dos expresiones
difieren precisamente por un signo; los términos son idénticos,
excepto que $v_{j}$ y $v_{j+1}$ aparecen intercambiados. Veamos que
sucede sobre el i-ésimo término cuando $i=j$ y $i=j+1$. En la primera
expresión tenemos que:
$$(-1)^{j}(\ldots v_{j-1} \widehat v_{j}v_{j+1}v_{j+2}\ldots)+(-1)^{j+1}(\ldots v_{j-1}v_{j}\widehat v_{j+1}v_{j+2}\ldots).$$
en la segunda expresión tenemos:
$$(-1)^{j}(\ldots v_{j-1}\widehat v_{j+1}v_{j}v_{j+2}\ldots)+(-1)^{j+1}(\ldots v_{j-1}v_{j+1}\widehat v_{j}v_{j+2}\ldots).$$
Comparando estas dos expresiones observamos que solo difieren por un signo.

\begin{example}
  De acuerdo a lo anterior, tenemos que
  \begin{enumerate}
  \item para un \emph{1-simplejo}: $\partial_{1}(v_{0}v_{1})= v_{1}-v_{0}$,
  \item para un \emph{2-simplejo}: $\partial_{2}(v_{0}v_{1}v_{2})=v_{1}v_{2}-v_{0}v_{2}+v_{0}v_{1}$,
  \item para un \emph{3-simplejo}:
    $\partial_{3}(v_{0}v_{1}v_{2}v_{3})=v_{1}v_{2}v_{3}-v_{0}v_{2}v_{3}+v_{0}v_{1}v_{3}-v_{0}v_{1}v_{2}$. 
  \end{enumerate}
\end{example}

\section{$\partial^{2}=0$ y Homología simplicial}

\begin{definition}
   El kernel de $\partial_{p}:C_{p}(\Delta)\rightarrow
   C_{p-1}(\Delta)$ es llamado el espacio de
   \textbf{\emph{p}-ciclos} y denotado por $Z_{p}(\Delta)$. La imagen
   de $\partial_{p+1}:C_{p+1}(\Delta)\rightarrow C_{p}(\Delta)$ es
   llamado el espacio de \textbf{\emph{p}-fronteras} y es denotado por $B_{p}(\Delta)$.
\end{definition}

\begin{theorem}
    $\partial_{p-1}\circ\partial_{p}=0$
\end{theorem}

\textit{Demostración.} Calculamos 
\begin{align*}
  \partial_{p-1}\partial_{p}(v_{0}\ldots
  v_{p})&=\sum_{i=0}^{p}(-1)^{i}\partial_{p-1}(v_{0}\ldots \widehat v_{i}\ldots v_{p})\\
  &=\sum_{j<i}(-1)^{i}(-1)^{j}(\ldots \widehat v_{j} \ldots \widehat v_{i} \ldots)\\
  &+\sum_{j>i}(-1)^{i}(-1)^{j-1}(\ldots\widehat v_{i}\ldots \widehat v_{j}\ldots).
\end{align*}

Los términos de estas dos sumas se cancelan a pares. $\qed$

\begin{corollary}
  $B_{p}(\Delta)$ es subespacio de $Z_{p}(\Delta)$.
\end{corollary}

\textit{Demostración.} Primero veamos que $B_{n}(\Delta)\subseteq Z_{n}(\Delta)$, tenemos que
$B_{n}(\Delta)=\partial[C_{n+1}(\Delta)]$, si $b\in B_{n}(\Delta)$,
podemos escribir a $b$ como $b=\partial_{n+1}(c)$ para algún $c\in
C_{n+1}(\Delta)$. Así que
$$\partial_{n}(b)=\partial_{n}(\partial_{n+1}(c))=0$$

por lo tanto $b\in Z_{n}(\Delta)$.
Las otras condiciones se siguen de que $\partial_{n+1}$ es una
transformación lineal. $\qed$
% \begin{enumerate}
%   \item Tomemos $0\in C_{n+1}$, así que $\partial_{n+1}(0)=\widehat 0$,
%     por lo tanto $\widehat 0\in B_{n}(\Delta)$
%   \item Sean $b_{1}\in B_{n}(\Delta)$ y $b_{2}\in B_{n}(\Delta)$,
%     donde $b_{1}=\partial_{n+1}(c_{1})$ y $b_{2}=\partial_{n+1}(c_{2})$ para
%     algún $c_{1},c_{2}\in C_{n+1}(\Delta)$, así que $b_{1}+b_{2}\in B_{n}$
%     pues $b_{1}+b_{2}=\partial_{n+1}(c_{1})+\partial_{n+1}(c_{2})=\partial_{n+1}(c_{1}+c_{2})$
%     ya que $c_{1}+c_{2}\in C_{n+1}(\Delta)$.
%   \item Por último sea $a\in \mathbb{C}$, y $b\in B_{n}(\Delta)$ con
%     $b=\partial_{n+1}(c)$ para algún $c\in C_{n+1}(\Delta)$, notemos
%     que $ac\in C_{n+1}(\Delta)$, así que $ab\in B_{n}(\Delta)$ pues $ab=a\partial_{n+1}(c)=\partial_{n+1}(ac)$.
% \end{enumerate}
\begin{definition}
   Definimos al espacio
   $$H_{p}(\Delta)=Z_{p}(\Delta)/B_{p}(\Delta)$$
   al cual llamamos la \textbf{\emph{p}-ésima homología de $\Delta$}
\end{definition}

\begin{example}
  Calculemos para $n=0, 1, 2$ los espacios $Z_{n}(\Delta)$,
  $B_{n}(\Delta)$ y $H_{n}$ para la superficie $\Delta$ del tetraedro.

  Como $C_{-1}(\Delta)=0$ por definición, se sigue que
  $$\boldsymbol{Z_{0}(\Delta)}=C_{0}(\Delta)$$
  Por otro lado, $C_{1}(\Delta)=\langle v_{0}v_{1},v_{0}v_{2},v_{0}v_{3},v_{1}v_{2},v_{1}v_{3},v_{2}v_{3}\rangle$,
  así que por el teorema $\ref{imT}$ tenemos 
  \begin{align}
    \label{frontera0}
    \boldsymbol{B_{0}(\Delta)}=&\partial_{1}[C_{1}(\Delta)]\nonumber\\
    &=\langle \partial_{1}(v_{0}v_{1}),\partial_{1}(v_{0}v_{2}),\partial_{1}(v_{0}v_{3}),\partial_{1}(v_{1}v_{2}),\partial_{1}(v_{1}v_{3}),\partial_{1}(v_{2}v_{3})\rangle\nonumber\\
    &=\langle v_{1}-v_{0},v_{2}-v_{0},v_{3}-v_{0},v_{2}-v_{1},v_{3}-v_{1},v_{3}-v_{2}\rangle\nonumber\\
    &=\langle v_{1}-v_{0},v_{2}-v_{0},v_{3}-v_{0}\rangle
  \end{align} 
  pues los vectores $v_{2}-v_{1}, v_{3}-v_{1}, v_{3}-v_{2}$ son
  combinación lineal de los vectores de $\ref{frontera0}$, es decir:
  $$v_{2}-v_{1}=(v_{2}-v_{0})-(v_{1}-v_{0})$$
  $$v_{3}-v_{1}=(v_{3}-v_{0})-(v_{1}-v_{0})$$
  $$v_{3}-v_{2}=(v_{3}-v_{0})-(v_{2}-v_{0})$$
  Así que $\dim(B_{0}(\Delta))=3$, además $C_{0}(\Delta)=\langle
  v_{0},v_{1},v_{2},v_{3}\rangle$, por lo que la
  $\dim(Z_{0}(\Delta))=\dim(C_{0}(\Delta))=4$, luego de el teorema
  $\ref{dim-esp-coc}$ tenemos que
  $\dim(Z_{0}(\Delta)/B_{0}(\Delta))=1$, así mismo la
  $\dim(\mathbb{C})=1$, por el resultado $\ref{esp-isomorfos}$ se sigue
  que $Z_{0}(\Delta)/B_{0}(\Delta)\cong \mathbb{C}$

  Por lo tanto:
  $$\boldsymbol{H_{0}(\Delta)}=Z_{0}(\Delta)/B_{0}(\Delta)\cong \mathbb{C}$$
  Un razonamiento similar se sigue para calcular las homologías restantes.
  Ahora calculemos $B_{1}(\Delta)$. Sabemos que:
  $$C_{2}(\Delta)=\langle
  v_{0}v_{1}v_{2},v_{0}v_{2}v_{3},v_{0}v_{1}v_{3},v_{1}v_{2}v_{3}\rangle$$
  Nuevamente por el teorema $\ref{imT}$ tenemos:
  \begin{align}  
    \label{generadores-B1}
    &\boldsymbol{B_{1}(\Delta)}=\partial_{2}[C_{2}(\Delta)]=\langle\partial_{2}(v_{0}v_{1}v_{2}),\partial_{2}(v_{0}v_{2}v_{3}),\partial_{2}(v_{0}v_{1}v_{3}),\partial_{2}
    (v_{1}v_{2}v_{3})\rangle \nonumber\\
    &=\langle v_{1}v_{2}-v_{0}v_{2}+v_{0}v_{1},v_{2}v_{3}-v_{0}v_{3}+v_{0}v_{2},\nonumber\\
    &\phantom{{}=v_{1}v_{2}-v_{0}v_{2}+v_{0}v_{1},v_{2}v_{3}}v_{1}v_{3}-v_{0}v_{3}+v_{0}v_{1},v_{2}v_{3}-v_{1}v_{3}+v_{1}v_{2}\rangle\nonumber\\
    &=\langle v_{1}v_{2}-v_{0}v_{2}+v_{0}v_{1},v_{2}v_{3}-v_{0}v_{3}+v_{0}v_{2},v_{1}v_{3}-v_{0}v_{3}+v_{0}v_{1}\rangle
  \end{align} 
 
  A continuación veremos que el conjunto generador de $Z_{1}(\Delta)$
  es el mismo que $B_{1}(\Delta)$. Sea $c\in C_{1}(\Delta)$, es decir,
 $$c=n_{1}v_{0}v_{1}+n_{2}v_{0}v_{2}+n_{3}v_{0}v_{3}+n_{4}v_{1}v_{2}+n_{5}v_{1}v_{3}+n_{6}v_{2}v_{3}$$
 tal que:
 \begin{align*}
   \partial_{1}(c)&=n_{1}(v_{1}-v_{0})+n_{2}(v_{2}-v_{0})+n_{3}(v_{3}-v_{0})\\
   &\phantom{{}=n_{1}}+n_{4}(v_{2}-v_{1})+n_{5}(v_{3}-v_{1})+n_{6}(v_{3}-v_{2})\\
   % \nonumber \\
   &=(-n_{1}-n_{2}-n_{3})v_{0}+(n_{1}-n_{4}-n_{5})v_{1}\\
   &\phantom{{}=-n_{1}}+(n_{2}+n_{4}-n_{6})v_{2}+(n_{3}+n_{5}+n_{6})v_{3}\\
   &=0
 \end{align*}
 Así que tenemos que resolver el siguiente sistema de ecuaciones:
 \[\begin{array}{rrrrr}
   -n_{1} & -n_{2} & -n_{3} & = & 0 \\
   n_{1} & -n_{4} & -n_{5} & = & 0 \\
   n_{2} & +n_{4} & -n_{6} & = & 0 \\
   n_{3} & +n_{5} & +n_{6} & = & 0 
 \end{array}\]
 Lo representamos en forma matricial.
 \[ \left(
   \begin{array}{rrrrrr}
  % n_{1} & n_{2} & n_{3} & n_{4} & n_{5} & n_{6} \\
     -1  & -1    & -1   & 0    & 0     & 0 \\
     1   & 0     &    0 & -1   & -1    & 0 \\
     0   & 1     &    0 & 1   & 0    & -1 \\
     0   & 0     &    1 & 0   & 1    & 1 
   \end{array} 
 \right)\]
 y lo llevamos a su forma escalonada reducida
 \[ \left(
   \begin{array}{rrrrrr}
     % n_{1} & n_{2} & n_{3} & n_{4} & n_{5} & n_{6} \\
     1     &    0  & 0     & -1    & -1    & 0 \\
     0     &    1  & 0     &  1    & 0     & -1 \\
     0     &    0  & 1     & 0     & 1     & 1 \\
     0     &    0  & 0     & 0     & 0     & 0 
   \end{array} 
 \right)\]
 De lo cual concluimos:
 \[\begin{array}{rrrrr}
   % n_{1}& = & n_{4} & n_{5} & n_{6} \\  
   n_{1} & = & n_{4} & +n_{5} & \\
   n_{2} & = & -n_{4} &      &+n_{6} \\
   n_{3} & = &        &-n_{5}&-n_{6} 
 \end{array}\]
 Entonces podemos escribir a $c\in Z_{1}(\Delta)$ como:
 \begin{align}
   \label{generadores-Z1}
   c&=(n_{4}+n_{5})v_{0}v_{1}+(-n_{4}+n_{6})v_{0}v_{2}+(-n_{5}-n_{6})v_{0}v_{3}\nonumber\\
   &\phantom{{}=n_{4}}+n_{4}v_{1}v_{2}+n_{5}v_{1}v_{3}+n_{6}v_{2}v_{3}\nonumber\\
   &=n_{4}(v_{0}v_{1}-v_{0}v_{2}+v_{1}v_{2})+n_{5}(v_{0}v_{1}-v_{0}v_{3}+v_{1}v_{3})\nonumber\\
   &\phantom{{}=n_{4}}+n_{6}(v_{0}v_{2}-v_{0}v_{3}+v_{2}v_{3})
 \end{align}
 Por lo que de la ecuación anterior $\ref{generadores-Z1}$ tenemos:
 $$\boldsymbol{Z_{1}(\Delta)}=\langle v_{0}v_{1}-v_{0}v_{2}+v_{1}v_{2},v_{0}v_{1}-v_{0}v_{3}+v_{1}v_{3},v_{0}v_{2}-v_{0}v_{3}+v_{2}v_{3}\rangle$$
 Notemos de las ecuaciones $\ref{generadores-B1}$ y
 $\ref{generadores-Z1}$ que  $Z_{1}(\Delta)$ y $B_{1}(\Delta)$ tienen
 el mismo conjunto generador, por lo tanto
 $Z_{1}(\Delta)=B_{1}(\Delta)$, en consecuencia,
 $$\boldsymbol{H_{1}(\Delta)}=Z_{1}(\Delta)/B_{1}(\Delta)=0$$
 Los simplejos de dimensión más alta son los \emph{2-simplejos}, así
 que $C_{3}(\Delta)=0$ por lo que 
 $$\boldsymbol{B_{2}(\Delta)}=\partial_{3}[C_{3}(\Delta)]=0$$ 
 Por determinar $Z_{2}(\Delta)$. Si $c\in C_{2}(\Delta)$, es decir 
 $$c=n_{1}v_{0}v_{1}v_{2}+n_{2}v_{0}v_{2}v_{3}+n_{3}v_{0}v_{1}v_{3}+n_{4}v_{1}v_{2}v_{3}$$
 tal que
 \begin{align*}
   \partial_{2}(c)&=n_{1}(v_{1}v_{2}-v_{0}v_{2}+v_{0}v_{1})+n_{2}(v_{2}v_{3}-v_{0}v_{3}+v_{0}v_{2})\\
   &+n_{3}(v_{1}v_{3}-v_{0}v_{3}+v_{0}v_{1})+n_{4}(v_{2}v_{3}-v_{1}v_{3}+v_{1}v_{2})\\
   &=(n_{1}+n_{3})v_{0}v_{1}+(-n_{1}+n_{2})v_{0}v_{2}+(-n_{2}-n_{3})v_{0}v_{3}\\
   & +(n_{1}+n_{4})v_{1}v_{2}+(n_{3}-n_{4})v_{1}v_{3}+(n_{2}+n_{4})v_{2}v_{3}\\
   &=0
 \end{align*}
 entonces $n_{1}=n_{2}=-n_{3}=-n_{4}$, así que podemos escribir a $c$
 de la siguiente forma: 
 $$c=n_{1}(v_{0}v_{1}v_{2}+v_{0}v_{2}v_{3}-v_{0}v_{1}v_{3}-v_{1}v_{2}v_{3})$$
 de donde vemos que
 $$Z_{2}(\Delta)=\langle v_{0}v_{1}v_{2}+v_{0}v_{2}v_{3}-v_{0}v_{1}v_{3}-v_{1}v_{2}v_{3}\rangle$$
 es decir, $\boldsymbol{Z_{2}(\Delta)}\cong \mathbb{C}$. Luego  $\boldsymbol{H_{2}}(\Delta)=Z_{2}(\Delta)/B_{2}(\Delta)\cong\mathbb{C}$.
\end{example}

\section{Complejo de emparejamiento}

\begin{definition}
Consideremos la gráfica completa de $n$ vértices $K_{n}$, tales
vértices son etiquetados como $1,2,\ldots,n$ e
$\overline{ij}$ denotará la arista que une al vértice $i$ con el
vértice $j$. Llamaremos \textbf{complejo de emparejamiento} de orden
$n$ al complejo simplicial $M_{n}$ de dimensión $n$ tal que:

\begin{enumerate}
  \item Su conjunto de vértices $\mathcal{V}$ consta de las aristas de la gráfica
  $K_{n}$. 
  \item Si $v_{i}=\overline{pq}$ y $v_{j}=\overline{rs}$ están en
  $\mathcal{V}$, $\{v_{i},v_{j}\}$ es  un 1-simplejo de $M_{n}$ si $v_{i}$
  y $v_{j}$ son ajenas
\end{enumerate} 
\end{definition}

\begin{example}
Considerese la gráfica $K_{4}$
\end{example}


\bigskip
\bigskip
\bigskip
\bigskip
\bigskip
\bigskip
\bigskip
\bigskip
\bigskip
\bigskip

\begin{center}
Tabla de S4

\begin{tabular}{c|r r r r r}
  No. Elementos& 1 & 6 & 8 & 6 & 3 \\
  Clase & (1) & (12) & (123) & (1234) &(12)(34)\\
    \hline
  $\chi_{{1}}$ & 1 & 1 & 1 & 1 & 1 \\
  $\chi_{{2}}$ & 1 & -1 & 1 & -1 & 1\\
  $\chi_{{3}}$ & 3 & 1 & 0 & -1 & -1\\
  $\chi_{{4}}$ & 3 & -1 & 0 & 1 & -1 \\
  $\chi_{{5}}$ & 2 & 0 & -1 & 0 & 2 \\
    \hline
  $\chi_{C_{0}(M_{4})}$ & 6 & 2 & 0 & 0 & 2 \\
  $\chi_{C_{1}(M_{4})}$ & 3 & 1 & 0 & -1 & -1
\end{tabular}
\end{center}

\bigskip

\begin{center}
\begin{small}
\begin{tabular}{c |r r r r r r r}
  No. Elementos& 1 & 10 & 20 & 30 & 24 & 15 & 20  \\
  Clase & (1) & (12) & (123) & (1234) & (12345) & (12)(34) & (123)(45) \\
    \hline
  $\chi_{\mathbb{C}}=\xi_{1}$ & 1 & 1 & 1 & 1 & 1 & 1 & 1 \\
  $\chi_{\mathbb{C}^{'}}=\xi_{2}$ & 1 & -1 & 1 & -1 & 1 & 1 & -1\\
  $\chi_{V_{5}}=\xi_{3}$ & 4 & 2 & 1 & 0 & -1 & 0 & -1\\
  $\chi_{V_{5}^{'}}=\xi_{4}$ & 4 & -2 & 1 & 0 & -1 & 0 & 1 \\
  $\chi_{W_{1}}=\xi_{5}$ & 6 & 0 & 0 & 0 & 1 & -2 & 0 \\
  $\chi_{W_{2}}=\xi_{6}$ & 5 & 1 & -1 & -1 & 0 & 1 & 1 \\
  $\chi_{W_{2}^{'}}=\xi_{7}$ & 5 & -1 & -1 & 1 & 0 & 1 & -1 \\
  \hline
  $\chi_{C_{0}(M_{5})}$ & 10 & 4 & 1 & 0 & 0 & 2 & 1 \\
  $\chi_{C_{1}(M_{5})}$ & 15 & 3 & 0 & -1 & 0 & -1 & 0
\end{tabular}
\end{small}
\end{center}

\bigskip

\begin{tabular}{c |r r r r r r}
  No. Elementos& 1 & 1 & 3 & 3 & 2 & 2 \\
  Clase & (1) & (45) & (12) & (12)(45) & (123) & (123)(45) \\
    \hline
  $\xi_{1}$ & 1 & 1 & 1 & 1 & 1 & 1 \\
  $\xi_{2}$ & 1 & -1 & -1 & 1 & 1 & -1 \\
  $\xi_{3}$ & 4 & 2 & 2 & 0 & 1 & -1 \\
  $\xi_{4}$ & 4 & -2 & -2 & 0 & 1 & 1 \\
  $\xi_{5}$ & 6 & 0 & 0 & -2 & 0 & 0 \\
  $\xi_{6}$ & 5 & 1 & 1 & 1 & -1 & 1 \\
  $\xi_{7}$ & 5 & -1 & -1 & 1 & -1 & -1 \\
  \hline
  $\phi_{1}$ & 1 & 1 & 1 & 1 & 1 & 1 \\
\end{tabular}

\bigskip

\begin{tabular}{c |r r r r r}
   & & (24) & (1432) & (14)(23) & \\
  Elementos & (1) & (13) & (1234) & (12)(34) & (13)(24) \\
    \hline
  $\xi_{1}$ & 1 & 1 & 1 & 1 & 1 \\
  $\xi_{2}$ & 1 & -1 & -1 & 1 & 1 \\
  $\xi_{3}$ & 4 & 2 & 0 & 0 & 0 \\
  $\xi_{4}$ & 4 & -2 & 0 & 0 & 0 \\
  $\xi_{5}$ & 6 & 0 & 0 & -2 & -2 \\
  $\xi_{6}$ & 5 & 1 & -1 & 1 & 1 \\
  $\xi_{7}$ & 5 & -1 & 1 & 1 & 1 \\
  \hline
  $\phi_{1}$ & 1 & 1 & -1 & -1 & 1 \\
\end{tabular}

\bigskip

\begin{tabular}{c |r r r r}
  Elementos & (1) & (12) & (45) & (12)(45) \\
    \hline
  $\xi_{1}$ & 1 & 1  & 1  & 1 \\
  $\xi_{2}$ & 1 & -1 & -1 & 1  \\
  $\xi_{3}$ & 4 & 2  & 2  & 0  \\
  $\xi_{4}$ & 4 & -2 & -2 & 0  \\
  $\xi_{5}$ & 6 & 0  & 0  & -2 \\
  $\xi_{6}$ & 5 & 1  & 1  & 1  \\
  $\xi_{7}$ & 5 & -1 & -1 & 1  \\
  \hline
  $\phi_{1}$ & 1 & 1 & -1 & -1 \\
\end{tabular}

\bigskip

\begin{tabular}{c |r r r r r r r r}
          &   & (12) & (12)(34) &          &             &              & &  \\        
          &   & (34) & (12)(56) &(13)(24)  &             & (13)(24)(56) & (2314)& (2314)(56)\\
Elementos &(1)& (56) & (34)(56) & (14)(23) & (12)(34)(56)& (14)(23)(56) &(2413)&(2413)(56)\\
    \hline
  $S^{{(6)}}$           & 1 & 1  & 1  & 1 & 1 & 1 & 1 & 1 \\
  $S^{{(1,1,1,1,1,1)}}$ & 1 & -1 & 1  & 1 &-1 &-1 &-1 & 1  \\
  $S^{{(5,1)}}$         & 5 & 3  & 1  & 1 &-1 &-1 & 1 &-1  \\
  $S^{{(2,1,1,1,1)}}$   & 5 & -3 &  1 & 1 & 1 & 1 &-1 &-1  \\
  $S^{{(4,1,1)}}$       & 10& 2  & -2 & -2&-2 &-2 & 0 & 0  \\
  $S^{{(3,1,1,1)}}$     & 10&-2  & -2 &-2 & 2 & 2 & 0 & 0  \\
  $S^{{(4,2)}}$         & 9 & 3  & 1  & 1 & 3 &3  &-1 & 1  \\
  $S^{{(2,2,1,1)}}$     & 9 & -3 & 1  & 1 & -3&-3 & 1 & 1  \\
  $S^{{(3,3)}}$         & 5 & 1  & 1  & 1 &-3 &-3 &-1 & -1 \\
  $S^{{(2,2,2)}}$       & 5 & -1 & 1  & 1 & 3 & 3 & 1 & -1 \\
  $S^{{(3,2,1)}}$       & 16& 0  & 0  & 0 & 0 & 0 & 0 &  0 \\
    \hline
  $\phi_{1}$            & 1 & 1  & 1  & -1& 1 & -1&-1 &-1  \\
\end{tabular}

\bigskip

\begin{tabular}{c |r r r r r r r r r r}
No. de Elementos  & 1 & 3 & 3 & 6 & 6 & 1 & 6 & 8 & 6 & 8 \\
 &(1)& (2) & (2,2) & (2,2) & (4)& (2,2,2) & (2,2,2) & (3,3) & (4,2) & (6)\\
    \hline
  $S^{{(6)}}$         & 1 & 1  & 1  & 1 & 1 &1  & 1 & 1 & 1 & 1 \\
  $S^{{(1,1,1,1,1,1)}}$ & 1 & -1 & 1  & 1 & -1&-1 &-1 & 1 & 1 & -1 \\
  $S^{{(5,1)}}$       & 5 & 3  & 1  & 1 & 1 &-1 &-1 &-1 &-1 & -1 \\
  $S^{{(2,1,1,1,1)}}$  & 5 & -3 &  1 & 1 &-1 & 1 & 1 &-1  &-1 & 1 \\
  $S^{{(4,1,1)}}$     & 10& 2  & -2 & -2& 0 &-2 &-2 & 1  & 0 & 1 \\
  $S^{{(3,1,1,1)}}$    & 10&-2  & -2 &-2 & 0 & 2 & 2 & 1 & 0 & -1 \\
  $S^{{(4,2)}}$       & 9 & 3  & 1  & 1 & -1& 3 & 3 & 0 & 1 &  0 \\
  $S^{{(2,2,1,1)}}$    & 9 & -3 & 1  & 1 & 1 &-3 &-3 & 0 & 1 &  0 \\
  $S^{{(3,3)}}$       & 5 & 1  & 1  & 1 &-1 &-3 & -3& 2 &-1 &  0 \\
  $S^{{(2,2,2)}}$     & 5  & -1& 1  & 1 & 1 & 3 & 3 & 2 & -1 & 0 \\
  $S^{{(3,2,1)}}$     & 16 & 0 & 0  & 0 & 0 & 0 & 0 & -2 & 0 & 0 \\
  \hline
  $\phi_{1}$ & 1 & 1 & 1 & -1 & -1 & 1 &-1 & 1 &- 1 & 1  \\
\end{tabular}

Vamos a citar \cite{MR0225619}

\bibliographystyle{plain}
\bibliography{labiblio}

\printindex


\end{document}
