
\documentclass[12pt]{book}

\usepackage[dvipsnames]{xcolor}
\usepackage{amssymb,latexsym}
\usepackage{graphicx}

\usepackage[spanish,mexico,es-nolayout]{babel}
\usepackage[utf8]{inputenc}
\usepackage{amsmath,amscd}
%\usepackage{amssymb}
\usepackage{amsthm}
%\usepackage{graphicx}
\usepackage{color}
\usepackage{tikz}
\usepackage{tkz-berge}
\usepackage{makeidx}
\usepackage{url}
\usepackage{xspace}
\usepackage{tocbibind}
% ver http://gilmation.com/articles/latex-margins-for-book-binding/
% y http://tex.stackexchange.com/questions/50258/margins-of-book-class
\usepackage[margin=3.5cm]{geometry}
\geometry{bindingoffset=1cm}

\usepackage{babelbib}

\usetikzlibrary{positioning,shapes,fit,arrows,decorations.pathmorphing}
\definecolor{myblue}{RGB}{56,94,141}


\newtheorem{theorem}{Teorema}[section]
\newtheorem{corollary}[theorem]{Corolario}
\newtheorem{proposition}[theorem]{Proposición}

\theoremstyle{definition}

\newtheorem{definition}[theorem]{Definición}
\newtheorem{notation}[theorem]{Notación}
\newtheorem{example}[theorem]{Ejemplo}
\newtheorem{lemma}[theorem]{Lema}

\DeclareMathOperator{\im}{im}
\DeclareMathOperator{\sgn}{sgn}

\newcounter{in}
\newcounter{ini}

\makeindex

\newcommand{\elespacio}{1.4cm}

\begin{document}
\mainmatter 
\begin{titlepage}
  \begin{center}
    \null
    \vspace*{\fill}

    \includegraphics[scale=1.2,bb=55 20 0 0]{escudouaeh.pdf}

    \vspace*{\elespacio}

    \textsc{Universidad Autónoma del Estado de Hidalgo}

    \textsc{Instituto de Ciencias Básicas e Ingeniería}

    \textsc{Área Académica de Matemáticas y Física}

    \vspace*{\elespacio}

    {\Huge\bfseries Representaciones del grupo simétrico en homologías\par}

    \vspace*{\elespacio}

    {\large Tesis que para obtener el título de}

    \vspace*{\elespacio}

    {\Large\textsc{Licenciada en Matemáticas Aplicadas}}

    \vspace*{\elespacio}

    {\large presenta}

    \vspace*{\elespacio}

    {\Huge Briseida Guadalupe Trejo Escamilla}

    \vspace*{\elespacio}

    {\large bajo la dirección de}

    \bigskip

    {\Large Dr.~Rafael Villarroel Flores}

    \bigskip

    {Pachuca, Hidalgo. Junio de 2013.}

    \vspace*{\fill}

  \end{center}
\end{titlepage}

\thispagestyle{empty}
\begin{flushleft}
  {\bfseries\Large Resumen}
\end{flushleft}

El presente trabajo tiene como objetivo analizar la estructura de
la homología de complejos simpliciales construidos a partir de una
gráfica simple finita en donde actúa el grupo simétrico $S_{n}$.

Se define la gráfica $M(G)$ de emparejamientos de la gráfica simple
$G$ como la gráfica cuyos vértices son las aristas de $G$ y dos
vértices son adyacentes si las correspondientes aristas son ajenas. En
la gráfica $G_{n}=M(K_{n})$, el grupo $S_{n}$ actúa de manera natural, por
lo que sus homologías con coeficientes complejos definen
representaciones de $S_{n}$. 

La descomposición en irreducibles de tales homologías ha sido exhibida
por Bouc (1992). En el presente trabajo se muestra el resultado
correspondiente para la gráfica de clanes $K(G_{6})$.

\vspace{2cm}

\begin{flushleft}
  {\bfseries\Large Abstract}
\end{flushleft}

In this thesis blah blah blah blah blah blah blah blah blah
blah blah blah blah blah blah blah blah blah.


\chapter*{Introducción}

(Resumen extendido)

En el primer capítulo...

\tableofcontents


 \newpage \thispagestyle{empty}

\chapter{Repaso de álgebra lineal}
%\label{cha:repaso-algebra-line}
\section{Espacios vectoriales}

En este capítulo se revisan algunos resultados de álgebra lineal, las
pruebas se omiten pero se puede consultar algún texto de álgebra
lineal como \cite{friedberg1982algebra}
  para más  detalles. En este trabajo todos los espacios vectoriales
  considerados serán dimensionalmente finitos y sobre los complejos.

\begin{definition}
  Sea $W$ un subespacio de un espacio vectorial $V$. Para toda $v\in V$ el conjunto $\{v\}+W=\{v+w:w\in W\}$ se
  llama \textbf{clase lateral de $W$ que contiene a $v$}. Es frecuente
  expresar esta clase lateral como $v+W$ en vez de $\{v\}+W$. 
\end{definition}

Se puede demostrar lo siguiente:
 
$v_{1}+W=v_{2}+W$ si y solo si $v_{1}-v_{2}\in W.$

La suma y el producto puede definirse en el conjunto $S=\{v+W:v\in
V\}$ de todos las clases laterales de $W$ como: 

$(v_{1}+W)+(v_{2}+W)=(v_{1}+v_{2})+W$ para toda $v_{1}$, $v_{2}\in V$ y

$a(v+W)=av+W$ para toda $v\in V$ y $a\in \mathbb{C}$.

Se puede demostrar que las operaciones anteriores están bien definidas.

\begin{definition}
  El conjunto $S$ es un espacio vectorial bajo las operaciones
  definidas anteriormente y se llama \textbf{espacio cociente de $V$ módulo $W$} y se denota mediante $V/W$. 
\end{definition}

\begin{theorem}
  \label{dim-esp-coc}
  Sea $(V/W)$ el espacio cociente de $V$ módulo $W$, se tiene:
  $$\dim(V/W)=\dim(V)-\dim(W)$$
\end{theorem}

\begin{theorem}
  \label{dim-esp-vec}
  Sean $W_{1}$ y $W_{2}$ subespacios de un espacio vectorial
  $V$. Entonces, 
  $$\dim(W_{1}+W_{2})=\dim(W_{1})+\dim(W_{2})-\dim(W_{1}\cap W_{2}).$$ 
\end{theorem}

\begin{definition}
  Si $W_{1}$ y $W_{2}$ son dos subconjuntos no vacíos de un espacio
  vectorial $V$, entonces la \textbf{suma} de $W_{1}$ y $W_{2}$, que se
  expresa como $W_{1}+W_{2}$, es el conjunto $\{x+y:x\in W_{1}$ y $y\in
  W_{_2}\}$. La suma de cualquier número finito de subconjuntos no
  vacíos de $V$, $W_{1},\cdots,W_{n}$, se define análogamente como el
  conjunto
  $$W_{1}+\ldots+W_{n}=\{x_{1}+\ldots+x_{n}: x_{i}\in W_{i} \mbox{ para }i=1,2,\ldots,n\}$$
\end{definition}

\begin{definition}
  Un espacio vectorial $V$ es la \textbf{suma directa de $W_{1}$ y
  $W_{2}$}, denotada como $V=W_{1}\oplus W_{2}$, si $W_{1}$ y $W_{2}$
son subespacios de $V$ tales que $W_{1}\cap W_{2}=\{0\}$ y $W_{1}+W_{2}=V.$ 
\end{definition}

\begin{theorem}
  Sean $W_{1}$ y $W_{2}$ subespacios de $V$ tales que
  $V=W_{1}+W_{2}$. Luego, $V=W_{1}\oplus W_{2}$ si y solo si 
  $$\dim(V)=\dim(W_{1})+\dim(W_{2}).$$
\end{theorem}

\begin{theorem}
  \label{clunica}
  Sea $V$ un espacio vectorial y $\beta=\{v_{1},\dots,v_{n}\}$ un
  subconjunto de $V$. Luego $\beta$ es una base de $V$ si y solo si
  cada vector $y\in V$ puede ser expresado de manera única como una
  combinación lineal de vectores de $\beta$, es decir, puede ser
  expresado de la forma
  $$y=a_{1}v_{1}+\ldots+a_{n}v_{n}$$
  para escalares únicos $a_{1},\ldots,a_{n}.$
\end{theorem}

\begin{theorem}
  \label{esp-iguales}
  Sea $W$ un subespacio de un espacio vectorial $V$ de dimensión
  $n$. Entonces, $W$ es dimensionalmente finito y $\dim(W)\leq
  n$. Además, si $\dim(W)=n$, entonces $W=V$.
\end{theorem}

\section{Transformaciones lineales}

\begin{theorem}
  \label{imT}
  Sean $V$ y $W$ espacios vectoriales y sea $T:V \rightarrow W$
  lineal. Si $V$ tiene una base $\beta=\{v_{1},\ldots,v_{n}\}$,
  entonces $\im(T)=\langle T(v_{1}),\ldots,T(v_{n})\rangle.$  
\end{theorem}

\begin{theorem}
  \label{esp-isomorfos}
  Sean $V$ y $W$ espacios vectoriales (sobre el mismo campo
  $F$). Entonces $V$ es isomorfo a $W$ si y solo si $\dim (V)=\dim(W).$ 
\end{theorem}

\begin{theorem}{(\textbf{Teorema de Isomorfismo})}
  \label{teorema-isomorfismo-esp}
  Si $f:V\rightarrow W$ es una transformación lineal, el espacio
  cociente $V/\ker(f)\cong \im(f)$. Un isomorfismo entre estos dos
  espacios es el siguiente:
  $$\phi:V/\ker(f)\rightarrow \im(f)$$
  definido por $\phi(v+\ker(f))=f(v).$
\end{theorem}
\begin{proof}[Demostración.]
  Hay que demostrar que $\phi$ está bien definida, es decir, que si
  $v_1+\ker(f)=v_2+\ker(f)$, entonces
  $f(v_1)=f(v_{2}).$ Pero
  $v_1+\ker(f)=v_2+\ker(f)$ si y sólo si
  $v_{1}-v_{2}\in \ker(f)$, es decir,
  $f(v_{1}-v_{2})=f(v_1)-f(v_2)=0$,
  como se quería. Queda por demostrar de forma directa que $\phi$ es
  una transformación lineal; y para demostrar que es isomorfismo falta
  demostrar que es inyectiva y sobreyectiva, lo cual también se hace
  de forma directa.
\end{proof}

\chapter{Representaciones de grupos}
\label{Representaciones de grupos}

\section{Grupos}

\begin{definition}
  Una \textbf{operación binaria} en un conjunto $G$ es una función
  de la forma $G  \times G \rightarrow G$. Para cada $(a,b)\in G
  \times G$, denotaremos al elemento $*((a,b))\in G$ por $ab$. 
\end{definition} 

\begin{definition} 
  Un \textbf{grupo} es un conjunto no vacío $G$, junto con una
  \textbf{operación binaria} $*$ que satisface las siguientes condiciones:
    \begin{enumerate}
    \item La operación es asociativa, es decir, $$a(bc)=(ab)c$$ para todo $a,b,c \in G$.
    \item Existe un elemento neutro $1 \in G$ que
      satisface $$a1=1a=a$$ para todo $a \in G$.
    \item Para cada elemento $a \in G$ existe otro elemento $a' \in G$
      tal que $$aa'=a'a=1$$ Al elemento $a'$ se le llama inverso del elemento $a$.
    \end{enumerate}
\end{definition}

\begin{definition}
  Si $G$ y $H$ son grupos, entonces un \textbf{homomorfismo} de $G$
  en $H$ es una función $\phi:G\rightarrow H$ la cual
  satisface $$\phi(ab)=\phi(a)\phi(b)$$ para todo $a,b \in G$
\end{definition}

\begin{theorem}
  Sea $\rho:G\rightarrow G^{'}$ un homomorfismo de grupos. Entonces
  \begin{enumerate}
    \item $\rho(1)=1^{'}$, donde $1\in G$ y  $1^{'}\in G^{'}$ son los
    neutros respectivos. 
    \item Si $g\in G$, entonces $\rho(g^{-1})=\rho (g)^{-1}$.
  \end{enumerate}
\end{theorem}

La demostración del teorema anterior es elemental, por lo tanto la omitiremos.

\begin{definition}
  Cuando un homomorfismo de grupos $\phi:G\rightarrow H$ es biyectivo,
  diremos que $\phi$ es un  \textbf{isomorfismo}. También diremos que
  $G$ y $H$ son grupos  \textbf{isomorfos} cuando exista un
    isomorfismo entre ellos, usaremos la notación $G\cong H$.
\end{definition}

\begin{theorem}
  Supongamos que $G$ y $H$ son grupos y sea $\phi:G\rightarrow H$ un
  homomorfismo. Entonces $$G/\ker \phi\cong \im \phi$$ 
\end{theorem}

  Denotemos $GL(n,\mathbb{C})$ al grupo de matrices
  invertibles  $n \times n$ con entradas en $\mathbb{C}$.  
  Y a $GL(V)$ a los operadores lineales invertibles en un
  $\mathbb{C}$-espacio vectorial $V$ de dimensión finita $n$.
  Además $GL(n,\mathbb{C})\cong GL(V)$ es un isomorfismo de grupos.

\section{Acciones de grupos}

\begin{definition}
  Sea $G$ un grupo y $Y$ un conjunto no vacío. Una  \textbf{acción de $G$
  en $Y$} es una función $*:G \times Y \rightarrow Y$ tal que
\begin{enumerate}
\item $1*x=x$ para todo $x\in Y.$
\item $(g_{1}g_{2})*x=g_{1}*(g_{2}*x)$ para todo $x\in Y$ y $g_{1},g_{2}\in G.$
\end{enumerate}
   Bajo estas condiciones, $Y$ es un $G$-conjunto. En ocasiones, por
   abuso de notación, en lugar de $g*x$ usaremos la notación $gx.$
\end{definition}

\begin{definition}\textbf{Acciones lineales.}
  Sea $G$ un grupo. Una acción de $G$ en un $K$-espacio
  vectorial $V$ de dimensión finita es una función
 $$*:G\times V \rightarrow V $$
que satisface los axiomas:
\begin{enumerate}
\item $1v=v$, para todo $v\in V$ (donde $1$ es el neutro de $G$).
\item $(g_{1}g_{2})v=g_{1}(g_{2}v)$ para todos $v\in V$ y
  $g_{1},g_{2}\in G$.
\item $g_{1}(v+w)=g_{1}v+g_{1}w$, para $g_{1}\in G$ y $v,w \in V .$
\item $g_{1}(\lambda v)=\lambda(g_{1}v)$, para $\lambda \in K$,
  $v\in V$ y $g_{1}\in G.$
\end{enumerate}
Diremos que la acción es $G$-lineal y que $V$ es un $G$-espacio
vectorial.
\end{definition} 

\begin{theorem}
  Existe una correspondencia biyectiva entre el conjunto de acciones
  lineales de un grupo $G$ en un $K$-espacio vectorial $V$ y el conjunto
  de homomorfismos de $G$ en $GL(V)$
\end{theorem}

  \textit{Demostración.} Supongamos primero que se tiene una acción lineal 
  $$*:G\times V \rightarrow V$$
  para cada $g\in G$, usando esta acción definamos la función $\rho
  g:V \rightarrow V$ dada por $(\rho g)v=gv$ para todo $v\in V$.

  Primero observemos que la función $\rho:G\rightarrow GL(V)$ es un
  homomorfismo de grupos ya que si $g,h\in G$, entonces para todo
  $v\in V$:

  $$(\rho(gh))v=(gh)v=g(hv)=g((\rho h)v)=(\rho g)((\rho h)v)=(\rho g \circ \rho h)v$$
  por lo que $\rho(gh)=\rho g \circ \rho h$

  Ahora demostremos que $\rho g$ es una transformación lineal
  invertible. Sean $v,w \in V$, $\lambda \in K$.

  \begin{flalign*}
   &(\rho g)(v+w)=g(v+w)=gv+gw=(\rho g)v+(\rho g)w\\
   &(\rho g)(\lambda v)=g(\lambda v)=\lambda(gv)=\lambda((\rho g)v)       
  \end{flalign*}

  Así que $\rho g$ es lineal, ahora veamos que es invertible. Como $G$
  es un grupo, existe $g^{-1}\in G$.

  \begin{eqnarray*}
     ((\rho g)(\rho g)^{-1})v=(\rho g)((\rho g)^{-1}v)=(\rho g)((\rho g^{-1})v)=(\rho g)(g^{-1}v)=g(g^{-1}v)=(gg^{-1})v=v\\
     ((\rho g)^{-1}(\rho g))v=(\rho g)^{-1}((\rho g)v)=(\rho g^{-1})((\rho g)v)=(\rho g^{-1})(gv)=g^{-1}(gv)=(g^{-1}g)v=v
   \end{eqnarray*}

Suponga ahora que se tiene un homomorfismo de grupos $\rho:G\rightarrow GL(V)$, por demostrar que $gv:=(\rho g)v$, define un
acción lineal de $G$ en $V$. 

\begin{enumerate} 
   \item $1$ es el neutro de $G$, entonces $$1v=(\rho 1)v=id_{V}=v$$ para
     todo $v\in V$. 
   \item Si $g,h\in G$ y $v\in V$, entonces $$(gh)v=(\rho g h)v=(\rho
     g \circ \rho h)v=\rho g((\rho h)v)=\rho g(hv)=g(hv)$$
   \item Para $g\in G$ y $v,w\in V$, $$g(v+w)=(\rho g)(v+w)=(\rho g)v+(\rho g)w=gv+gw$$
   \item Para $\lambda\in K$, $v\in V$ y $g\in G$, 
    $$g(\lambda v)=(\rho g)(\lambda v)=\lambda((\rho g)v)=\lambda(gv)$$
\end{enumerate}

$\qed$

\section{Representaciones de grupos}

\begin{definition}
  Si $G$ es un grupo y $V$ es un $K$-espacio vectorial, una
  \textbf{representación lineal} de $G$ en $V$ es un homomorfismo 
     $$\rho:G\rightarrow GL(V).$$
  Se denotará $(\rho,V)$ para enfatizar el hecho de que la
  representación $\rho$ de $G$  es sobre el espacio lineal $V$. A
  dicho espacio lineal $V$ se le llamará \emph{espacio de representación} y a
  su dimensión el \emph{grado} de la representación 
\end{definition}

Estudiar las representaciones lineales de un grupo $G$ en un espacio
vectorial $V$, es equivalente a estudiar las acciones lineales de $G$
en $V$. 

En este trabajo estudiaremos representaciones de grupos finitos en
espacios vectoriales complejos y de dimensión finita. 

\begin{example}(\textbf{Representación trivial})
  La representación trivial de un grupo $G$ es el homomorfismo
  $\rho:G\rightarrow \mathbb{C^{*}}$ dado por $\rho(g)=1$ para todo
  $g\in G$.
\end{example}

\begin{example}(\textbf{Representación signo})
  El grupo simétrico sobre $\mathbb{I}_{n}=\{1,2,\ldots,n\}$, denotado
  por $\mathcal{S}_{n}$ es el grupo de todas las permutaciones de $\mathbb{I}_{n}$ 
  Si $\pi=\tau_{1}\tau_{2}\cdots\tau_{k}$, donde $\tau_{i}$ son
  transposiciones, definimos la función signo
  $\sgn:\mathcal{S}_{n} \rightarrow\{\pm1\}$ mediante $$\sgn(\pi):=(-1)^{k}.$$
\end{example}

\begin{example}(\textbf{Representación regular})
  Sea $G$ cualquier grupo de orden $n$ y sea V el espacio vectorial de
  dimensión $n$ con base $\mathcal{B}=\{v_{g}\}_{g\in G}$ indexada por
  los elementos del grupos. Para cada $\sigma\in G$ definamos la función
  $\rho(\sigma):\mathcal{B}\rightarrow \mathcal{B}$ mediante $\rho(\sigma)(v_{g})=v_{\sigma g}$.
  Mostremos que la función $$\rho:G\rightarrow GL(V)$$ dada por
  $\sigma\mapsto\rho(\sigma)$ es un homomorfismo. Si $\sigma,\tau \in G$
  y si $v_{g}\in \mathcal{B}$, entonces $$\rho(\sigma
  \tau)v_{g}=v_{(\sigma\tau)g}=v_{\sigma(\tau g)}=\rho(\sigma)v_{\tau
    g}=\rho(\sigma)(\rho(\tau)v_{g})=\rho(\sigma)\rho(\tau)v_{g} $$
  así que $\rho(\sigma\tau)=\rho(\sigma)\rho(\tau)$. Por definición el 
  grado de la representación es el orden de $G$.
\end{example}

\section{Módulos irreducibles y submódulos}
\label{mod-irr-submodulos}
Denote $G$ un grupo y $V$ un espacio vectorial sobre
el campo de los complejos. 

\begin{definition}
  Sea $V$ un $G$-módulo no trivial. Decimos que $V$ es \textbf{irreducible} si
  los únicos submódulos de $V$ son $0$ y $V$.
\end{definition}

\begin{definition}
  Sea $V$ un $G$-módulo. Un \textbf{submódulo} de $V$ es un subespacio
  $W$ que es invariante bajo la acción de $G$, es decir, $ gw\in W$
  para todos $w\in W$, $g\in G$, tal que $W$ junto con la acción de
  $G$ es en sí mismo un $G$-módulo. Escribiremos $W\leq V$ si $W$ es
  submódulo de $V$. 
\end{definition}

% \begin{theorem}
%   \label{interseccion-submodulos}
%   Sea $V$ un $G$-módulo. Entonces la intersección de cualquier
%   colección de submódulos de $V$ es un submódulo de $V$.
% \end{theorem}

\begin{definition}
  Sea $V$ un $G$-módulo. Al submódulo de $V$ generado por las combinaciones lineales de los
  elementos de $W$ lo llamaremos \textbf{submódulo generado} por $W$ y lo denotaremos como $\langle W\rangle$.
\end{definition}

Sean $U$ y $V$ dos espacios vectoriales. Llamaremos \textbf{suma directa} de
$U$ y $V$ al conjunto $U\times V=\{(u,v):u\in U,v\in V\}$ con las operaciones
$$(u,v)+(u_{1},v_{1})=(u+u_{1},v+v_{1})$$
$$k(u,v)=(ku,kv),$$
donde $u,u_{1}\in U$, $v,v_{1}\in V$ y $k\in \mathbb{C}$. Con estas
operaciones $U\times V$ es un espacio vectorial, que designaremos por
$$U\oplus V.$$

Así que dados $V$ y $W$ $G$-módulos, podemos formar un tercer módulo a partir
de la suma directa $V\oplus W$, definiendo la acción como
$a(v,w)=(av,aw)$, con $a\in G$. Podemos extender esta definición a cualquier
cantidad finita de $G$-módulos.

Por otro lado, si $V$ es un $G$-módulo, se llama la \textbf{suma} de $W_{1}$ y $W_{2}$ y se denota con
$W_{1}+W_{2}$ al submódulo de $V$ generado por $W_{1}\cup W_{2}$. Ésta definición se puede extender a una
colección arbitraria $\{W_{i}\}_{i\in I}$ es
decir, $\sum_{i\in I}W_{i}$ es el submódulo generado por $\bigcup_{i\in I}W_{i}$.


\textbf{\emph{Observación:}} En el caso de que $W_{1}$ y $W_{2}$ son
submódulos de $V$ con $W_{1}\cap W_{2}=0$ tenemos que
$W_{1}+W_{2}\cong W_{1}\oplus W_{2}$.
% dado por el siguiente isomorfismo:
% \begin{align*}
%   \phi:W_{1}+W_{2}&\rightarrow W_{1}\oplus W_{2}\\
%   w_{1} & \mapsto  (w_{1},0)\\
%   w_{2} & \mapsto  (0,w_{2})\\
%   0 & \mapsto (0,0)
% \end{align*}
\begin{proposition}
  \label{modulos-iguales}
  Supongamos que $S\leq M$, $S$ irreducible, si $M=S\oplus T$ donde $S$
  tiene multiplicidad uno y sea $S^{'}\leq M$, tal que $S^{'}\cong S$, entonces $S=S^{'}$.
\end{proposition}

\begin{proof}[Demostración.]
  Consideremos $S\cap S^{'}\leq S$, así que $S\cap S^{'}=0$ o $S\cap
  S^{'}=S$, pues $S$ es irreducible.

  Si $S\cap S^{'}=0$, consideremos $U$ el submódulo generado por $S\cup S'$ al cual
  denotamos como $S+S^{'}$ y como $S\cap S^{'}=0$, entonces
  $S+S^{'}\cong S\oplus S^{'}$ como notamos en la observación anterior,
  así que $M$ tendría un submódulo isomorfo a dos copias de S, lo cual
  no es posible por hipótesis. 
  
 Por otro lado, si $S\cap S^{'}=S$, entonces $\dim (S\cap
 S^{'})=\dim(S)$, además por hipótesis $S\cong S^{'}$ así que
 $\dim(S)=\dim(S^{'})$. Por lo tanto $\dim(S\cap S^{'})=\dim(S^{'})$ y
 sea $S\cap S^{'}\leq S^{'}$ por el teorema \ref{esp-iguales} tenemos
 que $S\cap S^{'}=S^{'}$, con lo que $S=S^{'}$ como se quería demostrar.
\end{proof}

\begin{theorem}[Lema de Schur]
  \label{lema-schur}
  Si $V$ y $W$ son $G$-módulos irreducibles, y $\phi:V\rightarrow W$
  es un morfismo de $G$-módulos no trivial, entonces $\phi$ es un isomorfismo.
\end{theorem}

\begin{proof}[Demostración.]
  Como $\ker \phi\leq V$, y $V$ es irreducible, entonces $\ker \phi=0$
  o $\ker\phi=V$, pero  $\ker\phi\neq V$ pues $\phi$ es no trivial,
  así que $\ker \phi=0$ y por lo tanto $\phi$ es inyectiva. 

  Análogamente, $\im\phi\leq W$ y $\im \phi\neq 0$, así que $\im
  \phi=W$ y entonces $\phi$ es suprayectiva.
\end{proof}

\begin{proposition}
  \label{im-mod-irreducible}
   Si $V$ y $W$ son $G$-módulos, $f:V\rightarrow W$ un morfismo de $G$-módulos, $S\leq V$ con $S$
  irreducible, entonces $f(S)\cong S$ o $f(S)=0$.
\end{proposition}

\begin{proof}[Demostración.]
  Tomemos el siguiente morfismo de módulos
  $S\stackrel{i}{\hookrightarrow} V\stackrel{f}{\rightarrow}W$ donde
  $i$ es función inclusión. Así que $S/\ker(f\circ i)\cong\im(f\circ
  i)$ por el teorema \ref{teorema-isomorfismo-esp}. Como $\ker(f\circ
  i)\leq S$ entonces $\ker(f\circ i)=0$ o $\ker(f\circ i)=S$, pues $S$ es irreducible.

  Si $\ker(f\circ i)=0$, se sigue que
  $$S\cong S/0\cong\im(f\circ i)=f(S)$$

  Por otro lado, si $\ker(f\circ i)=S$, tenemos
  $$0=S/S\cong\im(f\circ i)=f(S)$$
\end{proof}
\bigskip
\begin{theorem}[Teorema de isomorfismo de módulos]
  \label{teorema-isomorfismo-mod}
  Sean.....
\end{theorem}
\chapter{Homología de complejos simpliciales}
%\label{cha:primer-capitulo}

\section{Complejos simpliciales abstractos}

\begin{definition}
Un \textbf{complejo simplicial abstracto} es una colección finita
$\Delta$ de conjuntos no vacíos, tal que si $A$ es un elemento de $\Delta$,
cada subconjunto no vacío de $A$ pertenece a $\Delta$.
\end{definition}

El elemento $A$ de $\Delta$ es llamado \textbf{simplejo} de
$\Delta$; Si $A$ tiene $p+1$ elementos, decimos que $A$ es un
\emph{p-simplejo} y su dimensión es $p$. La dimensión de $\Delta$
es el máximo de las dimensiones de los simplejos de $\Delta$. El
conjunto \textbf{vértices} $V$ de $\Delta$ es la unión de los
elementos de un punto de $\Delta$; no hacemos distinción entre los
vértices $v\in V$ y los $0$-simplejos $\{v\}\in \Delta$. 

Primero introducimos la idea de un \emph{n-simplejo orientado}. Un
\textbf{0-simplejo orientado} es un punto $v$. Un \textbf{1-simplejo
  orientado} es un segmento de línea dirigido $v_{1}v_{2}$ uniendo los
puntos $v_{1}$ y $v_{2}$ en dirección de $v_{1}$ a $v_{2}$ ver \setlength{\fboxsep}{0pt}\colorbox{green}{FIGURA(referencia)}, así que
$v_{1}v_{2}\neq v_{2}v_{1}$, pero estaremos de acuerdo en que
$v_{1}v_{2}=-v_{2}v_{1}$. Un \textbf{2-simplejo orientado} es una
región triangular $v_{1}v_{2}v_{3}$ con dirección de $v_{1}$ a $v_{2}$
a $v_{3}$ ver \setlength{\fboxsep}{0pt}\colorbox{green}{FIGURA(referencia)}, claramente $v_{1}v_{2}v_{3}$ tiene el mismo orden que
$v_{2}v_{3}v_{1}$ y $v_{3}v_{1}v_{2}$, pero con orientación opuesta a
$v_{1}v_{3}v_{2}$, $v_{3}v_{2}v_{1}$ y $v_{2}v_{1}v_{3}$, es decir, estaremos de acuerdo que:
$$v_{1}v_{2}v_{3}=v_{2}v_{3}v_{1}=v_{3}v_{1}v_{2}=-v_{1}v_{3}v_{2}=-v_{3}v_{2}v_{1}=-v_{2}v_{1}v_{3}$$

Note que $v_{i}v_{j}v_{k}$ es igual a $v_{1}v_{2}v_{3}$ si

\[ \left(
  \begin{array}{ccc}
    1 & 2 & 3 \\
    i & j & k 
  \end{array} 
\right)\] 

es una permutación par y es igual a $-v_{1}v_{2}v_{3}$ si la
permutación es impar.

Un \textbf{3-simplejo orientado} está dado por una secuencia ordenada
$v_{1}v_{2}v_{3}v_{4}$ de cuatro vértices de un tetraedro sólido
ver \setlength{\fboxsep}{0pt}\colorbox{green}{FIGURA(referencia)} y
acordaremos que $v_{1}v_{2}v_{3}v_{4}=\pm v_{i}v_{j}v_{r}v_{s}$,
dependiendo si la permutación es par o impar. Así que dos
\emph{n-simplejos}, son equivalentes si difieren el uno del otro
por una permutación par. 

\begin{center}
  \begin{tikzpicture}
    \GraphInit[vstyle=Classic][scale=.2]
    \Vertex[x=0,y=0,Math]{v_{0}}
    \Vertex[x=1.5,y=1.5,Math]{v_{1}}
    \Edge[style={->}](v_{0})(v_{1})
  \end{tikzpicture}\quad
  \begin{tikzpicture}[scale=.6]
    \draw[help lines] (-2,0);% grid (2,3);
    \SetGraphUnit{2}
    \GraphInit[vstyle=Classic]
    \Vertex[x=-2,y=0,Math,LabelOut,Lpos=180,]{v_{0}}
    \Vertex[x=2,y=0,Math]{v_{1}}
    \Vertex[x=0,y=2.5,Math]{v_{2}}
    \Edge[style={->}](v_{0})(v_{1})
    \Edge[style={->}](v_{1})(v_{2})
    \Edge[style={->}](v_{2})(v_{0})
 \end{tikzpicture}\qquad
\begin{tikzpicture}[scale=.25]
  \SetVertexNoLabel
  \GraphInit[vstyle=Classic]
  \grTetrahedral[RA=4]
\end{tikzpicture}
\end{center}

\begin{definition}
  Sea $\Delta$ un complejo simplicial. Una \textbf{\emph{p}-cadena} en
  $\Delta$ es una función $c$ de el conjunto de $p$-simplejos de
  $\Delta$ a los complejos, tal que:
  \begin{enumerate}
    \item $c(\sigma)=-c(\sigma^{'})$ si $\sigma$ y $\sigma^{'}$ tienen
      direcciones opuestas del mismo simplejo.
    \item $c(\sigma)=0$ para todo $p$-simplejo orientado $\sigma$,
      excepto en un número finito de ellos.
  \end{enumerate} 
\end{definition}

Dado que en nuestro caso sólo estudiamos complejos simpliciales cuyo
conjunto de vértices es finito, la segunda condición no se aplica.

Sumamos $p$-cadenas sumando sus valores; el $\mathbb{C}$-espacio vectorial resultante es
denotado por $C_{p}(\Delta)$ y es llamado el \textbf{espacio de
  \emph{p}-cadenas (orientadas)} de $\Delta$. Si $p<0$ o $p>dim \Delta$,
$C_{p}(\Delta)$ denota al espacio trivial.

\begin{definition}
  Si $\sigma$ es un simplejo orientado, la \textbf{cadena elemental} $c$
  correspondiente a $\sigma$ es la función definida como:
  \[ 
  \begin{array}{cl}
    c(\sigma)=1, & \\
    c(\sigma^{'})=-1 & \mbox{si $\sigma^{'}$ tiene orientación opuesta de $\sigma$}, \\
    c(\tau)=0 & \mbox{para todos los otros simplejos orientados $\tau$}, 
  \end{array}\] 
  \end{definition}

Por abuso de notación, muchas veces usamos el símbolo $\sigma$ para
denotar no solo a un simplejo, o a un simplejo orientado, también
denotamos a la $p$-cadena elemental $c$ correspondiente al simplejo
orientado $\sigma$. Con esta convención, si $\sigma$ y $\sigma^{'}$
tienen orientaciones opuestas del mismo simplejo, entonces podemos
escribir $\sigma^{'}=-\sigma$, pues ésta ecuación se mantiene cuando
nos referimos a $\sigma$ y $\sigma^{'}$ como cadenas elementales.

\begin{lemma}
   Una base para $C_{p}(\Delta)$ se puede obtener
   tomando una orientación por cada $p$-simplejo y usando las
   correspondientes cadenas elementales como elementos de la base.
\end{lemma}

\textit{Demostración.} Orientando (arbitrariamente) a cada
\emph{p}-simplejo de $\Delta$, toda \emph{p}-cadena se puede escribir
de manera única como una combinación lineal finita
$$c=\sum n_{i}\sigma_{i},$$
de las correspondientes cadenas elementales $\sigma_{i}$. La
cadena $c$ asigna el valor $(n_{i})$ al \emph{p}-simplejo orientado
$\sigma_{i}$ , el valor $(-n_{i})$ a la orientación opuesta de
$\sigma_{i}$ y el valor $0$ a todo \emph{p}-simplejo orientado que no
aparece en la suma, se sigue del teorema $\ref{clunica}$. $\qed$

\begin{corollary}
  Toda función $f$ de los p-simplejos orientados de $\Delta$ en
  un espacio vectorial $V$ puede extenderse de manera única a una
  trasformación lineal de $C_{p}(\Delta)\rightarrow V$, tal que $f(-
  \sigma)=-f(\sigma)$ para todo p-simplejo orientado $\sigma$.
\end{corollary}

\begin{definition}
  Sea $\sigma=(v_{0},\ldots ,v_{p})$ un simplejo orientado con $p>0$
  (sin embargo también será denotado como $\sigma=v_{0}\ldots v_{p}$
  cuando no tengamos problemas de confusión),
  definimos la trasformación lineal $\partial_{p}:C_{p}(\Delta)\rightarrow
  C_{p-1}(\Delta)$ como:

  \begin{equation}
    \label{ofrontera}
    \partial_{p}(\sigma)=\partial_{p}(v_{0}\ldots
    v_{p})=\sum^{p}_{i=0}(-1)^{i}(v_{0}\ldots \widehat v_{i}\ldots v_{p}),
  \end{equation}
  al que llamamos el \textbf{\emph{p}-ésimo operador frontera}, donde
  $\widehat v_{i}$ indica que el vértice $v_{i}$ es borrado del arreglo.
\end{definition}

Puesto que $C_{p}(\Delta)$ es el espacio trivial para $p<0$, diremos
que $\partial_{p}$ es la \textit{trasformación cero} para $p\leq
0$. Mostremos ahora que $\partial_{p}$ está bien definido y que
$\partial_{p}(-\sigma)=-\partial_{p}(\sigma)$. Para esto es suficiente
mostrar que la ecuación $(\ref{ofrontera})$ cambia de signo si intercambiamos dos
vértices adyacentes en el orden $v_{0}\ldots v_{p}$, entonces,
debemos comparar las expresiones:
$$\partial_{p}(v_{0}\ldots v_{j} v_{j+1} \ldots v_{p}) \mbox{ y } \partial_{p}(v_{0}\ldots v_{j+1} v_{j} \ldots v_{p}).$$

Para $i\neq j$, $j+1$, el $i$-ésimo término de estas dos expresiones
difieren precisamente por un signo; los términos son idénticos,
excepto que $v_{j}$ y $v_{j+1}$ aparecen intercambiados. Veamos que
sucede sobre el i-ésimo término cuando $i=j$ y $i=j+1$. En la primera
expresión tenemos que:
$$(-1)^{j}(\ldots v_{j-1} \widehat v_{j}v_{j+1}v_{j+2}\ldots)+(-1)^{j+1}(\ldots v_{j-1}v_{j}\widehat v_{j+1}v_{j+2}\ldots).$$
en la segunda expresión tenemos:
$$(-1)^{j}(\ldots v_{j-1}\widehat v_{j+1}v_{j}v_{j+2}\ldots)+(-1)^{j+1}(\ldots v_{j-1}v_{j+1}\widehat v_{j}v_{j+2}\ldots).$$
Comparando estas dos expresiones observamos que solo difieren por un signo.

\begin{example}
  De acuerdo a lo anterior, tenemos que
  \begin{enumerate}
  \item para un \emph{1-simplejo}: $\partial_{1}(v_{0}v_{1})= v_{1}-v_{0}$,
  \item para un \emph{2-simplejo}: $\partial_{2}(v_{0}v_{1}v_{2})=v_{1}v_{2}-v_{0}v_{2}+v_{0}v_{1}$,
  \item para un \emph{3-simplejo}:
    $\partial_{3}(v_{0}v_{1}v_{2}v_{3})=v_{1}v_{2}v_{3}-v_{0}v_{2}v_{3}+v_{0}v_{1}v_{3}-v_{0}v_{1}v_{2}$. 
  \end{enumerate}
\end{example}

\section{$\partial^{2}=0$ y Homología simplicial}

\begin{definition}
   El kernel de $\partial_{p}:C_{p}(\Delta)\rightarrow
   C_{p-1}(\Delta)$ es llamado el espacio de
   \textbf{\emph{p}-ciclos} y denotado por $Z_{p}(\Delta)$. La imagen
   de $\partial_{p+1}:C_{p+1}(\Delta)\rightarrow C_{p}(\Delta)$ es
   llamado el espacio de \textbf{\emph{p}-fronteras} y es denotado por $B_{p}(\Delta)$.
\end{definition}

\begin{definition}
  Sea $\varepsilon:C_{0}\rightarrow \mathbb{C}$ la transformación
  lineal sobreyectiva definida por $\varepsilon(v)=1$ para cada
  vértice $v\in \Delta$. Entonces si $c$ es una $0$-cadena,
  $\varepsilon(c)$ es igual a la suma de los valores de $c$ en los
  vértices de $\Delta$, es decir:
  $$\varepsilon(\sum \lambda_{i}v_{i})=\sum\lambda_{i}$$
  $\varepsilon$ es llamada la \textbf{función aumento} para
  $C_{0}(\Delta)$.
\end{definition}

\begin{theorem}
  $\partial_{p-1}\circ\partial_{p}=0$ para cualquier $p$ y también $\varepsilon\circ\partial_{1}=0$.
\end{theorem}

\textit{Demostración.} Calculamos 
\begin{align*}
  \partial_{p-1}\partial_{p}(v_{0}\ldots
  v_{p})&=\sum_{i=0}^{p}(-1)^{i}\partial_{p-1}(v_{0}\ldots \widehat v_{i}\ldots v_{p})\\
  &=\sum_{j<i}(-1)^{i}(-1)^{j}(\ldots \widehat v_{j} \ldots \widehat v_{i} \ldots)\\
  &+\sum_{j>i}(-1)^{i}(-1)^{j-1}(\ldots\widehat v_{i}\ldots \widehat v_{j}\ldots).
\end{align*}

Los términos de estas dos sumas se cancelan a pares. $\qed$

\begin{corollary}
  $B_{p}(\Delta)$ es subespacio de $Z_{p}(\Delta)$.
\end{corollary}

\textit{Demostración.} Primero veamos que $B_{n}(\Delta)\subseteq Z_{n}(\Delta)$, tenemos que
$B_{n}(\Delta)=\partial[C_{n+1}(\Delta)]$, si $b\in B_{n}(\Delta)$,
podemos escribir a $b$ como $b=\partial_{n+1}(c)$ para algún $c\in
C_{n+1}(\Delta)$. Así que
$$\partial_{n}(b)=\partial_{n}(\partial_{n+1}(c))=0$$

por lo tanto $b\in Z_{n}(\Delta)$.
Las otras condiciones se siguen de que $\partial_{n+1}$ es una
transformación lineal. $\qed$

De forma análoga se demuestra que $B_{0}(\Delta)$ es subespacio de $\ker(\varepsilon)$.
% \begin{enumerate}
%   \item Tomemos $0\in C_{n+1}$, así que $\partial_{n+1}(0)=\widehat 0$,
%     por lo tanto $\widehat 0\in B_{n}(\Delta)$
%   \item Sean $b_{1}\in B_{n}(\Delta)$ y $b_{2}\in B_{n}(\Delta)$,
%     donde $b_{1}=\partial_{n+1}(c_{1})$ y $b_{2}=\partial_{n+1}(c_{2})$ para
%     algún $c_{1},c_{2}\in C_{n+1}(\Delta)$, así que $b_{1}+b_{2}\in B_{n}$
%     pues $b_{1}+b_{2}=\partial_{n+1}(c_{1})+\partial_{n+1}(c_{2})=\partial_{n+1}(c_{1}+c_{2})$
%     ya que $c_{1}+c_{2}\in C_{n+1}(\Delta)$.
%   \item Por último sea $a\in \mathbb{C}$, y $b\in B_{n}(\Delta)$ con
%     $b=\partial_{n+1}(c)$ para algún $c\in C_{n+1}(\Delta)$, notemos
%     que $ac\in C_{n+1}(\Delta)$, así que $ab\in B_{n}(\Delta)$ pues $ab=a\partial_{n+1}(c)=\partial_{n+1}(ac)$.
% \end{enumerate}
\begin{definition}
   Definimos al espacio
   $$H_{p}(\Delta)=Z_{p}(\Delta)/B_{p}(\Delta)$$
   al cual llamamos la \textbf{\emph{p}-ésima homología de $\Delta$}
\end{definition}

\begin{definition}
  Definimos la \textbf{homología reducida} de $\Delta$ en
  dimensión $0$, denotado por $\widetilde H_{0}(\Delta)$, como
  \begin{equation*}
    \widetilde H_{0}(\Delta)=\ker\varepsilon/\im \partial_{1}
  \end{equation*}
  (Si $p>0$, $\widetilde H_{p}(\Delta)$ denota el espacio usual
  $H_{p}(\Delta)$.)
\end{definition}

\begin{example}
  Calculemos para $n=0, 1, 2$ los espacios $Z_{n}(\Delta)$,
  $B_{n}(\Delta)$ y $H_{n}$ para la superficie $\Delta$ del tetraedro.

  Como $C_{-1}(\Delta)=0$ por definición, se sigue que
  $$\boldsymbol{Z_{0}(\Delta)}=C_{0}(\Delta)$$
  Por otro lado, $C_{1}(\Delta)=\langle v_{0}v_{1},v_{0}v_{2},v_{0}v_{3},v_{1}v_{2},v_{1}v_{3},v_{2}v_{3}\rangle$,
  así que por el teorema $\ref{imT}$ tenemos 
  \begin{align}
    \label{frontera0}
    \boldsymbol{B_{0}(\Delta)}=&\partial_{1}[C_{1}(\Delta)]\nonumber\\
    &=\langle \partial_{1}(v_{0}v_{1}),\partial_{1}(v_{0}v_{2}),\partial_{1}(v_{0}v_{3}),\partial_{1}(v_{1}v_{2}),\partial_{1}(v_{1}v_{3}),\partial_{1}(v_{2}v_{3})\rangle\nonumber\\
    &=\langle v_{1}-v_{0},v_{2}-v_{0},v_{3}-v_{0},v_{2}-v_{1},v_{3}-v_{1},v_{3}-v_{2}\rangle\nonumber\\
    &=\langle v_{1}-v_{0},v_{2}-v_{0},v_{3}-v_{0}\rangle
  \end{align} 
  pues los vectores $v_{2}-v_{1}, v_{3}-v_{1}, v_{3}-v_{2}$ son
  combinación lineal de los vectores de $\ref{frontera0}$, es decir:
  $$v_{2}-v_{1}=(v_{2}-v_{0})-(v_{1}-v_{0})$$
  $$v_{3}-v_{1}=(v_{3}-v_{0})-(v_{1}-v_{0})$$
  $$v_{3}-v_{2}=(v_{3}-v_{0})-(v_{2}-v_{0})$$
  Así que $\dim(B_{0}(\Delta))=3$, además $C_{0}(\Delta)=\langle
  v_{0},v_{1},v_{2},v_{3}\rangle$, por lo que la
  $\dim(Z_{0}(\Delta))=\dim(C_{0}(\Delta))=4$, luego de el teorema
  $\ref{dim-esp-coc}$ tenemos que
  $\dim(Z_{0}(\Delta)/B_{0}(\Delta))=1$, así mismo la
  $\dim(\mathbb{C})=1$, por el resultado $\ref{esp-isomorfos}$ se sigue
  que $Z_{0}(\Delta)/B_{0}(\Delta)\cong \mathbb{C}$

  Por lo tanto:
  $$\boldsymbol{H_{0}(\Delta)}=Z_{0}(\Delta)/B_{0}(\Delta)\cong \mathbb{C}$$
  Un razonamiento similar se sigue para calcular las homologías restantes.
  Ahora calculemos $B_{1}(\Delta)$. Sabemos que:
  $$C_{2}(\Delta)=\langle
  v_{0}v_{1}v_{2},v_{0}v_{2}v_{3},v_{0}v_{1}v_{3},v_{1}v_{2}v_{3}\rangle$$
  Nuevamente por el teorema $\ref{imT}$ tenemos:
  \begin{align}  
    \label{generadores-B1}
    &\boldsymbol{B_{1}(\Delta)}=\partial_{2}[C_{2}(\Delta)]=\langle\partial_{2}(v_{0}v_{1}v_{2}),\partial_{2}(v_{0}v_{2}v_{3}),\partial_{2}(v_{0}v_{1}v_{3}),\partial_{2}
    (v_{1}v_{2}v_{3})\rangle \nonumber\\
    &=\langle v_{1}v_{2}-v_{0}v_{2}+v_{0}v_{1},v_{2}v_{3}-v_{0}v_{3}+v_{0}v_{2},\nonumber\\
    &\phantom{{}=v_{1}v_{2}-v_{0}v_{2}+v_{0}v_{1},v_{2}v_{3}}v_{1}v_{3}-v_{0}v_{3}+v_{0}v_{1},v_{2}v_{3}-v_{1}v_{3}+v_{1}v_{2}\rangle\nonumber\\
    &=\langle v_{1}v_{2}-v_{0}v_{2}+v_{0}v_{1},v_{2}v_{3}-v_{0}v_{3}+v_{0}v_{2},v_{1}v_{3}-v_{0}v_{3}+v_{0}v_{1}\rangle
  \end{align} 
 
  A continuación veremos que el conjunto generador de $Z_{1}(\Delta)$
  es el mismo que $B_{1}(\Delta)$. Sea $c\in C_{1}(\Delta)$, es decir,
 $$c=n_{1}v_{0}v_{1}+n_{2}v_{0}v_{2}+n_{3}v_{0}v_{3}+n_{4}v_{1}v_{2}+n_{5}v_{1}v_{3}+n_{6}v_{2}v_{3}$$
 tal que:
 \begin{align*}
   \partial_{1}(c)&=n_{1}(v_{1}-v_{0})+n_{2}(v_{2}-v_{0})+n_{3}(v_{3}-v_{0})\\
   &\phantom{{}=n_{1}}+n_{4}(v_{2}-v_{1})+n_{5}(v_{3}-v_{1})+n_{6}(v_{3}-v_{2})\\
   % \nonumber \\
   &=(-n_{1}-n_{2}-n_{3})v_{0}+(n_{1}-n_{4}-n_{5})v_{1}\\
   &\phantom{{}=-n_{1}}+(n_{2}+n_{4}-n_{6})v_{2}+(n_{3}+n_{5}+n_{6})v_{3}\\
   &=0
 \end{align*}
 Así que tenemos que resolver el siguiente sistema de ecuaciones:
 \[\begin{array}{rrrrr}
   -n_{1} & -n_{2} & -n_{3} & = & 0 \\
   n_{1} & -n_{4} & -n_{5} & = & 0 \\
   n_{2} & +n_{4} & -n_{6} & = & 0 \\
   n_{3} & +n_{5} & +n_{6} & = & 0 
 \end{array}\]
 Lo representamos en forma matricial.
 \[ \left(
   \begin{array}{rrrrrr}
  % n_{1} & n_{2} & n_{3} & n_{4} & n_{5} & n_{6} \\
     -1  & -1    & -1   & 0    & 0     & 0 \\
     1   & 0     &    0 & -1   & -1    & 0 \\
     0   & 1     &    0 & 1   & 0    & -1 \\
     0   & 0     &    1 & 0   & 1    & 1 
   \end{array} 
 \right)\]
 y lo llevamos a su forma escalonada reducida
 \[ \left(
   \begin{array}{rrrrrr}
     % n_{1} & n_{2} & n_{3} & n_{4} & n_{5} & n_{6} \\
     1     &    0  & 0     & -1    & -1    & 0 \\
     0     &    1  & 0     &  1    & 0     & -1 \\
     0     &    0  & 1     & 0     & 1     & 1 \\
     0     &    0  & 0     & 0     & 0     & 0 
   \end{array} 
 \right)\]
 De lo cual concluimos:
 \[\begin{array}{rrrrr}
   % n_{1}& = & n_{4} & n_{5} & n_{6} \\  
   n_{1} & = & n_{4} & +n_{5} & \\
   n_{2} & = & -n_{4} &      &+n_{6} \\
   n_{3} & = &        &-n_{5}&-n_{6} 
 \end{array}\]
 Entonces podemos escribir a $c\in Z_{1}(\Delta)$ como:
 \begin{align}
   \label{generadores-Z1}
   c&=(n_{4}+n_{5})v_{0}v_{1}+(-n_{4}+n_{6})v_{0}v_{2}+(-n_{5}-n_{6})v_{0}v_{3}\nonumber\\
   &\phantom{{}=n_{4}}+n_{4}v_{1}v_{2}+n_{5}v_{1}v_{3}+n_{6}v_{2}v_{3}\nonumber\\
   &=n_{4}(v_{0}v_{1}-v_{0}v_{2}+v_{1}v_{2})+n_{5}(v_{0}v_{1}-v_{0}v_{3}+v_{1}v_{3})\nonumber\\
   &\phantom{{}=n_{4}}+n_{6}(v_{0}v_{2}-v_{0}v_{3}+v_{2}v_{3})
 \end{align}
 Por lo que de la ecuación anterior $\ref{generadores-Z1}$ tenemos:
 $$\boldsymbol{Z_{1}(\Delta)}=\langle v_{0}v_{1}-v_{0}v_{2}+v_{1}v_{2},v_{0}v_{1}-v_{0}v_{3}+v_{1}v_{3},v_{0}v_{2}-v_{0}v_{3}+v_{2}v_{3}\rangle$$
 Notemos de las ecuaciones $\ref{generadores-B1}$ y
 $\ref{generadores-Z1}$ que  $Z_{1}(\Delta)$ y $B_{1}(\Delta)$ tienen
 el mismo conjunto generador, por lo tanto
 $Z_{1}(\Delta)=B_{1}(\Delta)$, en consecuencia,
 $$\boldsymbol{H_{1}(\Delta)}=Z_{1}(\Delta)/B_{1}(\Delta)=0$$
 Los simplejos de dimensión más alta son los \emph{2-simplejos}, así
 que $C_{3}(\Delta)=0$ por lo que 
 $$\boldsymbol{B_{2}(\Delta)}=\partial_{3}[C_{3}(\Delta)]=0$$ 
 Por determinar $Z_{2}(\Delta)$. Si $c\in C_{2}(\Delta)$, es decir 
 $$c=n_{1}v_{0}v_{1}v_{2}+n_{2}v_{0}v_{2}v_{3}+n_{3}v_{0}v_{1}v_{3}+n_{4}v_{1}v_{2}v_{3}$$
 tal que
 \begin{align*}
   \partial_{2}(c)&=n_{1}(v_{1}v_{2}-v_{0}v_{2}+v_{0}v_{1})+n_{2}(v_{2}v_{3}-v_{0}v_{3}+v_{0}v_{2})\\
   &+n_{3}(v_{1}v_{3}-v_{0}v_{3}+v_{0}v_{1})+n_{4}(v_{2}v_{3}-v_{1}v_{3}+v_{1}v_{2})\\
   &=(n_{1}+n_{3})v_{0}v_{1}+(-n_{1}+n_{2})v_{0}v_{2}+(-n_{2}-n_{3})v_{0}v_{3}\\
   & +(n_{1}+n_{4})v_{1}v_{2}+(n_{3}-n_{4})v_{1}v_{3}+(n_{2}+n_{4})v_{2}v_{3}\\
   &=0
 \end{align*}
 entonces $n_{1}=n_{2}=-n_{3}=-n_{4}$, así que podemos escribir a $c$
 de la siguiente forma: 
 $$c=n_{1}(v_{0}v_{1}v_{2}+v_{0}v_{2}v_{3}-v_{0}v_{1}v_{3}-v_{1}v_{2}v_{3})$$
 de donde vemos que
 $$Z_{2}(\Delta)=\langle v_{0}v_{1}v_{2}+v_{0}v_{2}v_{3}-v_{0}v_{1}v_{3}-v_{1}v_{2}v_{3}\rangle$$
 es decir, $\boldsymbol{Z_{2}(\Delta)}\cong \mathbb{C}$. Luego  $\boldsymbol{H_{2}}(\Delta)=Z_{2}(\Delta)/B_{2}(\Delta)\cong\mathbb{C}$.
\end{example}

\section{Complejo de cadenas}

\begin{definition}
  Un \textbf{complejo de cadenas} $(A,\partial)$ es una secuencia
  $$A=\{\cdots,A_{2},A_{1},A_{0},A_{-1},A_{-2},\cdots\}$$
  de espacios vectoriales $A_{k}$, junto con una colección
  $\partial=\{\partial_{k}\mid k \in \mathbb{Z}\}$ de transformaciones
  lineales tales que $\partial_{k}:A_{k}\rightarrow A_{k-1}$ y
  $\partial_{k-1}\partial_{k}=0.$

  Al complejo de cadenas en donde $A_{-1}=\mathbb{C}$ y $\partial_{0}=\varepsilon$,
  donde $\varepsilon$ es la función aumento, le llamamos \textbf{complejo de cadenas aumentado}.
\end{definition}

\begin{theorem}
  Si $(A,\partial)$ es un complejo de cadenas, entonces la imagen bajo
  $\partial_{k}$ es un subespacio de el kernel de $\partial_{k-1}$
\end{theorem}
\begin{proof}[Demostración.]
  Considere
  \begin{small}
    \[
    \begin{array}{ccccc}
      A_{k} & \stackrel{\partial_{k}}{\longrightarrow} & A_{k-1} &
      \stackrel{\partial_{k-1}}{\longrightarrow} & A_{k-2}.
    \end{array} 
    \]
  \end{small}
  $\partial_{k-1}\partial_{k}=0,$ pues $(A,\partial)$ es un complejo
  de cadenas. Esto es $\partial_{k-1}[\partial_{k}[A_{k}]]=0.$ Así que
  $\partial_{k}[A_{k}]$ está contenido en el kernel de
  $\partial_{k-1}$, como se quería demostrar.
\end{proof}

\begin{definition}
  Si $(A,\partial)$ es un complejo de cadenas, entonces el kernel
  $Z_{k}(A)$ de $\partial_{k}$ es el\textbf{ espacio de} $\boldsymbol{k}$\textbf{-ciclos,} y
  la imagen $B_{k}(A)=\partial_{k+1}[A_{k+1}]$ es el \textbf{espacio
    de} $\boldsymbol{k}$\textbf{-fronteras.} El espacio cociente $H_{k}(A)=Z_{k}(A)/B_{k}(A)$
  es la $\boldsymbol{k}$\textbf{-ésima homología de A.}
\end{definition}

\begin{theorem}
  Sean $(A,\partial)$ y $(A^{'},\partial^{'})$ complejos de cadenas, y
  supongamos que hay una colección $f$ de transformaciones lineales
  $f_{k}:A_{k}\rightarrow A^{'}_{k}$ como se indica en el diagrama  
  \[
  \begin{CD}
    \cdots @>{\partial_{k+2}}>> A_{k+1} @>{\partial_{k+1}}>> A_{k} @>{\partial_{k}}>> A_{k-1} @>{\partial_{k-1}}>> \cdots\\
    @.   @VVf_{k+1}V   @VVf_{k}V   @VVf_{k-1}V    \\
    \cdots @>{\partial^{'}_{k+2}}>> A^{'}_{k+1} @>{\partial^{'}_{k+1}}>> A^{'}_{k} @>{\partial^{'}_{k}}>> A^{'}_{k-1} @>{\partial^{'}_{k-1}}>> \cdots
  \end{CD}
  \]
  Supongamos, además que
  $$f_{k-1}\partial_{k}=\partial^{'}_{k}f_{k}$$
  para todo $k$. Entonces $f_{k}$ induce una transformación lineal
  $f_{*k}:H_{k}(A)\rightarrow H_{k}(A^{'}).$
\end{theorem}
\begin{proof}[Demostración.]
  Sea $z\in Z_{k}(A)$. Ahora
  \begin{equation*}
    \partial^{'}_{k}(f_{k}(z))=f_{k-1}(\partial_{k}(z))=f_{k-1}(0)=0,
  \end{equation*}
  así que $f_{k}(z)\in Z_{k}(A^{'})$. Definamos
  $f_{*k}:H_{k}(A)\rightarrow H_{k}(A^{'})$ por
  \begin{equation}
    \label{trans-lin-homologias}
    f_{*k}(z+B_{k}(A))=f_{k}(z)+B_{k}(A^{'}).
  \end{equation}
  
  Primero debemos mostrar que $f_{*k}$ está bien definida, es decir,
  independientemente de la elección de un representante de
  $z+B_{k}(A).$ Supongamos que $z_{1}\in (z+B_{k}(A)).$ Entonces
  $(z_{1}-z)\in B_{k}(A)$, así que existe $c\in A_{k+1}$ tal que
  $z_{1}-z=\partial_{k+1}(c).$ Pero entonces
  $$f_{k}(z_{1})-f_{k}(z)=f_{k}(z_{1}-z)=f_{k}(\partial_{k+1}(c))=\partial^{'}_{k+1}(f_{k+1}(c))$$
  este último término es un elemento de
  $\partial^{'}_{k+1}[A^{'}_{k+1}]=B_{k}(A^{'}).$ Por lo tanto
  $$f_{k}(z_{1})\in (f_{k}(z)+B_{k}(A^{'}))$$
  Así dos representantes de la misma clase lateral en
  $H_{k}(A)=Z_{k}(A)/B_{k}(A)$ son enviados en representantes de la misma
  clase lateral en $H_{k}(A^{'})=Z_{k}(A^{'})/B_{k}(A^{'}).$ Esto muestra
  que $f_{*k}:H_{k}(A)\rightarrow H_{k}(A^{'})$ está bien definida por
  la ecuación $(\ref{trans-lin-homologias})$.
  
  Demostrar que $f_{*k}$ es una transformación lineal se sigue de la
  linealidad de $f$.
\end{proof}
Notemos que si $f_{k}$ es un isomorfismo, entonces $f_{k}$ induce un
isomorfismo $f_{*k}$, es decir, $H_{k}(A)\cong H_{k}(A^{'})$.

\chapter{Gráficas}
%\label{cha:primer-capitulo}
% Una \textbf{gráfica} $G$ consiste de un conjunto finito no vacío $V$
% de $p$ \emph{vértices} junto con un conjunto preescrito $X$ de $q$ pares no
% ordenados de puntos distintos de $V$. Cada par $x=\{u,v\}$ de puntos
% en $X$ es una \emph{arista} de $G$, y se dice que $x$ une a $u$ y a
% $v$. Escribimos a $x=uv$ y decimos que $u$ y $v$ son vértices
% adyacentes.

% Completa, maximal
% \begin{definition}
%   Sea $S$ un conjunto y $F=\{S_{1},\cdots,S_{p}\}$ una familia de
%   subconjuntos distintos no vacíos de $S$ cuya unión es $S$. La
%   \textbf{gráfica de intersección} de $F$ es denotada
% \end{definition}

\section{Gráficas de Clanes}

\chapter{Homologías del complejo de emparejamiento}
%\label{cha:primer-capitulo}
\section{Complejo de emparejamiento}

\begin{definition}
Consideremos la gráfica completa de $n$ vértices $K_{n}$, tales
vértices son etiquetados como $1,2,\ldots,n$, además
$\overline{ij}$ denotará la arista que une al vértice $i$ con el
vértice $j$ (donde $\overline{ij}=\overline{ji}$). Llamaremos \textbf{complejo de emparejamiento} de orden
$n$ al complejo simplicial $M_{n}$ de dimensión $n$ tal que:

\begin{enumerate}
  \item Su conjunto de vértices $V$ consta de las aristas de la gráfica
  $K_{n}$. 
  \item Si $v_{i}=\overline{pq}$ y $v_{j}=\overline{rs}$ están en
  $V$, $\{v_{i},v_{j}\}$ es  un 1-simplejo de $M_{n}$ si $v_{i}$
  y $v_{j}$ son ajenas.
\end{enumerate} 
La \textbf{gráfica de emparejamiento} de $K_{n}$ será denotada por $G_{n}$.
\end{definition}

\begin{example}
Considérese la gráfica $K_{4}$; construiremos el complejo de
emparejamiento $M_{4}$ dado por el conjunto de vértices
$$V=\{\overline{12},\overline{13},\overline{14},\overline{23},\overline{24},\overline{34}\},$$
que es el conjunto de aristas de la gráfica de $K_{4}$. La familia de
$1$-simplejos estará dada por el siguiente conjunto:
$$\{\{\overline{12},\overline{34}\},\{\overline{13},\overline{24}\},\{\overline{14},\overline{23}\}\}.$$ 
Es decir,
\begin{equation*}
  M_{4}=\{\{\overline{12}\},\{\overline{13}\},\{\overline{14}\},\{\overline{23}\},\{\overline{24}\},\{\overline{34}\},\{\overline{12},\overline{34}\},\{\overline{13},\overline{24}\},\{\overline{14},\overline{23}\}\}.
\end{equation*}

\begin{center}
  \begin{minipage}{0.3\linewidth}
    \centering
    \begin{tikzpicture}[x=0.8 cm,y=0.8 cm]
      \draw[help lines] (-2,0);% grid (0,2);
      \GraphInit[vstyle=Classic] \SetUpVertex[MinSize=1pt]
      \Vertex[x=-2,y=0,Math,LabelOut,Lpos=180]{2}
      \Vertex[x=0,y=0,Math]{3}
      \Vertex[x=-2,y=2,Math,LabelOut,Lpos=180]{1}
      \Vertex[x=0,y=2,Math]{4} \Edge(1)(2) \Edge(1)(3) \Edge(1)(4)
      \Edge(2)(3) \Edge(2)(4) \Edge(3)(4)
    \end{tikzpicture}
  
    $K_{4}$
  \end{minipage}
  \begin{minipage}{0.3\linewidth}
    \centering
    \begin{tikzpicture}[x=0.8 cm,y=0.8 cm]
      \draw[help lines] (-2,0);% grid (0,2);
      \GraphInit[vstyle=Classic] \SetUpVertex[MinSize=1pt]
      \Vertex[x=-2,y=2,Math,LabelOut,Lpos=90,L=\overline{12}]{12}
      \Vertex[x=-2,y=0,Math,LabelOut,Lpos=-90,L=\overline{34}]{34}
      \Vertex[x=-1,y=0,Math,LabelOut,Lpos=-90,L=\overline{24}]{24}
      \Vertex[x=-1,y=2,Math,LabelOut,Lpos=90,L=\overline{13}]{13}
      \Vertex[x=0,y=2,Math,LabelOut,Lpos=90,L=\overline{14}]{14}
      \Vertex[x=0,y=0,Math,LabelOut,Lpos=-90,L=\overline{23}]{23} \Edge(12)(34)
      \Edge(24)(13) \Edge(14)(23)
    \end{tikzpicture}
  
    $G_{4}$
  \end{minipage}
  % \begin{minipage}{0.3\linewidth}
  %   \centering
  %   \begin{tikzpicture}[x=1 cm,y=0.8 cm]
  %     \draw[help lines] (-2,0);% grid (2,0);
  %     \GraphInit[vstyle=Classic] \SetUpVertex[MinSize=1pt]
  %     \Vertex[x=-2,y=0,Math,LabelOut,Lpos=-90]{12,34} 
  %     \Vertex[x=0,y=0,Math,LabelOut,Lpos=-90]{24,13} 
  %     \Vertex[x=2,y=0,Math,LabelOut,Lpos=-90]{14,23}
  %   \end{tikzpicture}
  
  %   $K(G_{4})$
  % \end{minipage}
\end{center}

Tomando la notación de $p$-simplejos orientados tenemos los espacios
de cadenas:
\begin{equation*}
  C_{0}(M_{4})=\langle(\overline{12}),(\overline{13}),(\overline{14}),(\overline{23}),(\overline{24}),(\overline{34})\rangle.
\end{equation*}
\begin{equation*}
 C_{1}(M_{4})=\langle(\overline{12},\overline{34}),(\overline{13},\overline{24}),(\overline{14},\overline{23})\rangle
\end{equation*}
Sean
\begin{center}
  \begin{tabular}{ccc}
    $a_{1}=\overline{12}$, & $a_{4}=\overline{23}$, & $b_{1}=(\overline{12},\overline{34})$,\\
    $a_{2}=\overline{13}$, & $a_{5}=\overline{24}$, & $b_{2}=(\overline{13},\overline{24})$,\\
    $a_{3}=\overline{14}$, & $a_{6}=\overline{34}$, & $b_{3}=(\overline{14},\overline{23})$,\\
  \end{tabular}
\end{center}
Consideremos a $\beta_{0}=\{a_{1},a_{2},a_{3},a_{4},a_{5},a_{6}\}$ y
$\beta_{1}=\{b_{1},b_{2},b_{3}\}$ como las bases de $C_{0}(M_{4})$ y
$C_{1}(M_{4})$ respectivamente. Notemos que $\dim C_{0}(M_{4})=6$ y
$\dim C_{1}(M_{4})=3$. A continuación se tiene la acción un
representante de cada clase de conjugación del grupo
simétrico $S_{4}$ sobre los elementos de la base de $C_{0}(M_{4})$ y
$C_{1}(M_{4})$.
\begin{center}
  \begin{tabular}{llll}
    $(12)a_{1}=a_{1}$ & $(123)a_{1}=a_{4}$ & $(1234)a_{1}=a_{4}$ & $(12)(34)a_{1}=a_{1}$ \\
    $(12)a_{2}=a_{4}$ & $(123)a_{2}=a_{1}$ & $(1234)a_{2}=a_{5}$ & $(12)(34)a_{2}=a_{5}$ \\
    $(12)a_{3}=a_{5}$ & $(123)a_{3}=a_{5}$ & $(1234)a_{3}=a_{1}$ & $(12)(34)a_{3}=a_{4}$ \\
    $(12)a_{4}=a_{2}$ & $(123)a_{4}=a_{2}$ & $(1234)a_{4}=a_{6}$ & $(12)(34)a_{4}=a_{3}$ \\
    $(12)a_{5}=a_{3}$ & $(123)a_{5}=a_{6}$ & $(1234)a_{5}=a_{2}$ & $(12)(34)a_{5}=a_{2}$ \\
    $(12)a_{6}=a_{6}$ & $(123)a_{6}=a_{3}$ & $(1234)a_{6}=a_{3}$ & $(12)(34)a_{6}=a_{6}$ \\
  \end{tabular}
\end{center}

\begin{center}
  \begin{tabular}{llll}
    $(12)b_{1}=b_{1}$  & $(123)b_{1}=-b_{3}$ & $(1234)b_{1}=-b_{3}$ & $(12)(34)b_{1}=b_{1}$ \\
    $(12)b_{2}=-b_{3}$ & $(123)b_{2}=b_{1}$  & $(1234)b_{2}=-b_{2}$ & $(12)(34)b_{2}=-b_{2}$ \\
    $(12)b_{3}=-b_{2}$ & $(123)b_{3}=-b_{2}$ & $(1234)b_{3}=b_{1}$  & $(12)(34)b_{3}=-b_{3}$ \\
  \end{tabular}
\end{center}
Entonces tenemos las representaciones $\theta_{1}$ y $\theta_{2}$ de
$S_{4}$ en $C_{0}(M_{4})$ y $C_{1}(M_{4})$ respectivamente como se
muestra enseguida:

\begin{center}
  $\theta_{1}(12)= \left(
    \begin{array}{rrrrrr}
      1 & 0 & 0 & 0 & 0 & 0\\
      0 & 0 & 0 & 1 & 0 & 0\\
      0 & 0 & 0 & 0 & 1 & 0\\
      0 & 1 & 0 & 0 & 0 & 0\\
      0 & 0 & 1 & 0 & 0 & 0\\
      0 & 0 & 0 & 0 & 0 & 1\\
    \end{array} 
  \right)$\quad 
  $\theta_{1}(123)= \left(
    \begin{array}{rrrrrr}
      0 & 1 & 0 & 0 & 0 & 0\\
      0 & 0 & 0 & 1 & 0 & 0\\
      0 & 0 & 0 & 0 & 0 & 1\\
      1 & 0 & 0 & 0 & 0 & 0\\
      0 & 0 & 1 & 0 & 0 & 0\\
      0 & 0 & 0 & 0 & 1 & 0\\
    \end{array} 
  \right)$
\end{center}

\begin{center}
  $\theta_{1}(1234)= \left(
    \begin{array}{rrrrrr}
      0 & 0 & 1 & 0 & 0 & 0\\
      0 & 0 & 0 & 0 & 1 & 0\\
      0 & 0 & 0 & 0 & 0 & 1\\
      1 & 0 & 0 & 0 & 0 & 0\\
      0 & 1 & 0 & 0 & 0 & 0\\
      0 & 0 & 0 & 1 & 0 & 0\\
    \end{array} 
  \right)$ \quad
  $\theta_{1}(12)(34)= \left(
    \begin{array}{rrrrrr}
      1 & 0 & 0 & 0 & 0 & 0\\
      0 & 0 & 0 & 0 & 1 & 0\\
      0 & 0 & 0 & 1 & 0 & 0\\
      0 & 0 & 1 & 0 & 0 & 0\\
      0 & 1 & 0 & 0 & 0 & 0\\
      0 & 0 & 0 & 1 & 0 & 1\\
    \end{array} 
  \right)$ 
\end{center}

\begin{center}
  $\theta_{2}(12)= \left(
    \begin{array}{rrr}
      1 & 0 & 0 \\
      0 & 0 & -1 \\
      0 & -1 & 0 \\
    \end{array} 
  \right)$ \quad
  $\theta_{2}(123)= \left(
    \begin{array}{rrr}
      0 & 1 & 0 \\
      0 & 0 & -1 \\
      -1 & 0 & 0 \\
    \end{array} 
  \right)$
\end{center}

\begin{center}
  $\theta_{2}(1234)= \left(
    \begin{array}{rrr}
      0 & 0 & 1 \\
      0 & -1 & 0 \\
      -1 & 0 & 0 \\
    \end{array} 
  \right)$ \quad
  $\theta_{2}(12)(34)= \left(
    \begin{array}{rrr}
      1 & 0 & 0 \\
      0 & -1& 0 \\
      0 & 0 & -1 \\
    \end{array} 
  \right)$
\end{center}
de tal forma que

\begin{tabular}{r r r}
  $\chi_{C_{0}(M_{4})}((1))=6$, & $\chi_{C_{0}(M_{4})}((12))=2$, & $\chi_{C_{0}(M_{4})}((123))=0$, \\
  $\chi_{C_{0}(M_{4})}((1234))=0$, & $\chi_{C_{0}(M_{4})}((12)(34))=2$, & \\
\end{tabular}
\bigskip	

\begin{tabular}{r r r}
  $\chi_{C_{1}(M_{4})}((1))=3$, & $\chi_{C_{1}(M_{4})}((12))=1$, & $\chi_{C_{1}(M_{4})}((123))=0$, \\
  $\chi_{C_{1}(M_{4})}((1234))=-1$, & $\chi_{C_{1}(M_{4})}((12)(34))=-1$. & \\
\end{tabular}
\medskip

Añadiendo estos dos últimos caracteres a la tabla de caracteres de
$S_{4}$ tenemos:
\begin{table}[htpb]
  \centering
  \begin{tabular}{c|r r r r r}
    No. Elementos & 1 & 6 & 8 & 6 & 3 \\
    Clase & (1) & (12) & (123) & (1234) &(12)(34)\\
    \hline
    $\chi_{\mathbb{C}}$ & 1 & 1 & 1 & 1 & 1 \\
    $\chi_{S^{(1,1,1,1)}}$ & 1 & -1 & 1 & -1 & 1\\
    $\chi_{S^{(3,1)}}$ & 3 & 1 & 0 & -1 & -1\\
    $\chi_{S^{(2,1,1)}}$ & 3 & -1 & 0 & 1 & -1 \\
    $\chi_{S^{(2,2)}}$ & 2 & 0 & -1 & 0 & 2 \\
    \hline
    $\chi_{C_{0}(M_{4})}$ & 6 & 2 & 0 & 0 & 2 \\
    $\chi_{C_{1}(M_{4})}$ & 3 & 1 & 0 & -1 & -1
  \end{tabular}

\caption{Tabla de caracteres de $S_{4}$, $C_{0}(M_{4})$ y $C_{1}(M_{4})$.}
\label{tab:S_4}
\end{table}

Queremos escribir a $C_{0}(M_{4})$ como suma directa de módulos
irreducibles de $S_{4}$, así que calculemos el producto interno de $\chi_{C_{0}(M_{4})}$ con los
caracteres de los módulos irreducibles de $S_{4}$ para conocer la
multiplicidad con la que aparecen estos últimos.
(\setlength{\fboxsep}{0pt}\colorbox{green}{COROLARIO(referencia)todavía
no escrito})
\begin{align*}
  \langle\chi_{C_{0}(M_{4})},\chi_{\mathbb{C}}\rangle &=\frac{1}{24}((1)(1\cdot6)+(6)(1\cdot2)+(8)(1\cdot0)+(6)(1\cdot0)+(3)(1\cdot2))\\
  &=\frac{1}{24}(6+12+6)=1\\
  \langle\chi_{C_{0}(M_{4})},\chi_{S^{(1,1,1,1)}}\rangle &=\frac{1}{24}((1)(1\cdot6)+(6)(-1\cdot2)+(8)(1\cdot0)+(6)(-1\cdot0)+(3)(1\cdot2))\\
  &=\frac{1}{24}(6-12+6)=0\\
  \langle\chi_{C_{0}(M_{4})},\chi_{S^{(3,1)}}\rangle &=\frac{1}{24}((1)(3\cdot6)+(6)(1\cdot2)+(8)(0\cdot0)+(6)(-1\cdot0)+(3)(-1\cdot2))\\
  &=\frac{1}{24}(18+12-6)=1\\
  \langle\chi_{C_{0}(M_{4})},\chi_{S^{(2,1,1)}}\rangle &=\frac{1}{24}((1)(3\cdot6)+(6)(-1\cdot2)+(8)(0\cdot0)+(6)(1\cdot0)+(3)(-1\cdot2))\\
  &=\frac{1}{24}(18-12-6)=0\\
  \langle\chi_{C_{0}(M_{4})},\chi_{S^{(2,2)}}\rangle &=\frac{1}{24}((1)(2\cdot6)+(6)(0\cdot2)+(8)(-1\cdot0)+(6)(0\cdot0)+(3)(2\cdot2))\\
  &=\frac{1}{24}(12+12)=1
\end{align*}
De lo anterior se sigue:
\begin{equation}
\label{eq:C_0(M_4)}
C_{0}(M_{4})\cong \mathbb{C}\oplus S^{(3,1)}\oplus S^{(2,2)}
\end{equation}
De la tabla \ref{tab:S_4}
podemos observar que $\chi_{C_{1}(M_{4})}=\chi_{S^{(3,1)}}$, así que  
\begin{equation}
\label{eq:C_1(M_4)}
C_{1}(M_{4})\cong S^{(3,1)}
\end{equation}
Con lo cual tenemos los siguientes complejos de cadenas aumentados
donde $\partial_{1},\partial_{2},\varepsilon$ son los correspondientes
operadores frontera y la función aumento;
$\widehat\partial_{2},\widehat\partial_{1},\widehat\varepsilon$ son
morfismos de módulos. 

\begin{small}
  \[
    \begin{array}{ccccccccccccc}
      \dots 0 & \rightarrow & 0 &
      \stackrel{\partial_{2}}{\rightarrow} & C_{1}(M_{4}) &
      \stackrel{\partial_{1}}{\rightarrow} & C_{0}(M_{4}) & \stackrel{\varepsilon}{\rightarrow} &
      \mathbb{C} & \rightarrow  & 0 & \rightarrow & 0 \dots
    \end{array} 
    \]
  \end{small}

\begin{small}
    \[
    \begin{array}{ccccccccccccc}
      \dots 0 & \rightarrow & 0 &
      \stackrel{\widehat\partial_{2}}{\rightarrow} &  S^{(3,1)} &
      \stackrel{\widehat\partial_{1}}{\rightarrow} & \mathbb{C} \oplus
      S^{(3,1)}\oplus S^{(2,2)} & \stackrel{\widehat\varepsilon}{\rightarrow} &
      \mathbb{C} & \rightarrow  & 0 & \rightarrow & 0 \dots
    \end{array} 
    \]
  \end{small}
Obteniendo el siguiente diagrama conmutativo, donde $f_{1}$ y $f_{0}$
son los isomorfimos obtenidos en las expresiones \ref{eq:C_1(M_4)}, \ref{eq:C_0(M_4)}
 respectivamente: \setlength{\fboxsep}{0pt}\colorbox{green}{(porqué conmuta)}
\begin{figure}[!hbtp]
  \centering
  \[
  \begin{CD}
    0 @>{\partial_{2}}>> C_{1}(M_{4}) @>{\partial_{1}}>> C_{0}(M_{4}) @>{\varepsilon}>> \mathbb{C}\\
    @VVV   @V{f_{1}}VV   @V{f_{0}}VV   @VVV    \\
    0 @>{\widehat\partial_{2}}>> S^{(3,1)} @>{\widehat\partial_{1}}>>
    \mathbb{C} \oplus S^{(3,1)}\oplus S^{(2,2)} @>{\widehat
      \varepsilon}>> \mathbb{C}
  \end{CD}
  \]
  
  \caption{Diagrama conmutativo de los complejos de cadenas de $M_{4}$}
\label{fig:diagrama-conmutativo4}
\end{figure}

Ahora calculemos las homologías reducidas $\widetilde H_{0}(M_{4})$ y
$\widetilde H_{1}(M_{4})$.

Por el teorema \ref{teorema-isomorfismo-mod} y como $\widehat\varepsilon$ es suprayectiva tenemos:
$$\mathbb{C} \oplus S^{(3,1)}\oplus S^{(2,2)}/\ker(\widehat\varepsilon)\cong\im\widehat\varepsilon=\mathbb{C}$$
entonces
\begin{equation}
\label{ker-0-4}
\ker\widehat\varepsilon\cong S^{(3,1)}\oplus S^{(2,2)}
\end{equation}

Por otra parte tenemos que $\im\widehat\partial_{1}\cong S^{(3,1)}$ o $\im\widehat\partial_{1}=0$ por
la proposición \ref{im-mod-irreducible}

Como $\partial_{1}((\overline{12},\overline{34}))=\overline{34}-\overline{12}$
tenemos que $\im\partial_{1}\neq 0$ (en general $\partial_{k}\neq 0$
por definición del operador frontera), se sigue que
$\im\widehat\partial_{1}\neq 0$ puesto que el diagrama es
conmutativo. Por lo tanto
\begin{equation}
\label{im-1-4}
\im\widehat\partial_{1}\cong S^{(3,1)}
\end{equation}
Un razonamiento similar usamos para cálculos
posteriores.

De las expresiones \ref{ker-0-4}, \ref{im-1-4} y de la
proposición \ref{modulos-iguales} concluimos:
\begin{equation}
\label{ker-0-4=}
\ker\widehat\varepsilon=S^{(3,1)}\oplus S^{(2,2)}
\end{equation}
\begin{equation}
\label{im-1-4=}
\im\widehat\partial_{1}=S^{(3,1)}
\end{equation}

Nuevamente por el teorema \ref{teorema-isomorfismo-mod}
$$S^{(3,1)}/\ker\widehat\partial_{1}\cong\im\widehat\partial_{1}= S^{(3,1)} $$
entonces
\begin{equation}
\ker\widehat\partial_{1}=0
\label{ker-1-4}
\end{equation}

$\widehat\partial_{2}$ es un morfismo de módulos, así que
\begin{equation}
\im\widehat\partial_{2}=\widehat\partial_{2}(0)=0
\label{im-2-4}
\end{equation}
Por lo tanto, de la ecuaciones \ref{ker-0-4=}, \ref{im-1-4=},
\ref{ker-1-4} y \ref{im-2-4} obtenemos:
\begin{align*}
\widetilde H_{0}(M_{4})&=\ker \widehat\varepsilon/\im
\widehat\partial_{1}=S^{(3,1)}\oplus S^{(2,2)}/S^{(3,1)}=S^{(2,2)},\\
\widetilde H_{1}(M_{4})&=\ker \widehat\partial_{1}/\im \widehat\partial_{2}=0/0=0.
\end{align*}

\end{example}

\begin{example}
  Considérese el complejo de emparejamiento $M_{5}$, que
  está dado por el conjunto (aristas de la gráfica de $K_{5}$) por el conjunto de vértices:
  $$V=\{a_{1},a_{2},a_{3},a_{4},a_{5},a_{6},a_{7},a_{8},a_{9},a_{10}\}$$
  donde
  \begin{table}[!hbtp]
    \centering
    \begin{tabular}{lllll}
    $a_{1}=\overline{12}$ & $a_{2}=\overline{13}$ & $a_{3}=\overline{14}$ & $a_{4}=\overline{15}$ & $a_{5}=\overline{23}$ \\
    $a_{6}=\overline{24}$ & $a_{7}=\overline{25}$ & $a_{8}=\overline{34}$ & $a_{9}=\overline{35}$ & $a_{10}=\overline{12}$
  \end{tabular}
\end{table}

Entonces la familia de $1$-simplejos orientados estará dada por:

\centering
  \begin{tabular}[h]{lll}
    $b_{1}=(\overline{12},\overline{34})$ & $b_{6}=(\overline{13},\overline{45})$ & $b_{11}=(\overline{15},\overline{24})$  \\
    $b_{2}=(\overline{12},\overline{35})$ & $b_{7}=(\overline{14},\overline{23})$ & $b_{12}=(\overline{15},\overline{34})$  \\
    $b_{3}=(\overline{12},\overline{45})$ & $b_{8}=(\overline{14},\overline{25})$ & $b_{13}=(\overline{23},\overline{45})$  \\
    $b_{4}=(\overline{13},\overline{24})$ & $b_{9}=(\overline{14},\overline{35})$ & $b_{14}=(\overline{24},\overline{35})$  \\
    $b_{5}=(\overline{13},\overline{25})$ & $b_{10}=(\overline{15},\overline{23})$ & $b_{15}=(\overline{25},\overline{34})$  
  \end{tabular}

la gráfica de emparejamiento $G_{5}$ se muestra en la figura \ref{fig:G_5},

\begin{figure}[!hbtp]
  \centering
  \begin{tikzpicture}[rotate=90,scale=0.8]
    \newcommand{\aset}[2]{$\{#1,#2\}$} \GraphInit[vstyle=Classic]
    \SetUpVertex[MinSize=17pt] \SetVertexNoLabel \SetVertexMath
    \grPetersen[RA=3,RB=1.5]
    \AssignVertexLabel{a}{\textsl{$\overline{12}$},\textsl{$\overline{34}$},\textsl{$\overline{15}$},\textsl{$\overline{23}$},\textsl{$\overline{45}$}}
    \AssignVertexLabel{b}{\textsl{$\overline{35}$},\textsl{$\overline{25}$},\textsl{$\overline{24}$},\textsl{$\overline{14}$},\textsl{$\overline{13}$}}
  \end{tikzpicture}
  
  \caption{Complejo de emparejamiento $G_{5}$}
  \label{fig:G_5}
\end{figure}

\begin{table}[!hbtp]
  \centering
  \begin{small}
    \begin{tabular}{c |r r r r r r r}
      No. Elementos& 1 & 10 & 20 & 30 & 24 & 15 & 20  \\
      Clase & (1) & (12) & (123) & (1234) & (12345) & (12)(34) & (123)(45) \\
      \hline
      $\chi_{S^{(5)}}$       & 1 & 1 & 1 & 1 & 1 & 1 & 1 \\
      $\chi_{S^{(1,1,1,1,1)}}$ & 1 & -1 & 1 & -1 & 1 & 1 & -1\\
      $\chi_{S^{(4,1)}}$      & 4 & 2 & 1 & 0 & -1 & 0 & -1\\
      $\chi_{S^{(2,1,1,1)}}$   & 4 & -2 & 1 & 0 & -1 & 0 & 1 \\
      $\chi_{S^{(3,1,1)}}$    & 6 & 0 & 0 & 0 & 1 & -2 & 0 \\
      $\chi_{S^{(3,2)}}$     & 5 & 1 & -1 & -1 & 0 & 1 & 1 \\
      $\chi_{S^{(2,2,1)}}$   & 5 & -1 & -1 & 1 & 0 & 1 & -1 
    \end{tabular}
  \end{small}

  \caption{Tabla de caracteres de $S_{5}$}
  \label{tab:S_5}
\end{table}

$$V_{0}=\langle\overline{45}\rangle$$
$$H_{0}=\{g\in S_{5}\mid
gV_{0}=V_{0}\}=\{\langle(1),(12),(123),(45),(12)(45),(123)(45)\rangle\}$$

\begin{table}[!hbtp]
  \centering
  \begin{tabular}{c |r r r r r r}
  No. Elementos& 1 & 1 & 3 & 3 & 2 & 2 \\
  Clase & (1) & (45) & (12) & (12)(45) & (123) & (123)(45) \\
    \hline
  $\chi_{S^{(5)}}$       & 1 & 1 & 1 & 1 & 1 & 1 \\
  $\chi_{S^{(1,1,1,1,1)}}$ & 1 & -1 & -1 & 1 & 1 & -1 \\
  $\chi_{S^{(4,1)}}$      & 4 & 2 & 2 & 0 & 1 & -1 \\
  $\chi_{S^{(2,1,1,1)}}$   & 4 & -2 & -2 & 0 & 1 & 1 \\
  $\chi_{S^{(3,1,1)}}$     & 6 & 0 & 0 & -2 & 0 & 0 \\
  $\chi_{S^{(3,2)}}$      & 5 & 1 & 1 & 1 & -1 & 1 \\
  $\chi_{S^{(2,2,1)}}$    & 5 & -1 & -1 & 1 & -1 & -1 \\
  \hline
  $\chi_{V_{0}}$ & 1 & 1 & 1 & 1 & 1 & 1 \\
\end{tabular}

\caption{Caracteres de $S_{5}$ restringidos a $H_{0}$ y carácter de $V_{0}$}
\label{tab:restriccion-H_0}
\end{table}

$$V_{1}=\langle(\overline{13},\overline{24})\rangle$$
$$H_{1}=\{g\in S_{5}\mid gV_{1}=V_{1}\}$$

\begin{align*}
  \langle\chi_{C_{0}(M_{5})},\chi_{S^{(5)}}\rangle_{S_{5}}&=\langle\chi_{V_{0}\uparrow^{S_{5}}_{H_0}},\chi_{S^{(5)}}\rangle_{S_{5}}=\langle\chi_{V_{0}},\chi_{S^{(5)}}\downarrow_{H_{0}}\rangle_{H_{0}}\\
  &=\frac{1}{12}(1+1+3+3+2+2)=1\\
  \langle\chi_{C_{0}(M_{5})},\chi_{S^{(1,1,1,1,1)}}\rangle_{S_{5}}&=\langle\chi_{V_{0}\uparrow^{S_{5}}_{H_0}},\chi_{S^{(1,1,1,1,1)}}\rangle_{S_{5}}=\langle\chi_{V_{0}},\chi_{S^{(1,1,1,1,1)}}\downarrow_{H_{0}}\rangle_{H_{0}}\\
  &=\frac{1}{12}(1-1-3+3+2-2)=0 \\
\langle\chi_{C_{0}(M_{5})},\chi_{S^{(4,1)}}\rangle_{S_{5}}&=\langle\chi_{V_{0}\uparrow^{S_{5}}_{H_0}},\chi_{S^{(4,1)}}\rangle_{S_{5}}=\langle\chi_{V_{0}},\chi_{S^{(4,1)}}\downarrow_{H_{0}}\rangle_{H_{0}}\\
  &=\frac{1}{12}(4+2+6+0+2-2)=1 \\
\langle\chi_{C_{0}(M_{5})},\chi_{S^{(2,1,1,1)}}\rangle_{S_{5}}&=\langle\chi_{V_{0}\uparrow^{S_{5}}_{H_0}},\chi_{S^{(2,1,1,1)}}\rangle_{S_{5}}=\langle\chi_{V_{0}},\chi_{S^{(2,1,1,1)}}\downarrow_{H_{0}}\rangle_{H_{0}}\\
  &=\frac{1}{12}(4-2-6+0+2+2)=0 \\
\langle\chi_{C_{0}(M_{5})},\chi_{S^{(3,1,1)}}\rangle_{S_{5}}&=\langle\chi_{V_{0}\uparrow^{S_{5}}_{H_0}},\chi_{S^{(3,1,1)}}\rangle_{S_{5}}=\langle\chi_{V_{0}},\chi_{S^{(3,1,1)}}\downarrow_{H_{0}}\rangle_{H_{0}}\\
  &=\frac{1}{12}(6+0+0-6+0+0)=0 \\
\langle\chi_{C_{0}(M_{5})},\chi_{S^{(3,2)}}\rangle_{S_{5}}&=\langle\chi_{V_{0}\uparrow^{S_{5}}_{H_0}},\chi_{S^{(3,2)}}\rangle_{S_{5}}=\langle\chi_{V_{0}},\chi_{S^{(3,2)}}\downarrow_{H_{0}}\rangle_{H_{0}}\\
  &=\frac{1}{12}(5+1+3+3-2+2)=1 \\
\langle\chi_{C_{0}(M_{5})},\chi_{S^{(2,2,1)}}\rangle_{S_{5}}&=\langle\chi_{V_{0}\uparrow^{S_{5}}_{H_0}},\chi_{S^{(2,2,1)}}\rangle_{S_{5}}=\langle\chi_{V_{0}},\chi_{S^{(2,2,1)}}\downarrow_{H_{0}}\rangle_{H_{0}}\\
  &=\frac{1}{12}(5-1-3+3-2-2)=0 
\end{align*}

\begin{table}[!hbtp]
  \centering
  \begin{tabular}{c |r r r r r}
   & & (24) & (1432) & (14)(23) & \\
  Elementos & (1) & (13) & (1234) & (12)(34) & (13)(24) \\
    \hline
  $\chi_{S^{(5)}}$       & 1 & 1 & 1 & 1 & 1 \\
  $\chi_{S^{(1,1,1,1,1)}}$ & 1 & -1 & -1 & 1 & 1 \\
  $\chi_{S^{(4,1)}}$      & 4 & 2 & 0 & 0 & 0 \\
  $\chi_{S^{(2,1,1,1)}}$   & 4 & -2 & 0 & 0 & 0 \\
  $\chi_{S^{(3,1,1)}}$     & 6 & 0 & 0 & -2 & -2 \\
  $\chi_{S^{(3,2)}}$      & 5 & 1 & -1 & 1 & 1 \\
  $\chi_{S^{(2,2,1)}}$    & 5 & -1 & 1 & 1 & 1 \\
  \hline
  $\chi_{V_{1}}$ & 1 & 1 & -1 & -1 & 1 \\
\end{tabular}

\caption{Caracteres de $S_5$ restringidos a $H_{1}$ y carácter de $V_{1}$}
\label{tab:restriccion-H_1}
\end{table}

\begin{align*}
  \langle\chi_{C_{1}(M_{5})},\chi_{S^{(5)}}\rangle_{S_{5}}&=\langle\chi_{V_{1}\uparrow^{S_{5}}_{H_1}},\chi_{S^{(5)}}\rangle_{S_{5}}=\langle\chi_{V_{1}},\chi_{S^{(5)}}\downarrow_{H_{1}}\rangle_{H_{1}}\\
  &=\frac{1}{8}(1+2-2-2+1)=0\\
  \langle\chi_{C_{1}(M_{5})},\chi_{S^{(1,1,1,1,1)}}\rangle_{S_{5}}&=\langle\chi_{V_{1}\uparrow^{S_{5}}_{H_1}},\chi_{S^{(1,1,1,1,1)}}\rangle_{S_{5}}=\langle\chi_{V_{1}},\chi_{S^{(1,1,1,1,1)}}\downarrow_{H_{1}}\rangle_{H_{1}}\\
  &=\frac{1}{8}(1-2+2-2+1)=0 \\
\langle\chi_{C_{1}(M_{5})},\chi_{S^{(4,1)}}\rangle_{S_{5}}&=\langle\chi_{V_{1}\uparrow^{S_{5}}_{H_1}},\chi_{S^{(4,1)}}\rangle_{S_{5}}=\langle\chi_{V_{1}},\chi_{S^{(4,1)}}\downarrow_{H_{1}}\rangle_{H_{1}}\\
  &=\frac{1}{8}(4+4+0+0+0)=1 \\
\langle\chi_{C_{1}(M_{5})},\chi_{S^{(2,1,1,1)}}\rangle_{S_{5}}&=\langle\chi_{V_{1}\uparrow^{S_{5}}_{H_1}},\chi_{S^{(2,1,1,1)}}\rangle_{S_{5}}=\langle\chi_{V_{1}},\chi_{S^{(2,1,1,1)}}\downarrow_{H_{1}}\rangle_{H_{1}}\\
  &=\frac{1}{8}(4-4+0+0+0)=0 \\
\langle\chi_{C_{1}(M_{5})},\chi_{S^{(3,1,1)}}\rangle_{S_{5}}&=\langle\chi_{V_{1}\uparrow^{S_{5}}_{H_1}},\chi_{S^{(3,1,1)}}\rangle_{S_{5}}=\langle\chi_{V_{1}},\chi_{S^{(3,1,1)}}\downarrow_{H_{1}}\rangle_{H_{1}}\\
  &=\frac{1}{8}(6+0+0+4-2)=1 \\
\langle\chi_{C_{1}(M_{5})},\chi_{S^{(3,2)}}\rangle_{S_{5}}&=\langle\chi_{V_{1}\uparrow^{S_{5}}_{H_1}},\chi_{S^{(3,2)}}\rangle_{S_{5}}=\langle\chi_{V_{1}},\chi_{S^{(3,2)}}\downarrow_{H_{1}}\rangle_{H_{1}}\\
  &=\frac{1}{8}(5+2+2-2+1)=1 \\
\langle\chi_{C_{1}(M_{5})},\chi_{S^{(2,2,1)}}\rangle_{S_{5}}&=\langle\chi_{V_{1}\uparrow^{S_{5}}_{H_1}},\chi_{S^{(2,2,1)}}\rangle_{S_{5}}=\langle\chi_{V_{1}},\chi_{S^{(2,2,1)}}\downarrow_{H_{1}}\rangle_{H_{1}}\\
  &=\frac{1}{8}(5-2-2-2+1)=0 
\end{align*}

\end{example}

Vamos a citar \cite{MR0225619}

\bibliographystyle{plain}
\bibliography{labiblio}

\printindex


\end{document}
