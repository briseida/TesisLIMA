
\documentclass[12pt]{book}

\usepackage[dvipsnames]{xcolor}
\usepackage{amssymb,latexsym}
\usepackage{graphicx}

\usepackage[spanish,mexico,es-nolayout]{babel}
\usepackage[utf8]{inputenc}
\usepackage{amsmath,amscd}
%\usepackage{amssymb}
\usepackage{amsthm}
%\usepackage{graphicx}
\usepackage{color}
\usepackage{tikz}
\usepackage{tkz-berge}
\usepackage{makeidx}
\usepackage{url}
\usepackage{xspace}
\usepackage{tocbibind}
\usepackage{rotating}
\usepackage{young}
% ver http://gilmation.com/articles/latex-margins-for-book-binding/
% y http://tex.stackexchange.com/questions/50258/margins-of-book-class
\usepackage[margin=3.5cm]{geometry}
\geometry{bindingoffset=1cm}

\usepackage{babelbib}

\usetikzlibrary{positioning,shapes,fit,arrows,decorations.pathmorphing}
\definecolor{myblue}{RGB}{56,94,141}


\newtheorem{theorem}{Teorema}[section]
\newtheorem{corollary}[theorem]{Corolario}
\newtheorem{proposition}[theorem]{Proposición}

\theoremstyle{definition}

\newtheorem{definition}[theorem]{Definición}
\newtheorem{notation}[theorem]{Notación}
\newtheorem{example}[theorem]{Ejemplo}
\newtheorem{lemma}[theorem]{Lema}

\DeclareMathOperator{\im}{im}
\DeclareMathOperator{\sgn}{sgn}

\newcounter{in}
\newcounter{ini}

\makeindex

\newcommand{\elespacio}{1.4cm}

\begin{document}
\chapter{Cartel}
%\label{cha:repaso-algebra-line}

\begin{small}
%\begin{sideways}        
 \[
  \begin{CD}
    C_{2}(M_{6}) @>{\partial_{2}}>> C_{1}(M_{6}) @>{\partial_{1}}>> C_{0}(M_{6}) @>{\varepsilon}>> \mathbb{C}\\
    @VVV   @VVV   @VVV   @VVV    \\
    S^{(4,1,1)}\oplus S^{(3,3)} @>{\widehat\partial_{2}}>>
    S^{(5,1)}\oplus S^{(4,1,1)}\oplus S^{(4,2)}\oplus S^{(3,3)}\oplus S^{(3,2,1)} @>{\widehat\partial_{1}}>> 
    \mathbb{C} \oplus S^{(5,1)}\oplus S^{(4,2)} @>{\widehat \varepsilon}>>  \mathbb{C}
  \end{CD}
  \]
%\end{sideways} 
\end{small}

$\widetilde H_{0}(M_{6})=\ker\widehat\varepsilon/\im\widehat\partial_{1}=0$ pues
$M_{6}$ es conexo entonces
$\ker\widehat\varepsilon=\im\widehat\partial_{1}$

$\mathbb{C}$ no pertenece a $\im\widehat\partial_{1}$
$\ker\widehat\varepsilon$ es $\mathbb{C} \oplus S^{(5,1)}\oplus
S^{(4,2)}$ ó $S^{(5,1)}\oplus S^{(4,2)}$, por lo tanto
\begin{align*}
  %\label{eq:2}
   \im\widehat\partial_{1}&=S^{(5,1)}\oplus S^{(4,1,1)}\oplus S^{(4,2)}\oplus S^{(3,3)}\oplus S^{(3,2,1)}/\ker\widehat\partial_{1}\\
   &=S^{(5,1)}\oplus S^{(4,2)}
\end{align*}
entonces 
$$\boldsymbol{\ker\widehat\partial_{1}}=S^{(4,1,1)}\oplus S^{(3,3)}\oplus S^{(3,2,1)}$$

$0=\widetilde H_{2}(M_{6})=\ker\widehat\partial_{2}/\im\widehat\partial_{3}$
asi que $\im\widehat\partial_{3}=0$ pues $C_{3}(M_{6})=0$
por lo tanto $\ker\widehat\partial_{2}=0$\\

por lo tanto 
\begin{align*}
  %\label{eq:3}
  \boldsymbol{\im\widehat\partial_{2}}&=S^{(4,1,1)}\oplus
  S^{(3,3)}/\ker\widehat\partial_{2}\\
  %&=S^{(4,1,1)}\oplus S^{(3,3)}/0\\
  &=S^{(4,1,1)}\oplus S^{(3,3)}
\end{align*}
Así que 
\begin{align}
  \label{eq:4}
  \widetilde
  H_{1}(M_{6})&=\ker\widehat\partial_{1}/\im\widehat\partial_{2}\\
  &=S^{(4,1,1)}\oplus S^{(3,3)}\oplus S^{(3,2,1)}/ S^{(4,1,1)}\oplus S^{(3,3)}\\
  &=S^{(3,2,1)}
\end{align}


\[
\begin{CD}
  F@>{h}>>G\\
@A{i} AA  @A{j}AA\\
X @>>{g}>Y
\end{CD}
\]

%\begin{sideways}                               
  \[
  \begin{CD}
    C_{2}(K(M_{6})) @>{\partial_{2}}>> C_{1}(K(M_{6})) @>{\partial_{1}}>> C_{0}(K(M_{6})) @>{\varepsilon}>> \mathbb{C}\\
    @VVV   @VVV   @VVV   @VVV    \\
    S^{(2,1,1,1,1)}\oplus S^{(3,1,1,1)} @>{\widehat\partial_{2}}>>
    S^{(2,1,1,1,1)}\oplus S^{(3,1,1,1)}\oplus S^{(4,2)}\oplus S^{(2,2,2)}\oplus S^{(3,2,1)} @>{\widehat\partial_{1}}>> 
    \mathbb{C} \oplus S^{(4,2)}\oplus S^{(2,2,2)} @>{\widehat \varepsilon}>>  \mathbb{C}
  \end{CD}
  \]
%\end{sideways}  
  
  $\widetilde H_{0}(K(M_{6}))=\ker\widehat\varepsilon/\im\widehat\partial_{1}=0$, pues
  $K(M_{6})$ es conexo, así que
  $\ker\widehat\varepsilon\cong\im\widehat\partial_{1}$

  $\ker\widehat\varepsilon$ puede ser $S^{(4,2)}\oplus S^{(2,2,2)}$ o $
  \mathbb{C} \oplus S^{(4,2)}\oplus S^{(2,2,2)}$ y $\mathbb{C}$ no
  pertenece a $\im\widehat\partial_{1}$
  
  \begin{align*}
    %\label{eq:1}
      \im\widehat\partial_{1}&=S^{(2,1,1,1,1)}\oplus S^{(3,1,1,1)}\oplus S^{(4,2)}\oplus S^{(2,2,2)}\oplus S^{(3,2,1)}/\ker\widehat\partial_{1}\\
      &=S^{(4,2)}\oplus S^{(2,2,2)}
  \end{align*}
 lo que $\Rightarrow$
 $$\boldsymbol{\ker\widehat\partial_{1}}=S^{(2,1,1,1,1)}\oplus S^{(3,1,1,1)}\oplus S^{(3,2,1)}$$

 $0=\widetilde H_{2}(K(M_{6}))=\ker\widehat\partial_{2}/\im\widehat\partial_{3}$

 $\im\widehat\partial_{3}=0$ pues $C_{3}(K(M_{6}))=0$ lo cual
 $\Rightarrow$ $\ker\widehat\partial_{2}=0$
 \begin{align*}
   %\label{eq:5}
   \boldsymbol{\im\widehat\partial_{2}}&=S^{(2,1,1,1,1)}\oplus
   S^{(3,1,1,1)}/\ker\widehat\partial_{2}\\
   %&=S^{(2,1,1,1,1)}\oplus S^{(3,1,1,1)}/0\\
   &=S^{(2,1,1,1,1)}\oplus S^{(3,1,1,1)}
 \end{align*}

 Así que
 \begin{align*}
    \widetilde H_{1}(K(M_{6}))&=\ker\widehat\partial_{1}/\im\widehat\partial_{2}\\
    &=S^{(2,1,1,1,1)}\oplus S^{(3,1,1,1)}\oplus S^{(3,2,1)}/S^{(2,1,1,1,1)}\oplus S^{(3,1,1,1)}\\
    &=S^{(3,2,1)}
\end{align*}

\begin{theorem}{(Fórmula Hook-length).} 
  Sea $\lambda\vdash n$ sea un diagrama de Young. Entonces
  $$\dim S^{\lambda}=\frac{n!}{\prod_{u\in \lambda}h_{\lambda}(u)}.$$
\end{theorem}
 
\begin{theorem}(Bouc)
  Para todo $n\geq1$ y $k\in \mathbb{Z}$,
  \begin{equation*}
    %\label{eq:6}
    \widetilde H_{k-1}(M_{n})\cong_{S_{n}}
    \bigoplus_{
      \substack{\lambda:\lambda\vdash n\\
        \lambda=\lambda^{'}\\
        r(\lambda)=\mid \lambda \mid-2k}} S^{\lambda}.
  \end{equation*}
  sobre los $\lambda$'s diagramas de Young estándar para $\lambda^{'}$
  el auto-conjugado de $\lambda$ y rango $n-2k$ (el cual puede ser
  obtenido de la fórmula de Hook-length).
\end{theorem}



% \bibliographystyle{plain}

% \bibliography{labiblio}

\printindex

\end{document}
