
\documentclass[12pt]{book}

\usepackage[dvipsnames]{xcolor}
\usepackage{amssymb,latexsym}
\usepackage{graphicx}

\usepackage[spanish,mexico,es-nolayout]{babel}
\usepackage[utf8]{inputenc}
\usepackage{amsmath,amscd}
\usepackage{amssymb}
\usepackage{amsthm}
\usepackage{graphicx}
\usepackage{color}
\usepackage{tikz}
\usepackage{tkz-berge}
\usepackage{makeidx}
\usepackage{url}
\usepackage{xspace}
\usepackage{tocbibind}
\usepackage{rotating}
\usepackage{young}
\usepackage{ytableau}
% ver http://gilmation.com/articles/latex-margins-for-book-binding/
% y http://tex.stackexchange.com/questions/50258/margins-of-book-class
\usepackage[margin=3.5cm]{geometry}
\geometry{bindingoffset=1cm}

\usepackage{babelbib}

\usetikzlibrary{positioning,shapes,fit,arrows,decorations.pathmorphing}
\definecolor{myblue}{RGB}{56,94,141}


\newtheorem{theorem}{Teorema}[section]
\newtheorem{corollary}[theorem]{Corolario}
\newtheorem{proposition}[theorem]{Proposición}

\theoremstyle{definition}

\newtheorem{definition}[theorem]{Definición}
\newtheorem{notation}[theorem]{Notación}
\newtheorem{example}[theorem]{Ejemplo}
\newtheorem{lemma}[theorem]{Lema}

\DeclareMathOperator{\im}{im}
\DeclareMathOperator{\sgn}{sgn}

\newcounter{in}
\newcounter{ini}

\makeindex

\newcommand{\elespacio}{1.4cm}

\begin{document}

\section{Cartel}

Se discute la relación entre tableros de Young y representaciones del
grupo simétrico $S_{n}$. Describimos la construcción de módulos de Specht los
cuales son representaciones irreducibles de $S_{n}$. blah blah...........
\begin{definition}
  Una \textit{partición} de un entero positivo $n$ es una secuencia de
  enteros positivos
  $\lambda=(\lambda_{1},\lambda_{2},\ldots,\lambda_{l})$ que satisface
  $\lambda_{1}\geq\lambda_{2}\geq\cdots\geq\lambda_{l}>0$ y
  $n=\sum^{m}_{i=1}\lambda_{i}$. Para indicar que  $\lambda$ es una partición de
  $n$ lo denotamos como $\lambda\vdash n$.
\end{definition}
Por ejemplo, el número $3$ tiene tres particiones:$(3)$, $(2,1)$, $(1,1,1)$.
\begin{definition}
  Si $\lambda=(\lambda_{1},\lambda_{2},\ldots,\lambda_{l})$ es una
  partición de $n$, entonces el \textit{diagrama de Young} de
  $\lambda$ consiste de $n$ cajas colocadas en $l$
  renglones donde el $i$-ésimo renglón tiene $\lambda_{i}$ cajas.
\end{definition}
Por ejemplo, los diagramas de Young correspondientes a las particiones
del $3$ son: 
\begin{center}
  \ytableausetup{mathmode, boxsize=1em}
  \ydiagram{3}\quad
  \ydiagram{2,1}\quad
  \ydiagram{1,1,1}
  \end{center}
\begin{definition}
  Si $\lambda\vdash n$, un \textit{$\lambda$-tablero} (o tablero de
  Young de forma $\lambda$) se obtiene llenando las cajas de un diagrama
  de Young para $\lambda$ colocando los números $1,2,\ldots,n$ exactamente
  una vez.
\end{definition}
Por ejemplo, enseguida mostramos todos los tableros correspondientes a la
partición $\lambda=(2,1)$:
\begin{center}
\begin{ytableau}
 1 & 2\\
  3
\end{ytableau} \quad
\begin{ytableau}
  2 & 1\\
  3
\end{ytableau}\quad
\begin{ytableau}
  1 & 3\\
  2
\end{ytableau}\quad
\begin{ytableau}
  3 & 1\\
  2
\end{ytableau}\quad
\begin{ytableau}
  2 & 3\\
  1
\end{ytableau}\quad
\begin{ytableau}
  3 & 2\\
  1
\end{ytableau}
\end{center}
\begin{definition}
  Un \textit{tablero estándar (Young)} es $\lambda$-tablero cuyas
  entradas son crecientes en cada renglón y en cada columna.
\end{definition}
Los únicos tableros estándar para $(2,1)$ son:
\begin{center}
  \begin{ytableau}
    1 & 2\\
    3
  \end{ytableau}\quad
  \begin{ytableau}
    1 & 3\\
    2
  \end{ytableau}
\end{center}
La \textbf{estructura cíclica} de una permutación $\pi$ es la
partición cuyas entradas son las longitudes de la descomposición en
ciclos. Por ejemplo $(123)(45)\in S_{5}$ tiene la estructura
cíclica (3,2). 

Dos elementos de $S_{n}$ están conjugados si y solo tienen la misma
estructura cíclica. 

Esto significa que las clases de conjugación de $S_{n}$ están
caracterizadas por la estructura cíclica, y en consecuencia
corresponden a las particiones de $n$, los cuales son equivalentes a
los diagramas de Young de tamaño $n$. 

El número de representaciones irreducibles de un grupo finito es igual
al número de sus clases de conjugación. Así que nuestra meta es
construir una representaciones irreducibles de $S_{n}$ correspondiente a cada diagrama de Young.



$S_{3}$ tiene tres representaciones irreducibles: trivial,
signo y estándar. Además, hay exactamente tres particiones del $3$: $(3)$, $(1,1,1)$,
$(2,1)$. Así que en este caso, las representaciones irreducibles son
exactamente los módulos de Specht. 

\begin{center}
\ydiagram{3}

representación trivial

\ydiagram{1,1,1}

representación signo

\ydiagram{2,1}

representación estándar
\end{center}


\begin{theorem}{(Fórmula Hook-length).} 
  Sea $\lambda\vdash n$ sea un diagrama de Young. Entonces
  $$\dim S^{\lambda}=\frac{n!}{\prod_{u\in \lambda}h_{\lambda}(u)}.$$
\end{theorem}
 
\begin{theorem}(Bouc)
  Para todo $n\geq1$ y $k\in \mathbb{Z}$,
  \begin{equation*}
    %\label{eq:6}
    \widetilde H_{k-1}(M_{n})\cong_{S_{n}}\bigoplus_{\lambda:\lambda\vdash n\\
      \lambda=\lambda^{'}\\r(\lambda)=\mid \lambda \mid-2k} S^{\lambda}.
  \end{equation*}
  sobre los $\lambda$'s diagramas de Young estándar para $\lambda^{'}$
  el auto-conjugado de $\lambda$ y rango $n-2k$ (el cual puede ser
  obtenido de la fórmula de Hook-length).
\end{theorem}


% \bibliographystyle{plain}

% \bibliography{labiblio}

\printindex

\end{document}
