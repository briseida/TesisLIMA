\documentclass[final,xcolor=svgnames]{beamer}
\usetheme{RJH}
\usepackage[orientation=portrait,size=a2,scale=1.4,debug]{beamerposter}
%\usepackage[orientation=portrait,size=custom,width=30,height=50,scale=1,debug]{beamerposter}
\usepackage[absolute,overlay]{textpos}
\setlength{\TPHorizModule}{1cm}
\setlength{\TPVertModule}{1cm}

\usepackage[T1]{fontenc}

\usefonttheme{professionalfonts}
\useinnertheme[shadow]{rounded}
%\usecolortheme{orchid}

%\usepackage{arev}
\usepackage{tikz}
\usepackage{tkz-graph}
\usepackage{tkz-berge}
\usepackage{ytableau}
%\usepackage{tkz-berge-add}

\usepackage[spanish,mexico,es-nolayout]{babel}
\usepackage[utf8]{inputenc}
\usepackage{amsmath,amscd}
\DeclareMathOperator{\sgn}{sgn}

\title{%\fontfamily{ptm}\fontseries{b}\fontsize{78}{80}\selectfont
  Representaciones en homologías}
\author{%\fontfamily{ppl}\fontseries{b}\fontsize{68}{70}\selectfont
  Briseida Trejo Escamilla}
\footer{\texttt{btbrisi@hotmail.com}}
\date{}

\begin{document}
\begin{frame}{}
  \begin{block}{Resumen}
    \renewcommand{\VertexLineWidth}{3pt}
    \renewcommand{\EdgeLineWidth}{3pt}
    \centering
    \begin{minipage}{0.15\linewidth}
      \begin{tikzpicture}[rotate=90]
        \GraphInit[vstyle=Hasse]
        \grPetersen[RA=2,RB=1]
      \end{tikzpicture}
    \end{minipage}
    \begin{minipage}{0.6\linewidth}
      \centering
      %\fontfamily{ptm}\fontseries{b}\fontsize{48}{50}\selectfont
      \begin{scriptsize}
        Se discute la relación entre tableros de Young y representaciones del
        grupo simétrico $S_{n}$. Describimos la construcción de módulos de Specht los
        cuales son representaciones irreducibles de $S_{n}$. blah blah...........
      \end{scriptsize}
    \end{minipage}
    \begin{minipage}{0.15\linewidth}
      \begin{tikzpicture}[rotate=90]
        \GraphInit[vstyle=Hasse]
        \grPrism[RA=2,RB=1]{5}
      \end{tikzpicture}
    \end{minipage}
  \end{block}

  \vfill
  
  \begin{columns}
    \begin{column}{0.5\textwidth}
      \centering
      \begin{block}{Representaciones de grupos}
        \begin{scriptsize}
          \begin{itemize}         
          \item La \textit{representación trivial} de $S_{3}$ es el homomorfismo
          $\rho:S_{3}\rightarrow \mathbb{C^{*}}$ dado por $\rho(g)=1$ para todo
          $g\in S_{3}$.
          
        \item Si $\pi=\tau_{1}\tau_{2}\cdots\tau_{k}$, donde $\tau_{i}$ son
          transposiciones, definimos la  \textit{función signo}
          $\sgn:\mathcal{S}_{n} \rightarrow\{\pm1\}$
          mediante $$\sgn(\pi):=(-1)^{k}.$$
          Con acción:
          \begin{center}
          $\sigma x=$
          \left\{
            \begin{array}{rl}
              x  & \mbox{ si $\sigma$ es par}, \\
              -x & \mbox{ si $\sigma$ es impar}, 
            \end{array}\right
          \end{center}

          para $\sigma \in S_{n}$, $x\in \mathbb{C}$. Esta representación de $S_{n}$ se
          llama \textit{representación signo.} 
        \item  Sea $E=\left\{(x_{1},x_{2},x_{3})\in
            \mathbb{C}^{3} \mid x_{1}+x_{2}+x_{3}=0\}\right$, con
          acción $\sigma(x_{1},x_{2},x_{3})=(x_{\sigma(1)},x_{\sigma(2)},x_{\sigma(3)})$,
          esta es la \textit{representación estándar}
        \item  Las representaciones irreducibles de un grupo finito
          están en correspondencia biyectiva con las clases de
          conjugación de sus elementos.
        \end{itemize}
      \end{scriptsize}
    \end{block}
      
      \begin{block}{Partición}
        \begin{scriptsize}
          Una \textit{partición} de un entero positivo $n$ es una secuencia de
          enteros positivos
          $\lambda=(\lambda_{1},\lambda_{2},\ldots,\lambda_{l})$ que satisface
          $\lambda_{1}\geq\lambda_{2}\geq\cdots\geq\lambda_{l}>0$ y
          $n=\sum^{m}_{i=1}\lambda_{i}$. Para indicar que  $\lambda$ es una partición de
          $n$ lo denotamos como $\lambda\vdash n$.

          Por ejemplo, el número $3$ tiene tres particiones:$(3)$,
          $(2,1)$, $(1,1,1)$. 
        \end{scriptsize}
      \end{block}

      \begin{block}{Diagrama de Young}
        \begin{scriptsize}
          Si $\lambda=(\lambda_{1},\lambda_{2},\ldots,\lambda_{l})$ es una
          partición de $n$, entonces el \textit{diagrama de Young} de
          $\lambda$ consiste de $n$ cajas colocadas en $l$
          renglones donde el $i$-ésimo renglón tiene $\lambda_{i}$ cajas.
          
          Del ejemplo anterior, los diagramas de Young correspondientes son:
          \begin{center}
            \ytableausetup{mathmode, boxsize=1em}
            \ydiagram{3}\quad
            \ydiagram{2,1}\quad
            \ydiagram{1,1,1}
          \end{center}
           \end{scriptsize}
      \end{block}

      \begin{block}{Tablero de Young}
        \begin{scriptsize}
          Si $\lambda\vdash n$, un \textit{$\lambda$-tablero} (o tablero de
          Young de forma $\lambda$) se obtiene llenando las cajas de un diagrama
          de Young para $\lambda$ colocando los números $1,2,\ldots,n$ exactamente
          una vez.
          Por ejemplo, enseguida mostramos todos los tableros correspondientes a la
          partición $\lambda=(2,1)$:
          \begin{center}
            \begin{ytableau}
              1 & 2\\
              3
            \end{ytableau} \quad
            \begin{ytableau}
              2 & 1\\
              3
            \end{ytableau}\quad
            \begin{ytableau}
              1 & 3\\
              2
            \end{ytableau}\quad
            \begin{ytableau}
              3 & 1\\
              2
            \end{ytableau}\quad
            \begin{ytableau}
              2 & 3\\
              1
            \end{ytableau}\quad
            \begin{ytableau}
              3 & 2\\
              1
            \end{ytableau}
          \end{center}
        \end{scriptsize}
      \end{block}

%      \begin{block}{Tablero estándar}
%        \begin{scriptsize}      
%          Un \textit{tablero estándar (Young)} es $\lambda$-tablero cuyas
%          entradas son crecientes en cada renglón y en cada columna.
%          Los únicos tableros estándar para $(2,1)$ son:
%          \begin{center}
%            \begin{ytableau}
%              1 & 2\\
%              3
%            \end{ytableau}\quad
%            \begin{ytableau}
%              1 & 3\\
%              2
%            \end{ytableau}
%          \end{center}
%        \end{scriptsize}
%      \end{block}

      \begin{block}{Desordenado}
        \begin{scriptsize}
          La \textbf{estructura cíclica} de una permutación $\pi$ es la
          partición cuyas entradas son las longitudes de la descomposición en
          ciclos. Por ejemplo $(123)(45)\in S_{5}$ tiene la estructura
          cíclica $(3,2)$. 

          Dos elementos de $S_{n}$ están conjugados si y solo tienen la misma
          estructura cíclica.

          %Por ejemplo $(12)$ y $(23)$ están conjugados pues
          %su estructura cíclica es $(2,1)$ 

         % Nuestra meta es construir representaciones irreducibles de
         % $S_{n}$ correspondientes a cada diagrama de Young.

          El grupo $S_{n}$ actúa en los tableros de Young de forma
          natural. Por ejemplo: 
          \begin{center}(23)
            \ytableausetup{mathmode, boxsize=1em}
            \begin{ytableau}
                1 & 2  \\
                3 \\
              \end{ytableau}
              =
              \begin{ytableau}
                1 & 3  \\
                2 \\
              \end{ytableau}
            \end{center}
          \end{scriptsize}
      \end{block}

      \begin{block}{Grupo columna}
        \begin{scriptsize}
          El \textit{grupo columna} $C_{t}$ es el
          subgrupo de $S_{n}$ que consiste de las permutaciones las cuales
          solo permutan los elementos entre cada columna de $t$.
          Por ejemplo, si
          \begin{center}$t=$
            \ytableausetup{mathmode,boxsize=1em}  
            \begin{ytableau}
              4 & 1 & 2\\
              3 & 5
            \end{ytableau}
          \end{center}
          $$C_{t}=S_{\{3,4\}}\times S_{\{1,5\}}\times S_{\{2\}}$$
        \end{scriptsize}
      \end{block}
    \end{column}

    \begin{column}{0.5\textwidth}
      \begin{block}{Politabloides}
        \begin{scriptsize}
          Si $t$ es un tablero, entonces el \textit{politabloide} asociado es
          $$e_{t}=\sum_{\pi\in C_{t}}\sgn(\pi)\pi\{t\}.$$
          Por ejemplo, si
          \begin{center}$t=$
            \ytableausetup{mathmode, boxsize=1em}
            \begin{ytableau}
              1 & 2 \\
              3
            \end{ytableau}\quad
            $e_{t}=$
            \ytableausetup{mathmode, boxsize=1em,tabloids}
            \ytableaushort{12,3}
            $-$ \ytableaushort{32,1}
          \end{center}
        \end{scriptsize}
      \end{block}     
 %\begin{block}{Lema}
       % \begin{scriptsize}
        %  Sea $t$ un tablero y $\pi$ una permutación. Entonces $e_{\pi t}=\pi e_{t}$.
         %\end{scriptsize}
      %\end{block}
      \begin{block}{Módulo de Specht}
        \begin{scriptsize}
          Para cualquier partición $\lambda$, el correspondiente
          \textit{módulo de Specht}, denotado $S^{\lambda}$, es el
          módulo generado por los politabloides $e_{t}$, donde $t$ es tomado sobre todos los tableros de forma $\lambda$.
        \end{scriptsize}
      \end{block}

      \begin{block}{Ejemplo}
        \begin{scriptsize}
            Considere $\lambda=(3)$. Hay un único politablide, es decir,
            \begin{center}
              \ytableausetup{mathmode, boxsize=1em,tabloids}    
              \ytableaushort{123} 
            \end{center}
            El politabloide está fijo por $S_{3}$, así que $S^{(3)}$ es la
            \textit{representación trivial}.
        
            Sea $\lambda=(1^{3})=(1,1,1)$. Sea
            \begin{center}$t=$
              \ytableausetup{mathmode, boxsize=1.5em,notabloids}
              \begin{ytableau}
                1\\
                2\\
                3
              \end{ytableau}
            \end{center}
            Para cualquier $\lambda$-tablero $t^{'}$,
            $e_{t}=e_{t^{'}}$ si $t^{'}$ es obtenida de $t$ mediante
            una permutación par, o $e_{t}=-e_{t^{'}}$ su $t^{'}$ es
            obtenida de $t$ mediante una permutación impar. Tenemos $\pi
            e_{t}=e_{\pi t}=\sgn(\pi)e_{t}$. Así que $S^{(1^{3})}$ es la
            \textit{representación signo}.
         
         Ahora sea $\lambda=(2,1)$, $\{t_{i}\}$ denota el $\lambda$-tabloide
         con $i$ en el segundo renglón, vemos que los politabloides tienen la
         forma  $\{t_{i}\}- \{t_{j}\}$. En efecto, el politabloide construido
         de el tablero
         \begin{center}$t=$
           \ytableausetup{mathmode, boxsize=1.5em,notabloids}
           \begin{ytableau}
             j & a & b \\
             i\\
           \end{ytableau}
         \end{center}
         es igual a $\{t_{i}\}- \{t_{j}\}$. Usemos temporalmente
         $\boldsymbol{e_{i}}$ para denotar el tabloide $\{t_{i}\}$. Entonces
         $S^{\lambda}$ es generado por los elementos de la forma
         $\boldsymbol{e_{i}}-\boldsymbol{e_{j}}$, y se sigue que
         $$S^{(2,1)}=\{c_{1}\boldsymbol{e_{1}}+c_{2}\boldsymbol{e_{2}}+c_{3}\boldsymbol{e_{3}}|c_{1}+c_{2}+c_{3}\}$$ 
         Esta es la representación es \textit{representación estándar}. 
       \end{scriptsize}
     \end{block}
     \begin{block}{lo que sigue}
       \begin{scriptsize}         
         Los módulos de Specht $S^{\lambda}$ para $\lambda\vdash n$ forman
         una lista completa de representaciones irreducibles de $S_{n}$ sobre $\mathbb{C}$
         
%$S_{3}$ tiene tres representaciones irreducibles: trivial,
%signo y estándar. Además, hay exactamente tres particiones del $3$: $(3)$, $(1,1,1)$,
%$(2,1)$. Así que en este caso, las representaciones irreducibles son
%exactamente los módulos de Specht. 

%\begin{center}
%\ydiagram{3}

%representación trivial

%\ydiagram{1,1,1}

%representación signo

%\ydiagram{2,1}

%representación estándar
%\end{center}

        \end{scriptsize}
      \end{block}

      \begin{block}{Bouc}
        \begin{scriptsize}
          Para todo $n\geq1$ y $k\in \mathbb{Z}$,
          \begin{equation*}
            % \label{eq:6}
            \widetilde H_{k-1}(M_{n})\cong_{S_{n}}\bigoplus_{\lambda:\lambda\vdash n\\
              \lambda=\lambda^{'}\\r(\lambda)=\mid \lambda \mid-2k} S^{\lambda}.
          \end{equation*}
          sobre los $\lambda$'s diagramas de Young estándar para $\lambda^{'}$
          el auto-conjugado de $\lambda$ y rango $n-2k$ (el cual puede ser
          obtenido de la fórmula de Hook-length).
        \end{scriptsize}
      \end{block}
    \end{column}
  \end{columns}

  \vfill

  \begin{block}{Blah blah}
    \begin{tiny}
    \[
    \begin{CD}
      C_{2}(M_{6}) @>{\partial_{2}}>> C_{1}(M_{6}) @>{\partial_{1}}>> C_{0}(M_{6}) @>{\varepsilon}>> \mathbb{C}\\
      @VVV   @VVV   @VVV   @VVV    \\
      S^{(4,1,1)}\oplus S^{(3,3)} @>{\widehat\partial_{2}}>>
      S^{(5,1)}\oplus S^{(4,1,1)}\oplus S^{(4,2)}\oplus S^{(3,3)}\oplus S^{(3,2,1)} @>{\widehat\partial_{1}}>> 
      \mathbb{C} \oplus S^{(5,1)}\oplus S^{(4,2)} @>{\widehat \varepsilon}>>  \mathbb{C}
    \end{CD}
    \]
    \end{tiny}
  \end{block}
\end{frame}
\end{document}
